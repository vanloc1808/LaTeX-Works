\section{Nội dung}
Trong mọi hoàn cảnh, nhất là trong điều kiện đảng cầm quyền, Bác Hồ nhấn mạnh tám chữ, đó là cần, kiệm, liêm, chính, chí công vô tư với nội dung đổi mới và cách mạng. Đây là những phẩm chất đạo đức cốt lõi đối với tất cả mọi người. \cite{soiduong} Vì vậy, Người đã nhiều lần đề cập đến phẩm chất này, xuyên suốt từ \textit{Đường kách mệnh} đến \textit{Di chúc}.\\
Người chỉ rõ: "Bọn phong kiến ngày xưa nêu ra cần, kiệm, liêm, chính nhưng không bao giờ làm mà lại bắt nhân dân phải tuân theo để phụng sự quyền lợi cho chúng. Ngày nay, ta đề ra cần, kiệm, liêm, chính cho cán bộ thực hiện làm gương cho nhân dân theo để lợi cho nước cho dân. \cite{HCMtt7}\\
Như vậy, ta có thể thấy cần, kiệm, liêm, chính, chí công vô tư cũng là một biểu hiện cụ thể của phẩm chất "trung với nước, hiếu với dân". \cite{syllabus}
Vậy theo tư tưởng của Người, cần, kiệm, liêm, chính, chí công vô tư là gì?

\subsection{Cần, kiệm, liêm, chính, chí công vô tư là gì?}
"Cần, kiệm, liêm, chính, chí công vô tư" là những khái niệm cũ trong đạo đức truyền thống của dân tộc, được Chủ tịch Hồ Chí Minh lọc bỏ những nội dung không phù hợp và đưa vào những nội dung mới đáp ứng yêu cầu của cách mạng. \cite{syllabus}

\subsubsection{Cần}
\textit{Cần} tức là siêng năng, chăm chỉ, cố gắng dẻo dai. \cite{HCMtt6}\\
Muốn cho chữ \textit{Cần} có nhiều kết quả hơn, thì phải có kế hoạch cho mọi công việc \cite{HCMtt6}.\\
Cần tức là lao động cần cù, siêng năng; lao động có kế hoạch, sáng tạo, có năng suất cao; lao động với tinh thần tự lực cánh sinh, không lười biếng. Phải thấy rõ "Lao động là \textit{nghĩa vụ thiêng liêng, là nguồn sống, nguồn hạnh phúc của chúng ta}"\cite{HCMtt13}.\\

\subsubsection{Kiệm}
"\textit{Kiệm} là thế nào? Là tiết kiệm, không xa xỉ, không hoang phí, không bừa bãi" \cite{HCMtt6}. Kiệm tức là tiết kiệm sức lao động, tiết kiệm thì giờ, tiết kiệm tiền của của dân, của nước, của bản thân mình; không phô trương hình thức, không liên hoan chè chén lu bù.\\
"\textit{Tiết kiệm} không phải là bủn xỉn. Khi không nên tiêu xài thì một đồng xu cũng không nên tiêu. Khi có việc đáng làm, việc ích lợi cho đồng bào, cho Tổ quốc, thì dù bao nhiêu công, tốn bao nhiêu của, cũng vui lòng. Như thế mới đúng là \textit{kiệm}. Việc đáng tiêu mà không tiêu, là bủn xỉn, chứ không phải là kiệm. Tiết kiệm phải kiên quyết \textit{không xa xỉ}" \cite{HCMtt6}.\\
"Cần với kiệm, phải đi đôi với nhau, như hai chân của con người" \cite{HCMtt6}. Hồ Chí Minh yêu cầu \textit{"Phải cần kiệm xây dựng nước nhà"} \cite{HCMtt13}.

\subsubsection{Liêm}
Liêm "là trong sạch, không tham lam" \cite{HCMtt6}; là liêm khiết, "luôn luôn tôn trọng giữ gìn của công, của dân" \cite{syllabus}.\\
"Liêm là không tham địa vị. Không tham tiền tài. Không tham sung sướng. Không ham người tâng bốc mình. Vì vậy mà quang minh chính đại, không bao giờ hủ hóa. Chỉ có một thứ ham là ham học, ham làm, ham tiến bộ" \cite{HCMtt5}.\\
"Chữ Liêm phải đi đôi với chữ Kiệm. Cũng như chữ Kiệm phải đi đôi với chữ Cần. Có Kiệm mới Liêm được" \cite{HCMtt6}.

\subsubsection{Chính}
"\textit{Chính} nghĩa là không tà, nghĩa là thẳng thắn, đứng đắn. Điều gì không đứng đắn, thẳng thắn, tức là tà" \cite{HCMtt6}.\\
Chính được thể hiện rõ trong ba mối quan hệ: "Đối với mình - Chớ tự kiêu, tự đại". "Đối với người:... Chớ nịnh hót người trên. Chớ xem khinh người dưới. Thái độ phải chân thành, khiêm tốn, ... Phải thực hành chữ Bác - Ái" \cite{HCMtt6}. "Đối với việc: Phải để công việc nước lên trên, trước việc tư, việc nhà" \cite{HCMtt6}. "Việc \textit{thiện} thì dù nhỏ mấy cũng làm. Việc \textit{ác} thì dù nhỏ mấy cũng tránh" \cite{HCMtt6}.\\
Hồ Chí Minh cho rằng, các đức tính cần, kiệm, liêm, chính có quan hệ chặt chẽ với nhau, ai cũng phải thực hiện, song cán bộ, đảng viên phải là người thực hành trước để làm kiểu mẫu cho dân. Người thường nhắc nhở cán bộ, công chức, những người trong các công sở đều có nhiều hoặc ít quyền hạn. Nếu không giữ đúng cần, kiệm, liêm, chính thì dễ trở nên hủ bại, biến thành sâu mọt của dân. \cite{syllabus}\\
Khẳng định cần, kiệm, liêm, chính là bốn đức tính căn bản nhất của con người, Chủ tịch Hồ Chí Minh viết:
\begin{center}
\textit{"Trời có bốn mùa: Xuân, Hạ, Thu, Đông.\\
Đất có bốn phương: Đông, Tây, Nam, Bắc.\\
Người có bốn đức: Cần, Kiệm, Liêm, Chính.\\
Thiếu một mùa, thì không thành trời.\\
Thiếu một phương, thì không thành đất.\\
Thiếu một đức, thì không thành người"\cite{HCMtt5}.}
\end{center}
Với quan niệm như trên, Người đã đảo ngược quan niệm của xã hội phong kiến, đưa ra quan niệm mới về cần, kiệm, liêm, chính và tạo ra thế vững chắc cho nền đạo đức mới, như người có hai chân đứng vững dưới đất, đầu ngẩng lên trời. \cite{soiduong}

\subsubsection{Chí công vô tư}
\textit{Chí công vô tư} là hoàn toàn vì lợi ích chung, không vì tư lợi; là hết sức công bằng, không chút thiên tư, thiên vị, công tâm, luôn đặt lợi ích của Đảng, của nhân dân, của dân tộc lên trên hết, trước hết; chỉ biết vì Đảng, vì dân tộc, "lo trước thiên hạ, vui sau thiên hạ". Chí công vô tư là chống chủ nghĩa cá nhân. Người nói: "Đem lòng chí công vô tư mà đối với người, với việc" \cite{HCMtt5}; "khi làm bất cứ việc gì cũng đừng nghĩ đến mình trước, ... khi hưởng thụ thì mình nên đi sau" \cite{HCMtt11}.\\
Về thực chất, chí công vô tư là sự tiếp nối của cần, kiệm, liêm, chính. Bác Hồ giải thích: "Trước nhất là cán bộ các cơ quan, các đoàn thể, cấp cao thì quyền to, cấp thấp thì quyền nhỏ. Dù to hay nhỏ, có quyền mà thiếu lương tâm là có dịp đục khoét, có dịp ăn của đút, có dịp "dĩ công vi tư". Vì vậy, cán bộ phải thực hành chữ \textit{Liêm} trước, để làm kiểu mẫu cho dân." \cite{HCMtt6}

\subsection{Học tập cần, kiệm, liêm, chính, chí công vô tư}
Học tập Người là ta phải tu dưỡng, rèn luyện theo tấm gương cần, kiệm, liêm, chính, chí công vô tư; đức khiêm tốn, trung thực của Người.\\
Hồ Chí Minh thường dạy cán bộ, đảng viên, đoàn viên và thanh niên phải cần, kiệm, liêm, chính, chí công vô tư, ít lòng ham muốn vật chất, đó là tư cách của Người cán bộ, và tự Người $-$ Người cũng đã gương mẫu thực hiện.\\
Theo Hồ Chí Minh, "Muốn hoàn thành nhiệm vụ được tốt thì chúng ta phải học tập, chúng ta phải trau dồi tư tưởng, ... phải trau dồi đạo đức cách mạng, trước hết là \textit{đức khiêm tốn}" \cite{HCMtt10}. "Khiêm tốn là một đạo đức mà mọi người cách mạng phải luôn luôn trau dồi" \cite{HCMtt8}; phải chân thành, khiêm tốn, không được tự mãn, chớ kiêu ngạo, luôn luôn cầu tiến bộ, phải \textit{"khiêm tốn, trong sạch và chính trực"} \cite{HCMtt7}. Bản thân Người là một tấm gương sáng ngời về trung thực, trách nhiệm với mình, với người, với việc, thể hiện trong tư tưởng và lẽ sống của Người.\\
Học tập Người, mọi người dân Việt Nam cần phải có ý thức dân tộc, nhưng trước hết là lớp trẻ, tương lai của đất nước. Trong lớp trẻ ấy, đặc biệt sinh viên phải có sự vun đắp tinh thần dân tộc, ý thức trách nhiệm với Tổ quốc thân yêu. Trên cơ sở có ý thức đúng đắn, tự giác, tích cực thực hiện trách nhiệm của mình là "có tinh thần trách nhiệm cao"
\cite{syllabus}.

\subsection{Vận dụng cần, kiệm, liêm, chính, chí công vô tư trong cuộc sống như thế nào?}
Chủ tịch Hồ Chí Minh đã dạy: "Một dân tộc biết cần, kiệm, biết liêm, là một dân tộc giàu về vật chất, mạnh về tinh thần, là một dân tộc văn minh tiến bộ" \cite{HCMtt6}. Cần, kiệm, liêm, chính là nền tảng của đời sống mới, của các phong trào thi đua yêu nước.\\
Tiếp thu lời dạy của Bác, thế hệ trẻ, đặc biệt là sinh viên, hôm nay luôn cố gắng tích cực trau dồi bản lĩnh chính trị, đạo đức cách mạng để đủ phẩm chất, trí tuệ phụng sự Tổ quốc, phục vụ nhân dân, thông qua ý thức "cần, kiệm" từ những công việc, nhiệm vụ hằng ngày. \cite{Hanh2019}\\
Học tập Người, mỗi sinh viên cần phải thấm nhuần giá trị tư tưởng đạo đức Hồ Chí Minh. Điều này không chỉ dừng ở việc đọc sách lý thuyết suông mà cần phải được chứng minh bằng hành động, bằng thực tế. Ta có thể học Bác từ những việc nhỏ nhất như tắt điện khi không sử dụng, có ý thức giữ gìn vệ sinh, bảo vệ của công, chăm chỉ ra sức học tập để mai sau xây dựng Tổ quốc... \\
Học tập đức tính \textit{cần} của Người, mỗi sinh viên khi còn ngồi trên giảng đường đại học cần tích cực học tập, nâng cao kiến thức, cả kiến thức chuyên môn và kiến thức xã hội, không học để đối phó, mà học với tinh thần tiếp thu kiến thức cho mình.\\
Học tập đức tính \textit{kiệm} của Bác, sinh viên cần biết tiết kiệm thời gian, dành thời gian vào những việc có ích, giúp phát triển bản thân, giúp đỡ cộng đồng như: tham gia các hoạt động tình nguyện, các hoạt động thể dục thể thao, tìm kiếm thông tin, nâng cao kiến thức, ... Ngoài ra, cần phải biết tiết kiệm tiền bạc, không keo kiệt nhưng cũng không xa xỉ.\\
Hiện nay, xã hội đang trên đà phát triển nhanh chóng, kéo theo đó là nhiều đòi hỏi với con người, và một trong số đó là đức tính \textit{liêm}. Khi còn là sinh viên, không nên nịnh hót người khác, cũng không nên ưa người khác nịnh hót mình, dẫn đến sự xem thường những người xung quanh. Ta cần sống một cuộc sống giản dị nhưng không thiếu những thứ quan trọng.\\
Để rèn luyện chữ \textit{chính}, chúng ta nên sống đúng với khả năng của mình, không nhận những thứ không thuộc về mình hoặc không phải do mình cố gắng mà được, ví dụ như khi kiểm tra, thi cử, không nên gian lận mà phải làm bằng chính khả năng của mình.\\
Đối với \textit{chí công vô tư}, ta cần kiên quyết chống chủ nghĩa cá nhân, chỉ biết vun vén cho bản thân mà không chăm lo cho nhân dân, cho đất nước. Cần phải thẳng thắn, trung thực, bảo vệ những giá trị tốt đẹp, kiên quyết chống cái xấu. Phải kiên quyết \textit{quét sạch chủ nghĩa cá nhân, nâng cao đạo đức cách mạng}, bồi dưỡng tư tưởng tập thể, tinh thần đoàn kết, tính tổ chức và tính kỷ luật. \citep{nangcao}\\
Ngoài ra, học tập Người, chúng ta cần rèn luyện thói quen nói đi đôi với làm, không chỉ nói suông mà phải chứng minh bằng hành động. Đơn cử một ví dụ, để chống nạn đói năm 1945, Người đã kêu gọi: “Lúc chúng ta nâng bát cơm mà ăn, nghĩ đến kẻ đói khổ, chúng ta không khỏi động lòng. Vậy tôi xin đề nghị với đồng bào cả nước, và tôi xin thực hành trước: Cứ 10 ngày nhịn ăn một bữa, mỗi tháng nhịn 3 bữa. Đem gạo đó (mỗi bữa một bơ) để cứu dân nghèo. Như vậy, thì những người nghèo sẽ có bữa rau, bữa cháo để chờ mùa lúa năm sau, khỏi đến nỗi chết đói” \cite{HCMtt4}. Không chỉ gương mẫu nhịn ăn, để dành gạo cho vào hũ gạo cứu đói, Người còn bán chiếc áo lụa đồng bào tặng lấy tiền mua áo ấm tặng cho chiến sỹ trong mùa đông giá rét và đem số tiền tiết kiệm ít ỏi vốn là tiền nhuận bút của mình để mua nước ngọt tặng cho các chiến sỹ trực phòng không trong những ngày hè nóng bức năm 1967. \cite{Mai2019}\\
Chủ tịch Hồ Chí Minh mãi mãi là tấm gương sáng ngời về đạo đức, đặc biệt là về cần, kiệm, liêm, chính, chí công vô tư. Cuộc đời Người là tập hợp những mẩu chuyện về lối sống giản dị, luôn vì dân, vì nước, để lại cho nhân dân ta một di sản khổng lồ để noi theo.\\
Ngày nay, khi đất nước bước vào giai đoạn mới của thời kỳ Đổi mới, đòi hỏi sự đoàn kết, thống nhất cao, "\textit{cán bộ, đảng viên và đoàn viên thanh niên} càng phải giữ vững lập trường giai cấp, nâng cao tinh thần trách nhiệm, trau dồi đạo đức cách mạng, quyết tâm hoàn thành nhiệm vụ mà Đảng và Nhà nước đã giao cho" \cite{doanket}.