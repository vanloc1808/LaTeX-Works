\chapter{Pháp luật $-$ Công cụ điều chỉnh các mối quan hệ xã hội}
\begin{ques}
Chỉ pháp luật mới mang tính quy phạm.
\end{ques}
\begin{ans}
\textbf{Sai.} Ngoài quy phạm pháp luật, ta còn các các quy phạm khác như quy phạm tôn giáo, quy phạm đạo đức.
\end{ans}

\begin{ques}
Ngôn ngữ pháp lý rõ ràng, chính xác thể hiện tính cưỡng chế nhà nước của pháp luật.
\end{ques}
\begin{ans}
\textbf{Sai.} Tính cưỡng chế của pháp luật thể hiện qua quyền lực nhà nước bằng các tổ chức trấn áp công an, quân đội, nhà tù
\end{ans}

\begin{ques}
Chỉ pháp luật mới mang tính cưỡng chế.
\end{ques}
\begin{ans}
\textbf{Sai.} Tính cưỡng chế của pháp luật mang bản chất quyền lực chính trị. Có thể có các thực thể khác cũng mang tính cưỡng chế.
\end{ans}

\begin{ques}
Chỉ pháp luật mới mang tính cưỡng chế nhà nước.
\end{ques}
\begin{ans}
\textbf{Đúng.} Pháp luật là hệ thồng các quy tắc xử sự chung (general rules of conduct) do nhà nước ban hành (hoặc thừa nhận) để điều chỉnh các quan hệ xã hội phù hợp với ý chí của giai cấp thống trị và được nhà nước đảm bảo thực hiện bằng quyền lực nhà nước.
\end{ans}

\begin{ques}
Tập quán pháp là những tập quán thông thường của người dân trong cuộc sống hàng ngày.
\end{ques}
\begin{ans}
\textbf{Sai.} Tập quán pháp là những tập quán thông thường của người dân trong cuộc sống hàng ngày, phù hợp với lợi ích của giai cấp thống trị, được nhà nước thừa nhận và áp dụng.
\end{ans}

\begin{ques}
Tập quán pháp không được thừa nhận tại Việt Nam.
\end{ques}
\begin{ans}
\textbf{Sai.} Điều 5 Bộ luật Dân sự 2015 quy định về việc áp dụng tập quán:\\
\textit{1. Tập quán là quy tắc xử sự có nội dung rõ ràng để xác định quyền, nghĩa vụ của cá nhân, pháp nhân trong quan hệ dân sự cụ thể, được hình thành và lặp đi lặp lại nhiều lần trong một thời gian dài, được thừa nhận và áp dụng rộng rãi trong một vùng, miền, dân tộc, cộng đồng dân cư hoặc trong một lĩnh vực dân sự.}\\
\textit{2. Trường hợp các bên không có thỏa thuận và pháp luật không quy định thì có thể áp dụng tập quán nhưng tập quán áp dụng không được trái với các nguyên tắc cơ bản của pháp luật dân sự quy định tại Điều 3 của Bộ luật này.}\\
\textbf{Nghĩa là tập quán pháp được thừa nhận tại Việt Nam.}
\end{ans}

\begin{ques}
Tiền lệ pháp không được thừa nhận tại Việt Nam.
\end{ques}
\begin{ans}
\textbf{Sai.} Khoản 3 điều 45 Bộ luật Tố tụng Dân sự 2015 quy định về việc áp dụng các nguyên tắc cơ bản của pháp luật dân sự, án lệ, lẽ công bằng, trong đó quy định về việc áp dụng án lệ như sau:\\
\textit{Án lệ được Tòa án nghiên cứu, áp dụng trong giải quyết vụ việc dân sự khi đã được Hội đồng Thẩm phán Tòa án nhân dân tối cao lựa chọn và được Chánh án Tòa án nhân dân tối cao công bố.}\\
\textbf{Như vậy, án lệ được thừa nhận tại Việt Nam, có nghĩa là tiền lệ pháp được thừa nhận tại Việt Nam.}
\end{ans}

\begin{ques}
Văn bản quy phạm pháp luật là hình thức pháp luật duy nhất được thừa nhận tại Việt Nam.
\end{ques}
\begin{ans}
\textbf{Sai.} Pháp luật Việt Nam thừa nhận ba hình thức:
\begin{itemize}
\item Tập quán pháp (legal practices): phong tục, tập quán lưu truyền trong xã hội, phù hợp với lợi ích của giai cấp thống trị.
\item Tiền lê pháp (legal precedents): nhà nước thùa nhận các bản án của Toàn án nhân dân hoặc quyết định của cơ quan hành chính nhà nước.
\item Văn bản quy phạm pháp luật (legislative documents): cơ quan nhà nước có thẩm quyền ban hành dưới hình thức văn bản (pháp luật thành văn).
\end{itemize}
\end{ans}

\begin{ques}
Mọi quy tắc xử sự tồn tại trong xã hội có nhà nước là quy phạm pháp luật.
\end{ques}
\begin{ans}
\textbf{Sai.} Quy phạm pháp luật là những quy tắc xử sự \textbf{có tính bắt buộc chung} do \textbf{Nhà nước đặt ra} và bảo đảm thực hiện, thể hiện ý ch, lợi ích của giai cấp thống trị và nhu cầu tồn tại của xã hội, nhằm điều chỉnh các quan hệ xã hội, tạo lập trật tự, ổn định cho sự phát triển của xã hội.
\end{ans}

\begin{ques}
Nhà nước ban hành pháp luật để điều chỉnh mọi quy tắc ứng xử của người dân trong cuộc sống hằng ngày.
\end{ques}
\begin{ans}
\textbf{Sai.} Pháp luật là hệ thồng các quy tắc xử sự chung (general rules of conduct) do nhà nước ban hành (hoặc thừa nhận) để điều chỉnh các \textbf{quan hệ xã hội phổ biến} phù hợp với ý chí của giai cấp thống trị và được nhà nước đảm bảo thực hiện bằng quyền lực nhà nước. 
\end{ans}

\begin{ques}
Một quy phạm pháp luật bắt buộc phải có ba bộ phận là giả định, quy định, chế tài.
\end{ques}
\begin{ans}
\textbf{Sai.} Có những quy phạm pháp luật chỉ có giả định và chế tài hoặc giả định và quy định.\\
Ví dụ, khoản 1 điều 9 Luật giao thông đường bộ 2008 quy định: \textit{Người tham gia giao thông phải đi bên phải theo chiều đi của mình, đi đúng làn đường, phần đường quy định và phải chấp hành hệ thống báo hiệu đường bộ.} Quy phạm pháp luật này chỉ gồm giả định và quy định.
\end{ans}

\begin{ques}
Văn bản do cơ quan nhà nước ban hành là văn bản quy phạm pháp luật.
\end{ques}
\begin{ans}
\textbf{Sai.} Văn bản quy phạm pháp luật là một hình thức văn bản do \textbf{cơ quan nhà nước có thẩm quyền} ban hành hoặc phối hợp ban hành theo \textbf{thẩm quyền, hình thức, trình tự, thủ tục luật định,} trong đó có các quy tắc xử sự chung mang tính bắt buộc chung, được nhà nước bảo đảm thực hiện nhằm điều chỉnh các quan hệ xã hội cơ bản và được áp dụng nhiều lần trong đời sống xã hội.
\end{ans}

\begin{ques}
Văn bản quy phạm pháp luật do các cơ quan nhà nước, cá nhân, tổ chức ban hành nhằm điều chỉnh các quan hệ xã hội.
\end{ques}
\begin{ans}
\textbf{Sai.} Văn bản quy phạm pháp luật là một hình thức văn bản do \textbf{cơ quan nhà nước có thẩm quyền} ban hành hoặc phối hợp ban hành theo \textbf{thẩm quyền, hình thức, trình tự, thủ tục luật định,} trong đó có các quy tắc xử sự chung mang tính bắt buộc chung, được nhà nước bảo đảm thực hiện nhằm điều chỉnh các quan hệ xã hội cơ bản và được áp dụng nhiều lần trong đời sống xã hội.
\end{ans}

\begin{ques}
Văn bản chứa đựng quy tắc xử sự chung cho mọi người là văn bản quy phạm pháp luật.
\end{ques}
\begin{ans}
\textbf{Sai.} Văn bản quy phạm pháp luật là một hình thức văn bản do \textbf{cơ quan nhà nước có thẩm quyền} ban hành hoặc phối hợp ban hành theo \textbf{thẩm quyền, hình thức, trình tự, thủ tục luật định,} trong đó có các quy tắc xử sự chung mang tính bắt buộc chung, được nhà nước bảo đảm thực hiện nhằm điều chỉnh các quan hệ xã hội cơ bản và được áp dụng nhiều lần trong đời sống xã hội.
\end{ans}

\begin{ques}
Mọi cơ quan nhà nước đều có quyền ban hành văn bản quy phạm pháp luật.
\end{ques}
\begin{ans}
\textbf{Sai.} Chỉ có cơ quan nhà nước có thẩm quyền mới có quyền ban hành văn bản quy phạm pháp luật theo quy định của Luật ban hành văn bản quy phạm pháp luật 2015 mới được ban hành hoặc phối hợp ban hành văn bản quy phạm pháp luật theo \textbf{thẩm quyền, hình thức, trình tự, thủ tục luật định.}
\end{ans}

\begin{ques}
Chỉ có các cơ quan nhà nước mới có quyền ban hành văn bản quy phạm pháp luật.
\end{ques}
\begin{ans}
\textbf{Sai.} Chỉ có cơ quan nhà nước có thẩm quyền mới có quyền ban hành văn bản quy phạm pháp luật theo quy định của Luật ban hành văn bản quy phạm pháp luật 2015 mới được ban hành hoặc phối hợp ban hành văn bản quy phạm pháp luật theo \textbf{thẩm quyền, hình thức, trình tự, thủ tục luật định.}
\end{ans}

\begin{ques}
Hệ thống văn bản quy phạm pháp luật bao gồm Hiến pháp và các văn bản dưới luật.
\end{ques}
\begin{ans}
\textbf{Sai.} Hệ thống văn bản quy phạm pháp luật bao gồm \textbf{văn bản quy phạm pháp luật có giá trị luật} và \textbf{văn bản quy phạm pháp luật có giá trị dưới luật.}\\
Theo điều 4 Luật ban hành văn bản quy phạm pháp luật 2015:\\
Văn bản quy phạm pháp luật có giá trị luật bao gồm:
\begin{itemize}
\item Hiến pháp
\item Bộ luật, luật, nghị quyết của Quốc hội.
\end{itemize}
Văn bản quy phạm pháp luật có giá trị dưới luật bao gồm:
\begin{itemize}
\item Pháp lệnh, nghị quyết của Ủy ban thường vụ Quốc hội; nghị quyết liên tịch giữa Ủy ban thường vụ Quốc hội với Đoàn Chủ tịch Ủy ban trung ương Mặt trận Tổ quốc Việt Nam.
\item Lênh, quyết định của Chủ tịch nước.
\item Nghị định của Chính phủ; nghị quyết liên tịch giữa Chính phủ với Đoàn Chủ tịch Ủy ban trung ương Mặt trận Tổ quốc Việt Nam.
\item Quyết định của Thủ tướng Chính phủ.
\item Nghị quyết của Hội đồng Thẩm phán Tòa án nhân dân tối cao.
\item Thông tư của Chánh án Tòa án nhân dân tối cao; thông tư của Viện trưởng Viện kiểm sát nhân dân tối cao; thông tư của Bộ trưởng, Thủ trưởng cơ quan ngang Bộ; thông tư liên tịch giữa Chánh án Tòa án nhân dân tối cao với Viện trưởng Viện kiểm sát nhân dân tối cao; thông tư liên tịch giữa Bộ trưởng, Thủ trưởng cơ quan ngang Bộ với Chánh án Tòa án nhân dân tối cao, Viện trưởng Viện kiểm sát nhân dân tối cao; quyết định của Tổng Kiểm toán nhà nước.
\item Nghị quyết của Hội đồng nhân dân tỉnh, thành phố trực thuộc trung ương (gọi chung là cấp tỉnh).
\item Quyết định của Ủy ban nhân dân cấp tỉnh.
\item Văn bản quy phạm pháp luật của chính quyền địa phương ở đơn vị hành chính $-$ kinh tế đặc biệt.
\item Nghị quyết của Hội đồng nhân dân huyện, quận, thị xã, thành phố thuộc tỉnh, thành phố trực thuộc trung ương (gọi chung là cấp huyện).
\item Quyết định của Ủy ba nhân dân cấp huyện.
\item Nghị quyết của Hội đồng nhân dân xã, phường, thị trấn (gọi chung là cấp xã).
\item Quyết định của Ủy ban nhân dân cấp xã.
\end{itemize}
\end{ans}

\begin{ques}
Quốc hội là cơ quan duy nhất có thẩm quyền ban hành văn bản quy phạm pháp luật.
\end{ques}
\begin{ans}
\textbf{Sai.} Ngoài Quốc hội ban hành Hiến pháp, bộ luật, luật, nghị quyết thì các cơ quan nhà nước có thẩm quyền ban hành văn bản quy phạm pháp luật bao gồm: Ủy ban thường vụ Quốc hội; Chính phủ; Đoàn Chủ tịch Ủy ban trung ương Mặt trận Tổ quốc Việt Nam; Hội đồng Thẩm phán Tòa án nhân dân tối cao; Hội đồng nhân dân, Ủy ban nhân dân cấp tỉnh; Hội đồng nhân dân, Ủy ban nhân dân cấp huyện; Hội đồng nhân dân, Ủy ban nhân dân cấp xã và các cơ quan khác theo quy định của Luật ban hành văn bản quy phạm pháp luật 2015.
\end{ans}

\begin{ques}
Quốc hội là cơ quan duy nhất có thẩm quyền ban hành văn bản luật.
\end{ques}
\begin{ans}
\textbf{Đúng.} Vì các văn bản quy phạm có giá trị luật bao gồm: \textbf{Hiến pháp, bộ luật, luật, nghị quyết do Quốc hội ban hành}. Các văn bản này đều do Quốc hội ban hành.
\end{ans}

\begin{ques}
Văn bản dưới luật là những văn bản pháp luật do Quốc hội và các cơ quan nhà nước khác có thẩm quyền ban hành.
\end{ques}
\begin{ans}
\textbf{Sai.} Văn bản quy phạm pháp luật có giá trị dưới luật là những văn bản do các cơ quan nhà nước khác Quốc hội có thẩm quyền như: \textbf{Ủy ban thường vụ Quốc hội, Chính phủ, Tòa án nhân dân tối cao, Viện kiểm sát nhân dân tối cao, Hội đồng nhân dân, Ủy ban nhân dân các cấp, ...} ban hành 
\end{ans}

\begin{ques}
Văn bản dưới luật có giá trị pháp lý thấp hơn văn bản luật.
\end{ques}
\begin{ans}
\textbf{Đúng.} Vì các văn bản quy phạm pháp luật có giá trị dưới luật phải \textbf{tuân thủ quy định, không được quy định trái} với các văn bản quy phạm pháp luật có giá trị luật (Hiến pháp, bộ luật, luật, nghị quyết do Quốc hội ban hành).
\end{ans}

\begin{ques}
Các văn bản dưới luật có giá trị pháp lý tương đương nhau.
\end{ques}
\begin{ans}
\textbf{Sai.} Các văn bản quy phạm pháp luật có giá trị dưới luật không có giá trị pháp lý tương đương nhau. Ví dụ: Thông tư của Bộ trưởng, Thủ trưởng cơ quan ngang Bộ có giá trị pháp lý thấp hơn Nghị định của Chính phủ.
\end{ans}

\begin{ques}
Việc ban hành Luật Thủ Đô thuộc thẩm quyền của Ủy ban nhân dân thành phố Hà Nội.
\end{ques}
\begin{ans}
\textbf{Sai.} Hiến pháp, bộ luật, luật đều thuộc thẩm quyền ban hành của \textbf{Quốc hội.}
\end{ans}

\begin{ques}
Hiến pháp, Luật, Nghị quyết của Quốc hội, Quyết định của Chủ tịch nước, Nghị định của Chính phủ là những văn bản luật.
\end{ques}
\begin{ans}
\textbf{Sai.} Văn bản luật bao gồm: Hiến pháp, bộ luật, luật, nghị quyết của Quốc hội.\\
Quyết định của Chủ tịch nước, Nghị định của Chính phủ là những \textbf{văn bản dưới luật}, không phải văn bản luật.
\end{ans}

\begin{ques}
Văn bản luật là văn bản chứa đựng các quy phạm pháp luật do các cơ quan, tổ chức cá nhân có thẩm quyền ban hành nhằm điều chỉnh các quan hệ xã hội theo ý chí của nhà nước.
\end{ques}
\begin{ans}
\textbf{Sai.} Văn bản quy phạm pháp luật có giá trị luật gồm: Hiến pháp, bộ luật, luật, nghị quyết của Quốc hội. Tất cả các văn bản này đều do \textbf{Quốc hội ban hành.}
\end{ans}

\begin{ques}
Chính phủ là cơ quan duy nhất có thẩm quyền ban hành \textbf{\textit{văn bản pháp luật}} là Nghị định.
\end{ques}
\begin{ans}
\textbf{Đúng.} Trong các văn bản quy phạm pháp luật (kể cả có giá trị luật và có giá trị dưới luật) thì Nghị định chỉ do Chính phủ ban hành.
\end{ans}

\begin{ques}
Quốc hội là cơ quan duy nhất có thẩm quyền ban hành \textbf{\textit{văn bản pháp luật}} là Nghị quyết.
\end{ques}
\begin{ans}
\textbf{Sai.} Các văn bản quy phạm pháp luật là Nghị quyết có thể do Ủy ban thường vụ Quốc hội, Hội đồng Thẩm phán Tòa án nhân dân tối cao, Hội đồng nhân dân các cấp ban hành.
\end{ans}

\begin{ques}
Quốc hội là cơ quan duy nhất có thẩm quyền ban hành \textbf{\textit{văn bản luật}} là Nghị quyết.
\end{ques}
\begin{ans}
\textbf{Đúng.} Văn bản luật là các văn bản quy phạm pháp luật có giá trị luật, bao gồm: Hiến pháp, bộ luật, luật, nghị quyết của Quốc hội.\\
Vậy Quốc hội là cơ quan duy nhất có thẩm quyền ban hành văn bản luật là Nghị quyết.
\end{ans}

\begin{ques}
Nghị quyết do các cơ quan, tổ chức, cá nhân có thẩm quyền ban hành là văn bản luật.
\end{ques}
\begin{ans}
\textbf{Sai.} Nghị quyết do Ủy ban thường vụ Quốc hội, Hội đồng nhân dân các cấp ban hành là văn bản dưới luật, không phải là văn bản luật.
\end{ans}

\begin{ques}
Các quan hệ nảy sinh trong cuộc sống hằng ngày là quan hệ pháp luật.
\end{ques}
\begin{ans}
\textbf{Sai.} Quan hệ pháp luật là các quan hệ xã hội \textbf{được các quy phạm pháp luật điều chỉnh.}\\
Quan hệ pháp luật là quan hệ giữa người với người do một quy phạm pháp luật điều chỉnh, biểu hiện thành quyền và nghĩa vụ pháp lý cụ thể của mỗi bên, được đảm bảo thực hiện bằng quyền cưỡng chế nhà nước.
\end{ans}

\begin{ques}
Mọi quan hệ nảy sinh trong cuộc sống hằng ngày đều chịu sự chi phối của pháp luật.
\end{ques}
\begin{ans}
\textbf{Sai.} Có những quan hệ xã hội do đạo đức hay tôn giáo điều chỉnh.\\
Chỉ có quan hệ pháp luật mới chịu sự chi phối của pháp luật.
\end{ans}

\begin{ques}
Chỉ quan hệ pháp luật mới mang tính ý chí của chủ thể tham gia.
\end{ques}
\begin{ans}
\textbf{Sai.} Các quan hệ xã hội khác cũng mang tính ý chí của chủ thể tham gia.
\end{ans}

\begin{ques}
Nếu không có quy phạm pháp luật điều chỉnh thì không có quan hệ pháp luật.
\end{ques}
\begin{ans}
\textbf{Đúng.} Vì quan hệ pháp luật là các quan hệ xã hội được các quy phạm pháp luật điều chỉnh, nếu không có quy phạm pháp luật thì quan hệ pháp luật sẽ không tồn tại.
\end{ans}

\begin{ques}
Năng lực chủ thể của các cá nhân khi tham gia vào một quan hệ pháp luật là giống nhau.
\end{ques}
\begin{ans}
\textbf{Sai.} Năng lực chủ thể của các cá nhâm khi tham gia vào một quan hệ pháp luật gồm:
\begin{itemize}
\item Năng lực pháp luật: là khả năng của chủ thể được hưởng quyền và nghĩa vụ pháp lý theo quy định của pháp luật, giống nhau giữa các cá nhân.
\item Năng lực hành vi: là khả năng của chủ thể được nhà nước xác nhận trong quy phạm pháp luật cụ thể, chủ thể thực hiện các quyền và nghĩa vụ pháp lý và độc lập chịu trách nhiệm pháp lý, \textbf{có thể khác nhau giữa các cá nhân tùy theo quy định của pháp luật.}
\end{itemize}
\end{ans}

\begin{ques}
Năng lực pháp luật của các cá nhân khác nhau là không giống nhau.
\end{ques}
\begin{ans}
\textbf{Sai.} Năng lực pháp luật là khả năng của chủ thể được hưởng quyền và nghĩa vụ pháp lý theo quy định của pháp luật, \textbf{giống nhau giữa các cá nhân.}
\end{ans}

\begin{ques}
Năng lực hành vi của các cá nhân khác nhau là không giống nhau tùy thuộc vào độ tuổi của họ.
\end{ques}
\begin{ans}
\textbf{Đúng.} Năng lực hành vi là khả năng của chủ thể được nhà nước xác nhận trong quy phạm pháp luật cụ thể, chủ thể thực hiện các quyền và nghĩa vụ pháp lý và độc lập chịu trách nhiệm pháp lý, \textbf{tùy thuộc vào độ tuổi, các cá nhân sẽ có năng lực hành vi khác nhau.}
\end{ans}

\begin{ques}
Năng lực hành vi có từ khi cá nhân được sinh ra và chỉ mất đi khi họ đã chết.
\end{ques}
\begin{ans}
\textbf{Sai.} \textbf{Năng lực pháp luật} của cá nhân xuất hiện kể từ khi cá nhân sinh ra và chỉ mất đi khi cá nhân đó chết.\\
Năng lực hành vi chỉ có khi một cá nhân đạt đến độ tuổi nhất định, có thể \textbf{bị mất đi khi cá nhân không có khả năng nhận thức và điều khiển hành vi} của mình khi tham gia quan hệ pháp luật.
\end{ans}

\begin{ques}
Năng lực hành vi của cá nhân có từ khi cá nhân đủ 18 tuổi.
\end{ques}
\begin{ans}
\textbf{Sai.} Các cá nhân từ 6 đến 18 tuổi có năng lực hành vi một phần, tùy theo luật định.
\end{ans}

\begin{ques}
Người từ đủ 18 tuổi trở lên mới có năng lực pháp luật đầy đủ.
\end{ques}
\begin{ans}
\textbf{Sai.} Năng lực pháp luật là khả năng của chủ thể được hưởng quyền và nghĩa vụ pháp lý theo quy định của pháp luật, \textbf{xuất hiện khi cá nhân sinh ra và chỉ mất đi khi cá nhân đó chết}.
\end{ans}

\begin{ques}
Người từ đủ 18 tuổi trở lên là người có năng lực hành vi dân sự đầy đủ.
\end{ques}
\begin{ans}
\textbf{Sai.} Người từ đủ 18 tuổi trở lên và \textbf{không bị tòa án tuyên bố mất năng lực hành vi dân sự}, có khó khăn trong nhận thức, làm chủ hành vi, hạn chế năng lực hành vi dân sự thì mới có năng lực hành vi dân sự đầy đủ.
\end{ans}

\begin{ques}
Người dưới 18 tuổi là người có năng lực hành vi dân sự hạn chế.
\end{ques}
\begin{ans}
\textbf{Sai.} Người dưới 6 tuổi không có năng lực hành vi dân sự, người từ 6 đến 18 tuổi có năng lực hành vi dân sự một phần.
\end{ans}

\begin{ques}
Người dưới 18 tuổi là người có năng lực hành vi dân sự một phần.
\end{ques}
\begin{ans}
\textbf{Sai.} Người dưới 6 tuổi không có năng lực hành vi dân sự, người từ 6 đến 18 tuổi có năng lực hành vi dân sự một phần.
\end{ans}

\begin{ques}
Người bị khiếm thính, khiếm thị là người hạn chế năng lực hành vi dân sự.
\end{ques}
\begin{ans}
\textbf{Sai.} Điều 24 Bộ luật Hình sự 2015 quy định người hạn chế năng lực hành vi dân sự: Người nghiện ma túy, nghiện các chất kích thích khác dẫn đến phá tán tài sản của gia đình thì theo yêu cầu của người có quyền, lợi ích liên quan hoặc của cơ quan, tổ chức hữu quan, Tòa án có thể ra quyết định tuyên bố người này là người bị hạn chế năng lực hành vi dân sự. \\
\textbf{Không có quy định người khiếm thính, khiếm thị bị hạn chế năng lực hành vi dân sự.}
\end{ans}

\begin{ques}
Người uống rượu bia say là người bị hạn chế năng lực hành vi dân sự.
\end{ques}
\begin{ans}
\textbf{Sai.} Điều 24 Bộ luật Hình sự 2015 quy định người hạn chế năng lực hành vi dân sự: Người nghiện ma túy, nghiện các chất kích thích khác dẫn đến phá tán tài sản của gia đình thì theo yêu cầu của người có quyền, lợi ích liên quan hoặc của cơ quan, tổ chức hữu quan, Tòa án có thể ra quyết định tuyên bố người này là người bị hạn chế năng lực hành vi dân sự. \\
\textbf{Không có quy định người uống rượu bia say là người bị hạn chế năng lực hành vi dân sự.}
\end{ans}

\begin{ques}
Người từ đủ 18 tuổi trở lên bị bệnh tâm thần là người không có năng lực hành vi dân sự.
\end{ques}
\begin{ans}
\textbf{Sai.} Người dưới 6 tuổi là người không có năng lực hành vi dân sự.
\end{ans}

\begin{ques}
Tư cách pháp nhân là tư cách con người theo quy định pháp luật của mọi tổ chức được thành lập trên lãnh thổ Việt Nam.
\end{ques}
\begin{ans}
\textbf{Sai.} Chỉ có \textbf{tổ chức} đủ 4 điều kiện theo quy định tại khoản 1 điều 74 Bộ luật Dân sự 2015 mới có tư cách pháp nhân.\\
\textbf{Điều 74. Pháp nhân}\\
1. Một tổ chức được công nhận là pháp nhân khi có đủ các điều kiện sau đây:\\
a. Được thành lập theo quy định của Bộ luật này, luật khác có liên quan;\\
b. Có cơ cấu tổ chức theo quy định của điều 83 Bộ luật này;\\
c. Có tài sản độc lập với cá nhân, pháp nhân khác và tự chịu trách nhiệm bằng tài sản của mình;\\
d. Nhân danh mình tham gia quan hệ pháp luật một cách độc lập.
\end{ans}

\begin{ques}
Tổ chức được thành lập hợp pháp là tổ chức có tư cách pháp nhân.
\end{ques}
\begin{ans}
\textbf{Sai.} Chỉ có \textbf{tổ chức} đủ 4 điều kiện theo quy định tại khoản 1 điều 74 Bộ luật Dân sự 2015 mới có tư cách pháp nhân.\\
\textbf{Điều 74. Pháp nhân}\\
1. Một tổ chức được công nhận là pháp nhân khi có đủ các điều kiện sau đây:\\
a. Được thành lập theo quy định của Bộ luật này, luật khác có liên quan;\\
b. Có cơ cấu tổ chức theo quy định của điều 83 Bộ luật này;\\
c. Có tài sản độc lập với cá nhân, pháp nhân khác và tự chịu trách nhiệm bằng tài sản của mình;\\
d. Nhân danh mình tham gia quan hệ pháp luật một cách độc lập.
\end{ans}

\begin{ques}
Chỉ tổ chức có tư cách pháp nhân mới được tham gia các quan hệ pháp luật một cách độc lập.
\end{ques}
\begin{ans}
\textbf{Sai.} Các tổ chức khác cũng có thể tham gia các quan hệ pháp luật một cách độc lập như doanh nghiệp tư nhân, hộ gia đình, tổ hợp tác, ...
\end{ans}

\begin{ques}
Khi tham gia quan hệ pháp luật, Nhà nước bình đẳng với các chủ thể khác về quyền và nghĩa vụ.
\end{ques}
\begin{ans}
\textbf{Sai.} Trong quan hệ pháp luật hình sự, quan hệ pháp luật hành chính, Nhà nước là một bên sử dụng quyền lực nhà nước để quản lý xã hội bằng phương pháp quyền uy phục tùng, điều chỉnh các quan hệ xã hội phù hợp với ý chí của giai cấp thống trị.
\end{ans}

\begin{ques}
Trong tất cả các quan hệ pháp luật mà Nhà nước tham gia, Nhà nước có địa vị pháp lý bình đẳng với các chủ thể khác.
\end{ques}
\begin{ans}
\textbf{Sai.} Trong quan hệ pháp luật hình sự, quan hệ pháp luật hành chính, Nhà nước là một bên sử dụng quyền lực nhà nước để quản lý xã hội bằng phương pháp quyền uy phục tùng, điều chỉnh các quan hệ xã hội phù hợp với ý chí của giai cấp thống trị.
\end{ans}

\begin{ques}
Trong quan hệ pháp luật dân sự, Nhà nước và các chủ thể khác bình đẳng với nhau về quyền và nghĩa vụ pháp lý.
\end{ques}
\begin{ans}
\textbf{Đúng.} Theo quy định tại điều 97 Bộ luật Dân sự 2015:\\
\textbf{Điều 97. Nhà nước Cộng hòa xã hội chủ nghĩa Việt Nam, cơ quan nhà nước ở Trung ương, ở địa phương trong quan hệ dân sự}\\
Nhà nước Cộng hòa xã hội chủ nghĩa Việt Nam, cơ quan nhà nước ở Trung ương, ở địa phương khi tham gia quan hệ dân sự thì bình đẳng với các chủ thể khác và chịu trách nhiệm dân sự theo quy định tại Điều 99 và Điều 100 của Bộ luật này.
\end{ans}

\begin{ques}
Ở nước ta, sự kiện pháp lý làm phát sinh quan hệ hôn nhân là sự kiện đăng ký kết hôn tại cơ quan nhà nước có thẩm quyền và tổ chức lễ cưới tại gia đình.
\end{ques}
\begin{ans}
\textbf{Sai.} Ở nước ta, sự kiện pháp lý làm phát sinh quan hệ hôn nhân là sự kiện đăng ký kết hôn tại cơ quan nhà nước có thẩm quyền.
\end{ans}

\begin{ques}
Ở nước ta, sự kiện pháp lý làm chấm dứt quan hệ hôn nhân là sự kiện Tòa án ra bản án, quyết định chấp nhận cho hai bên ly hôn.
\end{ques}
\begin{ans}
\textbf{Sai.} Ở nước ta, sự kiện pháp lý làm chấm dứt quan hệ hôn nhân là sự kiện bản án, quyết định chấp nhận cho hai bên ly hôn của Tòa án \textbf{có hiệu lực} (bản án sơ thẩm thì phải chờ 15 ngày để kháng cáo, kháng nghị).
\end{ans}

\begin{ques}
Anh A đốt nến, lửa bắt vào rèm cửa gây cháy nhà anh A và một số nhà hàng xóm là sự biến pháp lý.
\end{ques}
\begin{ans}
\textbf{Sai.} Sự biến pháp lý là những sự kiện pháp lý xảy ra trong tự nhiên, \textbf{không phụ thuộc vào ý chí của con người}, nhưng cũng có thể làm pahst sinh, thay đổi hay chấm dứt các quan hệ pháp luật. Sự biến pháp lý là những hiện tượng trong tự nhiên, rất đa dạng như: bão, lũ lụt, hạn hán, động đất, sóng thần, núi lửa, ...
\end{ans}

\begin{ques}
Hỏa hoạn, lũ lụt là sự biến pháp lý.
\end{ques}
\begin{ans}
\textbf{Sai.} Nếu hỏa hoạn, lũ lụt là do con người tạo ra, xảy ra theo ý chí của con người thì khi đó nó là \textbf{hành vi pháp lý}, không phải sự biến pháp lý.
\end{ans}

\begin{ques}
Mưa, gió, sấm, chớp là sự biến pháp lý.
\end{ques}
\begin{ans}
\textbf{Đúng.} Vì sự biến pháp lý là những sự kiện pháp lý xảy ra trong tự nhiên, không phụ thuộc vào ý chí của con người.
\end{ans}

\begin{ques}
Cháy rừng là sự biến pháp lý.
\end{ques}
\begin{ans}
Nếu cháy rừng là do con người tạo ra, xảy ra theo ý chí của con người thì khi đó nó là \textbf{hành vi pháp lý}, không phải sự biến pháp lý.
\end{ans}

\begin{ques}
Hành vi trái pháp luật là hành vi vi phạm pháp luật.
\end{ques}
\begin{ans}
\textbf{Sai.} Vì vi phạm pháp luật là \textbf{hành vi} (hành động hoặc không hành động) \textbf{trái pháp luật} và \textbf{có lỗi} do \textbf{chủ thể có năng lực trách nhiệm pháp lý} thực hiện, \textbf{xâm phạm} đến quan hệ xã hội được Nhà nước xác lập và bảo vệ.
\end{ans}

\begin{ques}
Hành vi vi phạm pháp luật là hành vi trái pháp luật.
\end{ques}
\begin{ans}
\textbf{Đúng.} Vì vi phạm pháp luật là \textbf{hành vi} (hành động hoặc không hành động) \textbf{trái pháp luật} và \textbf{có lỗi} do \textbf{chủ thể có năng lực trách nhiệm pháp lý} thực hiện, \textbf{xâm phạm} đến quan hệ xã hội được Nhà nước xác lập và bảo vệ.
\end{ans}

\begin{ques}
Thiệt hại do hành vi vi phạm pháp luật gây ra phải là thiệt hại về vật chất.
\end{ques}
\begin{ans}
\textbf{Sai.} Thiệt hại do hành vi vi phạm pháp luật gây ra có thể là thiệt hại về vật chất hoặc thiệt hại về tinh thần.
\end{ans}

\begin{ques}
Hành vi trái với đạo đức, chuẩn mực thông thường trong xã hội là hành vi vi phạm pháp luật.
\end{ques}
\begin{ans}
\textbf{Sai.} Vì vi phạm pháp luật là \textbf{hành vi} (hành động hoặc không hành động) \textbf{trái pháp luật} và \textbf{có lỗi} do \textbf{chủ thể có năng lực trách nhiệm pháp lý} thực hiện, \textbf{xâm phạm} đến quan hệ xã hội được Nhà nước xác lập và bảo vệ.
\end{ans}

\begin{ques}
Chủ thể có năng lực trách nhiệm pháp lý mà thực hiện hành vi trái pháp luật thì vi phạm pháp luật.
\end{ques}
\begin{ans}
\textbf{Sai.} Vì vi phạm pháp luật là \textbf{hành vi} (hành động hoặc không hành động) \textbf{trái pháp luật} và \textbf{có lỗi} do \textbf{chủ thể có năng lực trách nhiệm pháp lý} thực hiện, \textbf{xâm phạm} đến quan hệ xã hội được Nhà nước xác lập và bảo vệ.
\end{ans}

\begin{ques}
Cấu thành của vi phạm pháp luật bao gồm mặt chủ thể, mặt chủ quan, chủ thể và khách thể của vi phạm pháp luật.
\end{ques}
\begin{ans}
\textbf{Sai.} Cấu thành của vi phạm pháp luật bao gồm:
\begin{itemize}
\item Mặt khách quan của vi phạm pháp luật: trái pháp luật, gây ra hậu quả nguy hiểm cho xã hội (thể chất, tinh thần, vật chất).
\item Mặt chủ quan của vi phạm pháp luật: lỗi cố ý, lỗi vô ý.
\item Chủ thể của vi phạm pháp luật: cá nhân, tổ chức có năng lực trách nhiệm pháp lý.
\item Khách thể của vi phạm pháp luật: quan hệ xã hội được nhà nước xác lập và bảo vệ.
\end{itemize}
\end{ans}

\begin{ques}
Không biết trước hành vi của mình là nguy hiểm cho xã hội thì không được coi là có lỗi.
\end{ques}
\begin{ans}
\textbf{Sai.} Không thấy trước hành vi của mình có thể gây hậu quả mặc dù phải thấy trước và có thể thấy trước (vô ý vì cẩu thả) vẫn được coi là có lỗi.
\end{ans}

\begin{ques}
A đánh B gây thương tích thì khách thể bị xâm hại là B.
\end{ques}
\begin{ans}
\textbf{Sai.} Khách thể của vi phạm pháp luật là quan hệ xã hội được nhà nước xác lập và bảo vệ. Khi A đánh B gây thương tích thì khách thể bị xâm hại là \textbf{sức khỏe của B.}
\end{ans}

\begin{ques}
A có hành vi trộm cắp máy laptop của B thì khách thể bị xâm hại là cái máy laptop.
\end{ques}
\begin{ans}
\textbf{Sai.} Khách thể của vi phạm pháp luật là quan hệ xã hội được nhà nước xác lập và bảo vệ. Khi A có hành vi trộm cắp laptop của B thì khách thể bị xâm hại là \textbf{tài sản}, laptop là \textbf{đối tượng bị xâm hại}.
\end{ans}

\begin{ques}
C (tâm thần) dùng gậy đánh nhiều nhát vào người D là hành vi vi phạm pháp luật.
\end{ques}
\begin{ans}
\textbf{Sai.} Vì C bị tâm thần nên C bị mất năng lực trách nhiệm pháp lý, nên đây không là hành vi vi phạm pháp luật.
\end{ans}

\begin{ques}
N (13 tuổi) có hành vi trộm cắp tài sản (trị giá 250 triệu đồng) của gia đình ông P là hành vi vi phạm pháp luật hình sự.
\end{ques}
\begin{ans}
\textbf{Sai.} Vì theo quy định tại khoản 2 điều 12 Bộ luật Hình sự 2015, sửa đổi, bổ sung năm 2017, người từ đủ 14 tuổi trở lên mới phải chịu trách nhiệm về tội cướp tài sản.
\end{ans}

\begin{ques}
M (15 tuổi) có hành vi trộm cắp tài sản (trị giá 50 triệu đồng) của gia đình ông Q là hành vi vi phạm pháp luật.
\end{ques}
\begin{ans}
\textbf{Đúng.} Vì theo quy định tại khoản 2 điều 12 Bộ luật Hình sự 2015, sửa đổi, bổ sung năm 2017, người từ đủ 14 tuổi trở lên phải chịu trách nhiệm về tội cướp tài sản.
\end{ans}

\begin{ques}
M (15 tuổi) có hành vi trộm cắp tài sản (trị giá 50 triệu đồng) của gia đình ông Q là hành vi vi phạm pháp luật hình sự.
\end{ques}
\begin{ans}
\textbf{Đúng.} Vì theo quy định tại khoản 2 điều 12 Bộ luật Hình sự 2015, sửa đổi, bổ sung năm 2017, người từ đủ 14 tuổi trở lên phải chịu trách nhiệm về tội cướp tài sản.
\end{ans}

\begin{ques}
Hành vi trộm cắp tài sản là hành vi vi phạm pháp luật hình sự.
\end{ques}
\begin{ans}
\textbf{Sai.} Điều 173 Bộ luật Hình sự 2015, sửa đổi, bổ sung năm 2017 chỉ quy định hành vi vi phạm pháp luật hình sự đối với hành vi trộm cắp tài sản có giá trị từ 2 000 000 đồng trở lên.
\end{ans}

\begin{ques}
Hậu quả là yếu tố bắt buộc phải có đối với các hành vi vi phạm pháp luật.
\end{ques}
\begin{ans}
\textbf{Sai.} Đối với tội phạm có cấu thành tội phạm hình thức thì không cần có hậu quả.
\end{ans}

\begin{ques}
A (15 tuổi) đánh B gây tổn hại 5$\%$ sức khỏe của B là hành vi vi phạm pháp luật.
\end{ques}
\begin{ans}
\textbf{Đúng.} Trong trường hợp A (15 tuổi) đánh B gây tổn hại 5$\%$ sức khỏe của B thì đây sẽ được xem xét là hành vi vi phạm pháp luật hành chính.
\end{ans}

\begin{ques}
A (13 tuổi) đánh B gây tổn hại $50\%$ sức khỏe của B là hành vi vi phạm pháp luật.
\end{ques}
\begin{ans}
\textbf{Sai.} Vì A không có năng lực trách nhiệm pháp lý, do A mới 13 tuổi, chưa đáp ứng điều kiện về độ tuổi.
\end{ans}

\begin{ques}
A (18 tuổi) đánh B gây tổn hại $5\%$ sức khỏe của B là hành vi vi phạm pháp luật.
\end{ques}
\begin{ans}
\textbf{Đúng.} Trong trường hợp A (18 tuổi) đánh B gây tổn hại 5$\%$ sức khỏe của B thì đây sẽ được xem xét là hành vi vi phạm pháp luật hành chính.
\end{ans}

\begin{ques}
A (18 tuổi) đánh B gây tổn hại $5\%$ sức khỏe của B là hành vi vi phạm pháp luật hình sự.
\end{ques}
\begin{ans}
\textbf{Sai.} Theo điều 134 Bộ luật Hình sự 2015, sửa đổi, bổ sung năm 2017 thì hành vi đánh người gây tổn hại trên $11\%$ sức khỏe hoặc dưới $11\%$ nhưng thuộc vào những trường hợp nghiêm trọng thì mới được xem là hành vi vi phạm pháp luật hình sự.
\end{ans}

\begin{ques}
Hành vi cố ý đánh người gây thương tích là hành vi vi phạm pháp luật hình sự.
\end{ques}
\begin{ans}
\textbf{Sai.} Theo điều 134 Bộ luật Hình sự 2015, sửa đổi, bổ sung năm 2017 thì hành vi đánh người gây tổn hại trên $11\%$ sức khỏe hoặc dưới $11\%$ nhưng thuộc vào những trường hợp nghiêm trọng thì mới được xem là hành vi vi phạm pháp luật hình sự.
\end{ans}

\begin{ques}
Mỗi hành vi vi phạm pháp luật chỉ phải chịu một loại trách nhiệm pháp lý.
\end{ques}
\begin{ans}
\textbf{Sai.} Trong một số trường hợp, một hành vi vi phạm pháp luật có thể phải chịu cả trách nhiệm hình sự và trách nhiệm dân sự.
\end{ans}

\begin{ques}
Trách nhiệm hình sự là loại trách nhiệm pháp lý nghiêm khắc nhất.
\end{ques}
\begin{ans}
\textbf{Đúng.} Trách nhiệm hình sự gắn với các khung hình phạt/chế tài xử lý nặng nhất, nghiêm khắc nhất của Nhà nước.
\end{ans}

\begin{ques}
Một hành vi vi phạm pháp luật có thể phải chịu đồng thời nhiều loại trách nhiệm pháp lý khác nhau.
\end{ques}
\begin{ans}
\textbf{Đúng.} Trong một số trường hợp, một hành vi vi phạm pháp luật có thể phải chịu cả trách nhiệm hình sự và trách nhiệm dân sự.
\end{ans}

\begin{ques}
Một hành vi vi phạm pháp luật có thể phải chịu đồng thời trách nhiệm pháp lý hình sự và hành chính.
\end{ques}
\begin{ans}
\textbf{Sai.} Nếu hành vi vi phạm pháp luật đó không cấu thành tội phạm thì chỉ bị xử phạt hành chính. Nói cách khác, một hành vi vi phạm pháp luật hoặc là chịu trách nhiệm hình sự, hoặc là chịu trách nhiệm hành chính, \textbf{chứ không phải chịu đồng thời} cả hai.
\end{ans}

\begin{ques}
Một hành vi vi phạm pháp luật có thể phải chịu đồng thời trách nhiệm pháp lý hình sự và dân sự.
\end{ques}
\begin{ans}
\textbf{Đúng.} Trong một số trường hợp, một hành vi vi phạm pháp luật có thể phải chịu cả trách nhiệm hình sự và trách nhiệm dân sự.
\end{ans}

\begin{ques}
Mọi hành vi vi phạm pháp luật đều phải chịu trách nhiệm pháp lý.
\end{ques}
\begin{ans}
\textbf{Sai.} Chủ thể của hành vi vi phạm pháp luật phải có năng lực trách nhiệm pháp lý thì hành vi vi phạm pháp luật mới phải chịu trách nhiệm pháp lý.
\end{ans}

\begin{ques}
Toà án là cơ quan duy nhất có thẩm quyền truy cứu trách nhiệm pháp lý đối với các hành vi vi phạm pháp luật.
\end{ques}
\begin{ans}
\textbf{Sai.} Điều 102 Hiến pháp 2013 quy định:\\
\textit{1. Tòa án nhân dân là cơ quan xét xử của nước Cộng hòa Xã hội Chủ nghĩa Việt Nam, thực hiện quyền tư pháp.} \\
\textit{2. Tòa án nhân dân gồm Tòa án nhân dân tối cao và các Tòa án khác do luật định.}\\
\textit{3. Tòa án nhân dân có nhiệm vụ bảo vệ công lý, bảo vệ quyền con người, quyền công dân, bảo vệ chế độ xã hội chủ nghĩa, bảo vệ lợi ích của Nhà nước, quyền và lợi ích hợp pháp của tổ chức, cá nhân.}
\end{ans}

\begin{ques}
Hiến pháp là văn bản pháp luật có hiệu lực pháp lý tối cao trên lãnh thổ mỗi quốc gia.
\end{ques}
\begin{ans}
\textbf{Đúng.} Hiến pháp là đạo luật gốc, đạo luật cơ bản, là văn bản pháp luật có hiệu lực pháp lý cao nhất trên lãnh thổ một quốc gia. Các văn bản pháp luật khác không được trái với hiến pháp.
\end{ans}

\begin{ques}
Người làm việc trong cơ quan nhà nước là cán bộ, công chức nhà nước.
\end{ques}
\begin{ans}
\textbf{Sai.}\\
\begin{itemize}
\item Cán bộ là công dân Việt Nam, được bầu cử, phê chuẩn, bổ nhiệm giữ chức vụ, chức danh theo nhiệm kỳ trong cơ quan của Đảng Cộng sản Việt Nam, Nhà nước, tổ chức chính trị - xã hội ở Trung ương, cấp tỉnh, cấp huyện. (Khoản 1, điều 4 Luật Cán bộ, công chức 2008)
\item Công chức là công dân Việt Nam, được tuyển dụng, bổ nhiệm vào ngạch, chức vụ, chức danh tương ứng với vị trí việc làm trong biên chế và hưởng lương từ ngân sách Nhà nước trong:
\begin{itemize}
\item - Cơ quan của Đảng Cộng sản Việt Nam, Nhà nước, tổ chức chính trị - xã hội ở trung ương, cấp tỉnh, cấp huyện;
\item - Cơ quan, đơn vị thuộc Quân đội nhân dân mà không phải là sĩ quan, quân nhân chuyên nghiệp, công nhân quốc phòng;
\item - Cơ quan, đơn vị thuộc Công an nhân dân mà không phải là sĩ quan, hạ sĩ quan phục vụ theo chế độ chuyên nghiệp, công nhân công an.
\end{itemize}
(theo khoản 1 điều 1 Luật sửa đổi Luật Cán bộ, công chức 2019)
\end{itemize}
Trong cơ quan nhà nước còn có \textbf{người lao động.}
\end{ans}

\begin{ques}
Nếu xác định một hành vi vi phạm pháp luật là hành vi vi phạm pháp luật hành chính thì không được áp dụng hình phạt đối với người vi phạm.
\end{ques}
\begin{ans}
\textbf{Sai.} Nếu một hành vi vi phạm pháp luật hành chính thì người thực hiện sẽ bị áp dụng hình phạt hành chính/hình phạt bổ sung theo quy định của pháp luật.
\end{ans}

\begin{ques}
Chỉ chủ sở hữu tài sản mới có quyền chiếm hữu, sử dụng, định đoạt tài sản.
\end{ques}
\begin{ans}
\textbf{Sai.} Những người được chủ sở hữu tài sản ủy quyền cũng có quyền chiếm hữu, sử dụng, định đoạt tài sản.
\end{ans}

\begin{ques}
Chỉ có cha, mẹ đẻ, con đẻ của người để lại di sản mới được hưởng thừa kế của người đó theo quy định của pháp luật.
\end{ques}
\begin{ans}
\textbf{Sai.} Theo quy định thì con nuôi, cha mẹ nuôi cũng thuộc hàng thừa kế thứ I, nghĩa là cũng được hưởng thừa kế theo quy định của pháp luật.
\end{ans}

\begin{ques}
Con được hưởng thừa kế của cha mẹ phải là con của người vợ, người chồng hợp pháp.
\end{ques}
\begin{ans}
\textbf{Sai.} Theo quy định thì con thuộc hàng thừa kế thứ I không phân biệt con trong giá thú hay con ngoài giá thú.
\end{ans}

\begin{ques}
Cha, mẹ không được hưởng thừa kế của con nếu đã cho con làm con nuôi của người khác.
\end{ques}
\begin{ans}
\textbf{Sai.} Cha, mẹ đẻ có thể được hưởng thừa kế trong di chúc theo ý chí của con. Ngoài ra, cha, mẹ đẻ cũng thuộc hàng thừa kế thứ I theo quy định của pháp luật.
\end{ans}

\begin{ques}
Đứa trẻ đã được người khác nhận làm con nuôi theo đúng quy định của pháp luật thì không được hưởng thừa kế từ di sản do cha mẹ đẻ của mình để lại.
\end{ques}
\begin{ans}
\textbf{Sai.} Đứa trẻ có thể được hưởng thừa kế trong di chúc theo ý chí của cha, mẹ đẻ. Ngoài ra, con ruột cũng thuộc hàng thừa kế thứ I theo quy định của pháp luật.
\end{ans}

\begin{ques}
Người thừa kế chỉ có thể là cá nhân.
\end{ques}
\begin{ans}
\textbf{Sai.} Người thừa kế trong di chúc có thể là bất kì cá nhân, tổ chức nào theo ý chí của người để lại thừa kế.
\end{ans}

\begin{ques}
Thời hiệu khởi kiện về thừa kế là 10 năm kể từ thời điểm người để lại di sản chết.
\end{ques}
\begin{ans}
\textbf{Sai.} Khoản 1 điều 623 Bộ luật Dân sự 2015 quy đinh: \textit{Thời hiệu khởi kiện để người thừa kế yêu cầu chia di sản là 30 năm đối với bất động sản, 10 năm đối với động sản, kể từ thời điểm mở thừa kế.}
\end{ans}

\begin{ques}
Người có tài sản không được lập di chúc để lại tài sản của mình cho người không có mối quan hệ huyết thống hoặc quan hệ hôn nhân.
\end{ques}
\begin{ans}
\textbf{Sai.} Người thừa kế trong di chúc có thể là bất kì cá nhân, tổ chức nào theo ý chí của người để lại thừa kế.
\end{ans}

\begin{ques}
Ông A bị tai nạn giao thông ngày 01/4/2017. Ngày 02/5/2017, ông A có di chúc miệng hợp pháp để lại toàn bộ tài sản cho bà B. Ngày 05/8/2017 ông A chết. Trong trường hợp này, di sản của ông A được chia theo di chúc miệng ngày 02/5/2017.
\end{ques}
\begin{ans}
\textbf{Sai.} Sau 03 tháng mà người để lại di chúc miệng còn sống, minh mẫn, sáng suốt thì di chúc miệng mặc nhiên bị hủy bỏ.
\end{ans}

\begin{ques}
Độ tuổi để được kết hôn theo quy định của Luật Hôn nhân gia đình 2014 là nam, nữ phải từ 18 tuổi trở lên.
\end{ques}
\begin{ans}
\textbf{Sai.} Độ tuổi để được kết hôn theo quy định tại điều 8 Luật Hôn nhân gia đình 2014 là nam từ đủ 20 tuổi, nữ từ đủ 18 tuổi trở lên.
\end{ans}

\begin{ques}
Luật Hôn nhân gia đình Việt Nam cấm kết hôn giữa những người đồng giới.
\end{ques}
\begin{ans}
\textbf{Sai.} Luật Hôn nhân gia đình 2014 \textbf{không thừa nhận}, chứ không phải cấm, hôn nhân giữa những người đồng giới.
\end{ans}

\begin{ques}
Tất cả các tài sản có được trong quá trình hôn nhân là tài sản chung của vợ chồng.
\end{ques}
\begin{ans}
\textbf{Sai.} Những tài sản được xác định là tài sản riêng hoặc tài sản mà vợ chồng thỏa thuận là tài sản riêng thì là tài sản riêng của vợ/chồng.
\end{ans}

\begin{ques}
Nếu người vợ mang thai hoặc nuôi con dưới 36 tháng tuổi thì chồng không được quyền yêu cầu ly hôn.
\end{ques}
\begin{ans}
\textbf{Sai.} Khoản 3, điều 51 Luật Hôn nhân và gia đình 2014 quy định:\\
\textit{Chồng không có quyền yêu cầu ly hôn trong trường hợp vợ đang có thai, sinh con hoặc đang nuôi con dưới 12 tháng tuổi.}
\end{ans}

\begin{ques}
Khi vợ chồng ly hôn, việc giao nuôi dưỡng con chung sẽ được thực hiện theo nguyên tắc: con dưới 36 tháng tuổi sẽ được giao cho mẹ nuôi dưỡng, con trên 7 tuổi trở lên phải theo ý nguyện của con.
\end{ques}
\begin{ans}
\textbf{Đúng.} Điều 81 Luật Hôn nhân và gia đình 2014 quy định như sau:\\
\textbf{Điều 81. Việc trông nom, chăm sóc, nuôi dưỡng, giáo dục con sau khi ly hôn}\\
\textit{1. Sau khi ly hôn, cha mẹ vẫn có quyền, nghĩa vụ trông nom, chăm sóc, nuôi dưỡng, giáo dục con chưa thành niên, con đã thành niên mất năng lực hành vi dân sự hoặc không có khả năng lao động và không có tài sản để tự nuôi mình theo quy định của Luật này, Bộ luật dân sự và các luật khác có liên quan.\\
2. Vợ, chồng thỏa thuận về người trực tiếp nuôi con, nghĩa vụ, quyền của mỗi bên sau khi ly hôn đối với con; trường hợp không thỏa thuận được thì Tòa án quyết định giao cho một bên trực tiếp nuôi căn cứ vào quyền lợi về mọi mặt của con; nếu con từ đủ 07 tuổi trở lên thì phải xem xét nguyện vọng của con.\\
3. Con dưới 36 tháng tuổi được giao cho mẹ trực tiếp nuôi, trừ trường hợp người mẹ không đủ điều kiện để trực tiếp trông nom, chăm sóc, nuôi dưỡng, giáo dục con hoặc cha mẹ có thỏa thuận khác phù hợp với lợi ích của con.}
\end{ans}

\begin{ques}
Con được sinh ra sau khi hai vợ chồng đã ly hôn thì không được coi là con chung của vợ chồng.
\end{ques}
\begin{ans}
\textbf{Sai.} Khoản 1 điều 88 Luật Hôn nhân và gia đình 2014 quy định:\\
Con được sinh ra trong thời hạn 300 ngày kể từ thời điểm chấm dứt hôn nhân được coi là con do người vợ có thai trong thời kỳ hôn nhân.
\end{ans}

\begin{ques}
Con được sinh ra trong thời kỳ hôn nhân mặc nhiên được thừa nhận là con chung của vợ chồng.
\end{ques}
\begin{ans}
\textbf{Sai.} Trong trường hợp cha/mẹ không thừa nhận con theo xác định của Tòa án thì không phải là con chung của vợ chồng.
\end{ans}