\section{Mở đầu}
Năm 2011, tại Đại hội đại biểu toàn quốc lần thứ XI, Đảng ta đã đưa ra khái niệm: \textit{Tư tưởng Hồ Chí Minh} là một hệ thống quan điểm toàn diện và sâu sắc về những vấn đề cơ bản của cách mạng Việt Nam, kết quả của sự vận dụng và phát triển sáng tạo chủ nghĩa Mác - Lênin vào điều kiện cụ thể của nước ta, kế thừa và phát triển các giá trị truyền thống tốt đẹp của dân tộc, tiếp thu tinh hoa văn hóa nhân loại; là tài sản tinh thần vô cùng to lớn và quý giá của Đảng và dân tộc ta, mãi mãi soi đường cho sự nghiệp cách mạng của nhân dân ta giành thắng lợi. \cite{syllabus} \cite{vankienXI} \\
Tư tưởng Hồ Chí Minh bao quát trên nhiều lĩnh vực đa dạng như: độc lập dân tộc và chủ nghĩa xã hội; Đảng Cộng sản Việt Nam và Nhà nước của nhân dân, do nhân dân, vì nhân dân; đại đoàn kết toàn dân tộc và đoàn kết quốc tế; văn hóa, đạo đức, con người. Trong đó, tấm gương đạo đức sáng ngời của Người là một hình mẫu cho các thế hệ người Việt Nam noi theo. Trong tư tưởng của Bác về đạo đức, ta không thể không nhắc đến những chuẩn mực đạo đức cách mạng:
\begin{itemize}
\item Trung với nước, hiếu với dân.
\item Cần, kiệm, liêm, chính, chí công vô tư.
\item Thương yêu con người, sống có tình có nghĩa.
\item Tinh thần quốc tế trong sáng.
\end{itemize}
Tư tưởng của Người, tấm gương về lối sống cần, kiệm, liêm, chính, chí công vô tư trở thành một bài học quý giá cho mỗi con người Việt Nam chúng ta hôm nay trong sự nghiệp xây dựng và bảo vệ Tổ quốc. Nhận thức được điều này, dưới sự hướng dẫn của TS. Trương Thị Mai, em chọn thực hiện đề tài: \textbf{Sinh viên cần vận dụng tư tưởng Hồ Chí Minh về vấn đề cần, kiệm, liêm, chính, chí công vô tư như thế nào?} cho bài tiểu luận cuối kỳ.\\
Em xin chân thành cảm ơn Cô Trương Thị Mai vì những bài giảng quý báu, bổ ích về Tư tưởng Hồ Chí Minh. Em cảm ơn Cô vì những góp ý trong quá trình thực hiện tiểu luận này.
