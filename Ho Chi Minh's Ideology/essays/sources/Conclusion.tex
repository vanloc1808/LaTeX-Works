\section{Kết luận}
Chủ tịch Hồ Chí Minh đã đi xa hơn nửa thế kỷ nhưng những giá trị tư tưởng của Người vẫn còn sống mãi với thời gian. Giá trị và tầm quan trọng của đạo đức cách mạng được Người nhắc đến xuyên suốt cuộc đời hoạt động cách mạng của mình, từ \textit{Đường kách mệnh} năm 1927 đến tận bản \textit{Di chúc} được công bố khi Người ra đi. Cuộc đời Người là một tấm gương sáng ngời về đạo đức cách mạng, về cần, kiệm, liêm, chính, chí công vô tư, một tấm gương mà mỗi chúng ta hôm nay và mai sau sẽ luôn xem là hình mẫu cho mình noi theo.\\
Ngày 15/5/2016, Bộ Chính trị Ban Chấp hành Trung ương Đảng khóa XII đã ra Chỉ thị 05 \textit{"Về đẩy mạnh học tập và làm theo tư tưởng, đạo đức, phong cách Hồ Chí Minh"}, thúc đẩy quá trình học và làm theo Bác trong toàn Đảng nói riêng và toàn dân nói chung. Là sinh viên trong thời kỳ đất nước bước vào quá trình đẩy mạnh hội nhập quốc tế, đẩy mạnh công cuộc Đổi mới đất nước, mỗi chúng ta cần thực hành, trau dồi cần, kiệm, liêm, chính, chí công vô tư từ khi còn ngồi trên ghế nhà trường để mai sau có thể đem sức mình góp phần vào việc xây dựng Tổ quốc Việt Nam giàu mạnh, sánh vai cùng các cường quốc năm châu như lời mong mỏi của Bác.