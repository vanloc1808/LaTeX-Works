\chapter{Bảng hệ thống tuần hoàn và sự biến thiên tuần hoàn của tính chất của các nguyên tố hóa học}
\section{Tại sao lại cần bảng phân loại tuần hoàn?}
\begin{itemize}
\item Hiện nay, cỏ rất nhiều các nguyên tố hóa học được tìm thấy.
\item Do có nhiều nguyên tố như vậy khiến cho việc sắp xếp phân loại chúng theo các tính chất có tính tuần hoàn (ví dụ như tính chất lý hóa) rất cần thiết.
\item Một trong các thành tựu quan trọng nhất của các nhà hóa học thế kỉ XIX, đó chính là sự phân loại các nguyên tố để hình thành bảng phân loại tuần hoàn nhằm thể hiện tính chất hóa học của các nguyên tố đó.
\end{itemize}
\section{Phân loại các nguyên tố theo Mendeleev}
\subsection{Phân loại các nguyên tố theo Mendeleev}
Mendeleev đã phân loại các nguyên tố thành bảng hệ thống tuần hoàn như sau:
\begin{itemize}
\item Tại thời điểm mà Mendeleev phân loại có $63$ nguyên tố được tìm thấy theo chiều tăng của nguyên tử khối (khối lượng của nguyên tử).
\item Ông xây dựng một bảng có $12$ dòng, $9$ cột (được đánh dấu là 0, I, II, III, IV, V, VI, VII, VIII), gọi là các nhóm. Các nguyên tố có cùng tính chất giống nhau thì xếp chung một cột (chung một nhóm).
\end{itemize}
\subsection{Một vài điểm đáng lưu ý của cách sắp xếp theo Mendeleev}
Ở cách sắp xếp này, ta cần lưu ý một vài điểm như sau:
\begin{itemize}
\item Một vài chỗ đáng lẽ phải sắp xếp theo khối lượng nguyên tử tăng dần bị đổi để sắp xếp theo tính chất hóa lý của nguyên tố.
\item Một vài chỗ bị bỏ trống để cho các nguyên tố chưa được tìm ra.
\end{itemize}
\subsection{Các nhược điểm của cách sắp xếp theo Mendeleev}
Cách sắp xếp này cũng bộc lộ nhiều nhược điểm như sau:
\begin{itemize}
\item Những nguyên tố có cùng tính chất lại được sắp xếp những nhóm khác nhau.
\item Các nguyên tử có nguyên tử khối lớn hơn lại sắp ở vị trí nhỏ nhằm duy trì việc sắp xếp theo tính chất hóa học.
\item Các đồng vị (các nguyên tố có cùng số proton nhưng khác số neutron) không được đề cập trong bảng.
\end{itemize}
\section{Bảng phân loại tuần hoàn hiện đại $-$ mối liên quan giữa cấu hình electron nguyên tử và vị trí của các nguyên tố trong bảng phân loại tuần hoàn}
\subsection{Điện tích hạt nhân nguyên tử $-$ cơ sở của định luật tuần hoàn hiện đại}
Henry Moseley (1887 $-$ 1915) đã thực hiện các thí nghiệm về tia X. Ông nhận thấy rằng điện tích hạt nhân hay số hiệu nguyên tử của nguyên tố ($Z$) có liên quan trực tiếp đến tính chất của nguyên tử.\\
Từ đó, định luật tuần hoàn hiện đại được phát biểu như sau:\\
\textit{Tính chất hóa học và vật lý của các nguyên tố thay đổi một cách tuần hoàn theo số hiệu nguyên tử của chúng.}
\subsection{Bảng phân loại tuần hoàn hiện đại}
Từ định luật tuần haofn hiện đại, ta có cách sắp xếp sau đây cho Bảng phân loại tuần hoàn các nguyên tố hóa học:
\begin{itemize}
\item Các nguyên tố được sắp xếp theo chiều tăng dần điện tích hạt nhân nguyên tử, các nguyên tố cùng một nhóm có tính chất hóa học tương tự nhau.
\item Bao gồm $7$ chu kỳ ($7$ dòng) và $18$ nhóm ($18$ cột).
\item Nguyên tố đầu chu kỳ có cấu hình $ns^1,$ nguyên tố kết thúc chu kỳ có cấu hình $ns^2 np^6.$
\item Theo bảng này ta có $3$ khu vực:
\begin{itemize}
\item Khu vực bên trái: nhóm $1$ và nhóm $2$ (IA, IIA) là các kim loại kiềm và kim loại kiềm thổ, hay còn gọi là nguyên tố $s.$ Đây là các nguyên tố có tính kim loại mạnh nhất.
\item Khu vực bên phải: nhóm $13$ đến nhóm $18$ (IIIA - VIIIA) là các nguyên tố $p,$ bao gồm phần lớn là các phi kim.
\item Khu vực chính giữa, chia thành hai phần: các nguyên tố $d$ và các nguyên tố $f.$
\end{itemize}
\end{itemize}
\subsubsection{Các chu kỳ}
Các chu kỳ là các hàng ngang.\\
Các nguyên tố chùng chu kỳ tức là chúng cùng nằm trên một hàng ngang.
\subsubsection{Các nhóm}
Các nhóm và phân nhóm: Bảng phân loại tuần hoàn gồm $18$ cột, chia thành $2$ phân nhóm: phân nhóm chính A và phân nhóm phụ B.
\paragraph{Các nguyên tố $s$ và $p$ thuộc phân nhóm chính A.}
\begin{itemize}
\item Cách đánh số cũ: IA $\to$ VIIIA.
\item Cách đánh số mới: $1, 2$ và $13 \to 18.$
\end{itemize}
\paragraph{Các nguyên tố $d$ thuộc phân nhóm phụ B.}
\begin{itemize}
\item Cách đánh số cũ: IB $\to$ VIIIB.
\item Cách đanh số mới: $3 \to 12.$
\end{itemize}
\section{Tính chất hóa học cơ bản của các nguyên tố hóa học}
\subsection{Các nguyên tố $s$ và $p$}
Các nguyên tố $s$ và các nguyên tố $p$ có các tính chất hóa học cơ bản sau:
\begin{itemize}
\item Các nguyên tố $s$ thuộc nhóm $1$ và $2:$ là các kim loại hoạt động hóa học mạnh nhất. Dễ nhường electron tạo ion dương có lớp vỏ electron bão hòa, thể hiện tính khử.
\item Các nguyên tố $p$ có $3 - 4$ electron lớp ngoài cùng: có thể nhường hay nhận electron, thể hiện cả tính oxi hóa và tính khử. Càng ở chu kỳ lớn, bán kính nguyên tử càng lớn, thể hiện tính khử càng rõ.
\item Các nguyên tố $p$ có $5 - 7$ electron lớp ngoài cùng: có xu hướng nhận electron để tạo thành ion âm có lớp vỏ bão hòa.
\item Các nguyên tố nhóm $18$ (VIIIA) có cầu hình electron nguyên tử bão hòa với 8 electron lớp ngoài cùng: kém hoạt động hóa học.
\end{itemize}
\subsection{Các nguyên tố $d$}
Các nguyên tố $d$ đều là kim loại, có các tính chất hóa học cơ bản sau:
\begin{itemize}
\item Lớp electron ngoài cùng của các nguyên tố $d$ ($ns$) luôn chỉ có $1$ hoặc $2$ electron, các electron $ns^{1-2}$ có năng lượng cao hơn các electron $\left( {n - 1} \right)d^{1-10}.$ Khi tham gia phản ứng, electron của phân lớp $ns$ sẽ bị mất trước do phân lớp $d$ chắn sức hút của hạt nhân lên phân lớp $s.$
\item Các electron ở phân lớp $d$ đang xây dựng cũng có khả năng tham gia tạo liên kết hóa học nên hóa tính của các kim loại nguyên tố $d$ phức tạp hơn các kim loại nguyên tố $s$ và $p.$
\item Các nguyên tử nguyên tố $d$ có thể tạo thành các ion dương với cấu hình $d^x$ với $x$ thay đổi trong khoảng từ $0$ đến $10.$
\end{itemize}
\section{Biến thiên tuần hoàn một số tính chất của nguyên tử (nguyên tố) trong Bảng phân loại tuần hoàn}
\subsection{Bán kính nguyên tử và biến thiên bán kính nguyên tử}
\subsubsection{Trong một nhóm nguyên tử $s - p,$ bán kính nguyên tử tăng}
\begin{itemize}
\item Số lớp electron tăng làm bán kính tăng.
\item Điện tích hạt nhân tăng, electron được nhân hút mạnh hơn làm bán kính giảm.
\end{itemize}
Yếu tố số lớp electron tăng ảnh hưởng mạnh hơn nên làm bán kính nguyên tử tăng.
\subsubsection{Trong các chu kỳ ngắn, bán kính nguyên tử giảm}
\begin{itemize}
\item Số lớp electron không tăng.
\item Điện tích hạt nhân tăng, electron được nhân hút mạnh hơn, làm bán kính giảm.
\end{itemize}
Yếu tố điện tích hạt nhân ảnh hưởng mạnh hơn nên làm bán kính nguyên tử giảm.
\subsubsection{Trong chu kỳ dài, các nguyên tố $d$ và $f$ có bán kính giảm rất ít}
Lý do: electron điền vào các lớp bên trong $\left( {n - 1} \right)d$ hay $\left( {n - 2} \right)f$ làm chắn sức hút của hạt nhân lên nguyên tư $ns$ bên ngoài, bán kính sẽ giảm chậm, gọi là hiệu ứng chắn (shielding effect).
\begin{itemize}
\item Hiện tượng bán kính nguyên tử giảm chậm trong dãy các nguyên tố $d$ gọi là \textbf{hiện tượng co $d.$}
\item Hiện tượng bán kính nguyên tử giảm chậm trong dãy các nguyên tố $f$ gọi là \textbf{hiện tượng co $f.$}
\end{itemize}
\subsubsection{Trong phân nhóm của các nguyên tố $d$}
\begin{itemize}
\item Bán kính nguyên tử tăng ít từ chu kỳ $4$ qua chu kỳ $5$ (do điện tích hạt nhân nguyên tử tăng $18$ đơn vị từ chu kỳ $4$ qua chu kỳ $5$).
\item Bán kính nguyên tử của nguyên tố chu kỳ $5$ và chu kỳ $6$ gần như bằng nhau do điện tích hạt nhân tăng $32$ đơn vị giữa hai chu kỳ.
\item Hiện tượng bán kính nguyên tử của các nguyên tố $d$ ở chu kỳ $5$ và chu kỳ $6$ bằng nhau cũng gọi là \textbf{hiện tượng co $f.$}
\end{itemize}
\subsubsection{Bán kính $\mathrm{Ga}$ ở chu kỳ $4$ hơi nhỏ hơn $\mathrm{Al}$ ở chu kỳ $3.$}
Hiện tượng này cũng gọi là sự co $d:$ từ $\mathrm{Al}$ đến $\mathrm{Ga}$ tăng $18$ điện tích hạt nhân, nhưng chỉ tăng một lớp electron, nên \textbf{bán kính nguyên tử hơi giảm.}
\subsection{Bán kính ion và biến thiên bán kính ion}
\begin{itemize}
\item Trong cùng một nhóm, các ion có điện tích ion tương tư: bán kính ion biến thiên tương tự bán kính nguyên tử, nghĩa là bán kính các ion cùng điện tích tăng dần trong nhóm.
\item Các ion có số electron bằng nhau và cấu hình electron giống nhau được gọi là \textbf{các ion đẳng điện tử.} Các ion đẳng điện tử có bán kính giảm khi điện tích hạt nhân $Z$ tăng.
\end{itemize}
\subsection{Năng lượng ion hóa nguyên tử và biến thiên năng lượng ion hóa nguyên tử}
\subsubsection{Năng lượng ion hóa nguyên tử}
Năng lượng ion hóa nguyên tử, gọi tắt là năng lượng ion hóa, là năng lượng cần cung cấp để lấy electron ra khỏi nguyên tử (hoặc ion) \textbf{ở trạng thái khí.}
\subsubsection{Biến thiên năng lượng ion hóa nguyên tử}
\begin{itemize}
\item Năng lượng ion hóa thứ nhất của các nguyên tố trong cùng một chu kỳ có khuynh hướng tăng dần. Do số lớp electron không tăng nhưng điện tích hạt nhân tăng. Nói cách khacsm là do bán kính nguyên tử giảm.
\item Đối với các nguyên tố $s$ và $p$ trong cùng một nhóm, năng lượng ion hóa thứ nhất giảm do bán kính nguyên tử tăng.
\item Các nguyên tố $d$ trong cùng một nhóm có biến thiên năng lượng ion hóa thứ nhất lại tăng. Do bán kính các nguyên tố $d$ trong cùng nhóm biến đổi ít hay không biến đổi mà điện tích hạt nhân lại tăng lên đáng kể.
\item Cực đại nhỏ xảy ra ở những nguyên tố có cấu hình bán bão hòa, chỉ đề xét so với các nguyên tố kề cạnh nó.
\end{itemize}
\subsection{Ái lực điện tử và biến thiên ái lực điện tử của nguyên tử}
\subsubsection{Ái lực điện tử}
Ái lực điện tử đặc trưng cho khuynh hướng nhận thêm electron của nguyên tử, là năng lượng tương ứng với quá trình nguyên tử cô lập \textbf{ở trạng thái khí} kết hợp với electrong để tạo thành ion âm. 
\subsubsection{Biến thiên ái lực điện tử của nguyên tử}
\begin{itemize}
\item Trong một chu kỳ, ái lực điện tử của các nguyên tử tăng dần do bán kính giảm.
\item Trong một nhóm, ái lực điện tử của các nguyên tử giảm dần do bán kính tăng.
\item Ngoại lệ, các nguyên tử có cấu hình bão hòa hay bán bão hòa có ái lực kém.
\end{itemize}
\subsection{Độ âm điện và biến thiên độ âm điện của các nguyên tố}
\subsubsection{Độ âm điện}
Độ âm điện là đại lượng đặc trưng cho khả năng hút electron về phía mình của nguyên tử khi nó liên kết với nguyên tử khác.\\
Độ âm điện của các nguyên tố là một đại lượng hoàn toàn mang tính quy ước.
\subsubsection{Đánh giá một cách tương đối}
\begin{itemize}
\item Tính kim loại hay phi kim của nguyên tố: độ âm điện của nguyên tố càng cao, nguyên tố càng có tính phi kim cao.
\item Tính ion hay cộng hóa trị của liên kết hóa học: liên kết hóa học tạo thành giữa các nguyên tố có chênh lệch độ âm điện cao sẽ có tính ion cao.
\end{itemize}