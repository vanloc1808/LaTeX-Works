\chapter{Liên kết hóa học và các mô hình liên kết hóa học đơn giản}
\section{Khái niệm cơ sở về liên kết hóa học và ba kiểu liên kết hóa học chính}
\subsection{Khái niệm cơ sở về liên kết hóa học}
Các nguyên tử liên kết với nhau để tạo thành các chất có \textbf{năng lượng thấp hơn.}\\
Các nguyên tử liên kết với nhau bằng \textbf{liên kết hóa học.}
\subsection{Ba kiểu liên kết hóa học chính}
\begin{itemize}
\item Liên kết kim loại. Ví dụ: các kim loại như đồng, nhôm, ...
\item Liên kết cộng hóa trị. Ví dụ: kim cương, graphite, ...
\item Liên kết ion. Ví dụ: $\mathrm{NaCl},$ $\mathrm{MgCl_2}, ...$
\end{itemize}
\section{Liên kết trong kim loại $-$ thuyết khí quyển electron}
\begin{itemize}
\item Mô hình: nguyên tử kim loại phóng thích electron hóa trị tạo thành các cation kim loại và electron tự do.\\
\centerline{Nguyên tử kim loại $\to$ cation kim loại $+$ electron tự do}
\item Các electron di chuyển tự do trong toàn bộ tinh thể kim loại, chúng tạo thành \textbf{đám mây điện tử bất định xứ} liên kết các cation kim loại lại với nhau.
\item Chính tương tác điện giữa electron tự do và mạng lưới ion dương giữ các nguyên tử kim loại lại với nhau.
\item Tương tác này là tương tác mạnh nên kim loại có nhiệt độ nóng chảy cao.
\item Mô hình này giúp giải thích: tính dẫn điện, tính dễ dát mỏng, tính ánh kim.
\end{itemize}
\section{Hợp chất ion và liên kết trong hợp chất ion}
\subsection{Hợp chất ion và liên kết trong hợp chất ion}
Lực liên kết giữa các ion trái dấu với nhau mang tính chất của tương tác tĩnh điện và được gọi là \textbf{liên kết ion.}\\
Đặc tính chung của các hợp chất ion:
\begin{itemize}
\item Là các chất rắn tinh thể, có nhiệt độ nóng chảy và nhiệt độ sôi cao.
\item Dẫn điện ở trạng thái nóng chảy.
\item Thường dễ tan trong nước và dung dịch nước của chúng dẫn điện.
\end{itemize}
\subsection{Thành phần của các hợp chất ion}
Các loại cation, anion:
\begin{itemize}
\item Cation đơn giản: thường có số oxi hóa (điện tích ion) $\leqslant 3,$ ví dụ $\mathrm{Li ^ +}, \mathrm{Na^+}, \mathrm{Al^{3+}}, ...$
\item Cation phức tạp: amonium ($\mathrm{NH_4^+}),$ titanyl ($\mathrm{TiO_2^+}), ...$
\item Anion đơn giản: $\mathrm{F^-}, \mathrm{S^{2-}}, ...$
\item Anion phức tạp: $\mathrm{SO_4^{2-}}, ...$
\end{itemize}
Thành phần các hợp chất ion:
\begin{itemize}
\item Các nguyên tử có thể nhường hoặc nhận thêm electron để tạo thành ion với cấu hình của khí hiếm.
\item Các cation của nguyên tố $s, p,$ có thể có cấu hình của khí hiếm hoặc giả khí hiếm ($18$ electron lớp ngoài cùng), hoặc cấu hình $18 + 2$ electron.
\item Các cation của nguyên tố $d:$ cấu hình electron đa dạng.
\end{itemize}
\subsection{Cấu trúc hợp chất ion $-$ mạng tinh thể ion}
Các cation và anion xếp chặt vào nhau theo trật tự nhất định, sao cho tương tác đẩy giữa anion $-$ anion và cation $-$ cation là cực tiểu, tương tác hút giữa cation $-$ anion trong mạng tinh thể là cực đại.\\
Tương tác giữa các ion với nhau là tương tác tĩnh điện.
\subsubsection{Ô mạng cơ sở}
Ô mạng tinh thể (basic cell) là phần không gian nhỏ nhất có cấu trúc đặc trưng cho tinh thể.\\
Ô mạng tinh thể gồm $6$ thông số mạng:
\begin{itemize}
\item $a, b, c$ là ba thông số cạnh của ô mạng.
\item $\alpha, \beta, \gamma$ là ba thông số góc của ô mạng.
\end{itemize}
\subsubsection{Số phối trí của các ion trong mạng tinh thể}
Số phối trí của các ion trong mạng tinh thể là số lượng các ion trái dấu xung quanh nó.
\subsubsection{Số nguyên tử hay ion trong mỗi ô mạng cơ sở}
\begin{itemize}
\item Mỗi nguyên tử/ion ở đỉnh của ô mạng cơ sở đóng góp $\frac{1}{8}$ nguyên tử/ion vào ô mạng (do mỗi đỉnh là đỉnh chung của đúng $8$ ô).
\item Mỗi nguyên tử/ion ở mặt của ô mạng cơ sở đóng góp $\frac{1}{2}$ nguyên tử/ion vào ô mạng (do mỗi mặt là mặt chung của đúng $2$ ô).
\item Mỗi nguyên tử/ion ở cạnh của ô mạng cơ sở đóng góp $\frac{1}{4}$ nguyên tử/ion vào ô mạng (do mỗi cạnh là cạnh chung của đúng $4$ ô).
\item Mỗi nguyên tử/ion ở tâm của ô mạng cơ sở đóng góp $1$ nguyên tử/ion vào ô mạng (do mỗi tâm là tâm của đúng $1$ ô).
\end{itemize}
\subsubsection{Những lưu ý quan trọng}
\begin{itemize}
\item Các mạng tinh thể có cấu trúc khác nhau, nên không thể dự đoán sự phối trí, số nguyên tử/ion trong ô mạng cơ sở. \textbf{Phải dựa vào tính toán thực tế.}
\item Không có khái niệm phân tử ion, chỉ có mạng tinh thể ion.
\item Tỉ lệ nguyên tử giữa các ion tạo thành mạng tinh thể tạo nên công thức đơn giản của hợp chất ion.
\item Lực tương tác ion có thể hình thành theo bất cứ phương nào trong không gian, tùy thuộc vào vị trí ion trong mạng tinh thể, tạo nên tính bất định hướng và bất bão hòa.
\end{itemize}
\subsection{Lực tương tác ion $-$ năng lượng mạng tinh thể ion}
Năng lượng mạng tinh thể ion:
\begin{itemize}
\item Độ mạnh của lực liên kết ion, hay năng lượng liên kết ion, được đánh giá qua \textbf{năng lượng mạng tinh thể ion.}
\item Năng lượng mạng tinh thể ion được quy ước là năng lượng phát ra khi hình thành $1$ mole tinh thể ion từ các ion tương ứng \textbf{ở trạng thái khí.}
\item Là quá trình tỏa nhiệt nên năng lượng mạng tinh thể ion mang \textbf{dấu âm.}
\item $\left| {U_{mtt}} \right|$ càng lớn thì lực liên kết trong tinh thể ion càng mạng, mạng tinh thể ion càng khó bị phá vỡ, hợp chất ion có nhiệt độ nóng chảy càng cao.
\end{itemize}
\subsubsection{Chu trình Born $-$ Haber}
$$U_{mtt} = -k \frac{Z^+ Z^-}{r^+ + r^-}.$$
Nhận xét:
\begin{itemize}
\item Hằng số $k$ phụ thuộc vào kiểu mạng tinh thể.
\item Tích số $Z^+ Z^-$ ảnh hưởng mạnh lên $U_{mtt}.$
\item Tổng số $r^+ + r^-$ ảnh hưởng lên $U_{mtt}$ \textbf{ít hơn} tích số $Z^+ Z^-.$
\end{itemize}
\subsection{Tính cộng hóa trị của hợp chất ion}
Mô hình biến dạng ion: sự biến dạng lớp vỏ electron của ion làm tăng tính cộng hóa trị, giảm tính ion của hợp chất. \textbf{Nghĩa là không có liên kết ion $\mathbf{100 \%.}$}
\subsubsection{Các yếu tố ảnh hưởng đến sự biến dạng ion}
Các yếu tố ảnh hưởng đến sự biến dạng ion là:
\begin{itemize}
\item Mật độ điện tích $q^+, r^+$ của cation: điện tích $q^+$ càng lớn, bán kính $r^+$ càng nhỏ thì cation có khả năng làm biến dạng anion càng nhanh.
\item Cấu hình electron của cation: cation có lớp vỏ ngoài cùng càng nhiều electron thì càng dễ bị biến dạng.
\item Kích thước anion: anion có kích thước càng lớn càng dễ bị biến dạng.
\end{itemize}
Việc so sánh và dự đoán nhiệt độ nóng chảy giữa các hợp chất ion cũng có giới hạn, chúng ta chỉ có thể so sánh và dự đoán với các dãy hợp chất tương tự nhau.
\section{Mô hình liên kết cộng hóa trị đơn giản trên cơ sở thuyết Lewis}
\subsection{Liên kết cộng hóa trị theo Lewis}
Các nguyên tử có khuynh hướng kết hợp với nhau để đạt tới cấu trúc lớp vỏ của các khí hiếm, lớp vỏ này thường có $8$ electron (trừ khí hiếm $\mathrm{He}$ ở chu kỳ $1$ có lớp vỏ $2$ electron), còn được gọi là \textbf{thuyết bát tử}, hay \textbf{thuyết bát bộ.}\\
Cơ cấu bát tử cũng có thể đạt được khi các nguyên tử góp chung electron hóa trị thành các cặp electron dùng chung, còn gọi là các\textbf{ cặp electron liên kết.}\\
Các cặp electron không dùng chung sẽ nằm ở từng nguyên tử và chỉ dùng cho nguyên tử đó, gọi là các \textbf{cặp electron không liên kết.} Do sự phân bố electron như vậy mô hình này được gọi là \textbf{mô hình electron định vị.}
\subsection{Biểu diễn công thức phân tử theo Lewis}
\subsection{Liên kết cộng hóa trị phối trí}
Cặp electron liên kết cũng có thể \textbf{do một nguyên tử đóng góp}, trong trường hợp này, liên kết cộng hóa trị gọi là \textbf{liên kết phối trí}, còn gọi là \textbf{liên kết cộng hóa trị cho $\mathbf{-}$ nhận.}
\subsection{Hóa trị và số phối trí trong hợp chất cộng hóa trị}
\subsubsection{Hóa trị}
Hóa trị là cộng hóa trị, là số liên kết cộng hóa trị mà một nguyên tử liên kết với các nguyên tử khác trong phân tử (hoặc trong mạng tinh thể nguyên tử).
\subsubsection{Số phối trí}
Số phối trí là số nguyên tử liên kết trực tiếp với một nguyên tử nào đó trong phân tử.
\subsection{Công thức cộng hưởng}
\subsection{Các phân tử không theo đúng quy tắc bát tử của Lewis}
\subsection{Các phân tử thiếu electron}
\subsection{Hạn chế của quy tắc bát tử}
\begin{itemize}
\item Quy tắc bát tử chỉ đúng đối với các nguyên tố ở chu kỳ $2.$ Do các nguyên tố này chỉ có tối đa $4$ orbital hóa trị, nên tối đa xung quanh nguyên tử có $8$ electron.
\item Từ chu kỳ $3$ trở đi, có sự xuất hiện thêm của orbital $\mathbf{nd}$ là orbital hóa trị. Nên các nguyên tử có thể có nhiều hơn $4$ liên kết (nhiều hơn $8$ electron).
\item Các thuyết như VB hay MO sẽ giải thích được điều này, chúng ta sẽ tìm hiểu ở các chương sau.
\end{itemize}
\subsection{Liên kết cộng hóa trị phân cực}
Liên kết cộng hóa trị phân cực xảy ra khi các nguyên tử tạo liên kết cộng hóa trị với nhau có độ âm điện khác nhau. Tức là ${\Delta _\chi } \ne 0.$
\subsection{Năng lượng, độ bền, bậc liên kết, độ dài liên kết và góc liên kết của liên kết cộng hóa trị}
\subsubsection{Năng lượng và độ bền}
Năng lượng liên kết $E$ là năng lượng cần thiết để cắt đứt liên kết thành các nguyên tử cô lập.
\subsubsection{Bậc liên kết}
Bậc liên kết dùng để chỉ số liên kết tạo được giữa hai nguyên tử đang xét. Bậc liên kết càng cao, liên kết càng bền.
\subsubsection{Độ dài liên kết}
Độ dài liên kết là khoảng cách cân bằng giữa hai hạt nhân của hai nguyên tử liên kết với nhau. Độ dài liên kết càng ngắn thì liên kết càng bền.
\subsubsection{Góc liên kết}
Góc liên kết là góc tạo thành bởi hai liên kết. 
\section{Cấu trúc không gian của các phân tử cộng hóa trị $-$ Thuyết đẩy đôi điện tử tầng hóa trị}
\subsection{Thuyết đẩy đôi điện tử tầng hóa trị (Valence Shell Electron Pair Repulsion - VSEPR)}
Các cặp electron liên kết và không liên kết trong phân tử chiếm các vùng không gian nhất định và phân bố sao cho \textbf{tương tác đẩy giữa chúng là ít nhất.}\\
Trong thuyết này ta sẽ chú ý đến: cặp electron liên kết và căp electron không liên kết.\\
Thuyết đẩy đôi điện tử tầng hóa trị:
\begin{itemize}
\item Các cặp electron liên kết phân bố giữa hai nguyên tử nên ít tập trung trên nguyên tử trung tâm.
\item Các cặp electron không liên kết tập trung gần nguyên tử trung tâm hơn, dẫn đến các cặp electron không liên kết chiếm vùng không gian lớn hơn các cặp electron liên kết, "ép" các cặp electron liên kết lại gần nhau hơn.
\item Các nguyên tử biên có độ âm điện cao hơn nguyên tử trung tâm sẽ làm cặp electron liên kết phân bố lệch về phía các nguyên tử biên hơn, khi đó cặp electron không liên kết trên nguyên tử trung tâm có cơ hội chiếm vùng không gian rộng hơn, kết quả là góc nối giảm.
\end{itemize}
\subsection{Các dạng hình học căn bản}
\begin{itemize}
\item Dạng thẳng
\item Tam giác đều
\item Tứ diện đều
\item Lưỡng tháp tam giác
\item Bát diện đều
\end{itemize}
\subsection{Hình học của các phân tử cộng hóa trị}
\subsubsection{Dạng thẳng}
$3$ nguyên tử cùng nằm trên một đường thẳng. Phân tử $\mathrm{AX_2}.$
\subsubsection{Dạng cơ bản tam giác phẳng}
Phân tử $\mathrm{AX_3}.$ Dạng hình học: tam giác phẳng.\\
Phân tử $\mathrm{AX_2 E}.$ Dạng hình học: phân tử có góc.
\subsubsection{Dạng cơ bản tứ diện}
Phân tử $\mathrm{AX_4}.$ Dạng hình học: tứ diện đều.\\
Phân tử $\mathrm{AX_3 E}.$ Dạng hình học: tháp đáy tam giác.\\
Phân tử $\mathrm{AX_2 E_2}.$ Dạng hình học: phân tử có góc.
\subsubsection{Dạng cơ bản lưỡng tháp tam giác}
Phân tử $\mathrm{AX_5}.$ Dạng hình học: lưỡng tháp tam giác.\\
Phân tử $\mathrm{AX_4 E}.$ Dạng hình học: bập bênh (see-saw).\\
Phân tử $\mathrm{AX_3 E_2}.$ Dạng hình học: chữ T.\\
Phân tử $\mathrm{AX_2 E_3}.$ Dạng hình học: dạng thẳng.
\subsubsection{Dạng cơ bản bát diện đều}
Phân tử $\mathrm{AX_6}.$ Dạng hình học: bát diện đều.\\
Phân tử $\mathrm{AX_5 E}.$ Dạng hình học: tháp đáy vuông, có hình dạng như kim tự tháp.\\
Phân tử $\mathrm{AX_4 E_2}.$ Dạng hình học: vuông phẳng.
\section{Moment lưỡng cực của phân tử cộng hóa trị}
Tránh nhầm lẫn giữa phân tử phân cực và liên kết cộng hóa trị phân cực.