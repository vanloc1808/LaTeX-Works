\chapter{Dung dịch và tính chất của dung dịch}
\section{Một số khái niệm và thuật ngữ}
\textbf{Dung dịch:} là một hỗn hợp đồng thể, có thành phần và tính chất đồng nhất tại tất cả các điểm trong dung dịch.\\
\textbf{Dung môi:} chất với lượng chiều hơn.\\
\textbf{Chất tan:} chất với lượng ít hơn.\\
\textbf{Dung dịch loãng} nếu nồng độ chất tan trong dung dịch tương đối thấp. \textbf{Dung dịch đặc} nếu nồng độ chất tan trong dung dịch tương đối cao.\\
\textbf{Dung dịch khí:} thường được gọi là hỗn hợp khí (các chất khí không có phản ứng với nhau, áp dụng được định luật Dalton).\\
\textbf{Dung dịch rắn:} ví dụ như các hợp kim (hai kim loại không có phản ứng với nhau).
\section{Nồng độ dung dịch}
\subsection{Nồng độ phần trăm theo khối lượng, phần trăm theo thể tích}
\textbf{Nồng độ phần trăm theo khối lượng} là lượng chất tan có trong $100 \text{ } \mathrm{g}$ dung dịch.\\
\textbf{Nồng độ phần trăm theo thể tích} là tỉ số giữa thể tích chất tan và thể tích dung dịch.
\subsection{Nồng độ ppm, ppb và ppt}
Nồng độ ppm, ppb và ppt là nồng độ phần trăm khối lượng hay phần trăm thể tích (với lượng chất tan thấp).\\
ppm (parts per million, phần triệu): $1 \text{ } \mathrm{ppm} = 10^{-3} \mathrm{g/l}.$\\
ppb (parts per billion, phần tỷ): $1 \text{ } \mathrm{ppb} = 10^{-6} \mathrm{g/l}.$\\
ppm (parts per trillion, phần nghìn tỷ): $1 \text{ } \mathrm{ppt} = 10^{-9} \mathrm{g/l}.$
\subsection{Nồng độ phần mole và phần trăm mole}
$$x_i = \frac{n_i}{n_1 + n_2 + ... + n_n}$$
$$x_1 + x_2 + ... + x_n = 1.$$
\subsection{Nồng độ mole (molarity)}
Là số mole chất tan \textbf{trong một đơn vị thể tích.}
$$M = \frac{\text{số mol chất tan (mol)}}{\text{thể tích dung dịch (l)}}.$$
Phương pháp đo: dùng các dụng cụ để cân khối lượng và đo thể tích để pha dung dịch.
\subsection{Nồng độ molal (molality)}
Là số mole chất tan trong $1 \text{ } \mathrm{kg}$ \textbf{dung môi.}
$$\text{Molality} = \frac{\text{số mol chất tan (mol)}}{\text{Khối lượng dung môi (kg)}}.$$
Phương pháp đo: dùng cân để cân khối lượng của chất tan và dung môi.
\section{Sự hòa tan và nhiệt hòa tan của các chất}
\subsection{Quá trình hòa tan và nhiệt hòa tan của các hợp chất phân tử}
Các quá trình hòa tan thường kèm theo sự tỏa nhiệt hoặc thu nhiệt, gọi là nhiệt hòa tan, hay enthalpy hòa tan ($\Delta H_{\text{hòa tan}}$).
\begin{itemize}
\item Quá trình tách các phân tử dung môi ra khỏi nhau ($\Delta H_a > 0$);
\item Quá trình tách các phân tử chất tan ra khỏi nhau ($\Delta H_b > 0$);
\item Các phân tử chất tan và dung môi hút lẫn nhau để hòa tan vào nhau và tạo thành dung dịch ($\Delta H_c < 0$).
\end{itemize}
$$\Delta H_{\text{hòa tan}} = \Delta H_a + \Delta H_b + \Delta H_c.$$
Các trường hợp của $\Delta H_{\text{hòa tan}}$:
\begin{itemize}
\item $\Delta H_{\text{hòa tan}} = 0$: ta có thể dự đoán tính chất của dung dịch tạo thành qua tính chất của chất tan và dung môi tinh chất, và dung dịch tạo thành được gọi là \textbf{dung dịch lý tưởng.}
\item $\Delta H_{\text{hòa tan}} < 0:$ quá trình hòa tan kèm theo sự tỏa nhiệt. Thông thường, không thể dự đoán tính chất của dung dịch tạo thành qua tính chất của chất tan và dung môi tinh chất, và dung dịch tạo thành được gọi là \textbf{dung dịch không lý tưởng.}
\item $\Delta H_{\text{hòa tan}} > 0:$ quá trình hòa tan kèm theo sự thu nhiệt. Thông thường, không thể dự đoán tính chất của dung dịch tạo thành qua tính chất của chất tan và dung môi tinh chất, và dung dịch tạo thành được gọi là \textbf{dung dịch không lý tưởng.}
\item $\Delta H_{\text{hòa tan}} \gg 0:$ các chất sẽ không hòa tan vào nhau được, tọa thành các lớp.
\end{itemize}
\subsection{Sự hòa tan và nhiệt hòa tan các hợp chất ion trong nước}
$$\Delta H_{\text{hòa tan}} = \Delta H_a + \Delta H_b,$$
trong đó:
\begin{itemize}
\item Quá trình phá vỡ mạng tinh thể ion, tạo thành các ion ở trạng thái khí ($\Delta H_a = - U_{mtt} > 0$).
\item Quá trình hydrate hóa (solvate hóa) ion là quá trình chuyển các ion ở trạng thái khí thành dạng hydrate (solvate) ($\Delta H_b = \Delta H_{b+} + \Delta H_{b-} < 0$).
\end{itemize}
Quá trình hòa tan các hợp chất ion trong nước thuận lợi khi năng lượng mạng tinh thể ion không quá lớn.
\section{Cân bằng hòa tan $-$ kết tủa và độ tan của hợp chất ion trong nước}
Quá trình cân bằng hòa tan $ \rightleftarrows $ kết tủa là quá trình cân bằng động.\\
Lượng tan tối đa của chất tan trong nước hay trong dung môi nào đó ở nhiệt độ nhất định gọi là \textbf{độ tan.}\\
Khi $\Delta H_{\text{hòa tan}} < 0,$ cân bằng dịch chuyển theo chiều nghịch khi tăng nhiệt độ (độ tan muối giảm).\\
Khi $\Delta H_{\text{hòa tan}} > 0,$ cân bằng dịch chuyển theo chiều thuận khi tăng nhiệt độ (độ tan muối tăng).
\subsection{Kết tinh muối từ dung dịch nước}
\subsubsection{Phương pháp kết tinh đa nhiệt (đối với muối có độ tan thay đổi lớn theo nhiệt độ)}
Các bước kết tinh đa nhiệt:
\begin{itemize}
\item Tạo dung dịch bão hòa của muối ở nhiệt độ cao.
\item Làm lạnh dung dịch tới nhiệt độ thích hợp, đọ tan của muối giảm nên dung dịch trở thành quá bão hòa.
\item Muối sẽ kết tinh cho tới khi dung dịch đạt bão hòa.
\item Có thể đưa mầm tinh thể vào dung dịch quá bão hòa.
\end{itemize}
\subsubsection{Phương pháp kết tinh đẳng nhiệt (đối với muối có độ tan ít thay đổi theo nhiệt độ)}
Làm bay hơi dung môi ở nhiệt độ thích hợp.
\subsection{Tinh chế chất rắn bằng phương pháp kết tinh phân đoạn}
Nhằm thu được chất rắn có độ tinh khiết cao hơn ban đầu.\\
Chất cần tinh thể đạt điều kiện dung dịch bão hòa trước khi tạp chất đạt điều kiện dung dịch bão hòa.\\
Có thể lặp lại quá trình kết tinh lại nhiều lần để thu được chất rắn với độ tinh khiết cao.\\
Lựa chọn dung môi hữu cơ thích hợp để tinh chế các chất hữu cơ.
\section{Độ tan của các khí trong dung dịch}
\subsection{Ảnh hưởng của nhiệt độ đến độ tan của khí trong dung dịch}
Hầu hết các khí có độ tan trong nước giảm khi nhiệt độ tăng.\\
Độ tan trọng nước của các khí hiếm phức tạp hơn và thường giảm khi tăng tới nhiệt độ nào đó, sau đó độ tan các khí hiếm lại tăng theo nhiệt độ.\\
Độ tan các khí thường tăng theo nhiệt độ trong nhiều dung môi hữu cơ.
\subsection{Ảnh hưởng của áp suất đến độ tan của khí trong dung dịch $-$ Định luật Henry}
Định luật Henry biểu diễn sự phụ thuộc áp suất một chất khí lên độ tan cảu nó trong dung dịch nước.
$$C = k \times P_{\text{khí}}.$$
\section{Áp suất hơi của dung dịch và phương pháp chưng cất phân đoạn}
\subsection{Định luật Raoult}
Khi các dung dịch chỉ chứa dung môi A và chất tan B, với A và B là các chất lỏng dễ bay hơi, áp suất hơi riêng phần của A và B trên dung dịch luôn luôn thấp hơn áp suất hơi của A và B tinh chất.
$$P_A = x_A \times P_A^0.$$
Do $x_A + x_B = 1$ nên
$$x_B = \frac{P_A^0 - P_A}{P_A^0}.$$
Định luật Raoult chỉ áp dụng đúng cho dung dịch lý tương và các thành phần có thể bay hơi trong dung dịch.
\subsection{Cân bằng lỏng $-$ hơi của dung dịch lý tưởng}
Pha hơi luôn giàu chất dễ bay hơi hơn pha dung dịch.
\subsection{Quá trình chưng cất phân đoạn dung dịch lý tưởng}
Chưng cất phân đoạn là phương pháp dùng để tách các chất lỏng trong dung dịch ra khỏi nhau dựa vào khả năng bay hơi khác nhau của chúng.
\subsection{Cân bằng lỏng $-$ hơi của dung dịch không lý tưởng}
Điểm hằng phị: là điểm mà thành phần dung dịch trong pha lỏng và pha hơi là bằng nhau.\\
Có thể tiến hành chưng cất cho đến khi thu được dung dịch hằng phị, khi đó ta không tiếp tục chưng cất được nữa.
\section{Áp suất thẩm thấu của dung dịch}
Hiện tượng các phân tử nước đi qua màng bán thấm làm thay đổi nồng độ dung dịch hai bên màng gọi là \textbf{hiện tượng thẩm thấu.}\\
Áp suất cần thiết để chặn không cho hiện tượng thẩm thấu xảy ra được gọi là \textbf{áp suất thẩm thấu.}
$$\pi = \frac{n}{V}RT = CRT.$$
\section{Độ hạ nhiệt độ đông đặc và độ tăng nhiệt độ sôi của các dung dịch}
\subsection{Dung dịch chứa chất tan không điện ly}
Dung dịch có nhiệt độ đông đặc thấp hơn và nhiệt độ sôi cao hơn dung môi của chúng.\\
Độ hạ nhiệt độ đông đặc: $\Delta T_f = -m K_f.$\\
Độ tăng nhiệt độ sôi: $\Delta T_b = mK_b.$
\subsection{Dung dịch chứa chất tan điện ly}
Trong dung dịch các chất tan điện ly tạo thành các ion âm và ion dương, do vậy phải tính nồng độ tổng cộng của các ion trong dung dịch.
$$\text{Hệ số van Hoff } i = \frac{\Delta T_{f \text{ thực tế}}}{\Delta T_{f \text{ dự đoán}}}$$
$$\pi = i \frac{n}{V}RT = iCRT.$$
$$\Delta T_f = - i m K_f.$$
$$\Delta T_b = i mK_b.$$
\section{Dung dịch keo}
\textbf{Dung dịch keo} là dung dịch có chất phân tán lơ lửng mà không lắng, có kích thước ít nhất một trong các chiều trong khoảng $10 - 1000 \mathrm{nm}.$\\
Các hạt phân tán trong dung dịch keo được gọi là \textbf{các hạt keo.}\\
Các hạt keo có thể có hình cầu, hình dĩa, hoặc hình que, ...