\chapter{Cấu trúc electron trong nguyên tử}
\section{Một số khám phá vật lý quan trọng đầu thế kỷ XX}
\subsection{Sóng điện từ}
Bốn đặc trưng cơ bản của sóng điện từ:
\begin{itemize}
\item Tốc độ truyền sóng $C$,
\item Bước sóng $\lambda,$
\item Tần số $\nu,$
\item Cường độ.
\end{itemize}
$$C = \lambda \cdot \nu.$$
Hiện tượng giao thoa ánh sáng và hiện tượng nhiễu xạ tia X là bằng chứng cho thấy bản chất sóng của sóng điện từ.
\subsection{Quang phổ}
\subsubsection{Quang phổ ánh sáng khả kiến}
Hiện tượng tán sắc ánh sáng: khi chiếu ánh sáng mặt trời qua lăng kính, tia sáng đỏ bị lệch hướng ít nhất, tia sáng tím bị lệch hướng nhiều nhất.\\
Quang phổ của ánh sáng khả kiến là quang phổ liên tục.
\subsubsection{Quang phổ nguyên tử}
Khi cho dòng điện phóng qua các ống chứa khí hay hơi nguyên tử, thu được bức xạ có bước sóng rời rạc.
\begin{itemize}
\item Mỗi loại khí có phổ vạch đặc trưng khác nhau, giống như mỗi người có dấu vân tay khác nhau.
\item Quang phổ này do nguyên tử phát ra nên gọi là phổ nguyên tử, hay phổ phát xạ nguyên tử, hay phổ vạch.
\end{itemize}
\subsubsection{Quang phổ phát ra từ các vật nóng}
\begin{itemize}
\item Khi đốt nóng, các vật rắn phát ra bức xạ với bước sóng liên tục.
\item Khi tăng nhiệt độ, bước sóng với cường độ cực địa di chuyển về phía sóng ngắn hơn.
\end{itemize}
\subsection{Khái niệm lượng tử ánh sáng của Max Planck}
Năng lượng có tính gián đoạn tương tự như vật chất.\\
Năng lượng lượng tử của sóng điện từ;
$$E = h \nu,$$ trong đó:
\begin{itemize}
\item $\nu:$ là tần số ánh sáng,
\item $h:$ là hằng số Planck, có giá trị là $6.626 \cdot 10^{- 34} \mathrm{Js}.$
\end{itemize}
\subsection{Hiện tượng quang điện và thuyết lưỡng nguyên ánh sáng của Albert Einstein}
\subsubsection{Hiện tượng quang điện}
Kết quả thí nghiệm cho thấy:
\begin{itemize}
\item Khi $\nu > \nu_0$ thì mới có sự thoát ra của electron, gọi là tần số ngưỡng quang điện.
\item Cường độ dòng quang điện tăng theo cường độ ánh sáng chiếu vào.
\item Động năng của electron tăng theo tần số $\nu$ của ánh sáng chiếu vào.
\item Mỗi kim loại có tần số ngưỡng quang điện $\nu_0$ khác nhau.
\end{itemize}
\subsubsection{Thuyết lưỡng nguyên ánh sáng của Einstein}
\begin{itemize}
\item Bức xạ điện từ không chỉ có tính sóng mà còn có tính hạt.
\item Bức xạ điện từ là dòng các photon, mỗi photon có năng lượng nhất định $E = h \nu.$
\item Những photon có $\nu > \nu_0$ thì electron mới tách khỏi nguyên tử.
\item Khi cường độ photon tăng thì số lượng các photon càng nhiều, số lượng electron bật ra khỏi bề mặt kim loại càng nhiều, cường độ dòng quang điện càng lớn.
\item Động năng của electron được tính như sau:
$$E_{d} = \frac{1}{2}mv^2 = E - E_0 = h \nu - h \nu_0.$$
\item Photon không có khối lượng nghỉ, nhưng có khối lượng khi di chuyển. Năng lượng của photon được tính theo công thức:
$$E = mc^2.$$
\item Khi chiếu chùm photon vào chùm electron, khi có va chạm photon chuyển một phần năng lượng cho electron. Như vậy, ánh sáng có tính lưỡng nguyên: sóng hạt.
\end{itemize}
\section{Mô hình nguyên tử $\mathrm{H}$ của Bohr}
\subsection{Mô hình nguyên tử $\mathrm{H}$ của Bohr}
Năm 1923, Niels Bohr kết hợp vật lý cổ điển với khái niệm lượng tử của Planck để đưa ra mô hình nguyên tử của $\mathrm{H}.$
\begin{itemize}
\item Electron chỉ được phép chuyển động trên một số quỹ đạo nhất định, gọi là \textbf{quỹ đạo dừng.}
\item Khi ở một trạng thái dừng nào đó, electron có năng lượng xác định. Electron trên quỹ đạo dừng chỉ mang moment góc:
$$m \nu r = \frac{nh}{2 \pi}.$$
\item Nguyên tử chỉ háp thu hay phát xạ năng lượng khi electron di chuyển từ quỹ đạo dừng này sang quỹ đạo dừng khác. Năng lượng hấp thu hay phát xạ khi đó là:
$$\Delta E = E_{\text{trạng thái cuối}} - E_{\text{trạng thái đầu}}.$$
\end{itemize}
\subsection{Nhược điểm của mô hình nguyên tử $\mathrm{H}$ của Bohr}
\begin{itemize}
\item Kết quả không đúng với các nguyên tử khác.
\item Mô hình Bohr là sự kết hợp thô sơ giữa các định luật vật lý cổ điển và không cổ điển mà không dựa trên cơ sở khoa học.
\end{itemize}
\section{Những luận điểm cơ sở và những ý tưởng chính dẫn đến thuyết cơ học lượng tử}
\subsection{Giả thiết của Louis de Broglie $-$ Tính lưỡng nguyên của vật chất}
Nếu ánh sáng có cả tính sóng và hạt thì vật chất, đặc biệt là các hạt nhỏ, cũng có thể có đồng thời hai tính hạt và sóng.
$$\lambda = \frac{h}{mv},$$
trong đó:
\begin{itemize}
\item $h:$ hằng số Planck,
\item $m:$ khối lượng của vật,
\item $v:$ vận tốc di chuyển của vật.
\end{itemize}
Hiện tượng nhiễu xạ electron chứng tỏ vật chất $-$ hay ít nhất là electron $-$ có tính sóng.\\
Năng lượng vừa có tính sóng vừa có tính hạt; vật chất có tính hạt, cũng có tính sóng.
\subsection{Nguyên lý bất định Heisenberg}
Không thể xác định chính xác đồng thời tốc độ và vị trí của electron.
$$\Delta x \times \Delta p \geqslant \frac{h}{4 \pi}.$$
Thế giới vi hạt tuân theo nguyên lý bất định Heisenberg vì chúng có cả hai đặc tính sóng và hạt.
\subsection{Phương trình sóng Schrodinger mô tả chuyển động của electron trong nguyên tử hydrogen}
Phương trình sóng Schrodinger cho nguyên tử $\mathrm{H}$ có một nhân mang điện tích dương và một electrong mang điện tích âm có dạng sau:
$$\frac{{{\mathrm{d}^2}\Psi }}{{\mathrm{d}{x^2}}} = \left( {\frac{{2\pi }}{\lambda }} \right)\Psi .$$
\section{Mô hình nguyên tử hydrogen theo thuyết cơ học lượng tử}
\subsection{Kết quả giải phương trình Schrodinger cho nguyên tử hydrogen: ba số lượng tử, hàm sóng $\Psi$, và orbital nguyên tử}
Giải phương trình Schrodinger cho nguyên tử hydrogen dẫn đến các hàm sóng $\Psi$ với ba tham số $n, \ell, m_{\ell},$ gọi là ba số lượng tử, đặc trưng cho các hàm sóng:
\begin{itemize}
\item $n$ gọi là \textbf{số lượng tử chính,} nhận giá trị là các số tự nhiên dương $1, 2, 3, ...$
\item $\ell$ là \textbf{số lượng tử moment góc,} còn gọi là \textbf{số lượng tử phụ,} có các giá trị từ $0$ đến $n - 1.$
\item $m_{\ell}$ là \textbf{số lượng tử từ,} có giá trị từ $- \ell \to \ell.$
\end{itemize}
\subsection{Ý nghĩa của $\Psi$ $-$ orbital nguyên tử}
Ý nghĩa của hàm $\Psi:$
\begin{itemize}
\item Phương trình $\Psi$ mô tả chuyển động của electron trong nguyên tử.
\item $\Psi ^2$ đặc trưng cho xác suất bắt gặp electron (chính xác hơn là mật độ electron) tại vị trí nào đó quanh nhân nguyên tử.
\item Vùng không gian quanh nhân có khả năng tìm thấy điện tử nhiều nhatasm gọi là orbital nguyên tử (Atomic Orbital, AO), hay đám mây điện tử, vân đạo điện tử.
\item Bộ ba số lượng tử $n, \ell, m_{\ell}$ đặc trưng cho mỗi AO tương ứng. 
\end{itemize}
Số lượng tử chính $n:$
\begin{itemize}
\item Đặc trưng cho \textbf{kích thước và năng lượng} của orbital.
\item Các orbital có năng lượng bằng nhau gọi là \textbf{orbital suy biến năng lượng.}
\end{itemize}
Số lượng tử phụ $\ell:$
\begin{itemize}
\item Đặc trưng cho \textbf{moment động lượng của electron} và \textbf{hình dạng của orbital.}
\item Hình dạng của orbital được quy ước là hình dạng của vùng không gian có xác suất bắt gặp electron cao nhất.
\item Các orbital có $\ell = 0$ ($s$) có dạng hình cầu, các orbital có $\ell = 1$ ($p$) có dạng hình số tám nổi, các orbital có $\ell = 2$ ($d$) có hình dạng phức tạp hơn.
\end{itemize}
Số lượng tử phụ $m_{\ell}:$
\begin{itemize}
\item Đặc trưng cho \textbf{sự định hướng các orbital trong không gian.}
\end{itemize}
\subsection{Spin của electron $-$ số lượng tử thứ tư}
Bản thân các electron có moment từ nội tại, có thể định hướng theo hai kiểu khác nhau dưới tác dụng của từ trường ngoài, gọi là \textit{spin} của electron. \\
Do đó, ngoài ba số lượng tử xuất phát từ phương trình Schrodinger, cần có số lượng tử thứ tư biểu diễn cho đặc tính từ của electron, gọi là \textit{số lượng tử spin,} ký hiệu là $m_s.$
\subsection{Nguyên tử nhiều electron}
\begin{itemize}
\item Tương tác của electron trong nguyên tử nhiều electron rất phức tạp.
\item Không thể viết và giải phương trình Schrodinger một cách chính xác cho nguyên tử nhiều electron.
\item Chỉ có thể thiết lập và giải gần đúng phương trình Schrodinger cho nguyên tử nhiều electron bằng nhiều phương pháp khác nhau.
\end{itemize}
Các electron trên orbital $ns$ có mật độ electron gần nhân cao hơn electron trên orbital $np$ và $nd$ cùng lớp.
\subsection{Nguyên lý loại trừ Pauli $-$ Quy tắc Hund $-$ Quy tắc Klechkowski}
\subsubsection{Nguyên lý loại trừ Pauli}
Mỗi orbital nguyên tử chỉ có thể chứa tối đa hai electron với spin khác nhau. Để phân biệt hai electron này, ta dùng số lượng tử spin $m_s.$\\
Trong một nguyên tử \textbf{không thể có hai electron có cùng bốn số lượng tử.}
\subsubsection{Quy tắc Hund}
Các electron có khuynh hướng phân bố đều vào các orbital sao cho \textbf{tổng spin của các electron trong nguyên tử là cực đạ}i để tương tác đẩy giữa các electron trong cùng phân lớp là thấp nhất.
\subsubsection{Quy tắc Klechkowski}
Là quy tắc để ghi nhớ việc điền electron.