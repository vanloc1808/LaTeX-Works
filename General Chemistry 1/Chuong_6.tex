\chapter{Các mô hình liên kết hóa học hiện đại}
Các mô hình liên kết cộng hóa trị theo Lewis hay thuyết VSEPR không giúp giải thích được sự hình thành của các phân tử như $\mathrm{SF_6}, \mathrm{PCl_5}, ...$ (nhiều hơn $8$ electron).\\
Sự hình thành liên kết hóa học là do sự tham gia của các orbital nguyên tử, cần phải giải phương trình sóng Schrodinger cho các bài toán phân tử để xác định hàm sóng mô tả trạng thái các phân tử.\\
Việc giải các phương trình này quá khó, cần dùng đến các phương pháp gần đúng.\\
Có hai phương pháp chính:
\begin{itemize}
\item Phương pháp VB (Valence Bond)
\item Phương pháp MO (Molecular Orbital)
\end{itemize}
\section{Thuyết VB $-$ mô hình liên kết cộng hóa trị định chỗ}
\subsection{Sự tạo thành liên kết cộng hóa trị theo thuyết VB}
Liên kết hóa học tạo thành do sự xen phủ các orbital giữa các nguyên tử tham gia liên kết, liên kết tạo thành có hai electron trong vùng xen phủ.
\subsection{Các luận điểm cơ bản của thuyết VB}
Thuyết liên kết cộng hóa trị (Valence Bond $-$ VB) có các luận điểm chính sau:
\begin{itemize}
\item Liên kết được hình thành do sự xen phủ của $2$ orbital nguyên tử (AO) chứa $2$ electron độc thân.
\item Các electron liên kết phân bố ở vùng không gian giữa hai nguyên tử, nên mô hình VB được gọi là mô hình electron định vị (định chỗ, định xứ, khu trú).
\item Liên kết cộng hóa trị còn được hình thành do sự xen phủ của một orbital nguyên tử chứa hai electron với một orbital trống (không chứa electron) của nguyên tử thứ hai, gội là liên kết cộng hóa trị cho $-$ nhận (liên kết phối trí).
\item Các nguyên tử có thể kích thích các electron hóa trị ghép cập lên các orbital hóa trị trống để tạo thành orbital chứa electron độc thân để xen phủ tạo liên kết $\to$ \textbf{giải thích được tại sao quy tắc bát bộ không còn đúng với các nguyên tố ở chu kỳ lớn.}
\end{itemize}
\subsection{Hai kiểu xen phủ căn bản}
\subsubsection{Xen phủ $\sigma$ $-$ xen phủ trục}
Liên kết $\sigma$ được tạo thành do sự xen phủ dọc theo trục nối nhân của hai orbital hóa trị của hai nguyên tử tạo liên kết có hai electron trong vùng xen phủ, kiểu xen phủ như vậy còn được gọi là xen phủ trục.
\subsubsection{Xen phủ $\pi$ $-$ xen phủ bên}
Liên kết $\pi$ được tạo thành do sự xen phủ bên (xen phủ hông) giữa hai orbital định hướng song song với nhau và vuông góc với trục nối nhân.\\
Dấu $+, -$ trong xen phủ bên là dấu trong hàm sóng $\Psi.$ Hai hàm sóng ngược dấu sẽ triệt tiêu nhau, nên liên kết không hiệu quả.\\
\textbf{Mật độ electron của xen phủ $\sigma$ cao hơn xen phủ $\pi$ do nó nằm trên đường nối hai hạt nhân, nên liên kết $\sigma$ bền hơn liên kết $\pi.$}
\subsection{Đặc điểm của liên kết cộng hóa trị theo thuyết VB}
\textbf{Tính bão hòa:} do việc xen phủ tạo liên kết cộng hóa trị bị giới hạn ở số orbital hóa trị nên liên kết cộng hóa trị có tính bão hòa. Các số phối trí thường gặp: $6, 4, 3, 2.$\\
\textbf{Tính định hướng:}
\begin{itemize}
\item Muốn tạo liên kết cộng hóa trị bền thì sự xen phủ các orbital nguyên tử phải cực đại.
\item Sự xen phủ cực đại chỉ xảy ra theo hướng nhất định nên liên kết cộng hóa trị mang tính định hướng, lực liên kết cộng hóa trị cũng chỉ mạnh theo hướng tạo liên kết mà thôi.
\item Do tính định hướng, các phân tử cộng hóa trị có dạng hình học nhất định.
\end{itemize}
\subsection{Thuyết lai hóa (tạp chủng) vân đạo hóa trị}
Để giải thích sự tạo thành liên kết cộng hóa trị trong các phân tử cho phù hợp với dạng hình học của chúng, thuyết VB đã đề nghị cơ chế lai hóa (tạp chủng) của các orbital hóa trị tạo thành các orbital nguyên tử lai hóa có định hướng thích hợp cho sự xen phủ.\\
Nội dung cơ bản của thuyết lai hóa orbital hóa trị:
\begin{itemize}
\item Trong một nguyên tử, các orbital hóa trị có thể lai hóa với nhau để tạo thành các orbital lai hóa.
\item Sự lai hóa chỉ xảy ra đối với các orbital của cùng một nguyên tử.
\item Có bao nhiêu orbital nguyên tử tham gia lai hóa thì có bấy nhiêu orbital lai hóa tạo thành.
\item Các orbital lai hóa tạo thành có hình dạng và năng lượng giống nhau, phân bố đối xứng trong không gian phù hợp với góc liên kết tạo thành.
\end{itemize}
Thuyết lai hóa là một thuyết con của thuyết VB.\\
Từ thuyết VSEPR có thể suy ra dạng lai hóa.\\
Phương trình hàm sóng Schrodinger cho $4$ orbital $sp^3:$
$$\Psi_{sp^3} = a \Psi_{2s} + b \Psi_{2px} + b \Psi_{2py} + b \Psi_{2pz}.$$
\subsection{Hệ thống liên kết $\pi$ bất định vị trong các phân tử có sự cộng hưởng}
$$\text{Bậc liên kết trung bình} = \frac{\text{tổng số electron tham gia vào hệ liên kết}}{2 \text{ lần số liên kết } \sigma}.$$
\subsection{Acid và base Lewis}
\subsubsection{Acid Lewis}
Acid Lewis là những phân tử hay ion có orbital hóa trị trống có thể xen phủ được với orbital chứa hai electron của phân tử hay ion.
\subsubsection{Base Lewis}
Base Lewis là những phân tử hay ion có orbital chứa cặp electron chưa liên kết và có thể xen phủ với orbital trống của phân tử acid Lewis để tạo liên kết cộng hóa trị.
\subsubsection{Sự tạo phức}
\subsection{Liên kết cộng hóa trị trong mạng tinh thể $-$ các "đại phân tử"}
\subsubsection{Cách xác định trạng thái lai hóa của nguyên tử trung tâm}
Cách 1: dựa vào công thức Lewis
$$\text{Trạng thái lai hóa (số AO của nguyên tử trung tâm tham gia lai hóa) }$$ 
$$= \text{ số liên kết } \sigma + \text{ số cặp electron không liên kết.}$$
Cách 2: dùng công thức tính
$$\text{Tổng số cặp electron (liên kết + không liên kết)}$$
$$ = \frac{1}{2} \left( {\text{số electron hóa trị } + \text{ số electron đóng góp } - \text{ điện tích ion}} \right)$$
$\mathrm{S}, \mathrm{O}$ là nguyên tử biên thì không đóng góp electron.\\
$\mathrm{N}$ đóng góp $-1$ electron.
\subsubsection{Độ bền liên kết cộng hóa trị theo thuyết VB}
Độ bền của liên kết cộng hóa trị càng lớn khi độ xen phủ của các orbital càng lớn, nghĩa là mật độ electron giữa hai hạt nhân là lớn nhất.\\
Độ xen phủ phụ thuộc vào: hình dạng, kích thước, năng lượng của các orbital, hướng xen phủ và kiểu xen phủ giữa chúng.\\
Tóm lại, liên kết cộng hóa trị càng bền khi:
\begin{itemize}
\item Đồng năng: các orbital liên kết có mức năng lượng càng gần nhau.
\item Xen phủ: thể tích vùng xen phủ càng lớn.
\item Mật độ: mật độ electron trong vùng xen phủ càng lớn.
\end{itemize}
Trong các yếu tố đó thì \textbf{mật độ electron là yếu tố quan trọng nhất.}
\section{Thuyết orbital phân tử $-$ mô hình liên kết cộng hóa trị với electron bất định chỗ}
\subsection{Hạn chế của thuyết VB}
Ưu điểm:
\begin{itemize}
\item Giải thích được độ bền của liên kết cộng hóa trị (một cách định tính).
\item Giải thích được hình dạng phân tử thông qua sự lai hóa.
\end{itemize}
Nhược điểm:
\begin{itemize}
\item Không giải thích được từ tính của oxygen.
\item Không giải thích được sự tồn tại của phân tử $\mathrm{H_2^+}.$
\end{itemize}
\subsection{Sự tạo thành orbital phân tử theo thuyết MO}
Thuyết orbital phân tử là lý thuyết về cấu trúc điện tử của phân tử.\\
Orbital phân tử (molecular orbital) hình thành do sự tương tác của các orbital nguyên tử tham gia liên kết.\\
Orbital phân tử liên hệ với toàn bộ phân tử, không thuộc riêng nguyên tử nào.\\
Xác suất bắt gặp electron trong vùng nối hai hạt nhân của hàm sóng cùng pha cao hơn so với hàm sóng nghịch pha.\\
Electron trong MO \textbf{phản liên kết} sẽ hút hai hạt nhân lại và có năng lượng \textbf{cao hơn} so với orbital nguyên tử ban đầu.\\
Electron trong MO \textbf{liên kết} sẽ hút hai hạt nhân lại và có năng lượng \textbf{thấp hơn} so với orbital nguyên tử ban đầu.\\
Để có sự tạo liên kết hữu hiệu, các \textbf{orbital nguyên tử} tham gia liên kết phải:
\begin{itemize}
\item Có năng lượng xấp xỉ nhau, năng lượng quá chênh lệch thì sự liên kết không hiệu quả.
\item Có tính đối xứng tương tự nhau qua trục liên kết.
\item Phải gần nhau đáng kể để xen phủ hiệu quả.
\item Chỉ có các AO hóa trị mới đóng góp hình thành MO.
\item Các AO hóa trị không phù hợp tạo liên kết, được chuyển thẳng vào trong phân tử, trở thành MO không liên kết có năng lượng bằng với AO ban đầu.
\end{itemize}
\subsection{Tóm tắt nguyên tắc và quan điểm chung của thuyết MO}
\begin{itemize}
\item Các electron hóa trị bất định xứ trong toàn bộ phân tử, không bị giới hạn là nằm ở những nguyên tử hay liên kết riêng lẻ.
\item Orbital phân tử (khác với orbital nguyên tử) được hình thành do sự tổ hợp tuyến tính của các orbital nguyên tử.
\item Có bao nhiêu orbital nguyên tử tham gia tổ hợp sẽ tạo thành bấy nhiêu orbital phân tử.
\item Các electron phân bố vào các MO có năng lượng từ thấp lên cao. Các quy tắc Hund, loại trừ Pauli được tuân thủ như trường hợp cái AO.
\item Độ bền liên kết được đánh giá thông qua bậc liên kết:
$$\text{bậc liên kết } = \frac{1}{2} \left( {\Sigma \text{electron trên orbital liên kết } - \Sigma \text{electron trên orbital phản liên kết} } \right).$$
\item Theo mô hình này:
\subitem Không có nguyên tử riêng biệt trong phân tử.
\subitem Các electron và MO thuộc về phân tử mà không thuộc về nguyên tử.
\end{itemize}
Mô hình này được gọi là mô hình electron bất định chỗ (bất định vị).
\subsection{Các phân tử nhị nguyên đồng nhân thuộc chu kỳ 1}
\subsection{Các phân tử nhị nguyên đồng nhân thuộc chu kỳ 2}
Lưu ý trường hợp $Z \leqslant 7.$
\section{Thuyết dãy $-$ sự dẫn điện trong kim loại, bán dẫn và chất cách điện}