\chapter{Trạng thái khí}
\section{Sơ lược}
Sự chuyển pha là sự chuyển trạng thái của các hợp chất từ trạng thái tập hợp này sang trạng thái tập hợp khác. Các sự chuyển pha mà ta sẽ xem xét:
\begin{itemize}
\item Sự nóng chảy: rắn sang lỏng.
\item Sự đông đặc: lỏng sang rắn.
\item Sự hóa hơi (bay hơi): lỏng sang khí.
\item Sự ngưng tụ: khí sang lỏng/rắn.
\item Sự thăng hoa: rắn sang khí.
\end{itemize}
\section{Vài tính chất căn bản của chất khí}
Khí có thể bị nén lại thành thể tích nhỏ hơn, có nghĩa là khối lượng riêng của chúng tăng lên.
\begin{itemize}
\item Chất khí có khả năng khuếch tán, luôn chiếm đầy bình chứa, do đó có thể tích và hình dạng của bình chứa. 
\item Khoảng cách giữa các tiểu phân lớn hơn nhiều so với kích thước của các tiểu phân, nên ta có thể nén chất khí một cách dễ dàng.
\item Các tiểu phân va chạm vào thành bình tạo nên áp suất của chất khí. Các đơn vị đo áp suất của chất khí: $\mathrm{atm}, \mathrm{mmHg}, \mathrm{torr}, \mathrm{Bar}, ...$
\item Các thông số quan trọng của chất khí: lượng khí (số mole khí), thể tích, nhiệt độ, áp suất.
\end{itemize}
\section{Các định luật về khí đơn giản}
\subsection{Định luật Boyle}
Định luật Boyle:\\
\textit{Đối với một lượng khí nhất định ở nhiệt độ không đổi, thể tích khí tỉ lệ nghịch với áp suất khí.}
$$T = \mathrm{const} \Rightarrow pV = \mathrm{const}.$$
\subsection{Định luật Charles}
Định luật Charles:\\
\textit{Đối với một lượng khí nhất định ở áp suất không đổi, thể tích khí tỉ lệ thuận với nhiệt độ.}
$$p = \mathrm{const} \Rightarrow \frac{V}{T} = \mathrm{const}.$$
\subsection{Điều kiện tiêu chuẩn về nhiệt độ và áp suất của chất khí}
Điều kiện tiêu chuẩn về nhiệt độ và áp suất của chất khí (STP): \\
\centerline{$0 ^ \circ \mathrm{C}$ và $1 \mathrm{bar} = 10^5 \mathrm{Pa}.$}
\subsection{Định luật Avogadro}
Năm 1808, Gay-Lussac cho rằng các khí phản ứng với nhau theo các tỉ lệ thể tích nhất định.\\
Định luật Avogadro:\\
\textit{Ở nhiệt độ và áp suất nhất đinh, thể tích khí tỉ lệ thuận với số mole khí.}
$$V = c \times n, \text{ } c = \mathrm{const}.$$
\section{Định luật khí lý tưởng}
Ở áp suất không quá cao, nhiệt độ không quá thấp, các khí tuân theo định luật này được xem là \textit{khí lý tưởng.}\\
Thể tích khí tỉ lệ thuận với số mole khí và nhiệt độ Kelvin, tỉ lệ nghịch với áp suất khí.
Với cùng số mole khí $n,$ ta có:
$$pV = nRT \Rightarrow \frac{pV}{T} = nR = \mathrm{const}.$$
\section{Hỗn hợp khí $-$ định luật Dalton}
Khi một hỗn hợp khí có tổng số mol khí là:
$$n_{\text{tổng}} = n_1 + n_2 + ... n_n.$$
Áp suất tổng là:
$$p_{\text{tổng}} = p_1 + p_2 + ... p_n = n_{\text{tổng}} \frac{RT}{V}.$$
Phần mole của mỗi khí là:
$$x_i = \frac{n_i}{n_{\text{tổng}}} = \frac{p_i}{p_{\text{tổng}}},$$
và $x_1 + x_2 + ... + x_n = 1.$\\
Hỗn hợp khí mà các khí thành phần không có phản ứng với nhau ở điều kiện nhất định mới áp dụng được định luật này.
\section{Thuyết động học phân tử và sự phân bố tốc độ của các phân tử khí}
\subsection{Thuyết động học phân tử}
Thuyết động học phân tử có thể giải thích được các định luật Boyle, Charles và Avogadro.\\
Nội dung của thuyết động học phân tử:
\begin{itemize}
\item Bất kỳ thể tích khí nào cũng chứa một lượng rất lớn các tiểu phân.
\item Các tiểu phân có kích thước vô cùng nhỏ và chuyển động không ngừng theo mọi hướng.
\item Trong một thể tích khí, các phân tử ở rất xa nhau, có thể xem các phân tử khí là các điểm có khối lượng nhưng không có thể tích.
\item Khi va chạm, năng lượng của các phân tử khí có thể tăng hay giảm, nhưng nhiệt độ và áp suất của toàn bộ thể tích khí không thay đổi. Tức là \textbf{tổng năng lượng không thay đổi.}
\end{itemize}
\subsection{Phân bố tốc độ của các phân tử khí $-$ phương trình Maxwell}
Sự phân bố tốc độ của các phân tử khí:
\begin{itemize}
\item Các phân tử trong một chất khí chuyển động với tốc độ khác nhau.
\item Ta có thể dự đoán phần trăm các tiểu phân chuyển động ở một tốc độ nào đó.
\item Từ đó, Maxwell đã đưa ra phương trình mô tả sự phân bố của tốc độ các phân tử khí $u$ có khối lượng mole $M$ ở nhiệt độ $T.$
$$F\left( u \right) = 4\pi {\left( {\frac{M}{{2\pi RT}}} \right)^{\frac{3}{2}}}{u^2}{e^{ - \left( {\frac{{M{u^2}}}{{2RT}}} \right)}}.$$
\end{itemize}
\section{Sự khuếch tán và sự thoát khí qua lỗ nhỏ theo thuyết động học phân tử $-$ định luật Graham}
Chất khí liên tục chuyển động theo mọi hướng trong không gian, gọi là sự khuếch tán (diffusion) ra không gian xung quanh và sự trộn lẫn các khí khi chúng tiếp xúc với nhau.\\
Sự thoát khí qua lỗ nhỏ (effusion). Các phân tử nhẹ hơn thoát ra với tốc độ cao hơn.\\
Tốc độ thoát khí $R:$
$$\frac{{{R_A}}}{{{R_B}}} = \frac{{{u_{rms.A}}}}{{{u_{rms.B}}}} = \sqrt {\frac{{\frac{{3RT}}{{{M_A}}}}}{{\frac{{3RT}}{{{M_B}}}}}}  = \sqrt {\frac{{{M_B}}}{{{M_A}}}} .$$
Định luật Graham chỉ đúng với trường hợp khí ở áp suất thấp thoát ra ở lỗ nhỏ, không dùng để định lượng sự khuếch tán vì có thể có sự khuếch tán ngược dòng.
\section{Khí thực $-$ phương trình van der Waals}
\subsection{Khí thực}
Đối với khí lý tưởng:
$$pV = nRT \Rightarrow \frac{pV}{nRT} = 1,$$
tỉ số này gọi là hệ số nén.
\begin{itemize}
\item Ở áp suất thấp, khoảng dưới $1 \text{ } \mathrm{atm},$ có thể xem các khí là khí lý tưởng.
\item Ở áp suất cao, hệ số nén luôn khác $1,$ nghĩa là khí không còn lý tưởng.
\item Ở áp suất trung bình, lực tưởng tác đủ lớn thì hệ số nén nhỏ hơn $1.$
\item Các khí chỉ biểu hiện là khí lý tưởng ở nhiệt độ khá cao, áp suất thấp.
\end{itemize}
\subsection{Phương trình van der Waals}
Phương trình van der Waals:
$$\left( {p + \frac{an^2}{V^2}} \right) \left( {V - nb} \right) = nRT.$$
$p + \frac{an^2}{V^2}$ do có tương tác hút giữa các phân tử khí nên khí thực dễ nén hơn khí lý tưởng, khí thực tương tác lên thành bình yếu hơn khí lý tưởng, áp suất khí thực nhỏ hơn, $\frac{an^2}{V^2}$ biểu diễn cho tương tác giữa các phân tử khí thực.\\
$V - nb$ thì giá trị $nb$ đại diện cho thể tích thực của khí khi hóa lỏng.\\
Giá trị $a$ và $b$ được gọi là hệ số van der Waals, được xác định bằng thực nghiệm.