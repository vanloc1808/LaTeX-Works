\section{So sánh $\mu_1$ và $\mu_2$}
\begin{mybox}
\textbf{Bài tập 5.92} Hai máy được sử dụng để làm đầy các chai nhựa với khối lượng tịnh là $16.0$ ounce. Khối lượng làm đầy có thể được giả định có phân phối chuẩn, với độ lệch chuẩn $\sigma_1 = 0.020$ và $\sigma_2 = 0.025$ ounce. Một thành viên của đội ngũ nhân viên kỹ thuật chất lượng nghi ngờ rằng cả hai máy đều có cùng khối lượng trung bình, dù khối lượng này có là $16.0$ ounce hay không. Một mẫu ngẫu nhiên gồm $10$ chai được lấy từ đầu ra của mỗi máy:\\
\centering{Máy 1: $16.03 \text{ } 16.01 \text{ } 16.04 \text{ } 15.96 \text{ } 16.05 \text{ } 15.98 \text{ } 16.05 \text{ } 16.02 \text{ } 16.02 \text{ } 15.99$}\\
\centering{Máy 2: $16.02 \text{ } 16.03 \text{ } 15.97 \text{ } 16.04 \text{ } 15.96 \text{ } 16.02 \text{ } 16.01 \text{ } 16.01 \text{ } 15.99 \text{ } 16.00$}\\
Suy nghĩ của đội ngũ kỹ sư có đúng không? Sử dụng $\alpha = 0.05.$ Tính $p-\text{giá trị}.$
\end{mybox}
\textbf{Lời giải.}\\
Mẫu 1: $n_1 = 10, \text{ } \overline{x_1} = 16.015, \sigma_1 = 0.020.$\\
Mẫu 2: $n_2 = 10, \text{ } \overline{x_2} = 16.005, \sigma_2 = 0.025.$\\
Giả thuyết: $\begin{cases}
H_0: \mu_1 = \mu_2\\
H_1: \mu_1 \ne \mu_2
\end{cases}$\\
$$z = \frac{{\overline {{x_1}}  - \overline {{x_2}} }}{{\sqrt {\frac{{\sigma _1^2}}{{{n_1}}} + \frac{{\sigma _2^2}}{{{n_2}}}} }} \approx 0.9877$$
Tra bảng Gauss ta được: $z_{1 - \frac{\alpha}{2}} = 1.96.$\\
Do $\left| z \right| \leqslant z_{1 - \frac{\alpha}{2}}$ nên ta chưa đủ cơ sở để bác bỏ $H_0$ với mức ý nghĩa $\alpha = 0.05.$\\
$p-\text{giá trị} = 2 \left( {1 - Phi \left( {\left| z \right|} \right)} \right) = 0.3222.$\\
Vậy với độ tin cậy $95\%$ thì suy nghĩ của đội kỹ sư rằng cả hai máy đều có cùng khối lượng trung bình là đúng.

\begin{mybox}
\textbf{Bài tập 5.93} Hai loại nhựa phù hợp để sử dụng cho một nhà sản xuất linh kiện điện tử. Sức mạnh chịu sự phá hủy của loại nhựa này là quan trọng. Được biết $\sigma_1 = \sigma_2 = 1$ psi. Từ một mẫu ngẫu nhiên có kích thước $n_1 = 10$ và $n_2 = 12,$ ta có được $\overline{x_1} = 162.5$ và $\overline{x_2} = 155.0.$ Công ty sẽ không áp dụng nhựa loại 1 trừ khi sức chịu phá vỡ trung bình của nó vượt quá nhựa loại 2 ít nhất $10$ psi.\\
a. Trên cơ sở thông tin đó, ta có nên sử dụng nhựa loại 1 không? Sử dụng $\alpha = 0.05$ đưa ra câu trả lời. Tìm $p-\text{giá trị}.$\\
b. Giả sử sự khác nhau giữa chúng đúng là $12$ psi. Tính độ mạnh của kiểm định với $\alpha = 0.05.$\\
c. Giả sử sự khác nhau giữa chúng đúng là $12$ psi. Cỡ mẫu ở câu a có đủ để có khẳng định đúng đắn?
\end{mybox}
\textbf{Lời giải.}\\
a. Giả thuyết: $\begin{cases}
H_0: \mu_1 - \mu_2 \geqslant 10\\
H_1: \mu_1 - \mu_2 < 10
\end{cases}$\\
$$z = \frac{{\overline {{x_1}}  - \overline {{x_2}} } - 10}{{\sqrt {\frac{{\sigma _1^2}}{{{n_1}}} + \frac{{\sigma _2^2}}{{{n_2}}}} }} \approx -5.8387$$
Tra bảng Gauss ta được: $z_{1 - \frac{\alpha}{2}} = 1.96.$\\
Do $\left| z \right| > z_{1 - \frac{\alpha}{2}}$ nên ta bác bỏ $H_0$ với mức ý nghĩa $\alpha = 0.05.$\\
$p-\text{giá trị} = 2 \left( {1 - Phi \left( {\left| z \right|} \right)} \right) \approx 4 \cdot 10^{-4}.$\\
Vậy với độ tin cậy $95\%$ thì ta không nên sử dụng nhựa loại 1.\\
b. $$\beta  = \Phi \left( {\left| {\frac{{{\mu _1} - {\mu _2} - \mu }}{{\sqrt {\frac{{\sigma _1^2}}{{{n_1}}} + \frac{{\sigma _2^2}}{{{n_2}}}} }}} \right|} \right) \approx 0.9998 $$
Độ mạnh của kiểm định: $1 - \beta \approx 2 \cdot 10^{-4}.$\\

\begin{mybox}
\textbf{Bài tập 5.94} Tốc độ cháy của hai loại nguyên liệu rắn sử dụng trong động cơ tên lửa được nghiên cứu. Được biết tốc độ cháy của hai loại này có xấp xỉ phân phối chuẩn với $\sigma_1 = \sigma_2 = 3 {{cm} \mathord{\left/
 {\vphantom {{cm} s}} \right.
 \kern-\nulldelimiterspace} s}.$ Hai mẫu ngẫu nhiên với cỡ mẫu $n_1 = n_2 = 20$ được xem xét có tốc độ cháy trung bình $\overline{x_1} = 18 {{cm} \mathord{\left/
 {\vphantom {{cm} s}} \right.
 \kern-\nulldelimiterspace} s}$ và $\overline{x_2} = 24 {{cm} \mathord{\left/
 {\vphantom {{cm} s}} \right.
 \kern-\nulldelimiterspace} s}.$\\
Kiểm định xem hai loại này có cùng trung bình hay không? Với $\alpha = 0.05,$ hãy tìm $p-\text{giá trị}.$
\end{mybox}
\textbf{Lời giải.} \\
Giả thuyết: $\begin{cases}
H_0: \mu_1 = \mu_2\\
H_1: \mu_1 \ne \mu_2
\end{cases}$\\
$$z = \frac{{\overline {{x_1}}  - \overline {{x_2}} }}{{\sqrt {\frac{{\sigma _1^2}}{{{n_1}}} + \frac{{\sigma _2^2}}{{{n_2}}}} }} \approx -6.3246$$
Tra bảng Gauss ta được: $z_{1 - \frac{\alpha}{2}} = 1.96.$\\
Do $\left| z \right| > z_{1 - \frac{\alpha}{2}}$ nên ta bác bỏ $H_0$ với mức ý nghĩa $\alpha = 0.05.$\\
$p-\text{giá trị} = 2 \left( {1 - Phi \left( {\left| z \right|} \right)} \right) \approx  4 \cdot 10^{-4}.$\\
Vậy với độ tin cậy $95\%,$ ta có thể kết luận rằng hai loại này không có cùng trung bình.

\begin{mybox}
\textbf{Bài tập 5.95} Hai công thức khác nhau của nhiên liệu động cơ oxy hóa đang được thử nghiệm để nghiên cứu số octane của chúng. Phương sai chỉ số octane của công thức thứ nhất $\sigma_1^2 = 1.5$ và công thức thứ hai $\sigma_2^2 = 1.2.$ Hai mẫu ngẫu nhiên có cỡ mẫu $n_1 = 15$ và $n_2 = 20$ được nghiên cứu có chỉ số octane trung bình lần lượt là $\overline{x_1} = 89.6$ và $\overline{x_2} = 92.5.$ Với giả sử có phân phối chuẩn.\\
Nếu công thức 2 tạo ra một số octane cao hơn so với công thức 1, nhà sản xuất muốn phát hiện nó. Xây dựng và kiểm định giả thuyết thích hợp sử dụng $\alpha = 0.05$ và tính $p-\text{giá trị}.$
\end{mybox}
\textbf{Lời giải.} \\
Giả thuyết: $\begin{cases}
H_0: \mu_1 \geqslant \mu_2\\
H_1: \mu_1 < \mu_2
\end{cases}$\\
$$z = \frac{{\overline {{x_1}}  - \overline {{x_2}} }}{{\sqrt {\frac{{\sigma _1^2}}{{{n_1}}} + \frac{{\sigma _2^2}}{{{n_2}}}} }} \approx -7.25$$
Tra bảng Gauss ta được: $z_{1 - \frac{\alpha}{2}} = 1.96.$\\
Do $\left| z \right| > z_{1 - \frac{\alpha}{2}}$ nên ta bác bỏ $H_0$ với mức ý nghĩa $\alpha = 0.05.$\\
$p-\text{giá trị} = 2 \left( {1 - Phi \left( {\left| z \right|} \right)} \right) \approx  4 \cdot 10^{-4}.$\\
Vậy với độ tin cậy $95\%,$ nhà sản xuất có thể phát hiện công thức 2.

\begin{mybox}
\textbf{Bài tập 5.99} Đường kính của các thanh thép được sản xuất trên hai máy đúc khác nhau đang được nghiên cứu. Hai mẫu ngẫu nhiên có cỡ mẫu $n_1 = 15, n_2 = 17$ được chọn có trung bình và phương sai mẫu $\overline{x_1} = 8.73, s_1^2 = 0.35$ và $\overline{x_2} = 8.68, s_2^2 = 0.40.$ Giả sử rằng $\sigma_1^2 = \sigma_2^2$ và quan trắc lấy có phân phối chuẩn. Có bằng chứng để khẳng định rằng hai máy sản xuất thanh thép có đường kính trung bình khác nhau? Sử dụng $\alpha = 0.05$ khi đưa ra kết luận này. Tính $p-\text{giá trị}.$
\end{mybox}
\textbf{Lời giải.}\\
Ta đã biết $\sigma_1^2 = \sigma_2^2 $ và cỡ mẫu nhỏ.\\
Giả thuyết: $\begin{cases}
H_0: \mu_1 = \mu_2\\
H_1: \mu_1 \ne \mu_2
\end{cases}$\\
Đặt $$s_p^2 = \frac{{\left( {{n_1} - 1} \right)s_1^2 + \left( {{n_2} - 1} \right)s_2^2}}{{{n_1} + {n_2} - 2}} = \frac{{113}}{{300}}.$$
$$t = \frac{{\overline {{x_1}}  - \overline {{x_2}} }}{{\sqrt {s_p^2\left( {\frac{1}{{{n_1}}} + \frac{1}{{{n_2}}}} \right)} }} \approx 0.2300$$
Bậc tự do $n = n_1 + n_2 - 2 = 30.$\\
Tra bảng Student ta được: $t_{n - 1}^{1 - \frac{\alpha}{2}} = 2.0452.$\\
Do $\left| t \right| \leqslant t_{n}^{1 - \frac{\alpha}{2}}$ nên ta chưa đủ cơ sở để bác bỏ $H_0$ với mức ý nghĩa $\alpha = 0.05.$\\
$$p-\text{giá trị} = 2\mathbb{P}\left( {t\left( n \right) > \left| t \right|} \right) = 2\left( {1 - \mathbb{P}\left( {t\left( n \right) \leqslant \left| t \right|} \right)} \right) \approx 0.818$$
Vậy với độ tin cậy $95\%,$ ta có thể kết luận rừng hai nhà máy sản xuất thanh thép có đường kính trung bình giống nhau.

\begin{mybox}
\textbf{Bài tập 5.100} Hai chất xúc tác có thể được sử dụng trong một phản ứng hóa học hàng loạt. Mười hai lô được sử dụng chất xúc tác 1, dẫn đến năng suất trung bình là $86$ và độ lệch chuẩn mẫu là $3.$ Mười lăm lô được sử dụng chất xúc tác 2, và kết quả là năng suất trung bình $89$ với độ lệch chuẩn là $2.$ Giả sử năng suất các phép đo xấp xỉ thường được phân phối với cùng độ lệch chuẩn. Có bằng chứng để khẳng định rằng chất xúc tác 2 tạo ra năng suất trung bình cao hơn chất xúc tác 1 không? Sử dụng $\alpha = 0.01.$
\end{mybox}
\textbf{Lời giải.}\\
Ta đã biết $\sigma_1^2 = \sigma_2^2 $ và cỡ mẫu nhỏ.\\
Giả thuyết: $\begin{cases}
H_0: \mu_1 \geqslant \mu_2\\
H_1: \mu_1 < \mu_2
\end{cases}$\\
Đặt $$s_p^2 = \frac{{\left( {{n_1} - 1} \right)s_1^2 + \left( {{n_2} - 1} \right)s_2^2}}{{{n_1} + {n_2} - 2}} = 6.2.$$
$$t = \frac{{\overline {{x_1}}  - \overline {{x_2}} }}{{\sqrt {s_p^2\left( {\frac{1}{{{n_1}}} + \frac{1}{{{n_2}}}} \right)} }} \approx -3.1109$$
Bậc tự do $n = n_1 + n_2 - 2 = 25.$\\
Tra bảng Student ta được: $t_{n}^{1 - \alpha} = 2.4851.$\\
Do $ t  < -t_{n - 1}^{1 - \alpha}$ nên ta bác bỏ $H_0$ với mức ý nghĩa $\alpha = 0.01.$\\
Vậy với độ tin cậy $99\%,$ ta có thể khẳng định chất xúc tác 2 tạo ra năng suất trung bình cao hơn chất xúc tác 1.

\begin{mybox}
\textbf{Bài tập 5.102} Các điểm nóng chảy của hai hợp kim được sử dụng trong công thức hàn được điều tra bằng cách làm tan chảy $21$ mẫu của mỗi vật lên. Trung bình mẫu và độ lệch chuẩn mẫu của hợp kim thứ nhất là $\overline{x_1} = 420^\circ F, s_1 = 4^\circ F$ và của hợp kim thứ hai là $\overline{x_2} = 426^\circ F, s_2 = 3^\circ F.$\\
Dữ liệu mẫu có hỗ trợ cho rằng cả hai hợp kim có cùng điểm nóng chảy không? Sử dụng $\alpha = 0.05$ và giả định rằng cả hai tổng thể thường có phân phối chuẩn và có cùng độ lệch chuẩn. Tìm $p-\text{giá trị}$ cho kiểm định.
\end{mybox}
\textbf{Lời giải.}\\
Ta đã biết $\sigma_1^2 = \sigma_2^2 $ và cỡ mẫu nhỏ.\\
Giả thuyết: $\begin{cases}
H_0: \mu_1 = \mu_2\\
H_1: \mu_1 \ne \mu_2
\end{cases}$\\
Đặt $$s_p^2 = \frac{{\left( {{n_1} - 1} \right)s_1^2 + \left( {{n_2} - 1} \right)s_2^2}}{{{n_1} + {n_2} - 2}} = 12.5.$$
$$t = \frac{{\overline {{x_1}}  - \overline {{x_2}} }}{{\sqrt {s_p^2\left( {\frac{1}{{{n_1}}} + \frac{1}{{{n_2}}}} \right)} }} \approx -5.5$$
Bậc tự do $n = n_1 + n_2 - 2 = 40.$\\
Tra bảng Student ta được: $t_{n}^{1 - \frac{\alpha}{2}} = 2.0211.$\\
Do $\left| t \right| > t_{n}^{1 - \frac{\alpha}{2}}$ nên ta bác bỏ $H_0$ với mức ý nghĩa $\alpha = 0.05.$\\
$$p-\text{giá trị} = 2\mathbb{P}\left( {t\left( n \right) > \left| t \right|} \right) = 2\left( {1 - \mathbb{P}\left( {t\left( n \right) \leqslant \left| t \right|} \right)} \right) \approx 2 \cdot 10^{-4}.$$
Vậy với độ tin cậy $95\%$ thì ta có thể kết luận rằng cả hai hợp kim có điểm nóng chảy khác nhau.

\begin{mybox}
\textbf{Bài tập 5.114} Một nghiên cứu được thực hiện để xem việc tăng nồng độ cơ chất có tác động đáng kể đến tốc độ của một phản ứng hóa học hay không. Với một nồng độ cơ chất $1.5 \mathrm{{{mol} \mathord{\left/
 {\vphantom {{mol} l}} \right.
 \kern-\nulldelimiterspace} l}},$ phản ứng được chạy $15$ lần, với tốc độ trùng bình $7.5$ micromoles mỗi $30$ phút và độ lệch chuẩn $1.5.$ Với nồng độ cơ chất  $2.0 \mathrm{{{mol} \mathord{\left/
 {\vphantom {{mol} l}} \right.
 \kern-\nulldelimiterspace} l}},$ $12$ lần chạy được thực hiện, thu được tốc độ trung bình $8.8$ micromoles mỗi $30$ phút và độ lệch chuẩn mẫu $1.2.$ Có lý do để tin rằng việc tăng nồng độ cơ chất làm tăng tốc độ trung bình của phản ứng $0.5$ micromole mỗi $30$ phút hay không? Sử dụng mức ý nghĩa $0.01$ và giả sử rằng các tổng thể là xấp xỉ chuẩn với các phương sai bằng nhau.
\end{mybox}
\textbf{Lời giải.}\\
Ta đã biết $\sigma_1^2 = \sigma_2^2 $ và cỡ mẫu nhỏ.\\
Giả thuyết: $\begin{cases}
H_0: \mu_2 - \mu_1 = 0.5\\
H_1: \mu_2 - \mu_1 \ne 0.5
\end{cases}$\\
Đặt $$s_p^2 = \frac{{\left( {{n_1} - 1} \right)s_1^2 + \left( {{n_2} - 1} \right)s_2^2}}{{{n_1} + {n_2} - 2}} = 1.8936.$$
$$t = \frac{{\overline {{x_2}}  - \overline {{x_1}} } - 0.5}{{\sqrt {s_p^2\left( {\frac{1}{{{n_1}}} + \frac{1}{{{n_2}}}} \right)} }} \approx 1.5011$$
Bậc tự do $n = n_1 + n_2 - 2 = 25.$\\
Tra bảng Student ta được: $t_{n}^{1 - \frac{\alpha}{2}} = 2.7874.$\\
Do $\left| t \right| \leqslant t_{n}^{1 - \frac{\alpha}{2}}$ nên ta chưa có đủ cơ sở bác bỏ $H_0$ với mức ý nghĩa $\alpha = 0.01.$\\
Vậy với độ tin cậy $99\%,$ ta có thể kết luận rằng khi tăng nồng độ cơ chất thì tốc độ trung bình của phản ứng tăng lên $0.5$ micromoles mỗi $30$ phút.

\begin{mybox}
\textbf{Bài tập 5.115} Một nghiên cứu được thực hiện để xác định xem các chủ đề môn học trong môn vật lý có dđược hiểu tốt hơn không khi môn học có phần thực hành. Các sinh viên được lựa chọn ngẫu nhiên để tham gia vào khóa học $3$ giờ học không có thực hành hoặc một khóa học $4$ giờ học có thực hành. Trong khóa học có thực hành $11$ sinh viên có điểm trung bình $85$ với độ lệch chuẩn $4.7,$ và trong khóa học không có thực hành, $17$ sinh viên có điểm trung bình $79$ với độ lệch chuẩn $6.1.$ Bạn có nói được rằng khóa học có thực hành làm tăng điểm trung bình lên $8$ điểm hay không? Sử dụng $p-\text{giá trị}$ trong kết luận của bạn và giả sử các tổng thể là xấp xỉ chuẩn với các phương sai bằng nhau. 
\end{mybox}
\textbf{Lời giải.}\\
Ta đã biết $\sigma_1^2 = \sigma_2^2 $ và cỡ mẫu nhỏ.\\
Giả thuyết: $\begin{cases}
H_0: \mu_1 - \mu_2 = 8\\
H_1: \mu_1 - \mu_2 \ne 8
\end{cases}$\\
Đặt $$s_p^2 = \frac{{\left( {{n_1} - 1} \right)s_1^2 + \left( {{n_2} - 1} \right)s_2^2}}{{{n_1} + {n_2} - 2}} = 31.3946.$$
$$t = \frac{{\overline {{x_1}}  - \overline {{x_2}} } - 8}{{\sqrt {s_p^2\left( {\frac{1}{{{n_1}}} + \frac{1}{{{n_2}}}} \right)} }} \approx -0.9225$$
Bậc tự do $n = n_1 + n_2 - 2 = 26.$
$$p-\text{giá trị} = 2\mathbb{P}\left( {t\left( n \right) > \left| t \right|} \right) = 2\left( {1 - \mathbb{P}\left( {t\left( n \right) \leqslant \left| t \right|} \right)} \right) \approx 1.6261.$$
Do $\alpha < p-\text{giá trị}$ nên ta chưa đủ cơ sở để bác bỏ $H_0$ với mức ý nghĩa $\alpha = 0.05.$\\
Vậy với độ tin cậy $95\%,$ ta có thể kết luận rằng khóa học có thực hành làm tăng điểm trung bình lên $8$ điểm.

\begin{mybox}
\textbf{Bài tập 5.116} Để tìm ra liệu một loại huyết thanh huyết thanh mới có kìm hãm được bệnh bạch cầu hay không, $9$ con chuột, tất cả các con đều trong giai đoạn tiến triển của bệnh, được chọn. Năm con chuột nhận trị liệu và $4$ con không. Thời gian sống, theo năm, từ thời thí nghiệm bắt đầu là như sau:
\begin{table}[H]
\begin{tabular}{|c|c|c|c|c|c|}
\hline 
Trị liệu & $2.1$ & $5.3$ & $1.4$ & $4.6$ & $0.9$ \\ 
\hline 
Không trị liệu & $1.9$ & $0.5$ & $2.8$ & $3.1$ & • \\ 
\hline 
\end{tabular} 
\end{table}
Tại mức ý nghĩa $0.05,$ huyết thanh có thể được nói là có hiệu quả hay không? Giả sử hai tổng thể có phân phối chuẩn với các phương sai bằng nhau.
\end{mybox}
\textbf{Lời giải.}\\
Ta đã biết $\sigma_1^2 = \sigma_2^2 $ và cỡ mẫu nhỏ.\\
Giả thuyết: $\begin{cases}
H_0: \mu_1 \leqslant \mu_2\\
H_1: \mu_1 > \mu_2
\end{cases}$\\
Mẫu 1: \\
$n = 5.$\\
Trung bình mẫu:
$\overline x  = \frac{1}{n}\sum\limits_{i =1 1}^n {{x_i}}  \approx 2.86.$\\
Phương sai mẫu: $s_x^2 = \frac{1}{{n - 1}}\sum\limits_{i = 1}^n {{{\left( {{x_i} - \overline x } \right)}^2}}  \approx 3.883.$\\
Độ lệch chuẩn mẫu: ${s_x} \approx 1.9705.$\\
Mẫu 2: \\
$n = 4.$\\
Trung bình mẫu:
$\overline x  = \frac{1}{n}\sum\limits_{i =1 1}^n {{x_i}}  \approx 2.075.$\\
Phương sai mẫu: $s_x^2 = \frac{1}{{n - 1}}\sum\limits_{i = 1}^n {{{\left( {{x_i} - \overline x } \right)}^2}}  \approx 1.3625.$\\
Độ lệch chuẩn mẫu: ${s_x} \approx 1.1673.$\\
Đặt $$s_p^2 = \frac{{\left( {{n_1} - 1} \right)s_1^2 + \left( {{n_2} - 1} \right)s_2^2}}{{{n_1} + {n_2} - 2}} \approx 2.8028.$$
$$t = \frac{{\overline {{x_1}}  - \overline {{x_2}} }}{{\sqrt {s_p^2\left( {\frac{1}{{{n_1}}} + \frac{1}{{{n_2}}}} \right)} }} \approx 0.6990$$
Bậc tự do $n = n_1 + n_2 - 2 = 7.$\\
Tra bảng Student ta được: $t_{n}^{1 - \alpha} = 1.8946.$\\
Do $t < t_{n}^{1 - \alpha}$ nên ta chưa có đủ cơ sở bác bỏ $H_0$ với mức ý nghĩa $\alpha = 0.05.$\\
Vậy với độ tin cậy $95\%,$ ta có thể kết luận thuốc chưa có hiệu quả.