\section{Hoàn cảnh trước khi miền Bắc bước vào cách mạng xã hội chủ nghĩa}
"17 giờ 30 phút ngày 07/5/1954, tướng De Castrie và toàn bộ tham mưu tập đoàn cứ điểm bị bắt." \supercite{lsvnt4} Lá cờ chiến thắng tung bay trên nóc hầm De Castrie, đánh dấu sự kết thúc thắng lợi của chiến dịch Điện Biên Phủ "lừng lẫy năm châu, chấn động địa cầu". Chiến thắng này được ví như "một Bạch Đằng, một Chi Lăng hay một Đống Đa của thế kỷ XX". \supercite{danvan} Đồng thời, chiến thắng Điện Biên Phủ cũng đánh dấu sự kết thúc của cuộc kháng chiến chống thực dân Pháp xâm lược của nhân dân ta.\\
Ngày 21/7/1954, Hiệp định Geneve về chấm dứt chiến tranh, lập lại hòa bình ở Đông Dương được ký kết. "Đất nước ta tạm thời bị chia làm hai miền, sẽ tổ chức Tổng tuyển cử thống nhất Tổ quốc." \supercite{lsvnt4} "Miền Bắc được hoàn toàn giải phóng, phát triển theo con đường xã hội chủ nghĩa, miền Nam do chính quyền đối phương quản lý, trở thành thuộc địa kiểu mới của đế quốc Mỹ." \supercite{giaotrinh} Ở miền Nam, Đảng ta chủ trương tiếp tục thực hiện "cách mạng dân tộc dân chủ nhân dân, giải phóng miền Nam thống nhất Tổ quốc". \supercite{hvct} Vì vậy, cách mạng xã hội chủ nghĩa ở miền Bắc đóng vai trò "quyết định nhất đối với toàn bộ sự phát triển của cách mạng Việt Nam" \supercite{hvct}.

\subsection{Hoàn cảnh quốc tế khi miền Bắc tiến lên chủ nghĩa xã hội}
\subsubsection{Thuận lợi}
Hệ thống xã hội chủ nghĩa trên thế giới tiếp tục lớn mạnh về mọi mặt, "về kinh tế, quân sự, khoa học kỹ thuật, nhất là sự lớn mạnh của Liên Xô" \supercite{giaotrinh}.\\
Việt Nam nhận được sự hỗ trợ từ các nước trong khối xã hội chủ nghĩa như Liên Xô, Trung Quốc: "giúp đào tạo cán bộ, giúp về vật tư, thiết bị máy móc để xây dựng các công trình, giúp đỡ về vốn ban đầu, đưa chuyên gia của các lĩnh vực sang giúp Việt Nam, giúp về cơ chế, cách thức quản lý..." \supercite{hvct}.\\
Phong trào đấu tranh giải phóng dân tộc phát triển mạnh mẽ ở các nước thuộc địa. Phong trào đấu tranh vì "hòa bình, dân chủ" \supercite{giaotrinh} ở các nước tư bản tiếp tục lên cao.
\subsubsection{Khó khăn}
Đế quốc Mỹ xuất hiện với âm mưu làm "bá chủ thế giới" \supercite{giaotrinh}, các đời tổng thống nối tiếp nhau thực hiện các chiến lược phản cách mạng trên phạm vi toàn cầu.\\
Thế giới bước vào chiến tranh lạnh, chia thành hai phe: phe xã hội chủ nghĩa do Liên Xô đứng đầu và phe tư bản chủ nghĩa do Mỹ đứng đầu. Những cuộc chạy đua vũ trang liên tiếp diễn ra.\\
Trong nội bộ hệ thống xã hội chủ nghĩa đã xảy ra những sự "bất đồng, chia rẽ" \supercite{giaotrinh}, nhất là giữa hai nước lớn nhất: Liên Xô và Trung Quốc.

\subsection{Hoàn cảnh trong nước khi miền Bắc tiến lên chủ nghĩa xã hội}
\subsubsection{Thuận lợi}
Sau chiến thắng Điện Biên Phủ, miền Bắc đã hoàn toàn giải phóng, trở thành "căn cứ địa hậu phương cho cả nước" \supercite{giaotrinh}. Trải qua 9 năm kháng chiến gian khổ chống thực dân Pháp, cách mạng nước ta đã có "thế và lực lớn mạnh hơn trước" \supercite{giaotrinh}.\\
Nhân dân cả nước trên dưới một lòng, có cùng "ý chí độc lập thống nhất" \supercite{giaotrinh} đất nước.
\subsubsection{Khó khăn}
Đất nước bị chia làm hai miền, khác nhau về chế độ chính trị, "miền Nam trở thành thuộc địa kiểu mới của Mỹ" \supercite{gtduongloi}, do đế quốc Mỹ kiểm soát.\\
Sau khi bị chiến tranh tàn phá, "kinh tế miền Bắc nghèo nàn, lạc hậu" \supercite{giaotrinh}. "Tình hình xã hội miền Bắc phức tạp (các thế lực thù địch chống phá, dụ dỗ, cưỡng ép đồng bào Công giáo di cư vào miền Nam...)" \supercite{hvct}\\
Đế quốc Mỹ trở thành "kẻ thù trực tiếp" \supercite{giaotrinh} của dân tộc, của nhân dân Việt Nam.\\
Tình hình phức tạp nêu trên đã đặt ra một yêu cầu bức thiết cho Đảng ta là phải "vạch ra đường lối chiến lược đúng đắn để đưa cách mạng Việt Nam tiến lên phù hợp với tình hình mới của đất nước và phù hợp với xu thế phát triển chung của thời đại" \supercite{giaotrinh}. Trải qua nhiều cuộc họp, nhiều hội nghị của Ban Chấp hành Trung ương và Bộ Chính trị, Đảng đã từng bước hình thành chủ trương chiến lược cách mạng Việt Nam trong giai đoạn mới.
