\section{Đảng quyết định chuyển hướng phát triển kinh tế, tiếp tục xây dựng chủ nghĩa xã hội trong điều kiện cả nước có chiến tranh (1965 $-$ 1975)}
\subsection{Đường lối kháng chiến chống Mỹ, cứu nước của Đảng}
Trước nguy cơ thất bại trong chiến lược "chiến tranh đặc biệt" ở miền Nam Việt Nam, ngày 05/8/1964, đế quốc Mỹ leo thang, "tiến hành chiến tranh phá hoại miền Bắc xã hội chủ nghĩa" \supercite{hvct}. Cuộc chiến tranh phá hoại do đế quốc Mỹ gây ra đối với miền Bắc nước ta đã diễn ra rất ác liệt từ ngày 07/2/1965 nhằm đánh phá hậu phương miền Bắc, ngăn chặn sự chi viện của miền Bắc đối với miền Nam.\\
Chiến tranh lan rộng ra cả nước đã đặt vận mệnh dân tộc ta trước những thách thức nghiêm trọng. Trước tình hình đó, Hội nghị lần thứ 11 (tháng 3/1965) và Hội nghị lần thứ 12 (tháng 12/1965) của Ban Chấp hành Trung ương khóa III đã phát động cuộc kháng chiến chống Mỹ, cứu nước trên phạm vi toàn quốc, hạ quyết tâm chiến lược: "Động viên lực lượng của cả nước, kiên quyết đánh bại cuộc chiến tranh xâm lược của đế quốc Mỹ trong bất cứ tình huống nào, để bảo vệ miền Bắc, giải phóng miền Nam, hoàn thành cách mạng dân tộc dân chủ nhân dân trong cả nước, tiến tới thực hiện hòa bình thống nhất nước nhà" \supercite{vkd26}.\\
Nội dung đường lối kháng chiến chống Mỹ, cứu nước ở giai đoạn này là sự kế thừa và phát triển sáng tạo đường lối chiến lược chung của cách mạng Việt Nam, được Đảng ta đề ra tại Đại hội III (1960).\\
Đối với miền Bắc, Đảng ta có tư tưởng chỉ đạo là: "Chuyển hướng xây dựng kinh tế, bảo đảm tiếp tục xây dựng miền Bắc vững mạnh về kinh tế và quốc phòng trong điều kiện có chiến tranh, tiến hành cuộc chiến tranh nhân dân chống chiến tranh phá hoại của đế quốc Mỹ để bảo vệ vững chắc miền Bắc xã hội chủ nghĩa, động viên sức người sức của ở mức cao nhất để chi viện cho cuộc chiến tranh giải phóng miền Nam, đồng thời tích cực chuẩn bị đề phòng để đánh bại địch trong trường hợp chúng liều lĩnh mở rộng chiến tranh cục bộ ra cả nước" \supercite{giaotrinh}.

\subsection{Xây dựng hậu phương, chống chiến tranh phá hoại của đế quốc Mỹ ở miền Bắc (1965 $-$ 1968)}
Từ ngày 05/8/1964, Mỹ dựng lên "sự kiện Vịnh Bắc Bộ" nhằm lấy cớ dùng không quân và hải quân đánh phá miền Bắc Việt Nam \nocite{skvbb}. Cuộc chiến tranh phá hoại của đế quốc Mỹ với ý đồ của Tổng thống Mỹ Lyndon B. Johnson "đưa miền Bắc trở về thời kỳ đồ đá" \supercite{giaotrinh}; phá hoại công cuộc xây dựng chủ nghĩa xã hội ở miền Bắc, đồng thời ngăn sự chi viện cho miền Nam, đè bẹp ý chí quyết tâm chống Mỹ, cứu nước của cả dân tộc ta, buộc chúng ta phải kết thúc chiến tranh theo điều kiện của Mỹ đặt ra. Đế quốc Mỹ đã huy động các trang thiết bị và vũ khí tối tân nhằm mục đích phá hoại miền Bắc nước ta, gây nên những tội ác tày trời với nhân dân ta.\\
Trước tình hình đó, theo tinh thần của Nghị quyết Hội nghị Trung ương 11 và 12, Ban Chấp hành Trung ương Đảng đã kịp thời xác định chủ trương, nhiệm vụ cụ thể của miền Bắc cho phù hợp với yêu cầu, nhiệm vụ mới trong hoàn cảnh cả nước có chiến tranh:\\
\textit{Một là}, "kịp thời chuyển hướng xây dựng kinh tế cho phù hợp với tình hình có chiến tranh phá hoại" \supercite{giaotrinh};\\
\textit{Hai là}, "tăng cường lực lượng quốc phòng cho kịp với sự phát triển tình hình cả nước có chiến tranh" \supercite{giaotrinh};\\
\textit{Ba là}, "ra sức chi viện cho miền Nam với mức cao nhất để đánh bại địch ở chiến trường chính miền Nam" \supercite{giaotrinh};\\
\textit{Bốn là}, "phải kịp thời chuyển hướng tư tưởng và tổ chức cho phù hợp với tình hình mới" \supercite{giaotrinh}.\\
Chủ trương chuyển hướng xây dựng và phát triển kinh tế và những nhiệm vụ đặt ra cho miền Bắc phản ánh quyết tâm của Đảng và nhân dân ta trong việc kiên trình con đường xã hội chủ nghĩa, tiếp tục tăng cường sức mạnh của miền Bắc, làm chỗ dựa vững chắc cho sự nghiệp chống Mỹ xâm lược, giải phóng miền Nam, thống nhất đất nước. Quyết tâm đó đã được thể hiện trong \textit{Lời kêu gọi đồng bào và chiến sĩ cả nước} của Chủ tịch Hồ Chí Minh ngày 17/7/1966: "Chiến tranh có thể kéo dài 5 năm, 10 năm, 20 năm hoặc lâu hơn nữa. Hà Nội, Hải Phòng và một số thành phố, xí nghiệp có thể bị tàn phá, song nhân dân Việt Nam quyết không sợ! \textit{Không có gì quý hơn độc lập, tự do}" \supercite{HCMtt15}.\\
Thực hiện những nghị quyết của Đảng và theo Lời kêu gọi của Chủ tịch Hồ Chí Minh, quân và dân miền Bắc đã dấy lên cao trào chống Mỹ, cứu nước, vừa sản xuất, vừa chiến đấu, với niềm tin tưởng và quyết tâm cao độ.\\
Do bị thất bại nặng nề ở cả hai miền Nam $-$ Bắc, tháng 3/1968, đế quốc Mỹ tuyên bố hạn chế ném bom miền Bắc, ngày 01/11/1968, Mỹ buộc phải chấm dứt không điều kiện đánh phá miền Bắc bằng không quân và hải quân.\\
Sau bốn năm thực hiện chuyển hướng xây dựng và phát triển kinh tế, hậu phương lớn miền Bắc đã đạt được những thành tích đánh tự hào trên các mặt chính trị, kinh tế, văn hóa, xã hội, chi viện cho tiền tuyến lớn miền Nam.\\
Công cuộc xây dựng chủ nghĩa xã hội vẫn tiếp tục, miền Bắc ngày càng vững mạnh. Chế độ xã hội chủ nghĩa đang được xây dựng ở miền Bắc đã vượt qua được nhiều thử thách nghiêm trọng và ngày càng phát huy tính ưu việt trong chiến tranh. Chuyển hướng kinh tế, tiếp tục xây dựng chủ nghĩa xã hội trong hoàn cảnh có chiến tranh là nét đặc biệt chưa có tiền lệ. Sản xuất nông nghiệp không những không giảm sút mà còn có bước phát triển tiến bộ. Công nghiệp địa phương phát triển mạnh. Đời sống nhân dân được ổn định. Sự nghiệp văn hóa, giáo dục, y tế chẳng những không ngừng trệ mà còn phát triển mạnh mẽ trong thời chiến và đạt nhiều kết quả tốt. Công tác nghiên cứu khoa học, điều tra cơ bản, thăm dò tài nguyên được đẩy mạnh. \\
Trong chiến đấu, quân dân miền Bắc đã bắn rơi hơn 3200 máy bay, bắn cháy 140 tàu chiến của địch. Nhiệm vụ chi viện tiền tuyến được hoàn thành xuất sắc, góp phần cùng quân dân miền Nam đánh bại chiến lược "Chiến tranh cục bộ" của đế quốc Mỹ.

\subsection{Khôi phục kinh tế, bảo vệ miền Bắc, đẩy mạnh cuộc chiến đấu giải phóng miền Nam, thống nhất Tổ quốc (1969 $-$ 1975)}
Tranh thủ những điều kiện thuận lợi do đế quốc Mỹ vừa mới chấm dứt chiến tranh phá hoại miền Bắc, từ tháng 11/1968, Đảng ta đã lãnh đạo nhân dân miền Bắc tập trung thực hiện các kế hoạch ngắn hạn nhằm khắc phục hậu quả chiến tranh, tiếp tục công cuộc xây dựng miền Bắc và tăng cường chi viện sức người, sức của cho miền Nam.\\
Ngày 02/9/1969, Chủ tịch Hồ Chí Minh qua đời, một tổn thất rất lớn đối với cách mạng Việt Nam.\\
Nhân dân miền Bắc đã khẩn trương bắt tay khôi phục kinh tế, hàn gắn vết thương chiến tranh và đẩy mạnh sự nghiệp xây dựng chủ nghĩa xã hội. Chấp hành các nghị quyết của Đảng, sau ba năm phấn đấu gian khổ, từ năm 1969 đến năm 1972, tình hình khôi phục kinh tế và tiếp tục xây dựng chủ nghĩa xã hội có nhiều chuyển biến tốt đẹp trên nhiều mặt. "Trong nông nghiệp, năm 1969, diện tích các loại cây trồng đều vượt năm 1968, riêng diện tích và sản lượng lúa tăng khá nhanh, lúa xuân tăng hai lần so với năm 1968, chăn nuôi cũng phát triển mạnh. Trong công nghiệp, hầu hết các xí nghiệp bị địch đánh phá được khôi phục, sửa chữa. Hệ thống giao thông, cầu phà, bến bãi được khẩn trương khôi phục và xây dựng thêm. Trong lĩnh vực y tế, giáo dục có bước phát triển tốt so với trước, nhất là hệ thống giáo dục đại học, tăng lên 36 trường và phân hiệu với hơn 8 vạn sinh viên" \supercite{giaotrinh}.\\
Từ tháng 04/1972, để ngăn chặn cuộc tiến công chiến lược của nhân dân ta ở miền Nam, đế quốc Mỹ đã cho máy bay và tàu chiến tiến hành cuộc chiến tranh phá hoại miền Bắc lần thứ hai hết sức ác liệt, nhất là cuộc chiến đấu 12 ngày đêm trên bầu trời Hà Nội. Quân và dân miền Bắc đã kiên cường, anh dũng đánh bại hoàn toàn âm mưu phá hoại của Mỹ bằng trận "Điện Biên Phủ trên không". Ngày 15/1/1973, Chính phủ Mỹ phải tuyên bố ngừng mọi hoạt động phá hoại miền Bắc và trở lại bàn đàm phán ở Paris.\\
Ngày 27/1/1973, Hiệp định Paris chính thức được kí kết \supercite{hdpr}, miền Bắc lập lại hòa bình, Trung ương Đảng đề ra kế hoạch hai năm khôi phục và phát triển kinh tế 1974 $-$ 1975. Với khí thế chiến thắng, nhân dân miền Bắc đã lao động hăng hái, khẩn trương, thực hiện có hiệu quả kế hoạch. Đến năm 1975, hầu hết các cơ sở kinh tế đã hoạt động bình thường. Năng lực sản xuất nhiều ngành kinh tế như công nghiệp, nông nghiệp, xây dựng cơ bản, giao thông vận tải được tăng cường thêm một bước. Nhìn chung, sản xuất nông nghiệp và công nghiệp trên một số mặt quan trọng đã đạt và vượt mức năm 1965.\\
Miền Bắc đã hoàn thành nhiệm vụ hậu phương lớn đối với tiền tuyến miền Nam và hoàn thành nghĩa vụ quốc tế đối với cách mạng Lào và Campuchia. "Tính tổng thể, hậu phương miền Bắc xã hội chủ nghĩa đã bảo đảm $80\%$ bộ đội chủ lực, $70\%$ vũ khí và lương thực, $65\%$ thực phẩm cho chiến trường miền Nam, nhất là ở giai đoạn cuối" \supercite{giaotrinh}.