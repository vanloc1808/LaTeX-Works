\section{Lời mở đầu}
Năm 1986, tại Đại hội đại biểu toàn quốc lần thứ VI, Đảng Cộng sản Việt Nam đã khởi xướng công cuộc Đổi mới, đưa đất nước ta thoát khỏi hoàn cảnh khó khăn lúc đó, mở ra một chương mới trong lịch sử dân tộc.\\
Năm 2021, trong \textit{Báo cáo của Ban Chấp hành Trung ương Đảng khóa XII về các văn kiện trình Đại hội XIII của Đảng}, Đảng ta nhận định: "Qua 35 năm tiến hành công cuộc đổi mới, 30 năm thực hiện Cương lĩnh xây dựng đất nước trong thời kỳ quá độ lên chủ nghĩa xã hội, lý luận về đường lối đổi mới, về chủ nghĩa xã hội và con đường đi lên chủ nghĩa xã hội ở nước ta ngày càng được hoàn thiện và từng bước được hiện thực hóa. Chúng ta đã đạt được \textit{những thành tựu to lớn, có ý nghĩa lịch sử,} phát triển mạnh mẽ, toàn diện hơn so với những năm trước đổi mới. Với tất cả sự khiêm tốn, chúng ta vẫn có thể nói rằng: \textit{Đất nước ta chưa bao giờ có được cơ đồ, tiềm lực, vị thế} và \textit{uy tín quốc tế như ngày nay.}" \supercite{vk13t1}\\

Tầm quan trọng của công cuộc xây dựng chủ nghĩa xã hội và bảo vệ Tổ quốc xã hội chủ nghĩa đã thôi thúc em tìm hiểu lịch sử hình thành và lãnh đạo cách mạng của Đảng Cộng sản Việt Nam. Qua quá trình nghiên cứu, em nhận thấy \textbf{Cách mạng xã hội chủ nghĩa và những thành quả xây dựng chủ nghĩa xã hội ở miền Bắc giai đoạn 1954 $\mathbf{-}$ 1975} là một đề tài hay, thú vị, đem đến nhiều kiến thức bổ ích nên em quyết định chọn đề tài này cho bài tập.\\

Em xin gửi lời cảm ơn chân thành đến TS. Ngô Quang Định, giảng viên giảng dạy môn học Lịch sử Đảng Cộng sản Việt Nam, cũng là người hướng dẫn em thực hiện bài tập này. Từ tận đáy lòng mình, em biết ơn thầy về những bài giảng, những hướng dẫn, những lời khuyên bổ ích trong quá trình thực hiện bài tập.