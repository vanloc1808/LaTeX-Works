\section{Đảng lãnh đạo khôi phục kinh tế, cải tạo xã hội chủ nghĩa ở miền Bắc 1954 $-$ 1960}
\subsection{Đảng lãnh đạo hoàn thành những nhiệm vụ còn lại của cách mạng dân tộc dân chủ, khôi phục kinh tế (1954 $-$ 1957)}
Tháng 9/1954, Bộ Chính trị ra Nghị quyết \textit{về tình hình mới, nhiệm vụ mới và chính sách mới của Đảng}, trong đó đề ra nhiệm vụ chủ yếu trước mắt của miền Bắc là "hàn gắn vết thương chiến tranh, phục hồi kinh tế quốc dân, trước hết là phục hồi và phát triển sản xuất nông nghiệp, ổn định xã hội, ổn định đời sống nhân dân, tăng cường và mở rộng hoạt động quan hệ quốc tế..." \supercite{giaotrinh}, nhằm sớm đưa miền Bắc trở lại bình thường sau 9 năm kháng chiến.\\
Tại các Hội nghị lần thứ bảy (tháng 3/1955) và lần thứ tám (tháng 8/1955), Ban Chấp hành Trung ương Đảng khóa II đã nhận định: Muốn chống đế quốc Mỹ và tay sai, củng cố hòa bình, thực hiện thống nhất đất nước, hoàn thành độc lập và dân chủ, thì điều cốt lõi cần làm là "ra sức củng cố miền Bắc, đồng thời giữ vững và đẩy mạnh cuộc đấu tranh của nhân dân miền Nam" \supercite{giaotrinh}.\\
Ngay sau khi hòa bình được lập lại, Đảng đã lãnh đạo nhân dân miền Bắc đấu tranh đòi đối phương phải rút quân khỏi miền Bắc theo đúng lịch trình đã được Hiệp định Geneve quy định. Tuy nhiên, cuộc đấu tranh này đã vấp phải những âm mưu, thủ đoạn của Pháp và tay sai nhằm chống phá, làm rối loạn xã hội. Để chống lại những âm mưu, Đảng và Nhà nước ta đã kịp thời ban hành nhiều chính sách như "chính sách đối với tôn giáo, chính sách đối với công chức, trí thức trước đây làm việc cho địch, chính sách đối với ngụy quân" \supercite{giaotrinh}, ra sức tuyên truyền, vận động quần chúng đấu tranh chống những âm mưu, thủ đoạn của địch.\\
Trước tình hình đấu tranh quyết liệt của nhân dân ta, địch đã phải rút quân theo đúng Hiệp định. "Ngày 10/10/1954, người lính Pháp cuối cùng rút khỏi Hà Nội" \supercite{giaotrinh}, Trung ương Đảng và Chính phủ rời chiến khu Việt Bắc về tiếp quản thủ đô. "Ngày 16/5/1955, toàn bộ quân đội viễn chinh Pháp và tay sai đã phải rút khỏi miền Bắc" \supercite{giaotrinh}.\\
Nhận thức được rằng nông nghiệp là ngành cơ bản của kinh tế miền Bắc, Đảng ta đã chỉ đạo "lấy khôi phục và phát triển sản xuất nông nghiệp làm trọng tâm" \supercite{giaotrinh}. Việc khôi phục sản xuất nông nghiệp được kết hợp với cải cách ruộng đất và vận động đổi công, giúp nhau sản xuất, đồng thời chăm lo xây dựng cơ sở vật chất cho nông nghiệp. Đến năm 1957, nông nghiệp miền Bắc cơ bản đã đạt được năng suất và sản lượng của năm 1939 $-$ năm có sản lượng cao nhất dưới thời Pháp thuộc. Nhờ những thành tựu bước đầu này, nạn đói bị đẩy lùi, góp phần tạo điều kiện giải quyết những vấn đề cơ bản trong nền kinh tế quốc dân, góp phần ổn định chính trị, trật tự an ninh xã hội.\\
Cùng với nông nghiệp, việc khôi phục công nghiệp, tiểu thủ công nghiệp và giao thông vận tải cũng được hoàn thành. Hầu hết các xí nghiệp quan trọng đã được phục hồi sản xuất và tăng thêm thiết bị, một số nhà máy mới được xây dựng. Các lĩnh vực văn hóa, giáo dục, y tế được phát triển nhanh.\\
Công cuộc giảm tô, tức và cải cách ruộng đất được tiếp tục đẩy mạnh. Để đảm bảo thắng lợi của nhiệm vụ cải cách ruộng đất, Đảng đã chủ trương "dựa hẳn vào bần cố nông, đoàn kết với trung nông, đánh đổ giai cấp địa chủ, tịch thu ruộng đất của họ để chia cho dân cày nghèo" \supercite{giaotrinh}. Đến tháng 7/1956, cải cách ruộng đất đã căn bản hoàn thành ở đồng bằng, trung du và miền núi. Chế độ chiếm hữu ruộng đất phong kiến ở miền Bắc đến đây đã hoàn toàn bị xóa bỏ. Nghị quyết Hội nghị lần thứ 14 Ban Chấp hành Trung ương Đảng khóa II về tổng kết cải cách ruộng đất đã khẳng định kết quả của công cuộc này: "Trên 810.000 hecta ruộng đất của đế quốc và địa chủ, ruộng đất tôn giáo, ruộng đất công và nửa công nửa tư đã bị tịch thu, trưng thu, trưng mua và đem chia hẳn cho 2.220.000 hộ nông dân lao động và dân nghèo ở nông thôn, bao gồm trên 9.000.000 nhân khẩu" \supercite{hn14k2}.\\
Mặc dù vậy, trong quá trình cải cách ruộng đất, bên cạnh những kết quả đạt được, Đảng ta đã phạm phải "một số sai lầm nghiêm trọng, phổ biến và kéo dài trong chỉ đạo thực hiện" \supercite{giaotrinh}. Nguyên nhân chủ yếu dẫn đến sai lầm là "chủ quan, giáo điều, không xuất phát từ tình hình thực tiễn, nhất là những thay đổi quan trọng về quan hệ giai cấp, xã hội ở nông thôn miền Bắc sau ngày được hoàn toàn giải phóng" \supercite{giaotrinh}. Những sai lầm này đã gây ra một số tổn thất đối với Đảng và quan hệ giữa Đảng với nhân dân.\\
Tại Hội nghị lần thứ 10 khóa II (tháng 9/1956), Ban Chấp hành Trung ương Đảng đã nghiêm khắc kiểm điểm những sai lầm trong cải cách ruộng đất và chỉnh đốn tổ chức, công khai tự phê bình trước nhân dân, thi hành kỷ luật đối với một số Ủy viên Bộ Chính trị và Ủy viên Trung ương Đảng \supercite{hn10k2p1}. Trong năm 1956, công tác sửa sai đã được Đảng chỉ đạo một cách "thành khẩn, kiên quyết, khẩn trương, thận trọng và có kế hoạch chặt chẽ" \supercite{giaotrinh}, nên từng bước Đảng ta đã khắc phục được những sai lầm. Ngoài ra, trong năm 1956, Đảng cũng đã phê phán, uốn nắn, chấn chỉnh kịp thời vấn đề Nhân văn Giai phẩm \footnote{Nhân văn Giai phẩm: một số văn nghệ sĩ đã đăng những bài đăng không đúng quan điểm, đường lối, chính sách của Đảng, Nhà nước trên \textit{Báo văn và Giai phẩm mùa xuân}.}\\
\subsection{Đảng lãnh đạo cải tạo xã hội chủ nghĩa, phát triển văn hóa, xã hội (1958 $-$ 1960)}
Hội nghị lần thứ 13 Ban Chấp hành Trung ương khóa II họp tháng 12/1957 đã đánh giá thắng lợi về khôi phục kinh tế, đề ra nhiệm vụ "soạn thảo đường lối cách mạng trong giai đoạn mới" \supercite{giaotrinh}. Đến tháng 11/1958, Ban Chấp hành Trung ương họp Hội nghị lần thứ 14, đề ra "kế hoạch ba năm phát triển kinh tế, văn hóa và cải tạo xã hội chủ nghĩa đối với kinh tế cá thể và kinh tế tư bản tư doanh (1958 - 1960)" \supercite{giaotrinh}. Hội nghị xác định mục tiêu trước mắt là "xây dựng, củng cố miền Bắc thành cơ sở vững mạnh cho cuộc đấu tranh thống nhất nước nhà" \supercite{giaotrinh}.\\
Tháng 4/1959, Hội nghị lần thứ 16 Ban Chấp hành Trung ương thông qua Nghị quyết về vấn đề hợp tác hóa nông nghiệp, xác định hình thức và bước đi của hợp tác xã là: "hợp tác hóa đi trước cơ giới hóa, do vậy hợp tác hóa phải đi đôi với thủy lợi hóa và tổ chức lại lao động, để phát huy tính ưu việt và sức mạnh của tập thể" \supercite{giaotrinh}. Hội nghị chỉ rõ ba nguyên tắc cần được quán triệt trong suốt quá trình xây dựng hợp tác xã là: "tự nguyện, cùng có lợi và quản lý dân chủ" \supercite{hn16k2}. Về vấn đề cải tạo công thương nghiệp tư bản tư doanh, Hội nghị chủ trương "cải tạo hòa bình đối với giai cấp tư sản" \supercite{giaotrinh}. Về chính trị, "vẫn coi giai cấp tư sản là thành viên của Mặt trận Tổ quốc" \supercite{giaotrinh}. Về kinh tế, "không tịch thu tư liệu sản xuất của họ, mà dùng chính sách chuộc lại, thông qua hình thức công tư hợp doanh, sắp xếp công việc cho người tư sản trong xí nghiệp, dần dần cải tạo họ thành người lao động" \supercite{giaotrinh}.\\
Kết quả của ba năm phát triển kinh tế $-$ văn hóa và cải tạo xã hội chủ nghĩa (1958 - 1960) đã tạo nên những chuyển biến cách mạng trong nền kinh tế và xã hội ở miền Bắc nước ta. Miền Bắc được củng cố, từng bước đi lên chủ nghĩa xã hội và trở thành hậu phương ổn định, vững mạnh, đáp ứng yêu cầu của sự nghiệp cách mạng Việt Nam \supercite{giaotrinh}.