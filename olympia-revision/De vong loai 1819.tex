\documentclass[12pt,a4paper]{article}
\usepackage[utf8]{vietnam}
\usepackage{amsmath}
\usepackage{amsfonts}
\usepackage{amssymb}
\usepackage{etoolbox}
\usepackage[thref,thmmarks,standard,amsmath,hyperref]{ntheorem}
\usepackage{graphicx}
\usepackage[left=2cm,right=2cm,top=2cm,bottom=2cm]{geometry}
\author{Nguyễn Văn Lộc - 20120131}
\theorembodyfont{\normalfont}
\theoremseparator {.}
\newtheorem{ques}{Câu}
\theoremstyle{nonumberplain}
\newtheorem{ans}{Lời giải}
\begin{document}
\begin{center}
\textbf{VÒNG LOẠI\\}
\textbf{CUỘC THI ĐƯỜNG LÊN ĐỈNH OLYMPIA CẤP TRƯỜNG\\}
\textbf{TRƯỜNG THPT CHUYÊN LÊ QUÝ ĐÔN\\}
\textbf{NĂM HỌC 2018 - 2019}
\end{center}
\begin{ques}
Họa sĩ Bùi Xuân Phái nổi tiếng với dòng tranh về thành phố nào?
\end{ques}
\begin{ques}
Bốn cây đại thụ của nền mỹ thuật Việt Nam hiện đại là: \textit{"Nhất Trí, nhì Vân, tam Lân, tứ Cẩn"}. "Nhất Trí" ở đây là ai?
\end{ques}
\begin{ques}
Đại tướng Võ Nguyên Giáp sinh ra ở tỉnh nào ngày nay?
\end{ques}
\begin{ques}
Chùa Một Cột được vua Lý Thái Tông cho xây dựng vào năm nào?
\end{ques}
\begin{ques}
Ở cực Bắc của Trái Đất thì kim la bàn luôn chỉ về hướng nào?
\end{ques}
\begin{ques}
Quốc hội nước Việt Nam Dân chủ Cộng hòa quyết định đổi tên nước thành \textit{"Cộng hòa Xã hội Chủ nghĩa Việt Nam"} vào ngày tháng năm nào?
\end{ques}
\begin{ques}
Tính đến hết tháng 7 năm 2021, trong các nước sau: Thụy Điển, Thụy Sĩ, Phần Lan, Hà Lan, nước nào không phải là thành viên của Liên minh Châu Âu (EU)?
\end{ques}
\begin{ques}
Huệ biển là loài sinh vật thuộc giới nào?
\end{ques}
\begin{ques}
Máy tính điện tử đầu tiên trên thế giới có tên gọi là gì?
\end{ques}
\begin{ques}
Máy tính điện tử đầu tiên trên thế giới ra đời tại quốc gia nào?
\end{ques}
\begin{ques}
Thiết bị nào được ví như là bộ não của máy tính?
\end{ques}
\begin{ques}
Trong Windows, muốn xóa hoàn toàn một đối tượng nào đó mà không lưu lại trong Recycle Bin thì ta dùng tổ hợp phím nào?
\end{ques}
\begin{ques}
Who said: \textit{"Stay foolish. Stay hungry"}?
\end{ques}
\begin{ques}
Ai là người châu Á đầu tiên đạt giải Nobel văn học?
\end{ques}
\begin{ques}
Điền vào chỗ trống trong câu ca dao sau: \textit{"Gió đông là ... lúa chiêm, gió bấc là duyên lúa mùa.}
\end{ques}
\begin{ques}
Tác giả của \textit{"Nhật kí ở rừng"} là ai?
\end{ques}
\begin{ques}
Tác phẩm nào được Vũ Khâm Lân đánh giá là \textit{"một cuốn thiên cổ kì bút"}?
\end{ques}
\begin{ques}
Nhà văn nào được Nguyễn Minh Châu nhận xét là \textit{"một định nghĩa về người nghệ sĩ tài hoa"}?
\end{ques}
\begin{ques}
Kim loại nào là kim loại có nhiệt độ nóng chảy cao nhất, được dùng làm dây tóc bóng đèn?
\end{ques}
\begin{ques}
Khi làm kem que người ta thường làm như sau: Cắm que tre vào ô đựng nước trái cây rồi đặt cả vào khay đá có đựng nước đá hòa tan nhiều muối ăn. Tất cả cho vào làm lạnh. Nước trái cây sẽ nhanh chóng đông lại thành kem que. Người ta đã lợi dụng tính chất gì khi dùng muối làm kem que?
\end{ques}
\begin{ques}
Dmitri Ivanovich Mendeleev - cha đẻ của bảng tuần hoàn các nguyên tố hóa học sinh ra và lớn lên ở thành phố nào của Nga ngày nay?
\end{ques}
\begin{ques}
Thủy ngân dễ bay hơi và rất độc. Nếu chẳng may nhiệt kế thủy ngân bị vỡ thì có thể dùng chất nào để khử độc và thu gom thủy ngân?
\end{ques}
\begin{ques}
Ông là nhà Toán học lỗi lạc của thế kỉ XIX, được mệnh danh là \textit{"Hoàng tử của các nhà Toán học"}. Ông là ai?
\end{ques}
\begin{ques}
Bà Tư đi khoe khắp cả xóm rằng bà để dành được 1 000 000 đồng (một triệu đồng) gồm toàn những đồng xu mệnh giá 1000 đồng. Một hôm, thừa dịp bà Tư đi chợ, thằng Năm lẻn vào nhà lấy mất 750 000 (bảy mươi năm mươi ngàn đồng) từ chiếc hòm đựng của bà. Biết rằng cứ 30 giây nó đếm được 50 000 đồng. Hỏi cần ít nhất bao nhiêu thời gian để nó lấy được số tiền trên?
\end{ques}
\begin{ques}
Andrew Wiles là một Giáo sư Toán học người Mỹ. Cuối thế kỉ XX, ông được vinh danh như là người đầu tiên giải được trọn vẹn một bài toán đã tồn tại gần 4 thế kỉ. Đó là bài toán nào?
\end{ques}
\begin{ques}
Những số có dạng \(2^{2^n} + 1\) với \(n\) là số tự nhiên được gọi là số Fermat. Fermat đã từng suy đoán rằng tất cả các số có dạng như vậy đều là số nguyên tố, nhưng sau đó Euler đã chỉ ra sai lầm của ông. Hỏi với \(n\) nhỏ nhất là bao nhiêu thì một số Fermat \textbf{không} phải là số nguyên tố?
\end{ques}
\begin{ques}
Thế vận hội mùa hè năm 2020 được tổ chức tại quốc gia nào?
\end{ques}
\begin{ques}
Môn thể thao nào được coi là \textit{"môn thể thao nữ hoàng"}?
\end{ques}
\begin{ques}
Nam cầu thủ nào đã đoạt danh hiệu Cầu thủ xuất sắc nhất năm 2020 của FIFA?
\end{ques}
\begin{ques}
Tính đến hết Wimbledon năm 2021, 3 tay vợt nào đang giữ kỉ lục giành 20 Grand Slam trong sự nghiệp?
\end{ques}
\begin{ques}
Hiện tượng các tia sáng bị lệch phương (gãy) khi truyền xiên góc qua mặt phân cách giữa hai môi trường trong suốt khác nhau gọi là gì?
\end{ques}
\begin{ques}
Áp suất trong bình đựng chất lỏng lớn nhất ở vị trí nào của bình?
\end{ques}
\begin{ques}
\(1kg \times 1m \times 1{m \mathord{\left/
 {\vphantom {m {{s^2}}}} \right.
 \kern-\nulldelimiterspace} {{s^2}}}\) sẽ được đơn vị nào?
\end{ques}
\begin{ques}
Hiện tượng nào chứng tỏ ánh sáng có tính chất hạt?
\end{ques}
\begin{ques}
Quyển sách Vật lý học đầu tiên trong lịch sử loài người do ai viết?
\end{ques}
\end{document}