\documentclass[12pt,a4paper]{article}
\usepackage[utf8]{vietnam}
\usepackage{amsmath}
\usepackage{amsfonts}
\usepackage{xcolor}
\usepackage{titlesec}
\usepackage{mdframed}
\usepackage{amssymb}
\usepackage{graphicx}
\usepackage[left=2cm,right=2cm,top=2cm,bottom=2cm]{geometry}
\author{Nguyễn Văn Lộc - 20120131}
\newmdenv[linecolor=black,skipabove=\topsep,skipbelow=\topsep,
leftmargin=-5pt,rightmargin=-5pt,
innerleftmargin=5pt,innerrightmargin=5pt]{mybox}
\begin{document}
\begin{mybox}
\textbf{Họ và tên:} Nguyễn Văn Lộc\\
\textbf{MSSV:} 20120131\\
\textbf{STT:} 99b\\
\textbf{Lớp:} 20CTT1
\end{mybox}
\begin{center}
\textbf{BÀI TẬP ĐIỂM CỘNG CUỐI KÌ TUẦN 12}
\end{center}
\textbf{Bài toán 1.}
\begin{mybox}
Cho đường cong \(C\) với tham số hóa (đơn, chính qui) \(r:\left[ {a,b} \right] \to {\mathbb{R}^2}\) và hàm \(f\) liên tục trên \(C.\) Đặt \(m = \mathop {\min }\limits_C f\) và \(M = \mathop {\max }\limits_C f.\) Chứng minh 
\[m \leqslant \frac{1}{{{L_C}}}\int_C {fds}  \leqslant M,\]
trong đó \(L_C\) là độ dài đường cong \(C.\)
\end{mybox}
\textbf{Lời giải.}\\
Theo công thức tính tích phân đường loại \(1,\)
\[\int_C {fds}  = \int\limits_a^b {f\left( {\overrightarrow r \left( t \right)} \right)\left| {{{\overrightarrow r }^\prime }\left( t \right)} \right|dt} .\]
Độ dài \(L_C\) của đường cong \(C\) được tính theo công thức:
\[{L_C} = \int_C {1ds}  = \int\limits_a^b {\left| {{{\overrightarrow r }^\prime }\left( t \right)} \right|dt} .\]
Do \(m = \mathop {\min }\limits_C f\) và \(M = \mathop {\max }\limits_C f\) nên
\begin{equation}
m \leqslant f\left( {\overrightarrow r \left( t \right)} \right) \leqslant M,\forall t \in \left[ {a,b} \right].
\label{eq1}
\end{equation}
Do \(r\) là tham số hóa chính qui nên \(\left| {{{\overrightarrow r }^\prime }\left( t \right)} \right| > 0,\forall t \in \left[ {a,b} \right].\) Vì vậy, từ (\ref{eq1}) ta suy ra
\begin{equation}
m\left| {{{\overrightarrow r }^\prime }\left( t \right)} \right| \leqslant f\left( {\overrightarrow r \left( t \right)} \right)\left| {{{\overrightarrow r }^\prime }\left( t \right)} \right| \leqslant M\left| {{{\overrightarrow r }^\prime }\left( t \right)} \right|,\forall t \in \left[ {a,b} \right].
\label{eq2}
\end{equation}
Từ (\ref{eq2}), theo bất đẳng thức tích phân
\[\int\limits_a^b {m\left| {{{\overrightarrow r }^\prime }\left( t \right)} \right|dt}  \leqslant \int\limits_a^b {f\left( {\overrightarrow r \left( t \right)} \right)\left| {{{\overrightarrow r }^\prime }\left( t \right)} \right|} dt \leqslant \int\limits_a^b {M\left| {{{\overrightarrow r }^\prime }\left( t \right)} \right|dt} .\]
\[ \Rightarrow m\int\limits_a^b {\left| {{{\overrightarrow r }^\prime }\left( t \right)} \right|dt}  \leqslant \int\limits_a^b {f\left( {\overrightarrow r \left( t \right)} \right)\left| {{{\overrightarrow r }^\prime }\left( t \right)} \right|} dt \leqslant M\int\limits_a^b {\left| {{{\overrightarrow r }^\prime }\left( t \right)} \right|dt} .\]
\[ \Rightarrow m \leqslant \frac{{\int\limits_a^b {f\left( {\overrightarrow r \left( t \right)} \right)\left| {{{\overrightarrow r }^\prime }\left( t \right)} \right|} dt}}{{\int\limits_a^b {\left| {{{\overrightarrow r }^\prime }\left( t \right)} \right|dt} }} \leqslant M,\left( {\int\limits_a^b {\left| {{{\overrightarrow r }^\prime }\left( t \right)} \right|dt} = L_C  > 0} \right).\]
\[ \Rightarrow m \leqslant \frac{{\int_C {fds} }}{{{L_c}}} \leqslant M.\]
Vậy ta có điều phải chứng minh.\\
\textbf{Bài toán 2.}
\begin{mybox}
\textbf{a.} Chứng minh rằng với mỗi bộ số \(\left( {x,y} \right) \in {\mathbb{R}^2},\) tồn tại duy nhất số thực \(z\) thỏa mãn phương trình \(\frac{{5{z^3}}}{4} + \left( {2{x^2} + 1} \right)z + {x^2} + {y^2} - y - 2 = 0.\) Từ đó, hãy chứng minh tồn tại hàm \(f:{\mathbb{R}^2} \to \mathbb{R}\) sao cho
\[\frac{{5{z^3}}}{4} + \left( {2{x^2} + 1} \right)z + {x^2} + {y^2} - y - 2 = 0 \Leftrightarrow z = f\left( {x,y} \right).\]
Điều này có nghĩa rằng mặt cong \(\left( S \right):\frac{{5{z^3}}}{4} + \left( {2{x^2} + 1} \right)z + {x^2} + {y^2} - y - 2 = 0\) chính là đồ thị hàm \(f.\)\\
\textbf{b.} Xét hàm \(G:{\mathbb{R}^3} \to \mathbb{R}\) xác định bởi \(,G\left( {x,y,z} \right) = \frac{{5{z^3}}}{4} + \left( {2{x^2} + 1} \right)z + {x^2} + {y^2} - y - 2.\) Chứng minh rằng
\[G\left( {x,y,f\left( {x,y} \right)} \right) = 0,\forall \left( {x,y} \right) \in {\mathbb{R}^2}.\]
\textbf{c.} Tìm \(f\left( {0,2} \right).\)\\
\textbf{d.} Viết phương trình mặt tiếp xúc với \(\left( S \right)\) tại \(\left( {0,2,f\left( {0,2} \right)} \right).\)\\
\textbf{e.} Tìm \({f_x}\left( {0,2} \right)\) và \({f_y}\left( {0,2} \right).\)\\
\textbf{f.} Tìm xấp xỉ cho giá trị \(f\left( {0.03,1.99} \right).\)\\
\textbf{g.} Xét hàm \(g\left( {x,y} \right) = f\left( {{x^2} - y,{y^2} + x} \right).\) Tìm \({g_x}\left( {1,1} \right)\) và \({g_y}\left( {1,1} \right).\)\\
\textbf{h.} Tìm cực trị hàm \(f.\) 
\end{mybox}
\textbf{Lời giải.}\\
\textbf{h.} Với \(z = f \left( {x, y} \right),\) ta có
\begin{equation}
\frac{{5{z^3}}}{4} + \left( {2{x^2} + 1} \right)z + {x^2} + {y^2} - y - 2 = 0.
\label{eq3}
\end{equation}
Ta có các đạo hàm riêng của \(G\) như sau:
\[\left\{ \begin{gathered}
  {G_x}\left( {x,y,z} \right) = 4xz + 2x \hfill \\
  {G_y}\left( {x,y,z} \right) = 2y - 1 \hfill \\
  {G_z}\left( {x,y,z} \right) = \frac{{15{z^2}}}{4} + 2{x^2} + 1 \hfill \\ 
\end{gathered}  \right..\]
Vì vậy \(G_x, G_y, G_z\) xác định và liên tục trên \(\mathbb{R}^3.\)\\
Do \({G_z}\left( {x,y,z} \right) = \frac{{15{z^2}}}{4} + 2{x^2} + 1 > 0,\forall \left( {x,y,z} \right) \in {\mathbb{R}^3}\) nên theo định lí hàm ẩn
\[\left\{ \begin{gathered}
  {f_x}\left( {x,y} \right) =  - \frac{{{G_x}\left( {x,y,z} \right)}}{{{G_z}\left( {x,y,z} \right)}} \hfill \\
  {f_y}\left( {x,y} \right) =  - \frac{{{G_y}\left( {x,y,z} \right)}}{{{G_z}\left( {x,y,z} \right)}} \hfill \\ 
\end{gathered}  \right..\]
Điểm dừng của \(f\) thỏa mãn hệ phương trình sau
\begin{equation}
\left\{ \begin{gathered}
  {f_x}\left( {x,y} \right) = 0 \hfill \\
  {f_y}\left( {x,y} \right) = 0 \hfill \\ 
\end{gathered}  \right..
\label{hpt3}
\end{equation}
Hệ phương trình (\ref{hpt3}) tương đương với
\[\left\{ \begin{gathered}
  {G_x}\left( {x,y,z} \right) = 0 \hfill \\
  {G_y}\left( {x,y,z} \right) = 0 \hfill \\ 
\end{gathered}  \right.\]
\[ \Leftrightarrow \left\{ \begin{gathered}
  4xz + 2x = 0 \hfill \\
  2y - 1 = 0 \hfill \\ 
\end{gathered}  \right. \Leftrightarrow \left\{ \begin{gathered}
  2x\left( {2z + 1} \right) = 0 \hfill \\
  2y - 1 = 0 \hfill \\ 
\end{gathered}  \right.\]
\[ \Leftrightarrow \left\{ \begin{gathered}
  \left[ \begin{gathered}
  x = 0 \hfill \\
  z =  - \frac{1}{2} \hfill \\ 
\end{gathered}  \right. \hfill \\
  y = \frac{1}{2} \hfill \\ 
\end{gathered}  \right. \Leftrightarrow \left[ \begin{gathered}
  \left\{ \begin{gathered}
  x = 0 \hfill \\
  y = \frac{1}{2} \hfill \\ 
\end{gathered}  \right. \hfill \\
  \left\{ \begin{gathered}
  y = \frac{1}{2} \hfill \\
  z =  - \frac{1}{2} \hfill \\ 
\end{gathered}  \right. \hfill \\ 
\end{gathered}  \right..\]
Thay \(y = \frac{1}{2}, z = -\frac{1}{2}\) vào (\ref{eq3}), ta được
\[ - \frac{5}{{32}} - {x^2} - \frac{1}{2} + {x^2} + \frac{1}{4} - \frac{1}{2} - 2 = 0 \Leftrightarrow x \in \varnothing .\]
Vậy \(f\) có một điểm dừng là \(\left( {0,\frac{1}{2}} \right).\)\\
Thay \(\left( {x,y} \right) = \left( {0,\frac{1}{2}} \right)\) vào (\ref{eq3}), ta được
\[\frac{{5{z^3}}}{4} + z - \frac{9}{4} = 0 \Leftrightarrow z = 1.\]
Ta sẽ đi tìm các đạo hàm riêng cấp \(2\) của \(f.\)\\
Một vài đạo hàm riêng cấp \(2\) của \(G\) như sau:
\[\left\{ \begin{gathered}
  {G_{xx}}\left( {x,y,z} \right) = \frac{\partial }{{\partial x}}\left( {{G_x}\left( {x,y,z} \right)} \right) = 4z + 4x{f_x}\left( {x,y} \right) + 2 \hfill \\
  {G_{zx}}\left( {x,y,z} \right) = \frac{\partial }{{\partial x}}\left( {{G_z}\left( {x,y,z} \right)} \right) = \frac{{15}}{2}z{f_x}\left( {x,y} \right) + 4x \hfill \\
  {G_{yy}}\left( {x,y,z} \right) = \frac{\partial }{{\partial y}}\left( {{G_y}\left( {x,y,z} \right)} \right) = 2 \hfill \\
  {G_{zy}}\left( {x,y,z} \right) = \frac{\partial }{{\partial y}}\left( {{G_z}\left( {x,y,z} \right)} \right) = \frac{{15}}{2}z{f_y}\left( {x,y} \right) \hfill \\
  {G_{xy}}\left( {x,y,z} \right) = \frac{\partial }{{\partial y}}\left( {{G_x}\left( {x,y,z} \right)} \right) = 4xz{f_y}\left( {x,y} \right) \hfill \\ 
\end{gathered}  \right..\]
Thay \(\left( {x,y,z} \right) = \left( {0,\frac{1}{2},1} \right),{f_x}\left( {0,\frac{1}{2}} \right) = {f_y}\left( {0,\frac{1}{2}} \right) = 0,\) ta được
\[\left\{ \begin{gathered}
  {G_{xx}}\left( {0,\frac{1}{2},1} \right) = 6 \hfill \\
  {G_{zx}}\left( {0,\frac{1}{2},1} \right) = 0 \hfill \\
  {G_{yy}}\left( {0,\frac{1}{2},1} \right) = 2 \hfill \\
  {G_{zy}}\left( {0,\frac{1}{2},1} \right) = 0 \hfill \\
  {G_{xy}}\left( {0,\frac{1}{2},1} \right) = 0 \hfill \\
  {G_x}\left( {0,\frac{1}{2},1} \right) = 0 \hfill \\
  {G_y}\left( {0,\frac{1}{2},1} \right) = 0 \hfill \\
  {G_z}\left( {0,\frac{1}{2},1} \right) = \frac{{19}}{4} \hfill \\ 
\end{gathered}  \right..\]
Từ đó,
\[{f_{xx}}\left( {x,y} \right) = \frac{\partial }{{\partial x}}\left( {{f_x}\left( {x,y} \right)} \right) = \frac{\partial }{{\partial x}}\left( { - \frac{{{G_x}\left( {x,y,z} \right)}}{{{G_z}\left( {x,y,z} \right)}}} \right)\]
\[ \Rightarrow {f_{xx}}\left( {x,y} \right) =  - \frac{{\frac{\partial }{{\partial x}}\left( {{G_x}\left( {x,y,z} \right)} \right){G_z}\left( {x,y,z} \right) - \frac{\partial }{{\partial x}}\left( {{G_z}\left( {x,y,z} \right)} \right){G_x}\left( {x,y,z} \right)}}{{{{\left( {{G_z}\left( {x,y,z} \right)} \right)}^2}}}\]
\[ \Rightarrow {f_{xx}}\left( {x,y} \right) =  - \frac{{{G_{xx}}\left( {x,y,z} \right){G_z}\left( {x,y,z} \right) - {G_{zx}}\left( {x,y,z} \right){G_x}\left( {x,y,z} \right)}}{{{{\left( {{G_z}\left( {x,y,z} \right)} \right)}^2}}}.\]
\[ \Rightarrow {f_{xx}}\left( {0,\frac{1}{2}} \right) =  - \frac{{6 \cdot \frac{{19}}{4}}}{{{{\left( {\frac{{19}}{4}} \right)}^2}}} =  - \frac{{24}}{{19}}.\]
\[{f_{yy}}\left( {x,y} \right) = \frac{\partial }{{\partial y}}\left( {{f_y}\left( {x,y} \right)} \right) = \frac{\partial }{{\partial y}}\left( { - \frac{{{G_y}\left( {x,y,z} \right)}}{{{G_z}\left( {x,y,z} \right)}}} \right)\]
\[ \Rightarrow {f_{yy}}\left( {x,y} \right) =  - \frac{{\frac{\partial }{{\partial y}}\left( {{G_y}\left( {x,y,z} \right)} \right){G_z}\left( {x,y,z} \right) - \frac{\partial }{{\partial y}}\left( {{G_z}\left( {x,y,z} \right)} \right){G_y}\left( {x,y,z} \right)}}{{{{\left( {{G_z}\left( {x,y,z} \right)} \right)}^2}}}\]
\[ \Rightarrow {f_{yy}}\left( {x,y} \right) =  - \frac{{{G_{yy}}\left( {x,y,z} \right){G_z}\left( {x,y,z} \right) - {G_{zy}}\left( {x,y,z} \right){G_y}\left( {x,y,z} \right)}}{{\left( {{G_z}} \right){{\left( {x,y,z} \right)}^2}}}.\]
\[ \Rightarrow {f_{yy}}\left( {0,\frac{1}{2}} \right) =  - \frac{{2 \cdot \frac{{19}}{4}}}{{{{\left( {\frac{{19}}{4}} \right)}^2}}} =  - \frac{8}{{19}}.\]
\[{f_{xy}}\left( {x,y} \right) = \frac{\partial }{{\partial y}}\left( {{f_x}\left( {x,y} \right)} \right) = \frac{\partial }{{\partial y}}\left( { - \frac{{{G_x}\left( {x,y,z} \right)}}{{{G_z}\left( {x,y,z} \right)}}} \right)\]
\[ \Rightarrow {f_{xy}}\left( {x,y} \right) =  - \frac{{\frac{\partial }{{\partial y}}\left( {{G_x}\left( {x,y,z} \right)} \right){G_z}\left( {x,y,z} \right) - \frac{\partial }{{\partial y}}\left( {{G_z}\left( {x,y,z} \right)} \right){G_x}\left( {x,y,z} \right)}}{{{{\left( {{G_z}\left( {x,y,z} \right)} \right)}^2}}}\]
\[ \Rightarrow {f_{xy}}\left( {x,y} \right) =  - \frac{{{G_{xy}}\left( {x,y,z} \right){G_z}\left( {x,y,z} \right) - {G_{zy}}\left( {x,y,z} \right){G_y}\left( {x,y,z} \right)}}{{{{\left( {{G_z}\left( {x,y,z} \right)} \right)}^2}}}.\]
\[ \Rightarrow {f_{xy}}\left( {0,\frac{1}{2}} \right) = 0.\]
Ta có \(D\left( {0,\frac{1}{2}} \right) = {f_{xx}}\left( {0,\frac{1}{2}} \right){f_{yy}}\left( {0,\frac{1}{2}} \right) - {\left[ {{f_{xy}}\left( {0,\frac{1}{2}} \right)} \right]^2} = \frac{{192}}{{361}} > 0.\) \\
Mà \({f_{xx}}\left( {0,\frac{1}{2}} \right) =  - \frac{{24}}{{19}} < 0\) nên điểm \(\left( {0,\frac{1}{2}} \right)\) là cực đại của \(f.\)\\
Vậy \(f\) có một điểm cực đại là \(\left( {0,\frac{1}{2}} \right).\)
\end{document}