\documentclass[12pt,a4paper]{article}
\usepackage[utf8]{vietnam}
\usepackage{amsmath}
\usepackage{amsfonts}
\usepackage{xcolor}
\usepackage{titlesec}
\usepackage{mdframed}
\usepackage{amssymb}
\usepackage{graphicx}
\usepackage[left=2cm,right=2cm,top=2cm,bottom=2cm]{geometry}
\author{Nguyễn Văn Lộc - 20120131}
\newmdenv[linecolor=black,skipabove=\topsep,skipbelow=\topsep,
leftmargin=-5pt,rightmargin=-5pt,
innerleftmargin=5pt,innerrightmargin=5pt]{mybox}
\begin{document}
\begin{mybox}
\textbf{Họ và tên:} Nguyễn Văn Lộc\\
\textbf{MSSV:} 20120131\\
\textbf{STT:} 99b\\
\textbf{Lớp:} 20CTT1
\end{mybox}
\begin{center}
\textbf{BÀI TẬP ĐIỂM CỘNG CUỐI KÌ TUẦN 14}
\end{center}
\textbf{Bài toán.}
\begin{mybox}
Cho \(D\) là một miền con của \(\mathbb{R}^2\) và \(\vec F: D \to {\mathbb{R}^2}\) là trường trơn và bảo toàn trên \(\mathbb{R}^2.\) Cho \(f\) và \(g\) là các hàm thế của \(\vec F.\)\\
a. Khi \(D = \mathbb{R}^2,\) chứng minh rằng \(f - g\) là hàm hằng.\\
b. Với tập \(D\) như thế nào thì kết luận ở câu a. vẫn còn đúng?
\end{mybox}
\textbf{Lời giải.}\\
a. Theo đề bài, ta có
\begin{center}
\(f\) và \(g\) đều là hàm thế của \(\vec F\) trên \(\mathbb{R}^2\)
\end{center}
\[ \Rightarrow \nabla f = \vec F = \nabla g.\]
\[ \Rightarrow \nabla f = \nabla g.\]
\begin{equation}
 \Rightarrow \left\{ \begin{gathered}
  {f_x}\left( {x,y} \right) = {g_x}\left( {x,y} \right) \hfill \\
  {f_y}\left( {x,y} \right) = {g_y}\left( {x,y} \right) \hfill \\ 
\end{gathered}  \right., \forall \left( {x,y} \right) \in \mathbb{R}^2.
\label{eq1}
\end{equation}
Từ phương trình đầu tiên của hệ (\ref{eq1}), ta được
\[\int {{f_x}\left( {x,y} \right)dx}  = \int {{g_x}\left( {x,y} \right)dx}, \forall \left( {x,y} \right) \in \mathbb{R}^2 \]
\[ \Rightarrow f\left( {x,y} \right) = g\left( {x,y} \right) + D\left( y \right), \forall \left( {x,y} \right) \in \mathbb{R}^2.\]
\[ \Rightarrow \frac{\partial }{{\partial y}}\left( {f\left( {x,y} \right)} \right) = \frac{\partial }{{\partial y}}\left( {g\left( {x,y} \right) + D\left( y \right)} \right), \forall \left( {x,y} \right) \in \mathbb{R}^2\]
\[ \Rightarrow {f_y}\left( {x,y} \right) = {g_y}\left( {x,y} \right) + D'\left( y \right), \forall \left( {x,y} \right) \in \mathbb{R}^2.\]
Từ phương trình thứ hai của hệ (\ref{eq1}), ta suy ra 
\[D'\left( y \right) = 0 \Rightarrow D\left( y \right) = C, \forall \left( {x,y} \right) \in \mathbb{R}^2.\]
\[ \Rightarrow f\left( {x,y} \right) - g\left( {x,y} \right) = C, \forall \left( {x,y} \right) \in \mathbb{R}^2.\]
Vậy ta có điều phải chứng minh.\\
b. Để kết luận ở câu a. vẫn còn đúng, \(D\) phải là miền mở, hình sao.\\
Tức là, tồn tại điểm \(P_0 \in D\) sao cho với mọi \(P \in D\) thì đoạn nối \(P_0\) và \(P\) nằm trong \(D.\)
\end{document}