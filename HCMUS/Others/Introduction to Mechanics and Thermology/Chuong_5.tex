\part{Nhiệt học}
\chapter{Khí lý tưởng}
\section{Một số khái niệm}
\subsection{Khí lý tưởng}
Là chất khí thỏa mãn hai điều kiện sau:
\begin{itemize}
\item Lực tương tác giữa các phân tử tạo thành chất khí không đáng kể (trừ khi chúng va chạm với nhau hoặc khi va chạm với thành bình).
\item Kích thước các phân tử không đáng kể và có thể bỏ qua.
\end{itemize}
Nói một cách chính xác, các khí thực không phải là các khí lý tưởng, nhưng các khí thực khi khá loãng có các tính chất rất gần với khí lý tưởng. Nhiều khí thực như oxygen, hydrogen, nitrogen, ... ở nhiệt độ phòng và áp suất khí quyển có thể coi là khí lý tưởng.
\subsection{Thông số trạng thái}
\subsubsection{Nhiệt độ}
Theo quan điểm cổ điển, nhiệt độ đặc trưng cho mức độ nóng lạnh của một vật, thang đo nhiệt thường sử dụng là: thang nhiệt độ bách phân (Celsius) hoặc thang nhiệt độ tuyệt đối (Kelvin).\\
Liên hệ giữa thang nhiệt độ Kelvin và thang nhiệt độ Celsius là: $T \text{ } \mathrm{K} = t^ \circ \mathrm{C} + 273.$ Như vậy $-273 ^\circ \mathrm{C}$ ứng với $0 \mathrm{K}$ và trong thang nhiệt độ Kelvin không có nhiệt độ âm, do đó thang nhiệt độ này còn được gọi là \textit{thang nhiệt độ tuyệt đối.}
\subsubsection{Áp suất}
Áp suất đặc trưng cho mức độ tác dụng của các phân tử khí lên thành bình. Nếu gọi $F$ là lực nén vuông góc lên một diện tích $S$ của thành bình thì áp suất $p$ là:
$$p = \frac{F}{S}.$$
Trong hệ SI, đơn vị của áp suất là Pascal ($\mathrm{Pa}$).
\subsubsection{Thể tích}
Miền không gian mà các phân tử khí chuyển động, đối với khí lí tưởng, thể tích của bình chứa là thể tích của khối khí.
\section{Phương trình trạng thái của khí lý tưởng}
\subsection{Phương trình trạng thái}
Kết quả thực nghiệm cho thấy đối với một khối khí cho trước thì nhiệt độ, thể tích, áp suất thỏa mãn phương trình sau đây:
$$ pV = \frac{M}{\mu}RT,$$
với:
\begin{itemize}
\item $M$ là khối lượng của chất khí mà ta đang xét tính theo $\mathrm{kg},$
\item $\mu$ là khối lượng của một kilomol chất khí đó,
\item $V$ là thể tích khối khí đang xét, tính theo đơn vị $\mathrm{m^3},$
\item $R = 8.31 \cdot 10^3$ ($\mathrm{\frac{J}{kmol \cdot K}}$) là hằng số, gọi là hằng số khí lí tưởng,
\item $T$ là nhiệt độ của khối khí theo thang nhiệt độ tuyệt đối $\mathrm{K}.$ 
\end{itemize}
Phương trình trên gọi là \textit{phương trình trạng thái của khí lý tưởng.}
\subsection{Một số trường hợp riêng}
\begin{itemize}
\item Quá trình đẳng nhiệt ($T = \mathrm{const}$) (Boyle - Mariotte): là quá trình biến đổi trong đó nhiệt độ $T$ của khối khí được giữ nguyên không dổi. Từ phương trình trạng thái khí lý tưởng ta có:
$$pV = \mathrm{const}.$$
\item Quá trình đẳng áp ($p = \mathrm{const}$) (Gay - Lussac): là quá trình biến đổi trong đó áp suất $p$ của khối khí được giữ nguyên không đổi. Từ phương trình trạng thái khí lý tưởng ta có:
$$\frac{V}{T} = \mathrm{const}.$$
\item Quá trình đẳng tích ($V = \mathrm{const}$) (Charles): là quá trình biến đổi trong đó thể tích $V$ của khối khí được giữ nguyên không đổi. Từ phương trình trạng thái khí lý tưởng ta có:
$$\frac{p}{T} = \mathrm{const}.$$
\end{itemize}
\section{Thuyết động học phân tử các chất khí}
\subsection{Nội dung}
Thuyết này là một trong những thuyết đầu tiên của chất khí gồm các giả thiết sau:
\begin{itemize}
\item Các chất khí được tạo thành từ các phân tử khí.
\item Phân tử khí chuyển động không ngừng và có kích thước rất nhỏ.
\item Các phân tử khí không tương tác với nhau trừ khi va chạm.
\item Va chạm giữa các phân tử khí với nhau và giữa các phân tử khí với thành bình là va chạm đàn hồi.
\end{itemize}
\subsection{Phương trình cơ bản của thuyết động học phân tử các chất khí}
$$p = \frac{2}{3}n \overline{E_d},$$
trong đó $\overline{E_d} = \frac{1}{2} m \overline{v}^2$ là động năng tịnh tiến trung bình của một phân tử.
\subsection{Các hệ quả}
$$p = n k_B T$$
$$\overline{E_d} = \frac{3}{2} k_B T.$$
\subsection{Luật phân bố đều năng lượng theo các bậc tự do}
\subsubsection{Bậc tự do}
Là số tọa độ độc lập cần thiết để xác định vị trí của phân tử khí ở trong không gian. Ký hiệu bậc tự do là $i.$
\begin{itemize}
\item Trường hợp phân tử chỉ có một nguyên tử (các hơi kim loại) thì bậc tự do của chúng là $i = 3.$
\item Trường hợp phân tử gồm hai nguyên tử (các khí oxygen, nitrogen, hydrogen, ...) thì bậc tự do của chúng là $i = 5.$
\item Trường hợp phân tử gồm $\geqslant 3$ nguyên tử (ví dụ $\mathrm{CO_2}$) thì bậc tự do của chúng là $i = 6.$
\end{itemize}
\subsubsection{Luật phân bố đều năng lượng theo các bậc tự do}
Maxwell: \textit{"Động năng trung bình của phân tử được phân bố đều cho các bậc tự do của phân tử"}.\\
Tổng quát có thể nói rằng: phân tử có bậc tự do là $i$ thì năng lượng của phân tử là $\frac{1}{2} k_B T.$
\subsection{Nội năng của khí lý tưởng}
Nội năng của khí lý tưởng:
$$U = \frac{M}{\mu}  \frac{i}{2} RT.$$
Ta thấy nội năng của khí lý tưởng chỉ phụ thuộc vào nhiệt độ của khối khí. Do trong một quá trình biến đổi bất kỳ, nếu nhiệt độ của khối khí thay đổi một lượng là $\Delta T = T_2 - T_1,$ thì độ biến thiên nội năng là:
$$\Delta U = \frac{M}{\mu}  \frac{i}{2} R \Delta T.$$