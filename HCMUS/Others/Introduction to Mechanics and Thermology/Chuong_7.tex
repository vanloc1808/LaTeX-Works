\chapter{Nguyên lý thứ hai nhiệt động học}
\section{Những hạn chế của nguyên lý thứ nhất nhiệt động học}
\begin{itemize}
\item Không xác định chiều truyền tự nhiên của nhiệt lượng. Nhiệt truyền tự nhiên từ vật nóng hơn sang vật lạnh hơn. Không có quá trình tự nhiên ngược lại.
\item Không xác định chiều chuyển hóa tự nhiên của năng lượng. Thế năng tự nhiên chuyển hóa thành động năng rồi thành nhiệt năng tỏa ra. Không có quá trình tự nhiên ngược lại.
\item Mặc dù các quá trình tự nhiên ngược lại của hai ví dụ trên đều thỏa mãn nguyên lý thứ nhất nhiệt động học.
\item Không đánh giá được chất lượng nhiệt.
\item Không phân biệt khác nhau giữa công và nhiệt.
\end{itemize}
\section{Quá trình thuận nghịch và không thuận nghịch}
Quá trình đưa một hệ nhiệt động từ trạng thái 1 sang trạng thái 2 được gọi là \textit{thuận nghịch} nếu ta có thể thực hiện được quá trình ngược lại, tức là đưa hệ từ trạng thái 2 trở về trạng thái 1 và \textit{đi qua đúng mọi trạng thái trung gian} giống hệ như đi theo chiều thuận từ 1 sang 2.\\
Nếu không thực hiện được quá trình ngược đi qua đúng các trạng thái trung gian như cũ, thì quá trình đó được gọi là \textit{không thuận nghịch,} các quá trình có ma sát là không thuận nghịch.
\section{Nguyên lý thứ hai nhiệt động học}
\subsection{Máy nhiệt}
Máy nhiệt là một hệ hoạt động tuần hoàn để biến công thành nhiệt hoặc nhiệt thành công.
\subsubsection{Động cơ nhiệt}
\begin{itemize}
\item \textit{Nguyên tắc:} là loại máy nhiệt \textit{biến đổi nhiệt lượng thành công.} Ví dụ: động cơ hơi nước, động cơ đốt trong, ...
\item \textit{Tác nhân:} chất vận chuyển (khí, hơi nước, xăng, ...), biến nhiệt thành công: tuần hoàn.
\item \textit{Hiệu suất của động cơ nhiệt:} Là tỉ số giữa công sinh ra $A'$ và nhiệt lượng nhận vào $Q_1:$
$$\eta  = \frac{{A'}}{{{Q_1}}}.$$
Theo nguyên lý thứ nhất, trong một chu trình, nhiệt lượng mà hệ nhận vào $Q_1$ bằng công $A'$ do tác nhân sinh ra cộng với nhiệt $Q_2 ^\prime$ mà hệ nhả ra cho nguồn lạnh:
$${Q_1} = A' + {Q_2}^\prime .$$
$$\eta  = \frac{{{Q_1} - {Q_2}^\prime }}{{{Q_1}}}$$
Vậy hiệu suất của động cơ nhiệt là:
$$\eta  = 1 - \frac{{{Q_2}^\prime }}{{{Q_1}}}.$$
\end{itemize}
\subsubsection{Máy làm lạnh}
\begin{itemize}
\item \textit{Nguyên tắc:} là máy nhiệt \textit{biến công thành nhiệt.} Tác nhân trong máy làm lạnh biến đổi theo quá trình ngược với động cơ nhiệt.
\item Trong quá trình hoạt động, tác nhân nhận công $A$ từ ngoại vật, lấy nhiệt lượng $Q_2$ của nguồn lạnh, nhả nhiệt lượng $Q_1 ^\prime$ cho nguồn nóng. Hệ số làm lạnh của máy làm lạnh là:
$$\varepsilon = \frac{Q_2}{A}$$
$$\Leftrightarrow \varepsilon = \frac{Q_2}{Q_1 ^\prime - Q_2}.$$
\end{itemize}
\subsection{Phát biểu nguyên lý thứ hai}
\subsubsection{Phát biểu của Thompson (liên quan đến động cơ nhiệt)}
Phát biểu: \textit{Một động cơ nhiệt không thể sinh công nếu nó chỉ trao đổi nhiệt với một nguồn nhiệt duy nhất.}\\
Người ta gọi động cơ nhiệt hoạt động tuần hoàn bằng cách chỉ trao đổi nhiệt với một nguồn nhiệt duy nhất là \textit{động cơ vĩnh cửu loại hai,} nói cách khác, động cơ vĩnh cửu loại hai là động cơ nhiệt có hiệu suất $100 \%.$ Nguyên lý thứ hai khẳng định không thể chế tạo động cơ vĩnh cửu loại hai.
\subsubsection{Phát biểu của Clausius (liên quan đến máy làm lạnh)}
Phát biểu: \textit{Không thể tồn tại một quá trình nhiệt động mà kết quả duy nhất là sự truyền nhiệt lượng từu vật lạnh hơn sang vật nóng hơn.}\\
Tóm lại, theo nguyên lý thứ hai nhiệt động học thì \textbf{công có thể biến hoàn toàn thành nhiệt} như trong các quá trình có sự tham gia của ma sát, nhưng ngược lại \textbf{nhiệt chỉ có thể biến một phần của nó thành công cơ học}.
\section{Chu trình Carnot và định lý Carnot}
\subsection{Chu trình Carnot thuận nghịch}
Chu trình Carnot gồm hai quá trình đẳng nhiệt và hai quá trình đoạn nhiệt xen kẽ nhau: dãn nở đẳng nhiệt $\to$ dãn nở đoạn nhiệt $\to$ nén đẳng nhiệt $\to$ nén đoạn nhiệt, đây là chu trình Carnot với động cơ nhiệt. Nếu tiến hành ngược lại, ta được chu trình Carnot với máy làm lạnh.
\subsection{Hiệu suất của chu trình Carnot thuận nghịch}
$$\eta_{Carnot} = 1 - \frac{T_2}{T_1}.$$
\subsection{Định lý Carnot}
\begin{itemize}
\item Hiệu suất của tất cả các động cơ nhiệt làm việc theo chu trình Carnot thuận nghịch với cùng \textit{nguồn nóng như nhau} và \textit{nguồn lạnh như nhau} thì \textit{bằng nhau}, không phụ thuộc bản chất của tác nhân, \textit{chỉ phụ thuộc vào nhiệt độ của hai nguồn nhiệt.}
\item Hiệu suất của động cơ không thuận nghịch thì nhỏ hơn hiệu suất của động cơ thuận nghịch:
$$\eta_{ktn} < \eta_{tn}.$$
\item Trong cùng điều kiện như nhau, chu trình Carnot luôn có hiệu suất lớn hơn các chu trình không phải là Carnot.
\end{itemize}