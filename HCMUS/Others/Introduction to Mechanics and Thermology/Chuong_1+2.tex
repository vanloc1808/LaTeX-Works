\part{Cơ học}
\chapter{Động học chất điểm}
\section{Một số khái niệm mở đầu}
\subsection{Chuyển động cơ học}
Là sự thay đổi vị trí của vật này so với vật khác.\\
Có tính tương đối
\subsection{Động học}
Là phần cơ học, nghiên cứu về hình thái chuyển động của các vật mà không xét đến các lực là nguyên nhân làm thay đổi trạng thái chuyển động.
\subsection{Chất điểm}
Vật có kích thước nhỏ so với quãng đường mà nó chuyển động.
\subsection{Không gian và thời gian}
\subsection{Hệ quy chiếu}
Vật được chọn làm mốc và xem là đứng yên để xét chuyển động của các vật khác trong không gian.
\subsection{Hệ tọa độ}
Là hệ thống các đường thẳn có định vector đơn vị và các góc định hướng dùng để xác định vị trí và chuyển động của các vật.
\subsubsection{Hệ tọa độ Descartes}
\subsubsection{Hệ tọa độ cầu}
\subsubsection{Hệ tọa độ cong}
\subsection{Phương trình chuyển động và phương trình quỹ đạo}
\subsubsection{Phương trình chuyển động của chất điểm}
$$x = x \left( t \right), y = y \left( t \right)$$
\subsubsection{Phương trình quỹ đạo của chất điểm}
Không phụ thuộc vào tham số thời gian, có thể thu được bằng cách \textit{khử tham số $t$} từ phương trình chuyển động.
\section{Vector vận tốc của chất điểm}
\subsection{Định nghĩa}
\subsubsection{Giá trị của vận tốc}
Vận tốc trung bình
$$\overline v  = \frac{{\Delta s}}{{\Delta t}}.$$
Vận tốc tức thời
$$v = \frac{\mathrm{d}s}{\mathrm{d}t}.$$
\subsubsection{Vector vận tốc}
$$\overrightarrow{v} = \frac{\mathrm{d}\overrightarrow{r}}{\mathrm{d}t}.$$
\subsection{Thành phần, độ lớn, phương chiều của vận tốc}
$$v = \sqrt{v_x^2 + v_y^2 + v_z^2}$$
\section{Vector gia tốc của chất điểm}
\subsection{Định nghĩa}
Gia tốc tức thời
$$\overrightarrow{a} = \frac{\mathrm{d}\overrightarrow{v}}{\mathrm{d}t} = \frac{\mathrm{d}^2 \overrightarrow{r}}{\mathrm{d}{t^2}}.$$
\subsection{Thành phần của gia tốc}
$$a = \sqrt{a_x^2 + a_y^2 + a_z^2}$$
\subsection{Gia tốc tiếp tuyến và gia tốc pháp tuyến của chất điểm chuyển động cong}
Gia tốc pháp tuyến
$$a_n = \frac{v^2}{R}.$$
Gia tốc tiếp tuyến
$$a_\tau = \frac{\mathrm{d}v}{\mathrm{d}t}.$$
Trị tuyệt đối của gia tốc toàn phần
$$a = \sqrt{{a_n^2} + {a_\tau^2}} = \sqrt{\left( {\frac{v^2}{R}} \right)^2 + \left( {\frac{\mathrm{d}v}{\mathrm{d}t}} \right)^2}.$$
\section{Vận tốc góc và gia tốc góc trong chuyển động tròn}
\subsection{Vận tốc góc}
$$\omega = \frac{\mathrm{d}\varphi}{\mathrm{d}t}.$$
\subsubsection{Liên hệ giữa vận tốc góc $\omega$ và vận tốc dài $v$}
$$v = R \omega \Leftrightarrow \omega = \frac{v}{R}.$$
\subsection{Gia tốc góc}
$$\beta = \frac{\mathrm{d}\omega}{\mathrm{d}t}.$$
\subsubsection{Liên hệ giữa gia tốc góc $\beta$ và gia tốc tiếp tuyến $a_\tau$}
$$a_\tau = R\beta \Leftrightarrow \beta = \frac{a_\tau}{R}.$$
\subsection{Vector vận tốc góc và vector gia tốc góc}
$$\overrightarrow v  = \overrightarrow \omega   \times \overrightarrow R .$$
$$\overrightarrow {{a_\tau }}  = \overrightarrow \beta   \times \overrightarrow R .$$
$$\overrightarrow {{a_n}}  = \overrightarrow \omega   \times \overrightarrow v .$$
\section{Rơi tự do}
Người ta gọi sự rơi của các vật chỉ do tác dụng của sức hút Trái Đất với vận tốc đầu bằng không là sự rơi tự do.
$$ v = gt \text{ do (} v_0 = 0 \text{)}.$$
$$ h = \frac{1}{2}gt^2.$$
Trong trường hợp vật được ném từ dưới lên, thông thường người ta chọn chiều dương từ dưới lên và gốc tọa độ $O$ tại mặt đất.
\section{Chuyển động của vật bị ném}
Phương trình chuyển động:
$$\left\{ \begin{gathered}
  x = {v_0}\left( {\cos \alpha } \right)t \hfill \\
  y = {v_0}\left( {\sin \alpha } \right)t - \frac{1}{2}g{t^2} \hfill \\ 
\end{gathered}  \right..$$
Phương trình quỹ đạo:
$$y =  - \frac{g}{{2v_0^2{{\cos }^2}\alpha }}{x^2} + x\tan \alpha .$$
\section{Phép cộng vận tốc và gia tốc cổ điển}

\chapter{Động lực học chất điểm}
\section{Ba định luật Newton}
\subsection{Định luật 1 Newton}
Một vật cô lập (không chịu tác dụng bởi các lực bên ngoài hoặc hợp lực tác dụng lên nó bằng không) nếu nó:
\begin{itemize}
\item Đang đứng yên thì sẽ đứng yên mãi.
\item Đang chuyển động thì sẽ chuyển động thẳng đều mãi.
\end{itemize}
Do đó một vật bất kì có khả năng bảo toàn trạng thái đứng yên hay chuyển động của nó, nên người ta gọi nó là có quán tính. Định luật này còn gọi là \textit{định luật quán tính.}\\
Định luật 1 Newton chỉ đúng với hệ quy chiếu quán tính.
\subsection{Định luật 2 Newton}
Một chất điểm có khối lượng $m$ chịu tác dụng của một lực $\overrightarrow{F},$ sẽ chuyển động với một gia tốc $\overrightarrow{a}$ thỏa phương trình:
$$\overrightarrow{F} = m \overrightarrow{a}.$$
Định luật 1 là trường hợp riêng của định luật 2.\\
Tương tự như định luật 1, định luật 2 Newton cũng chỉ đúng với hệ quy chiếu quán tính. 
\subsection{Định luật 3 Newton}
Khi một vật tác dụng lên một vật khác bằng một lực $\overrightarrow{F_{12}}$ (tác lực) thì ngược lại nó cũng sẽ chịu tác dụng từ vật kia một lực $\overrightarrow{F_{21}}$ (phản lực) đối kháng (cùng phương, cùng trị số, ngược chiều).
$$\overrightarrow{F_{12}} = - \overrightarrow{F_{21}}.$$
Định luật 3 Newton cũng chỉ đúng với hệ quy chiếu quán tính.
\section{Hệ quy chiếu không quán tinh $-$ lực quán tính $-$ nguyên lý tương đối của Galilee}
\subsection{Hệ quy chiếu không quán tính}
Bất kì một hệ quy chiếu nào chuyển động có gia tốc so với hệ quy chiếu quán tính đều là \textit{hệ quy chiếu không quán tính.}
\subsection{Lực quán tính}
\subsection{Nguyên lí tương đối Galilee}
Một hiện tượng cơ học bất kì thì xảy ra như nhau đối với các hệ quy chiếu quán tính khác nhau.
\section{Một số lực trong cơ học}
\subsection{Trọng lực và trọng lượng}
\subsubsection{Trọng lực}
Là lực làm cho mọi vật đều rơi về phía Trái Đất với gia tốc trọng trường $\overrightarrow{g}.$\\
Xét trong hệ quy chiếu Trái Đất quay, trọng lực là tổng hợp lực của lực hấp dẫn và lực ly tâm.\\
\textit{Lực hấp dẫn.}
$$F = G \frac{mM}{r^2}.$$
$G = 6.67 \cdot 10^{-11} \left( {\mathrm{\frac{Nm^2}{kg^2}}} \right):$ là hằng số hấp dẫn.\\
\textit{Lực li tâm.} 
$${\overrightarrow F _{LT}} = m\omega _0^2{\overrightarrow r _ \bot }.$$ 
Hợp lực: $\overrightarrow{P} = \overrightarrow{F} + \overrightarrow{F}_LT = m \overrightarrow{g}.$
\subsubsection{Trọng lượng}
Là lực mà vật tác dụng lên giá đỡ nó hay dây treo nó.\\
Khi giá đỡ hay dây treo đứng yên thì trọng lượng bằng trọng lực:
$$P = P' = mg.$$
\subsection{Lực đàn hồi}
Khi ngoại lực tác dụng làm biến dạng vật, trong vật sẽ xuất hiện một lực có xu hướng chống lại biến dạng đó. Lực ấy gọi là \textit{lực đàn hồi.}\\
Xét biến dạng một chiều, lực đàn hồi tuân theo định luật Hooke: \textit{Trong giới hạn đàn hồi, lực đàn hồi tỉ lệ với độ biến dạng của vật.}
$$\overrightarrow{F}_{\mathrm{dh}} = -k \Delta \overrightarrow{x},$$
với $k$ $\left( {\mathrm{\frac{N}{m}}} \right)$ là \textit{hệ số đàn hồi} hay \textit{độ cứng.}\\
Thể hiện rõ nhất ở lò xo.
\subsection{Lực ma sát}
Lực ma sát là lực xuất hiện trên hai mật tiếp xúc giữa hai vật và có xu hướng cản trở sự chuyển động tương đối giữa hai vật đó.\\
Độ lớn:
$$F_{\mathrm{ms}} = k \cdot N$$
với:
\begin{itemize}
\item $k$ là hệ số ma sát,
\item $N$ là phản lực.
\end{itemize}
Điểm chung của lực ma sát:
\begin{itemize}
\item Có xu hướng cản trở sự chuyển động của vật nên ngược chiều chuyển động của vật.
\item Có độ lớn tỉ lệ thuận với phản lực $N$ hoặc vận tốc $v.$
\item Điểm đặt: trên vật.
\end{itemize}
\subsection{Lực căng của sợi dây}
Khi một vật bị buộc chặt vào một sợi dây treo tại một điểm cố định nào đó trên giá treo thì dưới tác dụng của ngoại lực, sợi dây bị kéo căng. Tại các điểm trên dây xuất hiện các lực gọi là \textit{lực căng của dây.}\\
Thường được xác định bằng định luật 2 Newton.\\
Nếu dây đồng chất lí tưởng thì ở mọi điểm trên dây lực căng dây đều như nhau.\\
Dây là vật không chống lại lực nén mà chỉ chống lại lực kéo.