\chapter{Những vấn đề chung về nhà nước và pháp luật}
\begin{ques}
Theo quan điểm của chủ nghĩa Marx – Lenin, Nhà nước là hiện tượng xã hội có tính vĩnh cửu, bất biến.
\end{ques}
\begin{ans}
\textbf{Sai}. Theo học thuyết Marx:
\begin{itemize}
\item Nhà nước xuất hiện tại xã hội tồn tại \textbf{chế độ tư hữu} và \textbf{phân chia thành các giai cấp} đối kháng. Nhà nước là sản phẩm của những đối kháng giai cấp không thể điều hòa được.
\item Nhà nước là một phạm trù lịch sử, \textbf{xuất hiện khách quan} nhưng không vĩnh cửu và bất biến. Nhà nước luôn vận động, phát triển và \textbf{sẽ tiêu vong} khi điều kiện khách quan cho sự tồn tại của nó không còn nữa.
\item Trong hình thái kinh tế \textbf{cộng sản chủ nghĩa} không tồn tại nhà nước.
\end{itemize}
\end{ans}

\begin{ques}
Theo chủ nghĩa Marx – Lenin, nguyên nhân hình thành nhà nước là do ba lần phân công lao động trong xã hội công xã nguyên thủy.
\end{ques}
\begin{ans}
\textbf{Sai.} Theo quan điểm của chủ nghãi Marx - Lenin, nhà nước xuất hiện tại xã hội tồn tại chế độ tư hữu và phân chia thành các giai cấp đối kháng. Nhà nước là sản phẩm của những đối kháng giai cấp không thể điều hòa được.
\end{ans}

\begin{ques}
Khi lí giải nguồn gốc ra đời của Nhà nước, các học thuyết đều dựa trên việc phân tích tiền đề kinh tế, tiền đề xã hội cho sự ra đời của nhà nước.
\end{ques}
\begin{ans}
\textbf{Sai.} Học thuyết về thần quyền, học thuyết gia trưởng và học thuyết khế ước xã hội không đề cập đến tiền đề kinh tế $-$ xã hội.
\begin{itemize}
\item Học thuyết thần quyền cho rằng mọi sự vật trên thế giới đều do Thượng đế sáng tạo ra, và \textbf{Thượng đế tạo ra nhà nước} để duy trì trật tựt hế giới bằng cách trao quyền lực tối thượng, siêu nhiên, vô hạn cho nhà nước.
\item Học thuyết gia trưởng cho rằng nhà nước tiến hóa theo thời gian, ban đầu là từ các gia đình riêng lẻ rồi đến các gia tộc, sau đó tập trung lại thành các bộ lạc, dần dần hình thành nên nhà nước. Nhà nước là kết quả từu \textbf{"gia đình"} và \textbf{"quyền gia trưởng"}.
\end{itemize}
\end{ans}

\begin{ques}
Nhà nước xã hội chủ nghĩa tôn trọng tối đa quyền làm chủ của nhân dân nên không mang bản chất giai cấp.
\end{ques}
\begin{ans}
\textbf{Sai.} Nhà nước là một cơ quan thống trị giai cấp, là một cơ quan áp bức của một giai cấp này với một giai cấp khác, đó là sự kiến lập một \textbf{"trật tự"}, trật tự này hợp pháp hóa và củng cố sự áp bức kia bằng cách làm dịu bớt xung đột giai cấp. \textbf{Bất cứ nhà nước nào cũng mang bản chất giai cấp.}
\end{ans}

\begin{ques}
Tùy vào các nhà nước khác nhau mà bản chất nhà nước có thể là bản chất giai cấp hay xã hội.
\end{ques}
\begin{ans}
\textbf{Sai.} Bản chất của nhà nước \textbf{luôn bao gồm tính giai cấp (class) và tính xã hội (social)}. \\
Tính giai cấp:
\begin{itemize}
\item Nhà nước là sản phẩm và biểu hiện của những mâu thuẫn giai cấp không thể điều hòa được.
\item Nhà nước là một cơ quan thống trị giai cấp, là một cơ quan áp bức của một giai cấp này với một giai cấp khác, đó là sự kiến lập một \textbf{"trật tự"}, trật tự này hợp pháp hóa và củng cố sự áp bức kia bằng cách làm dịu bớt xung đột giai cấp.
\end{itemize}
Tính xã hội: nhà nước phản ánh ý chí chung, lợi ích chung của xã hội và cũng thể hiện qua các nhiệm vụ chung của nhà nước.
\end{ans}

\begin{ques}
Chức năng lập pháp của nhà nước là hoạt động xây dựng pháp luật và tổ chức thực thi pháp luật.
\end{ques}
\begin{ans}
\textbf{Sai.} \textbf{Chức năng lập pháp của nhà nước chỉ bao gồm hoạt động xây dựng và ban hành pháp luật.} Tất cả các quy định của nhà nước đều được thể hiện trong những quy định của pháp luật và được đảm bảo thực hiện bằng nhiều biện pháp, trong đó có biện pháp cưỡng chế nhà nước.\\
\textbf{Hoạt động tổ chức thực thi pháp luật là do cơ quan hành pháp chịu trách nhiệm.}
\end{ans}

\begin{ques}
Nguồn gốc ra đời của pháp luật và nhà nước là giống nhau.
\end{ques}
\begin{ans}
\textbf{Đúng.} Nguồn gốc của nhà nước: nhà nước xuất hiện tại xã hội tồn tại chế độ tư hữu và phân chia thành các giai cấp đối kháng. Nhà nước là sản phẩm của những đối kháng giai cấp không thể điều hòa được.\\
Nguồn gốc của pháp luật: sự hình thành xã hội giai cấp dẫn đến sự xung đột lợi ích giữa các nhóm, các tập đoàn người, dẫn đến sự đấu tranh gay gắt giữa các giai cấp trong xã hội, do đó cần phải có pháp luật.\\
Vậy nguồn gốc ra đời của pháp luật và nhà nước đều là \textbf{chế độ tư hữu} và \textbf{sự phân chia thành các giai cấp đối kháng}.
\end{ans}

\begin{ques}
Pháp luật và nhà nước ra đời cùng một thời điểm.
\end{ques}
\begin{ans}
\textbf{Sai.} Để bảo đảm xã hội được ổn định, giai cấp cầm quyền đã thiết lập một thiết chế là nhà nước, do giai cấp thống trị nắm giữ. Để bảo đảm trật tự xã hội, bảo vệ lợi ích kinh tế $-$ chính trị của mình, họ đặt ra các quy tắc ứng xử mang tính bắt buộc đối với mọi người trong xã hội, khi đó pháp luật mới xuất hiện.\\
\textbf{Nghĩa là nhà nước hình thành trước, rồi sau đó nhà nước đật ra pháp luật. Pháp luật là công cụ điều chỉnh xã hội của nhà nước.}
\end{ans}

\begin{ques}
Pháp luật chỉ ra đời khi xã hội có sự tư hữu, phân hóa giai cấp và đấu tranh giai cấp.
\end{ques}
\begin{ans}
\textbf{Đúng.} Nguồn gốc ra đời của pháp luật:
\begin{itemize}
\item Sự hình thành xã hội giai cấp dẫn đến sự xung đột lợi ích giữa các nhóm, các tập đoàn người, dẫn đến sự đấu tranh gay gắt giữa các giai cấp trong xã hội.
\item Để bảo đảm xã hội được ổn định, giai cấp cầm quyền đã thiết lập một thiết chế là nhà nước, do giai cấp thống trị nắm giữ. Để bảo đảm trật tự xã hội, bảo vệ lợi ích kinh tế $-$ chính trị của mình, họ đặt ra các quy tắc ứng xử mang tính bắt buộc đối với mọi người trong xã hội, khi đó pháp luật xuất hiện.
\end{itemize}
Vậy \textbf{pháp luật chỉ ra đời khi có sự tư hữu, phân hóa giai cấp và đấu tranh giai cấp.}
\end{ans}

\begin{ques}
Pháp luật tồn tại song hành với sự tồn tại của nhà nước.
\end{ques}
\begin{ans}
\textbf{Đúng.} Để đảm bảo trật tự xã hội, bảo vệ lợi ích kinh tế $-$ chính trị của mình, nhà nước đặt ra các quy tắc ứng xử mang tính bắt buộc đối với mọi người trong xã hội $-$ đó là pháp luật. Pháp luật là công cụ quản lý xã hội của nhà nước và chỉ có nhà nước mới có thẩm quyền ban hành pháp luật.
\end{ans}

\chapter{Hình thái nhà nước, bộ máy nhà nước Cộng hòa Xã hội Chủ nghĩa Việt Nam}
\begin{ques}
Trong chính thể cộng hòa tổng thống, Tổng thống do nhân dân trực tiếp bầu ra, là nguyên thủ quốc gia còn Thủ tướng là người đứng đầu Chính phủ.
\end{ques}
\begin{ans}
\textbf{Sai.} Trong chính thể cộng hòa tổng thống, Tổng thống do nhân dân trực tiếp bầu ra, \textbf{vừa là nguyên thủ quốc gia vừa là người đứng đầu Chính phủ}; \textbf{không có chức vụ Thủ tướng}.\\
Chỉ có trong chính thể \textbf{cộng hòa lưỡng tính} thì Tổng thống do nhân dân bầu ra, là nguyên thủ quốc gia còn Thủ tướng là người đứng đầu Chính phủ.
\end{ans}

\begin{ques}
Tại các quốc gia theo hình thức chính thể cộng hòa dân chủ đại nghị, cơ quan quyền lực nhà nước (Quốc hội, Nghị viện) và nguyên thủ quốc gia (Tổng thống, Chủ tịch nước) do nhân dân bầu ra và chịu trách nhiệm trước nhân dân.
\end{ques}
\begin{ans}
\textbf{Sai.} Tại các quốc gia theo hình thức chính thể cộng hòa dân chủ đại nghị, cơ quan quyền lực nhà nước (Nghị viện, Quốc hội) là một thiết chế quyền lực trung tâm, do nhân dân bâu ra, nguyên thủ quốc gia (Tổng thống, Chủ tịch nước) \textbf{do cơ quan quyền lực nhà nước (Nghị viện, Quốc hội) bầu ra và chịu trách nhiệm trước cơ quan quyền lực nhà nước (Nghị viện, Quốc hội)}.
\end{ans}

\begin{ques}
Tại các nhà nước theo chế độ quân chủ, quyền lực nhà nước tập trung hoàn toàn trong tay người đứng đầu nhà nước đó.
\end{ques}
\begin{ans}
\textbf{Sai.} Chỉ có trong chính thể quân chủ chuyên chế thì quyền lực nhà nước mới tập trung hoàn toàn vào tay người đứng đầu nhà nước $-$ nhà vua.\\
Còn tại các quốc gia theo chính thể \textbf{quân chủ lập hiến}, quyền lực của nhà vua bị hạn chế, phải nhường quyền lực cho các thiết chế khác (Nghị viện, Quốc hội) $-$ là các cơ quan do nhân dân bầu ra.
\end{ans}

\begin{ques}
Các quốc gia theo hình thức chính thể quân chủ thì chế độ chính trị là phản dân chủ.
\end{ques}
\begin{ans}
\textbf{Sai.} Tại các quốc gia theo hình thức chính thể quân chủ lập hiến (quân chủ hạn chế) như Thái Lan, Nhật Bản, Anh, ... thì vẫn có chế độ chính trị dân chủ.
\end{ans}

\begin{ques}
Tại các quốc gia theo hình thức chính thể cộng hòa, mọi người dân đều có quyền tham gia bầu cử ra cơ quan quyền lực nhà nước.
\end{ques}
\begin{ans}
\textbf{Sai.} Tại các quốc gia theo hình thức chính thể cộng hòa, tất cả mọi người \textbf{đủ tuổi bầu cử theo quy định} thuộc các tầng lớp nhân dân mới được đi bầu cử. Ví dụ, ở nước ta, theo Luật bầu cử đại biểu Quốc hội và đại biểu Hội đồng Nhân dân 2015, công dân Việt Nam đủ 18 tuổi trở lên mới được đi bầu cử.
\end{ans}

\begin{ques}
Tại các quốc gia theo hình thức cấu trúc nhà nước liên bang, mặc dù có hai hệ thống cơ quan nhà nước nhưng chỉ có một hệ thống pháp luật áp dụng chung cho toàn liên bang.
\end{ques}
\begin{ans}
\textbf{Sai.} Tại các quốc gia theo hình thức cấu trúc nhà nước liên bang, có các đặc điểm sau:
\begin{itemize}
\item Được hợp thành từ hai nhà nước thành viên trở lên.
\item CÓ hai loại chủ quyền quốc gia: chủ quyền nhà nước liên bang và chủ quyền nhà nước thành viên mỗi bang.
\item Có hai hệ thống cơ quan nhà nước: nhà nước liên bang và nhà nước thành viên mỗi bang.
\item \textbf{Có hai hệ thống pháp luật: hệ thống pháp luật liên bang và hệ thống pháp luật mỗi bang.}
\end{itemize}
\end{ans}

\begin{ques}
Tại Việt Nam, mọi công dân đều có quyền bầu cử Quốc hội và Hội đồng nhân dân các cấp.
\end{ques}
\begin{ans}
\textbf{Sai.} Theo Điều 2 Luật bầu cử đại biểu Quốc hội và đại biểu Hội đồng Nhân dân 2015, công dân nước Cộng hòa Xã hội Chủ nghĩa Việt Nam \textbf{đủ mười tám tuổi trở lên} có quyền bầu cử.
\end{ans}

\begin{ques}
Ở nước ta, người từ đủ 18 tuổi trở lên có thể ứng cử làm Đại biểu Quốc hội.
\end{ques}
\begin{ans}
\textbf{Sai.} Theo Điều 2 Luật bầu cử đại biểu Quốc hội và đại biểu Hội đồng Nhân dân 2015, công dân nước Cộng hòa Xã hội Chủ nghĩa Việt Nam \textbf{đủ hai mươi mốt tuổi trở lên} có quyền ứng cử vào Quốc hội, Hội đồng nhân dân các cấp.
\end{ans}

\begin{ques}
Ở nước ta, tất cả mọi người từ đủ 18 tuổi trở lên có quyền đi bầu cử đại biểu Quốc hội.
\end{ques}
\begin{ans}
\textbf{Sai.} Tại khoản 1, Điều 30 Luật bầu cử đại biểu Quốc hội và đại biểu Hội đồng Nhân dân 2015, người đang bị tước quyền bầu cử theo bản án, quyết định của Tòa án đã có hiệu lực pháp luật, người bị kết án tử hình đang trong giai đoạn chờ thi hành án, người đang chấp hành hình phạt tù mà không được hưởng án treo, người mất năng lực hành vi dân sự thì không được ghi tên vào danh sách cử tri.
\end{ans}

\begin{ques}
Đoàn Thanh niên Cộng sản Hồ Chí Minh, Đảng Cộng sản Việt Nam là các cơ quan nhà nước.
\end{ques}
\begin{ans}
\textbf{Sai.}\\
Đảng Cộng sản Việt Nam là tổ chức chính trị.\\
Đoàn Thanh niên Cộng sản Hồ Chí Minh là tổ chức chính trị $-$ xã hội.
\end{ans}

\begin{ques}
Ở nước ta, Quốc hội là cơ quan quyền lực nhà nước cao nhất, đại diện cho ý chí và nguyện vọng của người dân cả nước, nắm trong tay quyền lập pháp, hành pháp và tư pháp.
\end{ques}
\begin{ans}
\textbf{Sai.} Theo Điều 69 Hiến pháp 2013:\\
\textit{Quốc hội là cơ quan đại biểu cao nhất của Nhân dân, là cơ quan quyền lực nhà nước cao nhất của nước Cộng hòa Xã hội Chủ nghĩa Việt Nam.}\\
\textit{Quốc hội thực hiện quyền lập hiến, quyền lập pháp, quyết định các vấn đề quan trọng của đất nước và giám sát tối cao đối với hoạt động của Nhà nước.}\\
\textbf{Quốc hội không thực hiện quyền hành pháp và tư pháp.}
\end{ans}

\begin{ques}
Ở nước ta, Quốc hội và Hội đồng nhân dân các cấp là cơ quan quyền lực nhà nước cao nhất, đại diện cho ý chí và nguyện vọng của người dân cả nước.
\end{ques}
\begin{ans}
\textbf{Sai.} Hội đồng nhân dân là cơ quan quyền lực nhà nước ở địa phương, đại diện cho ý chí, nguyện vọng và quyền làm chủ của Nhân dân địa phương, do Nhân dân địa phương bầu ra, chịu trách nhiệm trước Nhân dân địa phương và cơ quan nhà nước cấp trên.
\end{ans}

\begin{ques}
Quốc hội là cơ quan hành chính cao nhất của nước ta.
\end{ques}
\begin{ans}
\textbf{Sai.} Theo điều 69 Hiến pháp 2013:\\
\textit{Quốc hội là cơ quan đại biểu cao nhất của Nhân dân, là cơ quan quyền lực nhà nước cao nhất của nước Cộng hòa Xã hội Chủ nghĩa Việt Nam.}\\
Theo Điều 94 Hiến pháp 2013:\\
\textit{Chính phủ là cơ quan hành chính cao nhất của nước Cộng hòa Xã hội Chủ nghĩa Việt Nam, thực hiện quyền hành pháp, là cơ quan chấp hành của Quốc hội.}
\end{ans}

\begin{ques}
Ở nước ta, Chính phủ là cơ quan hành chính nhà nước cao nhất, đại diện cho ý chí, nguyện vọng của nhân dân cả nước.
\end{ques}
\begin{ans}
\textbf{Sai.} Theo Điều 69 Hiến pháp 2013:\\
\textit{Quốc hội là cơ quan đại biểu cao nhất của Nhân dân, là cơ quan quyền lực nhà nước cao nhất của nước Cộng hòa Xã hội Chủ nghĩa Việt Nam.}\\
\textit{Quốc hội thực hiện quyền lập hiến, quyền lập pháp, quyết định các vấn đề quan trọng của đất nước và giám sát tối cao đối với hoạt động của Nhà nước.}\\
Theo Điều 94 Hiến pháp 2013:\\
\textit{Chính phủ là cơ quan hành chính cao nhất của nước Cộng hòa Xã hội Chủ nghĩa Việt Nam, thực hiện quyền hành pháp, là cơ quan chấp hành của Quốc hội.}\\
\textit{Chính phủ chịu trách nhiệm trước Quốc hội và báo cáo công tác trước Quốc hội, Ủy ban thường vụ Quốc hội, Chủ tịch nước.}
\end{ans}

\begin{ques}
Ở nước ta, người đứng đầu Chính phủ là người có quyền lực nhà nước cao nhất.
\end{ques}
\begin{ans}
\textbf{Sai.} Theo Điều 69 Hiến pháp 2013:\\
\textit{Quốc hội là cơ quan đại biểu cao nhất của Nhân dân, là cơ quan quyền lực nhà nước cao nhất của nước Cộng hòa Xã hội Chủ nghĩa Việt Nam.}\\
\textit{Quốc hội thực hiện quyền lập hiến, quyền lập pháp, quyết định các vấn đề quan trọng của đất nước và giám sát tối cao đối với hoạt động của Nhà nước.}\\
Theo Điều 94 Hiến pháp 2013:\\
\textit{Chính phủ là cơ quan hành chính cao nhất của nước Cộng hòa Xã hội Chủ nghĩa Việt Nam, thực hiện quyền hành pháp, là cơ quan chấp hành của Quốc hội.}\\
\textit{Chính phủ chịu trách nhiệm trước Quốc hội và báo cáo công tác trước Quốc hội, Ủy ban thường vụ Quốc hội, Chủ tịch nước.}
\end{ans}

\begin{ques}
Ở nước ta, Chủ tịch Quốc hội là người có quyền lực nhà nước cao nhất.
\end{ques}
\begin{ans}
\textbf{Sai.} Theo Điều 69 Hiến pháp 2013:\\
\textit{Quốc hội là cơ quan đại biểu cao nhất của Nhân dân, là cơ quan quyền lực nhà nước cao nhất của nước Cộng hòa Xã hội Chủ nghĩa Việt Nam.}\\
\textit{Quốc hội thực hiện quyền lập hiến, quyền lập pháp, quyết định các vấn đề quan trọng của đất nước và giám sát tối cao đối với hoạt động của Nhà nước.}
\end{ans}

\begin{ques}
Ở nước ta, Chính phủ là cơ quan hành chính nhà nước cao nhất.
\end{ques}
\begin{ans}
\textbf{Đúng.} Theo Điều 94 Hiến pháp 2013:\\
\textit{Chính phủ là cơ quan hành chính cao nhất của nước Cộng hòa Xã hội Chủ nghĩa Việt Nam, thực hiện quyền hành pháp, là cơ quan chấp hành của Quốc hội.}\\
\textit{Chính phủ chịu trách nhiệm trước Quốc hội và báo cáo công tác trước Quốc hội, Ủy ban thường vụ Quốc hội, Chủ tịch nước.}
\end{ans}

\begin{ques}
Ở nước ta, các thành viên Chính phủ do Quốc hội bầu, miễn nhiệm, bãi nhiệm.
\end{ques}
\begin{ans}
\textbf{Sai.} Theo Điều 28 Luật Tổ chức Chính phủ năm 2015, \textbf{Thủ tướng Chính phủ trình Quốc hội phê chuẩn} đề nghị bổ nhiệm, miễn nhiệm, cách chức Phó Thủ tướng Chính phủ, Bộ trưởng và thành viên khác của Chính phủ; trong thời gian Quốc hội không họp, trình Chủ tịch nước quyết định tạm đình chỉ công tác của Phó Thủ tướng Chính phủ, Bộ trưởng và thành viên khác của Chính phủ.\\
Sau đó, Điều 88 Hiến pháp 2013 quy định, \textbf{Chủ tịch nước} căn cứ vào nghị quyết của Quốc hội,\textbf{ bổ nhiệm, miễn nhiệm}, cách chức Phó Thủ tướng Chính phủ, Bộ trưởng và các thành viên khác của Chính phủ.
\end{ans}

\begin{ques}
Ở nước ta, Chủ tịch nước phải là đại biểu Quốc hội.
\end{ques}
\begin{ans}
\textbf{Đúng.} Theo Điều 87 Hiến pháp 2013:\\
\textit{Chủ tịch nước do Quốc hội bầu \textbf{trong số đại biểu Quốc hội.}}\\
\textit{Chủ tịch nước chịu trách nhiệm và báo cáo công tác trước Quốc hội.}\\
Nhiệm kỳ của Chủ tịch nước theo nhiệm kỳ của Quốc hội. Khi Quốc hội hết nhiệm kỳ, Chủ tịch nước tiếp tục làm nhiệm vụ cho đến khi Quốc hội khóa mới bầu ra Chủ tịch nước mới.
\end{ans}

\begin{ques}
Ở nước ta, các thành viên Chính phủ đều phải là đại biểu Quốc hội.
\end{ques}
\begin{ans}
\textbf{Sai.} Điều 98 Hiến pháp 2013 quy định: \textit{Thủ tướng Chính phủ do Quốc hội bầu trong số đại biểu Quốc hội.}\\
Không có quy định Phó Thủ tướng Chính phủ, Bộ trưởng, Thủ trưởng cơ quan ngang Bộ phải là đại biểu Quốc hội.
\end{ans}

\begin{ques}
Ở nước ta, Chủ tịch nước là người có quyền lực nhà nước cao nhất.
\end{ques}
\begin{ans}
\textbf{Sai.} Theo Điều 69 Hiến pháp 2013:\\
\textit{Quốc hội là cơ quan đại biểu cao nhất của Nhân dân, là cơ quan quyền lực nhà nước cao nhất của nước Cộng hòa Xã hội Chủ nghĩa Việt Nam.}\\
\textit{Quốc hội thực hiện quyền lập hiến, quyền lập pháp, quyết định các vấn đề quan trọng của đất nước và giám sát tối cao đối với hoạt động của Nhà nước.}
\end{ans}

\begin{ques}
Ở nước ta, Thủ tướng Chính phủ do Chủ tịch nước bổ nhiệm, miễm nhiệm, bãi nhiệm.
\end{ques}
\begin{ans}
\textbf{Sai.} Khoản 7 Điều 70 Hiến pháp 2013 quy định Quốc hội có quyền hạn bầu, miễn nhiệm, bãi nhiệm Chủ tịch nước, Phó Chủ tịch nước, Chủ tịch Quốc hội, Phó Chủ tịch Quốc hội, Ủy viên Ủy ban thường vụ Quốc hội, Chủ tịch Hội đồng dân tộc, Chủ nhiệm Ủy ban của Quốc hội, \textbf{Thủ tướng Chính phủ,} Chánh án Tòa án nhân dân tối cao, Viện trưởng Viện kiểm sát nhân dân tối cao, Chủ tịch Hội đồng bầu cử quốc gia, Tổng Kiểm toán nhà nước, người đứng đầu cơ quan khác do Quốc hội thành lập; phê chuẩn đề nghị bổ nhiệm, miễn nhiệm, cách chức Phó Thủ tướng Chính phủ, Bộ trưởng và thành viên khác của Chính phủ, Thẩm phán Tòa án nhân dân tối cao, phê chuẩn danh sách thành viên Hội đồng quốc phòng và an ninh, Hội đồng bầu cử quốc gia.
\end{ans}

\begin{ques}
Ở nước ta, Chánh án Tòa án nhân dân tối cao, Viện trưởng Viện kiểm sát nhân dân tối cao được Chủ tịch nước bổ nhiệm.
\end{ques}
\begin{ans}
\textbf{Sai.} Khoản 7 Điều 70 Hiến pháp 2013 quy định Quốc hội có quyền hạn bầu, miễn nhiệm, bãi nhiệm Chủ tịch nước, Phó Chủ tịch nước, Chủ tịch Quốc hội, Phó Chủ tịch Quốc hội, Ủy viên Ủy ban thường vụ Quốc hội, Chủ tịch Hội đồng dân tộc, Chủ nhiệm Ủy ban của Quốc hội, Thủ tướng Chính phủ, \textbf{Chánh án Tòa án nhân dân tối cao, Viện trưởng Viện kiểm sát nhân dân tối cao}, Chủ tịch Hội đồng bầu cử quốc gia, Tổng Kiểm toán nhà nước, người đứng đầu cơ quan khác do Quốc hội thành lập; phê chuẩn đề nghị bổ nhiệm, miễn nhiệm, cách chức Phó Thủ tướng Chính phủ, Bộ trưởng và thành viên khác của Chính phủ, Thẩm phán Tòa án nhân dân tối cao, phê chuẩn danh sách thành viên Hội đồng quốc phòng và an ninh, Hội đồng bầu cử quốc gia.
\end{ans}

\begin{ques}
Theo quy định của Hiến pháp 2013, trong cơ cấu tổ chức của Chính phủ thì chỉ duy nhất Thủ tướng Chính phủ mới được là đại biểu Quốc hội.
\end{ques}
\begin{ans}
Sai, Hiến pháp 2013 quy định Thủ tướng Chính phủ được Quốc hội bầu trong số đại biểu Quốc hội, không có quy định về việc các Phó Thủ tướng Chính phủ, Bộ trưởng, Thủ trưởng cơ quan ngang Bộ có phải là đại biểu Quốc hội hay không. Nghĩa là \textbf{Phó Thủ tướng Chính phủ, Bộ trưởng, Thủ trưởng cơ quan ngang Bộ vẫn có thể là đại biểu Quốc hội.}
\end{ans}

\begin{ques}
Ở nước ta, Viện kiểm sát nhân dân là cơ quan duy nhất có chức năng thực hành quyền công tố.
\end{ques}
\begin{ans}
\textbf{Đúng.} Điều 107 Hiến pháp 2013 quy định:\\
\textit{1. Viện kiểm sát nhân dân thực hành quyền công tố, kiểm sát hoạt động tư pháp.}\\
\textit{2. Viện kiểm sất nhân dân gồm Viện kiểm sát nhân dân tối cao và các Viện kiểm sát khác do luật định.}\\
\textit{3. Viện kiểm sát nhân dân có nhiệm vụ bảo vệ pháp luật, bảo vệ quyền con người, quyền công dân, bảo vệ chế độ xã hội chủ nghĩa, bảo vệ lợi ích của Nhà nước, quyền và lợi ích hợp pháp của tổ chức, cá nhân, góp phần bảo đảm pháp luật được chấp hành nghiêm chỉnh và thống nhất.}
\end{ans}

\begin{ques}
Ở nước ta, Viện kiểm sát nhân dân là cơ quan duy nhất có chức năng thực hành quyền công tố và xét xử các vụ án hình sự.
\end{ques}
\begin{ans}
\textbf{Sai.} Điều 102 Hiến pháp 2013 quy định:\\
\textit{1. Tòa án nhân dân là cơ quan xét xử của nước Cộng hòa Xã hội Chủ nghĩa Việt Nam, thực hiện quyền tư pháp.} \\
\textit{2. Tòa án nhân dân gồm Tòa án nhân dân tối cao và các Tòa án khác do luật định.}\\
\textit{3. Tòa án nhân dân có nhiệm vụ bảo vệ công lý, bảo vệ quyền con người, quyền công dân, bảo vệ chế độ xã hội chủ nghĩa, bảo vệ lợi ích của Nhà nước, quyền và lợi ích hợp pháp của tổ chức, cá nhân.}
\end{ans} 

\begin{ques}
Ở nước ta, Tòa án nhân dân là cơ quan duy nhất có chức năng xét xử.
\end{ques}
\begin{ans}
\textbf{Đúng.} Điều 102 Hiến pháp 2013 quy định:\\
\textit{1. Tòa án nhân dân là cơ quan xét xử của nước Cộng hòa Xã hội Chủ nghĩa Việt Nam, thực hiện quyền tư pháp.} \\
\textit{2. Tòa án nhân dân gồm Tòa án nhân dân tối cao và các Tòa án khác do luật định.}\\
\textit{3. Tòa án nhân dân có nhiệm vụ bảo vệ công lý, bảo vệ quyền con người, quyền công dân, bảo vệ chế độ xã hội chủ nghĩa, bảo vệ lợi ích của Nhà nước, quyền và lợi ích hợp pháp của tổ chức, cá nhân.}
\end{ans}

\begin{ques}
Ở nước ta, Tòa án nhân dân là cơ quan duy nhất có chức năng xét xử và thi hành bản án, quyết định do mình ban hành.
\end{ques}
\begin{ans}
\textbf{Sai.} Điều 102 Hiến pháp 2013 quy định:\\
\textit{1. Tòa án nhân dân là cơ quan xét xử của nước Cộng hòa Xã hội Chủ nghĩa Việt Nam, thực hiện quyền tư pháp.} \\
\textit{2. Tòa án nhân dân gồm Tòa án nhân dân tối cao và các Tòa án khác do luật định.}\\
\textit{3. Tòa án nhân dân có nhiệm vụ bảo vệ công lý, bảo vệ quyền con người, quyền công dân, bảo vệ chế độ xã hội chủ nghĩa, bảo vệ lợi ích của Nhà nước, quyền và lợi ích hợp pháp của tổ chức, cá nhân.}\\
\textbf{Tòa án nhân dân không có quyền thi hành bản án, quyết định do mình ban hành.}
\end{ans}

\begin{ques}
Ở nước ta, Hội đồng nhân dân do nhân dân bầu ra và chịu trách nhiệm trước nhân dân.
\end{ques}
\begin{ans}
\textbf{Đúng.} Điều 113 Hiến pháp 2013 quy định: \\
\textit{1. Hội đồng nhân dân là cơ quan quyền lực nhà nước ở địa phương, đại diện cho ý chí, nguyện vọng và quyền làm chủ cảu Nhân dân, do Nhân dân địa phương bầu ra, chịu trách nhiệm trước Nhân dân địa phương và cơ quan nhà nước cấp trên.}\\
\textit{2. Hội đồng nhân dân quyết định các vấn đề của địa phương do luật định; giám sát việc tuân theo Hiến pháp và pháp luật ở địa phương và việc thực hiện nghị quyết của Hội đồng nhân dân.}
\end{ans}

\begin{ques}
Ở nước ta, Hội đồng nhân dân là cơ quan hành chính nhà nước, do nhân dân bầu ra và chịu trách nhiệm trước nhân dân.
\end{ques}
\begin{ans}
\textbf{Sai.} Điều 113 Hiến pháp 2013 quy định: \\
\textit{1. Hội đồng nhân dân là \textbf{cơ quan quyền lực nhà nước ở địa phương}, đại diện cho ý chí, nguyện vọng và quyền làm chủ cảu Nhân dân, do Nhân dân địa phương bầu ra, chịu trách nhiệm trước Nhân dân địa phương và cơ quan nhà nước cấp trên.}\\
\textit{2. Hội đồng nhân dân quyết định các vấn đề của địa phương do luật định; giám sát việc tuân theo Hiến pháp và pháp luật ở địa phương và việc thực hiện nghị quyết của Hội đồng nhân dân.}
\end{ans}

\begin{ques}
Ở nước ta, Ủy ban nhân dân là cơ quan quyền lực nhà nước ở địa phương.
\end{ques}
\begin{ans}
\textbf{Sai.} Điều 114 Hiến pháp 2013 quy định:\\
\textit{1. Ủy ban nhân dân ở cấp chính quyền địa phương do Hội đồng nhân dân cùng cấp bầu là cơ quan chấp hành của Hội đồng nhân dân, \textbf{cơ quan hành chính nhà nước ở địa phương}, chịu trách nhiệm trước Hội đồng nhân dân và cơ quan hành chính nhà nước cấp trên.}\\
\textit{2. Ủy ban nhân dân tổ chức việc thi hành Hiến pháp và pháp luật ở địa phương; tổ chức thực hiện nghị quyết của Hội đồng nhân dân và thực hiện các nhiệm vụ do cơ quan nhà nước cấp trên giao.}
\end{ans}

\begin{ques}
Ở nước ta, Ủy ban nhân dân các cấp là cơ quan có quyền quyết định mọi vấn đề quan trọng nảy sinh tại địa phương cấp đó.
\end{ques}
\begin{ans}
\textbf{Sai.} Điều 113 Hiến pháp 2013 quy định: \\
\textit{1. Hội đồng nhân dân là \textbf{cơ quan quyền lực nhà nước ở địa phương}, đại diện cho ý chí, nguyện vọng và quyền làm chủ cảu Nhân dân, do Nhân dân địa phương bầu ra, chịu trách nhiệm trước Nhân dân địa phương và cơ quan nhà nước cấp trên.}\\
\textit{2. Hội đồng nhân dân quyết định các vấn đề của địa phương do luật định; giám sát việc tuân theo Hiến pháp và pháp luật ở địa phương và việc thực hiện nghị quyết của Hội đồng nhân dân.}\\
\textbf{Nghĩa là thẩm quyền quyết định các vấn đề quan trọng nảy sinh tại địa phương là của Hội đồng nhân dân các cấp.}
\end{ans}