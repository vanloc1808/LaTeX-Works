\chapter{Giới thiệu}
\section{Giới thiệu ngành Hóa học}
\section{Phương pháp nghiên cứu khoa học}
\chapter{Nguyên tử $-$ Nguyên tố hóa học $-$ Đồng vị}
\section{Sơ lược lịch sử Hóa học đến thế kỉ thứ XIX, khái niệm về nguyên tử, nguyên tố hóa học, đơn chất và hợp chất}
\subsection{Định luật bảo toàn khối lượng (1789)}
Antoine Lavoisier (1743 $-$ 1794):\\
\textit{Tổng khối lượng các sản phẩm thu được đúng bằng với tổng khối lượng của các tác chất ban đầu.}
\subsection{Định luật thành phần không đổi (1799)}
Joseph Proust (1754 $-$ 1826):\\
\textit{Một hợp chất có thành phần không đổi cho dù điều chế bằng cách nào.}\\
Ví dụ, khí $\mathrm{CO_2}$ có thể được điều chế bằng cách đốt cháy than hoặc cho acid phản ứng với đá vôi, nhưng tỉ lệ khối lượng giữa $\mathrm{C}$ và $\mathrm{O}$ trong $\mathrm{CO_2}$ luôn giống nhau, không phụ thuộc vào cách điều chế.\\
Định luật này \textbf{chỉ áp dụng được cho chất khí, chất lỏng có khối lượng phân tử thấp.} VỚi chất rắn, do có sự xuất hiện của các \textit{khuyết tật mạng tinh thể,} thành phần của chúng không nhất thiết không đổi.
\subsection{Định luật tỉ lệ bội (1803)}
John Dalton (1766 $-$ 1844):\\
\textit{Nếu hai nguyên tố kết hợp với nhau cho một số hợp chất ứng với cugnf một khối lượng nguyên tố này, các khối lượng nguyên tố kia tỉ lệ với nhau như những số nguyên đơn giản.}
\subsection{Thuyết nguyên tử của Dalton (1808)}
\begin{itemize}
\item Vật chất được tạo thành từ những hạt rất nhỏ không thể phân chia thành những phần nhỏ hơn, cũng không phá hủy được chúng, các hạt rất nhỏ đó là \textit{nguyên tử.}
\item Các nguyên tử của cùng một nguyên tố thì giống nhau, các nguyên tố khác nhau có nguyên tử khác nhau.
\item Nguyên tử của các nguyên tố khác nhau kết hợp với nhau theo những tỉ lệ xác định để tạo thành các hợp chất.
\item Có sự sắp xếp lại của các nguyên tử trong các chât khi phản ứng hóa học xảy ra.
\end{itemize}
\section{Hóa học hiện đại $-$ các thí nghiệm khám phá cấu tạo của nguyên tử}
\subsection{Thí nghiệm khám phá electron $-$ Thuyết nguyên tử của Thomson}
\subsubsection{Khám phá electron}
Dụng cụ góp phần khám phá cấu tạo của nguyên tửu là ống phát tia âm cực, còn được gọi là ống phát tia cathode (cathode-ray tube, CRT).\\
Kết quả thu được:
\begin{itemize}
\item Do chùm tia lệch về cực dương nên phải có điện tích âm, sau này gọi là electron.
\item Thay đổi cường độ điện trường $\mathrm{E},$ thay đổi độ lệch.
\item Thomson tính được tỉ số:
$$\frac{m}{q} = -5.6857 \cdot 10^{-9} \text{ } \left( {\mathrm{\frac{g}{C}}} \right).$$
\item Như vậy, người ta cho rằng tất cả các nguyên tử đều chứa electron.
\item Mà nguyên tử lại không tích điện, vậy phải có hạt mang điện tích dương trong nguyên tử để trung hòa điện.
\end{itemize}
\subsubsection{Mô hình nguyên tử của Thomson}
Nguyên tử như một \textbf{đám mây hình cầu tích điện dương,} các \textbf{electron mang điện âm nằm rải rác} trong đám mây.\\
Đây chỉ là kết luận từ những nhận định đơn giản để đưa ra mô hình nguyên tử, cần phải được kiểm chứng bằng thực nghiệm.
\subsubsection{Điện tích và khối lượng electron}
Thí nghiệm giọt dầu rơi của Millikan năm 1909.\\
$$\frac{m}{q} = -5.6857 \cdot 10^{-9} \text{ } \left( {\mathrm{\frac{g}{C}}} \right).$$
$$m_e = 9.11 \cdot 10^{-31} \text{ } \mathrm{kg}.$$
$$q_e = -1.6 \cdot 10^{-19} \text{ } \mathrm{C}.$$
\subsection{Hiện tượng phóng xạ tự nhiên}
\begin{itemize}
\item Tia $\alpha:$ là chùm các hạt mang điện tích $+2$ ($\mathrm{He^{2+}}$), lệch về bản cực âm khi đi qua điện trường.
\item Tia $\beta:$ là chùm các electron tốc độ cao, lệch về bản cực dương khi đi qua điện trường.
\item Tia $\gamma:$ là chùm các sóng điện từ năng lượng cao, đi thẳng khi qua điện trường.
\end{itemize}
Hiện tượng phóng xạ tự nhiên đi kèm với sự thay đổi vật chất ở cấp độ nhỏ hơn nguyên tử, tức là nguyên tử được tạo thành từ các hạt nhỏ hơn.
\subsection{Thuyết cấu tạo nguyên tử theo Rutherford (1911)}
Mô hình nguyên tử theo kiểu đám mây hình cầu tích điện dương của Thomson không hợp lý, hay nói cách khác, nguyên tử phải "rỗng". Dựa vào kết quả thí nghiệm của mình và cộng sự, năm 1911, Rutherford đã đề nghị mô hình nguyên tử mới như sau:
\begin{itemize}
\item Nguyên tử gồm hạt nhân mang điện tích dương, có kích thước rất nhỏ và nằm ở tâm nguyên tửu, phần không gian còn lại của nguyên tử là rỗng.
\item Các electron mang điện tích âm chuyển động quanh nhân và ở khoảng cách khá xa so với hạt nhân nguyên tử.
\item Các nguyên tử khác nhau có điện tích hạt nhân nguyên tử khác nhau, điện tích hạt nhân của nguyên tử bằng tổng điện tích của các electron trong nguyên tử, nhưng trái dấu, do đó \textbf{nguyên tử trung hòa điện.}
\end{itemize}
\subsection{Sự khám phá proton và neutron trong nhân nguyên tử}
Nghiên cứu tia X từ các nguyên tử khác nhau cho thấy rằng điện tích hạt nhân nguyên tử khác nhau từng đơn vị một, đơn vị đó bằng điện tích electron nhưng mang điện tích dương.\\
Năm 1918, Rutherford khám phá ra proton.\\
Năm 1932, Chadwick khám phá ra neutron.\\
Đến thập niên 1930, các nhà khoa học đã biết nhân nguyên tử có hai loại hạt chính là proton và neutron.
\subsection{Cấu tạo và các đặc trưng căn bản của nguyên tử}
Nguyên tử được cấu tạo từ: proton mang điện dương, electron mang điện âm và neutron không mang điện.\\
Trong ba loại hạt kể trên, electron được xem là một loại \textit{hạt cơ bản} ($1$ trong $6$ loại hạt \textit{lepton}), tức là không thể phân chia thành các hạt nhỏ hơn. Tuy nhiên, Vật lý hiện đại không xem proton và neutron là các hạt cơ bản.\\
Khối lượng nguyên tử gần bằng tổng khối lượng các hạt proton và neutron trong nhân nguyên tử.\\
\centerline{Số khối của nguyên tử ($A$) $=$ số hạt proton ($Z$) $+$ số hạt neutron ($N$).}
\section{Nguyên tố hóa học $-$ đồng vị $-$ nguyên tử lượng}
\subsection{Nguyên tố hóa học}
\begin{itemize}
\item Tất cả các nguyên tử có cùng điện tích hạt nhân, tức là cùng số proton trong nhân và số electron ở lớp vỏ, đều có tính chất hóa học giống nhau.
\item Những nguyên tử như vậy tạo thành một nguyên tố hóa học (gọi vắn tắt là nguyên tố).
\item Mỗi nguyên tố được đặc trưng bởi số hiệu nguyên tử của nguyên tố đó, chính là điện tích hạt nhân ($Z$) của các nguyên tử tạo nên nguyên tố hóa học.
\end{itemize}
\subsection{Đồng vị (isotopes)}
\begin{itemize}
\item Các nguyên tử có \textbf{cùng số proton} nhưng \textbf{khác số neutron.}
\item Do có cùng số proton nên chúng cùng thuộc một nguyên tố.
\item Tập hợp các nguyên tử có cùng khối lượng của một nguyên tố được gọi là một \textbf{đồng vị} của nguyên tố đó.
\item Thuật ngữ "đồng vị" có nghĩa là "cùng một vị trí" trong bảng phân loại tuần hoàn.
\item Hầu hết các nguyên tố hóa học đều có nhiều đồng vị tự nhiên khác nhau.
\end{itemize}
\subsection{Xác định khối lượng nguyên tử $-$ phố khối lượng}
\begin{itemize}
\item Khối lượng chính xác của nguyên tử nhỏ hơn tổng khối lượng của các proton, neutron và electron.
\item Sự khác biệt này được gọi là \textbf{độ hụt khối lượng} hay \textbf{độ hụt khối.}
\item Giải thích cho hiện tượng này: do khi các proton và neutron kết hợp tạo thành hạt nhân nguyên tử, một phần khối lượng của các hạt được chuyển thành năng lượng liên kết hạt nhân.
\item Vì vậy, khối lượng thực tế của nguyên tử phải xác định từ thực nghiệm.
\end{itemize}
\subsection{Nguyên tử lượng của nguyên tố}
Do mỗi nguyên tố trong tự nhiên thường là tập hợp của nhiều đồng vị nên khối lượng nguyên tử dùng để cân đong trong thực tế thường là khối lượng nguyên tử trung bình có tính đến thành phần của các đồng vị, được gọi là \textbf{nguyên tử lượng trung bình}, hay \textbf{nguyên tử lượng.}
\subsection{Mole, khối lượng mole, số Avogadro}
\subsubsection{Mole}
Là số hạt vi mô bằng với số lượng nguyên tử carbon có trong $12.00 \text{ } \mathrm{g}$ đồng vị $\mathrm{^{12}C}.$\\
Trong $12.00 \text{ } \mathrm{g}$ đồng vị $\mathrm{^{12}C}$ có $N_A = 6.022137 \cdot 10^{23}$ nguyên tử carbon.
\subsubsection{Số Avogadro}
$N_A$ là số Avogadro.
\subsubsection{Khối lượng mol}
Là khối lượng (tính bằng gram) của $N_A$ tiểu phân của một chất.