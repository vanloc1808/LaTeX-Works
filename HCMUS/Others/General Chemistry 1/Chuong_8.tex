\chapter{Lực liên kết liên phân tử $-$ Trạng thái lỏng $-$ Trạng thái rắn}
\section{Lực liên kết liên phân tử}
Do có sự có mặt của lực liên kết liên phân tử làm cho khí không ở trạng thái lý tưởng.\\
Khi lực tương tác này đủ mạnh, chất khí sẽ ngưng tụ tạo thành chất lỏng hoặc chất rắn.\\
Cũng chính lực tương tác này là nguyên nhân gây ra các tính chất của chất lỏng: nhiệt độ sôi, áp suất hơi, độ nhớt và nhiệt hóa hơi.\\
Lực tương tác này yếu hơn lực liên kết hóa học (cộng hóa trị, ion).\\
Nguồn gốc của lực này là do sự phân bố không đồng đều của electron trong các phân tử.
\subsection{Lực tương tác van der Waals giữa các phân tử}
Giữa các tiểu phân (phân tử, ion, gốc tự do, ...) dù đã bão hòa hóa trị hay chưa đều luôn luôn tồn tại một tương tác điện yếu được gọi là liên kết van der Waals hay lực liên kết liên phân tử.
Electron trong các nguyên tử và phân tử không phân cực vẫn luôn luôn chuyển động.\\
Tương tác này gọi là tương tác phân tán (dispersion force), tương tác khuếch tán hay tương tác London.\\
Tương tác khuếch tán (tương tác London):
\begin{itemize}
\item Xuất hiện giữa các phân tử \textbf{phân cực hay không phân cực.} 
\item Các lưỡng cực luôn luôn xuất hiện, chuyển đổi, biến mất và tạo ra tương tác cảm ứng, nên các phân tử hút lẫn nhau.
\end{itemize}
Tương tác cảm ứng (tạm thời):
\begin{itemize}
\item Xuất hiện giữa các phân tử \textbf{phân cực hay không phân cực.}
\item Các phân tử phân cực làm \textbf{phân cực tạm thời} các phân tử không phân cực, xuất hiện lực hút lẫn nhau giữa hai loại phân tử này.
\end{itemize}
Tương tác khuếch tán và tương tác cảm ứng thường đi với nhau.\\
Tương tác định hướng (tương tác lưỡng cực $-$ lưỡng cực) (thường trực):
\begin{itemize}
\item Xuất hiện giữa các \textbf{phân tử phân cực.}
\item Các phân tử này sắp xếp theo một hướng xác định.
\end{itemize}
Lực phân tán luôn luôn hiện diện giữa các phân tử cho dù phân tử có phân cực hay không.\\
Lực phân tán giữa các nguyên tử/phân tử tăng khi khả năng xuất hiện lưỡng cực (còn gọi là khả năng phân cực) của phân tử tăng.\\
Khả năng phân cực của phân tử tỉ lệ thuận với kích thước của phân tử và nói một cách không chính xác thì nó thường tăng theo khối lượng phân tử. Đây là yếu tố quan trọng nhất.\\
Tương tác khuếch tán (tương tác London) là quan trọng nhất.
\subsubsection{Tổng quát}
\begin{itemize}
\item Các phân tử có kích thước và khối lượng xấp xỉ nhau: phân tử càng có moment lưỡng cực lớn, nhiệt độ nóng chảy, nhiệt độ sôi, nhiệt nóng chảy, nhiệt bay hơi của hóa chất càng cao.
\item Các phân tử khác biệt đáng kể về kích thước và khối lượng: tương tác London là yếu tố quyết định nhiệt độ nóng chảy, nhiệt độ sôi, nhiệt nóng chảy, nhiệt bay hơi của hóa chất.
\end{itemize}
Tính chất liên kết van der Waals:
\begin{itemize}
\item Có bản chất bất định hướng và bất bão hòa.
\item Là tương tác yếu có năng lượng liên kết $< 40 \mathrm{kJ/mol}.$ Lực liên kết giảm rất nhanh khi khoảng cách giữa các phân tử tăng lên.
\end{itemize}
Cường độ liên kết van der Waals:
\begin{itemize}
\item Tương tác định hướng:
\begin{itemize}
\item Vai trò quyết định đối với các phân tử phân cực.
\item Càng mạnh khi phân tử phân cực càng mạnh.
\end{itemize}
\item Tương tác cảm ứng: thường yếu nên bỏ qua
\item Tương tác khuếch tán:
\begin{itemize}
\item Quan trọng đối với mọi phân tử (bất kể phân cực hay không phân cực).
\item Càng mạnh khi kích thước phân tử càng lớn.
\end{itemize}
\end{itemize}
\subsection{Liên kết hydrogen}
$$\mathrm{{A^{\delta  - }} \leftarrow {H^{\delta  + }} \bullet  \bullet  \bullet {B^{\delta  - }}}$$
$A$ là nguyên tố có độ âm điện lớn như $\mathbb{F}, \mathrm{O}, \mathrm{N}$ hay nguyên tử gắn với nguyên tử có độ âm điện lớn như $\mathrm{F_3C ^-}, \mathrm{NC^-}, ...$\\
$B$ là nguyên tố có độ âm điện lớn như $\mathbb{F}, \mathrm{O}, \mathrm{N}$ và $\mathrm{Cl}$ hay các nhóm có liên kết $\pi$ như nối đôi, nhân thơm, ...\\
$\mathrm{A}$ có độ âm điện lớn nên có khả năng kéo điện tử về phía mình làm $\mathrm{H}$ thiếu điện tử và phân cực dương.\\
$\mathrm{H^{\delta  + }}$ tương tác tĩnh điện với $\mathrm{B}$ phân cực âm.\\
Liên kết này là trường hợp đặc biệt của liên kết van der Waals, gọi là liên kết hydrogen, ký hiệu là $\bullet  \bullet  \bullet$\\
Đây là liên kết vật lý yếu, không phải là liên kết hóa học.\\
Độ dài liên kết này dài hơn liên kết hóa học.\\
Liên kết hydrogen làm nhiệt độ sôi tăng lên một cách "bất thường".\\
Liên kết hydrogen được thực hiện giữa các phân tử khiến cho các phân tử này liên kết lại với nhau.
\section{Một số tính chất của chất lỏng}
\subsection{Sức căng bề mặt và sự thấm ướt bề mặt của chất lỏng}
\subsubsection{Sức căng bề mặt}
Các phân tử trong chất lỏng luôn luôn hút nhau bằng các lực liên phân tử.\\
Các phân tử trên bề mặt chất lỏng được các phân tử chung quanh và trong lòng chất lỏng hút, tạo thành lớp phân tử bề mặt liên kết chặt chẽ mà các vật bên ngoài phải bẻ gãy các liên kết này mới có thể rơi vào lòng chất lỏng.\\
Lực liên kết giữa các phân tử ở bề mặt chất lỏng được gọi là \textbf{sức căng bề mặt} của chất lỏng.\\
Nói cách khác, sức căng bề mặt là năng lượng cần thiết để làm tăng diện tích bề mặt của chất lỏng.
\subsubsection{Sự thấm ướt bề mặt}
\subsection{Độ nhớt của chất lỏng}
Độ nhớt của chất lỏng là đại lượng đặc trưng cho tính \textbf{chống lại sự cháy,} cũng chính là ma sát nội của dòng chảy của chất lỏng.\\
Lực tương tác liên phân tử của chất lỏng càng mạnh, độ nhớt của chất lỏng càng cao.
\subsection{Sự bay hơi và áp suất hơi của chất lỏng}
Những yếu tố ảnh hưởng đến sự bay hơi của chất lỏng:
\begin{itemize}
\item Nhiệt độ: nhiệt độ càng cao thì càng nhiều phân tử có động năng đủ lớn để thắng lực tương tác liên phân tử, chất lỏng bay hơi càng mạnh.
\item Diện tích bề mặt của chất lỏng: diện tích càng rộng thì càng nhiều phân tử ở bề mặt chất lỏng, khả năng thoát ra ngoài của các phân tử càng cao, làm chất lỏng bay hơi càng nhanh.
\item Lực tương tác giữa các phân tử của chất lỏng: lực này càng nhỏ thì càng nhiều phân tử ở bề mặt chất lỏng dễ thoát ra, nên chất lỏng càng dễ bay hơi.
\end{itemize}
Quá trình bay hơi và ngưng tụ của chất lỏng trong bình kín là quá trình cân bằng.\\
Áp suất hơi:
\begin{itemize}
\item Các chất lỏng có áp suất hơi bão hòa ở nhiệt độ phòng cao được gọi là các chất lỏng dễ bay hơi.
\item Áp suất hơi của chất lỏng chỉ phụ thuộc vào bản chất của chất lỏng và nhiệt độ.
\end{itemize}
Ở cùng một nhiệt độ, nếu lực tương tác liên phân tử của chất lỏng càng mạnh thì áp suất hơi của nó càng thấp.
Phương trình Clausius $-$ Clayperon:
$$\ln \left( {\frac{P_2}{P_1}} \right) = - \frac{\Delta H_{\text{hóa hơi}}}{R} \left( {\frac{1}{T_2} - \frac{1}{T_1}} \right).$$
\subsection{Nhiệt hóa hơi và nhiệt ngưng tụ của chất lỏng}
Phân biệt thuật ngữ: nhiệt và nhiệt độ.\\
Nhiệt (heat) để chỉ năng lượng khác với nhiệt độ.\\
Nhiệt hóa hơi $\Delta H_{\text{hóa hơi}}$ chính là năng lượng ứng với quá trình chuyển $1$ mole chất lỏng thành hơi ở nhiệt độ không đổi.
$$A \text{ (lỏng) } \to A \text{ (hơi)}$$
$$\Delta H_{\text{hóa hơi}} = H_{\text{hơi}} - H_{\text{lỏng}} > 0$$
Lực liên phân tử giữa các phân tử trong chất lỏng cành mạnh thì nhiệt hóa hơi của chất lỏng càng cao.\\
Nhiệt ngưng tụ $\Delta H_{\text{ngưng tụ}}$ chính là năng lượng ứng với quá trình chuyển $1$ mole chất ở thể hơi thành lỏng ở nhiệt độ không đổi.
$$A \text{ (hơi)} \to A \text{ (lỏng) }$$
$$\Delta H_{\text{ngưng tụ}} = H_{\text{lỏng}} - H_{\text{hơi}} = - \Delta H_{\text{hóa hơi}} < 0$$
\subsection{Sự sôi và nhiệt độ sôi của chất lỏng}
Chất lỏng bắt đầu sôi khi áp suất hơi của nó bằng áp suất bên ngoài. Nhiệt độ khi $P_{\text{hơi}} = P_{\text{ngoài}}$ gọi là \textbf{nhiệt độ sôi.}\\
Khi áp suất bên ngoài càng cao thì nhiệt độ sôi càng cao.\\
Trong quá trình sôi, năng lượng cung cấp chỉ để chuyển chất lỏng thành hơi, nhiệt độ chất lỏng không đổi cho đến khi tất cả các phân tử chất lỏng chuyển thành hơi.
\subsection{Điểm tới hạn của chất lỏng}
Nhiệt độ $T_c$ và áp suất $P_c$ mà ở đó hai pha lỏng và  hơi bắt đầu trở thành không thể phân biệt được gọi là \textbf{nhiệt độ tới hạn} và\textbf{ áp suất tới hạn} của chất lỏng (chữ c ở đây là viết tắt của \textit{critical}). Cặp giá trị này được gọi là \textbf{điểm tới hạn} của chất lỏng.
\section{Sự giới hạn và sự thăng hoa của chất rắn}
\subsection{Nhiệt nóng chảy và nhiệt độ nóng chảy của chất rắn}
Nhiệt nóng chảy là nhiệt lượng (năng lượng) cần thiết để làm nóng chảy một mole chất rắn ở nhiệt độ nóng chảy.\\
Tương tác giữa các tiểu phân bên trong chất rắn càng mạnh thì nhiệt độ nóng chảy càng cao.
\subsection{Sự thăng hoa của chất rắn}
Quá trình chuyển trực tiếp chất rắn thành hơi được gọi là thăng hoa.\\
Quá trình ngược lại, hơi chuyển thành rắn, được gọi là quá trình ngưng tụ.
$$\Delta H_{\text{thăng hoa}} = \Delta H_{\text{nóng chảy}} + \Delta H_{\text{hóa hơi}}.$$
Phương trình Clausius $-$ Clayperon có thể dùng cho cả quá trình bay hơi của chất lỏng và thăng hoa của chất rắn.
\section{Giản đồ pha biểu diễn điều kiện tồn tại và chuyển pha giữa các pha rắn $-$ lỏng $-$ hơi}
Giản đồ pha ở trong học phần Hóa đại cương 1 là giản đồ chuyển đổi giữa ba trạng thái: rắn, lỏng, khí.\\
Các quá trình chuyển đổi trạng thái là quá trình cân bằng.\\
Giản đồ pha được dùng để biểu diễn mối tương quan giữa các pha của vật chất trong khoảng nhiệt độ và áp suất chúng có thể tồn tại.\\
\textbf{Pha} là một phần giới hạn của vật chất mà trong đó tính chất của nó hoàn toàn đồng nhất, không có bề mặt phân cách trong một pha.\\
\textbf{Trạng thái khí:} luôn chỉ có một pha. Do các tính chất khuếch tán và đồng nhất.\\
\textbf{Trạng thái rắn:} các chất khác nhau có tính chất giống nhau là một pha, nếu như cùng một chất mà có dạng thù hình khác nhau được xem là các pha khác nhau.\\
\textbf{Trạng thái lỏng:} các chất lỏng khác nhau nhưng lại hòa tan vào nhau tạo thành dung dịch đồng nhất được xem là cùng một pha. Ngược lại, chúng được xem là các pha khác nhau.
\subsection{Giản đồ pha của $\mathrm{CO_2}$}
\subsection{Sự hóa lỏng chất khí và trạng thái tới hạn}
Điểm mà tại đó tồn tại đồng thời cả ba pha rắn, lỏng và khí được gọi là \textbf{điểm ba.}\\
Điểm mà tại đó ứng với nhiệt độ cao nhất có thể hóa lỏng chất khí bằng cách nén ở áp suất cao, không phân biệt pha lỏng và pha khí được gọi là \textbf{điểm tới hạn.}
\subsection{Giản đồ pha của nước}
\section{Phân loại chất rắn tinh thể theo đặc điểm liên kết trong tinh thể}
Trạng thái rấn của vật chất còn được gọi là trạng thái liên kết.\\
Ở trạng thái rắn, các phần tử tạo nên chất rắn sắp xếp khá trật tự.\\
Nếu trật tự của các phần tử đó lặp lại trong khoảng không gian lớn, chất rắn đó được gọi là tinh thể (crystal).\\
Nếu trật tự trên chỉ giới hạn trong khoảng không gian hẹp, ta có chất rắn vô định hình.\\
Trạng thái rắn tinh thể luôn bền hơn trạng thái vô định hình.\\
Phân loại tinh thể dựa trên lực tương tác giữa các tiểu phân tạo nên mạng tinh thể: mạng tinh thể kim loại, mạng tinh thể ion, mạng tinh thể phân tử và mạng tinh thể nguyên tử.
\begin{table}[H]
\begin{tabular}{ | p{\dimexpr 0.2\linewidth-2\tabcolsep} |
                   p{\dimexpr 0.2\linewidth-2\tabcolsep} |
                   p{\dimexpr 0.2\linewidth-2\tabcolsep} |
                   p{\dimexpr 0.2\linewidth-2\tabcolsep} | 
                   p{\dimexpr 0.2\linewidth-2\tabcolsep} |} \hline
\hline 
Tính chất & Mạng tinh thể kim loại & Mạng tinh thể ion & Mạng tinh thể phân tử & Mạng tinh thể nguyên tử \\ 
\hline 
Vị trí các nút mạng & Các nguyên tử kim loại & Các ion trái dấu & Phân tử & Nguyên tử \\ 
\hline 
Loại liên kết & Liên kết kim loại & Liên kết ion & Liên kết van der Waals (hay liên kết hydrogen) & Liên kết cộng hóa trị \\ 
\hline 
Độ mạnh yếu liên kết & Liên kết mạnh & Liên kết mạnh & Liên kết yếu & Liên kết mạnh \\ 
\hline 
Để kim loại nóng chảy & Cung cấp năng lượng cắt đứt những liên kết này & Cung cấp năng lượng cắt đứt những liên kết này & Cung cấp năng lượng cắt đứt những liên kết này & Cung cấp năng lượng cắt đứt những liên kết này \\ 
\hline 
Nhiệt độ nóng chảy & Cao & Cao & Thấp & Cao \\ 
\hline 
Ví dụ & $\mathrm{Cu}, \mathrm{Ag}, \mathrm{Au}$ & $\mathrm{NaCl}, \mathrm{CuSO_4}$ & Kim cương, $\mathrm{Si}, \mathrm{SiO_2}$ & $\mathrm{H_2}, \mathrm{O_2}, \mathrm{S_8}, \mathrm{I_2},$ benzene \\ 
\hline 
\end{tabular}
\caption{Tóm tắt tính chất của bốn kiểu mạng tinh thể}
\end{table} 
\section{Một số kiểu cấu trúc tinh thể}
\subsection{Sự xếp chặt các nguyên tử kim loại trong mạng tinh thể kim loại}
Cấu trúc tinh thể của các kim loại, nguyên tử kim loại thường được xem như các quả cầu cứng có bán kính $r.$\\
Các nguyên tử kim loại trong mạng tinh thể là xếp chặt các quả cầu với nhau, còn gọi là sắp xếp đặc khít.\\
Có hai kiểu sắp xếp đặc khít chính:
\begin{itemize}
\item Xếp chặt lục phương (hcp).
\item Xếp chặt lập phương (ccp).
\end{itemize}
\subsubsection{Xếp chặt lục phương (hexagonal close packed - hcp)}
\subsubsection{Xếp chặt lập phương (cubic close packed - ccp)}
\subsubsection{Lỗ trống tứ diện và lỗ trống bát diện}
\subsection{Các cấu trúc xếp chặt khác trong mạng tinh thể kim loại}
\begin{table}[H]
\begin{tabular}{p{\dimexpr 0.25\linewidth-2\tabcolsep} |
                   p{\dimexpr 0.25\linewidth-2\tabcolsep} |
                   p{\dimexpr 0.25\linewidth-2\tabcolsep} |
                   p{\dimexpr 0.25\linewidth-2\tabcolsep} | }
\hline 
 & Lập phương đơn giản (sc) & Lập phương chính tâm (bcc) & Lập phương tâm diện (fcc) \\ 
\hline 
Số nguyên tử trong ô mạng cơ sở & $1$ & $2$ & $4$ \\ 
\hline 
Số phối trí & $6$ & $8$ & $12$ \\ 
\hline 
Hướng xếp chặt của nguyên tử & Cạnh của ô mạng cơ sở & Đường chéo tâm của ô mạng cơ sở & Đường chéo mặt của ô mạng cơ sở \\ 
\hline 
Phần nguyên tử chiếm không gian & $52 \%$ & $68 \%$ & $74 \%$ \\ 
\hline 
\end{tabular} 
\caption{Ba kiểu sắp xếp căn bản của các nguyên tử kim loại trong ô mạng lập phương}
\end{table}
\subsection{Một số cấu trúc của mạng tinh thể phân tử}
Vì một số phân tử không có đối xứng cầu, các phân tử có thể tạo thành các ô mạng cơ sở bị bóp méo so với ô mạng lập phương, trong đó các thông số góc của mạng tinh thể lệch khỏi $90 ^\circ,$ và/hoặc chiều dài cạnh không bằng nhau.