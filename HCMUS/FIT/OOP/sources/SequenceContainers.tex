\section{Sequence containers}
\label{seqcon}
Như đã nói trong phần \ref{contain}, sequence containers lưu trữ các phần tử theo thứ tự giống với thứ tự chúng được đưa vào container. Sequence containers gồm: \lstinline{array}, \lstinline{vector}, \lstinline{list}, \lstinline{forward_list}, và \lstinline{deque}. 
\subsection{array}
Cấu trúc \lstinline{array} cung cấp container chứa một mảng tĩnh với số phần tử cố định. \\
Để sử dụng cấu trúc \lstinline{array}, ta cần chỉ thị \lstinline{#include <array>}.\\
Cấu trúc \lstinline{array} được khai báo trong header \lstinline{<array>} như sau:\cite{array}
\begin{lstlisting}
template<class T, std::size_t N>
struct array;
\end{lstlisting}
\subsubsection{Các hàm thành viên}
Implicitly-defined member functions: các constructor, destructor, toán tử gán.\\
Các phương thức truy cập phần tử:
\begin{itemize}
    \item \lstinline{at(size_type pos)} : Trả về tham chiếu tới phần tử đối tượng thứ \lstinline{pos} (bắt đầu từ 0). Hàm sẽ ném ngoại lệ (throw exception) \lstinline{std::out_of_range } nếu như pos vượt quá phạm vi của mảng (\lstinline{pos >= N}).
    \item \lstinline{operator[](size_type pos)} : Tương tự như trên nhưng không ném ngoại lệ.
    \item \lstinline{front()}: Trả về tham chiếu tới phần tử đầu tiên.
    \item \lstinline{back()}: Trả về tham chiếu tới phần tử cuối cùng.
    \item \lstinline{data()}: Trả về constant pointer (\lstinline{T*}) của tới mảng nằm trong \lstinline{array}.
\end{itemize}
Các iterators như trong phần hình \ref{popite}.\\
Các phương thức kiểm tra số lượng (capacity):
\begin{itemize}
    \item \lstinline{empty()}: trả về \lstinline{true} nếu mảng trống và \lstinline{false} nếu ngược lại.
    \item \lstinline{size()}: trả về số lượng phần tử hiện tại của mảng.
    \item \lstinline{max_size()}: trả về số lượng phần tử tối đa có thể có của mảng.
\end{itemize}
Một số phương thức khác:
\begin{itemize}
    \item \lstinline{fill(const T& value)}: gán toàn bộ giá trị của mảng bằng \lstinline{value}.
    \item \lstinline{swap(array& other)}: thay đổi nội dung array này với array khác.
\end{itemize}
\textbf{Lưu ý:} Phải tạo mảng có kích thước lớn hơn 0, nếu không sẽ gây Undefined Behaviour  khi truy cập phần tử.\\
Độ phức tạp của các hàm thành viên có thể được tìm thấy tại \cite{array}.
\subsubsection{Ví dụ}
\begin{lstlisting}
#include <array>

//default initialization (non-local = static storage)
std::array<int, 3> global; //{0, 0, 0}

int main() {
    //default initialization (local = automatic storage)
    std::array<int, 3> first; //{?, ?, ?}
    
    //initializer-list initializations
    std::array<int, 3> second = {10, 20}; //{10, 20, 0}
    std::array<int, 3> third = {1, 2, 3}; //{1, 2, 3}
    
    //copy initialization
    std::array<int, 3> fourth(third); //copy
    
    return 0;
}
\end{lstlisting}
\subsection{forward$\_$list}
Lớp \lstinline{forward_list} triển khai kiểu dữ liệu danh sách liên kết đơn, hỗ trợ các thao tác chèn, xóa nhanh của phần tử ở bất cứ đâu trong container, tuân theo quy tắc RAII để đảm bảo chương trình không bị memory leak.Tuy nhiên không thể truy cập ngẫu nhiên trên container này được.\\
Để sử dụng lớp \lstinline{forward_list}, ta cần chỉ thị \lstinline{#include <forward_list>}.\\
Lớp \lstinline{forward_list} được khai báo trong header \lstinline{<forward_list>} như sau: \cite{forlist}
\begin{lstlisting}
template<class T, class Allocator = std::allocator<T>> 
class forward_list;
namespace pmr {
    template <class T>
    using forward_list = std::forward_list<T, std::pmr::polymorphic_allocator<T>>;
}
\end{lstlisting}
\subsubsection{Các hàm thành viên}
Các constructor, destructor, toán tử gán.\\
Phương thức truy cập phần tử: \lstinline{front()}: trả về tham chiếu giá trị của HEAD trong DSLK đơn. \\
Các iterators như trong phần hình \ref{popite}.\\
Các phương thức kiểm tra số lượng (capacity):
\begin{itemize}
    \item \lstinline{empty()}: trả về \lstinline{true} nếu mảng trống và \lstinline{false} nếu ngược lại.
    \item \lstinline{max_size()}: trả về số lượng phần tử tối đa có thể có của danh sách.
\end{itemize}
Các phương thức thao tác trên phần tử:
\begin{itemize} 
    \item \lstinline{insert_after(pos, (n), value)}: chèn \lstinline{n} giá trị \lstinline{value} vào sau vị trí iterator \lstinline{pos}, \lstinline{value} ở đây có thể là một giá trị hoặc là một\lstinline{initializer_list}, nếu không có tham số \lstinline{n} thì mặc định chèn 1 giá trị. Trả về iterator trỏ tới phần tử cuối cùng được thêm vào.
    \item \lstinline{insert_after(pos, beginIt, endIt)}: chèn giá trị copy từ iterator \lstinline{beginIt} đến \lstinline{endIt} vào sau vị trí iterator \lstinline{pos}. Trả về iterator trỏ tới phần tử cuối cùng được thêm vào.
    \item \lstinline{erase_after(pos)}: xoá phần tử phía sau vị trí iterator \lstinline{pos}, trả về iterator trỏ tới phần tử sau phần tử đã xoá.
    \item \lstinline{erase_after(begin_pos, end_pos)}: xoá phần tử từ vị trí iterator \lstinline{begin_pos} đến trước \lstinline{end_pos}, trả về iterator trỏ tới \lstinline{end_pos}.
    \item \lstinline{push_front(val)}: tương đương việc \lstinline{addHead} của DSLK đơn, đưa giá trị \lstinline{val} lên đầu.
    \item \lstinline{pop_front()}: tương đương việc deleteHead của DSLK đơn, xoá giá trị cuối cùng của danh sách. Nếu danh sách đang rỗng thì sẽ gây Undefined Behavior.
    \item \lstinline{swap(f)}: đổi chỗ các phần tử DSLK đơn hiện tại với DSLK đơn\lstinline{ f} (cùng kiểu).
    \item \lstinline{remove(val)}: xoá toàn bộ các Node có giá trị là val trong DSLK đơn.
    \item \lstinline{remove_if(f)}: xoá toàn bộ các Node có giá trị thoả điều kiện của function \lstinline{f}.
\end{itemize}
Các phương thức với DSLK đơn:
\begin{itemize}
    \item \lstinline{merge(l, comparator)}: chèn DSLK \lstinline{l} vào DSLK hiện tại, 2 danh sách này phải được sắp xếp từ trước, sau đó được chèn theo điều kiện của \lstinline{comparator} (mặc định là “<” cho tăng dần).
    \item \lstinline{reverse()}: lật ngược toàn bộ DSLK.
    \item \lstinline{unique()}: xoá các phần tử trùng nhau trong DSLK (giữ lại 1 phần tử).
    \item \lstinline{sort()}: sắp xếp DSLK tăng dần.
    \item \lstinline{clear()}: xóa toàn bộ các phần tử trong DSLK đơn.
\end{itemize}
Độ phức tạp của các hàm thành viên có thể được tìm thấy tại \cit{forlist}.
\subsubsection{Ví dụ}
\begin{lstlisting}
#include <iostream>
#include <forward_list>

int main() {
    std::forward_list<int> first; //default: empty
    std::forward_list<int> second(3, 77); //fill: 3 seventy-sevens
    std::forward_list<int> third(second.begin(), second.end()); //range initialization
    std::forward_list<int> fourth(third); //copy constructor
    std::forward_list<int> fifth(std::move(fourth)); //move ctor. (fourth wasted)
    std::forward_list<int> sixth = {3, 52, 25, 90}; //initializer_list constructor
    
    std::forward_list<int> mylist;
    std::forward_list<int>::iterator it; //it -> ()
    
    it = mylist.insert_after(mylist.before_begin(), 10); //it -> (10)
    it = mylist.insert_after(it, 2, 20); //10 20 (20)
    std::array<int, 3> myarray = {11, 22, 33};
    it = mylist.insert_after(it, myarray.begin(), myarray.end()); //10 20 30 11 22 (33)
    it = mylist.begin(); //(10) 20 30 11 22 33
    it = mylist.erase_after(it); //10 (30) 11 22 33
    it = mylist.erase_after(it, mylist.end()); //10 30 ()
    
    return 0;
}
\end{lstlisting}

\subsection{list}
Lớp \lstinline{list}, triển khai kiểu dữ liệu danh sách liên kết đôi, có những ưu điểm so với \lstinline{std::forward_list<T>} nhưng tốn bộ nhớ hơn cho iterator hai chiều (bidirectional Iterator).\\
Để sử dụng lớp \lstinline{list}, ta cần chỉ thị \lstinline{#include <list>}.\\
Lớp \lstinline{list} được khai báo trong header \lstinline{<list>} như sau: \cite{list}
\begin{lstlisting}
template<class T, class Allocator = std::allocator<T>> 
class list;
namespace pmr {
    template <class T>
    using list = std::list<T, std::pmr::polymorphic_allocator<T>>;
}
\end{lstlisting}

\subsubsection{Các hàm thành viên}
Các constructor, destructor, toán tử gán.\\
Các phương thức truy cập phần tử:
\begin{itemize}
    \item \lstinline{front()}: trả về tham chiếu giá trị của HEAD trong DSLK đôi.
    \item \lstinline{back()}: trả về tham chiếu giá trị của TAIL trong DSLK đôi
\end{itemize}
Các iterators như trong phần hình \ref{popite}.\\
Các phương thức kiểm tra số lượng (capacity):
\begin{itemize}
    \item \lstinline{empty()}: trả về \lstinline{true} nếu mảng trống và \lstinline{false} nếu ngược lại.
    \item \lstinline{max_size()}: trả về số lượng phần tử tối đa có thể có của danh sách.
\end{itemize}
Các phương thức thao tác trên phần tử:
\begin{itemize} 
    \item \lstinline{push_front(val)},  \lstinline{pop_front()}: tương tự \lstinline{forward_list}.
    \item \lstinline{push_back(val)}: tương đương việc \lstinline{addTail} của DSLK đôi, đưa giá trị \lstinline{val} về cuối danh sách.
    \item \lstinline{pop_back()}: tương đương việc \lstinline{deleteTail} của DSLK đôi, xoá giá trị cuối cùng của danh sách. Nếu danh sách đang rỗng thì sẽ gây Undefined Behavior.
    \item \lstinline{insert(pos, (n), value)}: chèn \lstinline{n} giá trị \lstinline{value} vào trước vị trí iterator \lstinline{pos}, \lstinline{value} ở đây có thể là một giá trị hoặc là một \lstinline{initializer_list}, nếu không có tham số \lstinline{n} thì mặc định chèn 1 giá trị. Trả về iterator trỏ tới phần tử đầu tiên được thêm vào.
    \item \lstinline{insert(pos, beginIt, endIt)}: chèn giá trị copy từ iterator\lstinline{ beginIt} đến \lstinline{endIt} vào trước  vị trí iterator \lstinline{pos}. Trả về iterator trỏ tới phần tử đầu tiên được thêm vào.
    \item \lstinline{erase(pos)}: xoá phần tử ở vị trí iterator \lstinline{pos}, trả về iterator trỏ tới phần tử sau phần tử đã xoá.
    \item \lstinline{erase(begin_pos, end_pos)}: xoá phần tử từ vị trí iterator \lstinline{begin_pos} đến trước \lstinline{end_pos}, trả về iterator trỏ tới \lstinline{end_pos}.
\end{itemize}
Các phương thức với DSLK đôi: tương tự như \lstinline{forward_list}.\\
Độ phức tạp của các hàm thành viên có thể được tìm thấy tại \cite{list}.
\subsubsection{Ví dụ}
\begin{lstlisting}
#include <list>

int main() {
    std::list<int> mylist = {1, 2, 3, 4};
    std::list<int>:: it;
    
    it = mylist.begin(); //(1) 2 3 4
    it = mylist.insert(it, 10); //(10) 1 2 3 4
    it++; //10 (1) 2 3 4
    it = mylist.insert(it, 2, 3); //10 (3) 3 1 2 3 4
    
    return 0;
}
\end{lstlisting}

\subsection{vector}
Lớp \lstinline{vector} cung cấp mảng động với khả năng thay đổi số lượng phần tử tùy ý. Do là một associative container nên các phần tử của \lstinline{vector} được lưu trữ một cách tuần tự, theo thứ tự đúng với thứ tự được nhập vào.\\
Để sử dụng lớp \lstinline{vector}, ta sử dụng chỉ thị \lstinline{#include <vector>}.\\
Lớp \lstinline{vector} được khai báo trong header \lstinline{<vector>} như sau: \cite{anothervector}
\begin{lstlisting}
template<class T, class Allocator = std::allocator<T>> class vector;
namespace pmr {
    template <class T>
    using vector = std::vector<T, std::pmr::polymorphic_allocator<T>>;
}
\end{lstlisting}
\subsubsection{Các hàm thành viên}
Các constructor, destructor, toán tử gán.\\
Các phương thức truy cập phần tử: tương tự như \lstinline{array}.\\
Các iterators như trong phần hình \ref{popite}.\\
Các phương thức kiểm tra số lượng phần tử:
\begin{itemize}
    \item \lstinline{empty()} \lstinline{size()}, \lstinline{max_size()}: tương tự như \lstinline{array}.
    \item \lstinline{capacity()}: trả về số phần tử nhiều nhất hiện tại mà vector có thể chứa mà không cần phải cấp phát lại.
    \item \lstinline{shrink_to_fit()}:  giảm vùng nhớ bằng cách giải phóng những ô nhớ không sử dụng.
\end{itemize}
Các phương thức thao tác trên phần tử:
\begin{itemize}
    \item  \lstinline{push_back(val)}: thêm phần tử có giá trị \lstinline{val} vào vị trí cuối vào \lstinline{vector}.
    \item \lstinline{pop_back()}: xoá phần tử cuối khỏi \lstinline{vector}.
    \item \lstinline{insert()}, \lstinline{erase()}, \lstinline{clear()}: tương tự như các phần bên trên.
\end{itemize}
Độ phức tạp của các hàm thành viên có thể được tìm thấy tại \cite{vector}.
\subsubsection{Ví dụ}
\begin{lstlisting}
#include <iostream>
#include <vector>

int main() {
    std::vector<int> a;
    for (int i = 0; i <= 10; i++) {
        a.push_back(i);
        std::cout << a[i] << " ";
    }
    //0 1 2 3 4 5 6 7 8 9 10
    std::cout << "\n";
    
    int sum = 0;
    for (auto it = a.begin(); it != a.end(); it++) {
        if (*it % 2 == 0) {
            sum += *it;
        }
    }
    std::cout << "Sum of odd number: " << sum << "\n"; //25
}
\end{lstlisting}
Do cách tạo và hủy của \lstinline{vector}, khi các phần tử trong \lstinline{vector} là các đối tượng phức tạp, việc gọi nhiều lần các constructor và destructor sẽ rất tốn kém. Các hướng giải quyết cho vấn đề này được đề cập ở \cite{tdtfit}.
\subsection{Deque}
Lớp \lstinline{deque}, viết tắt của double-ended queue (tạm dịch: hàng đợi kết thúc kép, hàng đợi hai đầu), là một cấu trúc dữ liệu mở rộng của hàng đợi, cho phép thêm và xóa các phần tử ở cả hai đầu.\\
Để sử dụng lớp \lstinline{deque}, ta sử dụng chỉ thị \lstinline{#include <deque>}.\\
Lớp \lstinline{deque} được khai báo trong header \lstinline{<deque>} như sau:\cite{deque}
\begin{lstlisting}
template<class T, class Allocator = std::allocator<T>> 
class deque;
namespace pmr {
    template <class T>
    using deque = std::deque<T, std::pmr::polymorphic_allocator<T>>;
}
\end{lstlisting}
\subsubsection{Các hàm thành viên}
Các constructor, destructor, toán tử gán.\\
Các phương thức truy cập phần tử: tương tự như \lstinline{array}.\\
Các iterators như trong phần hình \ref{popite}.\\
Các phương thức kiểm tra số lượng phần tử:
\begin{itemize}
    \item \lstinline{empty()} \lstinline{size()}, \lstinline{max_size()}: tương tự như \lstinline{array}.
    \item \lstinline{shrink_to_fit()}:  giảm vùng nhớ bằng cách giải phóng những ô nhớ không sử dụng.
\end{itemize}
Các phương thức thao tác trên phần tử:
\begin{itemize}
    \item \lstinline{push_back(val)}, \lstinline{pop_back()}: tương tự như \lstinline{vector}.
    \item \lstinline{push_front(val)}: thêm phần tử có giá trị \lstinline{val} vào đầu \lstinline{deque}.
    \item \lstinline{pop_front()}: xóa phần tử đầu khỏi \lstinline{deque}.
    \item \lstinline{insert()}, \lstinline{erase()}, \lstinline{clear()}: tương tự như các phần bên trên.
\end{itemize}
Độ phức tạp của các hàm thành viên có thể được tìm thấy tại \cite{deque}.
\subsubsection{Ví dụ}
\begin{lstlisting}
#include <deque>

int main() {
    std::deque<int> d; // 
    d.push_front(8); // 8
    d.push_front(18); //18 8
    d.push_back(9); //18 8 9
    d.push_back(20); //18 8 9 20
    
    return 0;
}
\end{lstlisting}