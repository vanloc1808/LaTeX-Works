\begin{mybox}
\textbf{Câu 1:} Trình bày những nội dung cơ bản của thời kỳ quá độ lên chủ nghĩa xã hội. Trên cơ sở đó, phân tích những thuận lợi, khó khăn và đặc trưng của thời kỳ quá độ lên chủ nghĩa xã hội ở Việt Nam. Sinh viên cần làm gì để góp phần đẩy nhanh tiến trình của thời kỳ quá độ lên chủ nghĩa xã hội ở nước ta hiện nay?
\end{mybox}
\textbf{Trả lời.}\\
\textbf{Nội dung cơ bản của thời kỳ quá độ lên chủ nghĩa xã hội:}\\
Thực chất của thời kỳ quá độ lên chủ nghĩa xã hội là thời kỳ cải biên cách mạng từ xã hội tiền tư bản chủ nghĩa và tư bản chủ nghĩa sang xã hội xã hội chủ nghĩa. Xã hội của thời kỳ quá độ là xã hội có sự đan xen của nhiều tàn dư về mọi phương diện kinh tế, đạo đức, tinh thần của chủ nghĩa tư bản và những yếu tố mới mang tính chất xã hội chủ nghĩa của chủ nghĩa xã hội mới phát sinh, chưa phải là chủ nghĩa xã hội phát triển trên cơ sở của chính nó.\\
Về nội dung, thời kỳ quá độ lên chủ nghĩa xã hội là thời kỳ cải tạo cách mạng sâu sắc, triệt để xã hội tư bản chủ nghĩa trên tất cả các lĩnh vực, kinh tế, chính trị, văn hóa, xã hội, xây dựng từng bước cơ sở vật chất $-$ kỹ thuật và đời sống tinh thần của chủ nghĩa xã hội. Đó là thời kỳ lâu dài, gian khổ, "thời kỳ những cơn đau đẻ kéo dài" \supercite{Lenintt33}, bắt đầu từ khi giai cấp công nhân và nhân dân lao động giành được chính quyền đến khi xây dựng thành công chủ nghĩa xã hội. Có thể khái quát những đặc điểm cơ bản của thời kỳ quá độ lên chủ nghĩa xã hội như sau:
\begin{itemize}
\item Trên lĩnh vực kinh tế\\
Thời kỳ quá độ từ chủ nghĩa tư bản lên chủ nghĩa xã hội, về phương diện kinh tế, tất yếu tồn tại nền kinh tế nhiều thành phần, trong đó có thành phần đối lập. Đề cập tới đặc trưng này, V. I. Lenin cho rằng: "Vậy thì danh từ quá độ có nghĩa là gì? Vận dụng vào kinh tế, có phải nó có nghĩa là trong chế độ hiện nay có những thành phần, những bộ phận, những mảnh của chủ nghĩa tư bản lẫn chủ nghĩa xã hội không? Bất cứ ai cũng thừa nhận là có. Song không phải mọi người thừa nhận điểm yếu ấy đều suy nghĩ xem các thành phần của kết cấu kinh tế $-$ xã hội khác nhau hiện có ở Nga, chính là như thế nào? Mà tất cả then chốt của vấn đề lại chính là ở đó" \supercite{Lenintt36}. Tương ứng với nước Nga, V. I. Lenin cho rằng thời kỳ quá độ tồn tại 5 thành phần kinh tế: kinh tế gia trưởng; kinh tế hàng hóa nhỏ; kinh tế tư bản; kinh tế tư bản nhà nước; kinh tế xã hội chủ nghĩa.
\item Trên lĩnh vực chính trị\\
Thời kỳ quá độ từ chủ nghĩa tư bản lên chủ nghĩa xã hội về phương diện chính trị, là việc thiết lập, tăng cường chuyên chính vô sản mà thực chất của nó là việc giai cấp công nhân nắm và sử dụng quyền lực nhà nước trấn áp giai cấp tư sản, tiến hành xây dựng một xã hội không giai cấp. Đây là sự thống trị về chính trị của giai cấp công nhân với chức năng thực hiện dân chủ đối với nhân dân, tổ chức xây dựng và bảo vệ chế độ mới, chuyên chính với những phần tử thù địch, chống lại nhân dân; là tiếp tục cuộc đấu tranh giai cấp giữa giai cấp vô sản đã chiến thắng nhưng chưa phải đã toàn thắng với giai cấp tư sản đã thất bại nhưng chưa phải thất bại hoàn toàn. Cuộc đấu tranh diễn ra trong điều kiện mới $-$ giai cấp công nhân đã trở thành giai cấp cầm quyền, với nội dung mới $-$ xây dựng toàn diện xã hội mới, trọng tâm là xây dựng nhà nước có tính kinh té, và hình thức mới $-$ cơ bản là "hòa bình tổ chức xây dựng" \supercite{giaotrinh}.
\item Trên lĩnh vực tư tưởng $-$ văn hóa\\
Thời kỳ quá độ từ chủ nghĩa tư bản lên chủ nghĩa xã hội còn tồn tại nhiều tư tưởng khác nhau, chủ yếu là tư tưởng vô sản và tư tưởng tư sản. Giai cấp công nhân thông qua đội tiền phong của mình là Đảng Cộng sản từng bước xây dựng nền văn hóa vô sản, nền văn hóa mới xã hội chủ nghĩa, tiếp thu giá trị văn hóa dân tộc và tinh hoa văn hóa nhân loại, bảo đảm đáp ứng nhu cầu văn hóa $-$ tinh thần ngày càng tăng của nhân dân.
\item Trên lĩnh vực xã hội\\
Do kết cấu của nền kinh tế nhiều thành phần quy định nên trong thời kỳ quá độ còn tồn tại nhiều giai cấp, tầng lớp và sự khác biệt giữa các giai cấp, tầng lớp xã hội, các giai cấp, tầng lớp vừa hợp tác, vừa đấu tranh với nhau. Trong xã hội của thời kỳ quá độ còn tồn tại sự khác biệt giữa nông thôn, thành thị, giữa lao động trí óc và lao động chân tay. Bởi vậy, thời kỳ quá độ từ chủ nghĩa tư bản lên chủ nghĩa xã hội, về phương diện xã hội là thời kỳ đấu tranh giai cấp chống áp bức, bất công, xóa bỏ tệ nạn xã hội và những tàn dư của xã hội cũ để lại, thiết lập công bằng xã hội trên cơ sở thực hiện nguyên tắc phân phối theo lao động là chủ đạo.
\end{itemize}
\textbf{Việt Nam tiến lên chủ nghĩa xã hội trong điều kiện vừa thuận lợi, vừa khó khăn đan xen, có những đặc trưng cơ bản:}
\begin{itemize}
\item Xuất phát từ một xã hội vốn là thuộc địa, nửa phong kiến, lực lượng sản xuất rất thấp. Đất nước trải qua chiến tranh ác liệt, kéo dài nhiều thập kỷ, hậu quả để lại còn nặng nề. Những tàn dư thực dân, phong kiến còn nhiều. Các thế lực thù địch thường xuyên tìm cách phá hoại chế độ xã hội chủ nghĩa và nền độc lập dân tộc của nhân dân ta.
\item Cuộc cách mạng khoa học công nghệ hiện đại đang diễn ra mạnh mẽ, cuốn hút tất cả các nước ở các mức độ khác nhau. Nền sản xuất vật chất và đời sống xã hội đang trong quá trình quốc tế hóa sâu sắc, ảnh hưởng lớn tới nhịp độ phát triển lịch sử và cuộc sống các dân tộc. Những xu thế đó vừa tạo thời cơ phát triển nhanh cho các nước, vừa đặt ra những thách thức gay gắt.
\item Thời đại ngày nay vẫn là thời đại quá độ từ chủ nghĩa tư bản lên chủ nghĩa xã hội, cho dù chế độ xã hội chủ nghĩa ở Liên Xô và Đông Âu sụp đổ. Các nước với chế độ xã hội và trình độ phát triển khác nhau cùng tồn tại, vừa hợp tác vừa đấu tranh, cạnh tranh gay gắt vì lợi ích quốc gia, dân tộc. Cuộc đấu tranh của nhân dân các nước vì hòa bình, độc lập dân tộc, dân chủ, phát triển và tiến bộ xã hội dù gặp nhiều khó khăn, thách thức, song theo quy luật tiến hóa của lịch sử, loài người nhất định sẽ tiến tới chủ nghĩa xã hội.
\end{itemize}
Quá độ lên chủ nghĩa xã hội bỏ qua chế độ tư bản chủ nghĩa là sự lựa chọn duy nhất đúng, khoa học, phản ánh đúng quy luật phát triển khách quan của cách mạng Việt Nam trong thời đại ngày nay. Cương lĩnh năm 1930 của Đảng chỉ rõ: "làm tư sản dân quyền cách mạng và thổ địa cách mạng để đi tới xã hội cộng sản" \supercite{vkdtt2}. Đây là sự lựa chọn dứt khoát và đúng đắn của Đảng, đáp ứng nguyện vọng thiết tha của dân tộc, nhân dân, phản ánh xu thế phát triển của thời đại, phù hợp với quan điểm khoa học, cách mạng và sáng tạo của chủ nghĩa Mác $-$ Lênin.
\textbf{Xã hội xã hội chủ nghĩa mà nhân dân Việt Nam xây dựng có các đặc trưng về mục tiêu, bản chất, nội dung là:} \supercite{vk11}
\begin{itemize}
\item \textit{Một là}, dân giàu, nước mạnh, dân chủ, công bằng, văn minh.
\item \textit{Hai là}, do nhân dân làm chủ.
\item \textit{Ba là}, có nền kinh tế phát triển cao dựa trên lực lượng sản xuất hiện đại và quan hệ sản xuất tiến bộ phù hợp.
\item \textit{Bốn là}, có nền văn hóa tiên tiến, đậm đà bản sắc dân tộc.
\item \textit{Năm là}, con người có cuộc sống ấm no, tự do, hạnh phúc, có điều kiện phát triển toàn diện.
\item \textit{Sáu là}, các dân tộc trong cộng đồng Việt Nam bình đẳng, đoàn kết, tôn trọng và giúp nhau cùng phát triển.
\item \textit{Bảy là}, có Nhà nước pháp quyền xã hội chủ nghĩa của nhân dân, do nhân dân, vì nhân dân do Đảng Cộng sản lãnh đạo.
\item \textit{Tám là}, có quan hệ hữu nghị và hợp tác với các nước trên thế giới.
\end{itemize}
\textbf{Trách nhiệm của sinh viên để góp phần đẩy nhanh tiến trình của thời kỳ quá độ lên chủ nghĩa xã hội ở nước ta hiện nay:}\\
Không ngừng tích cực học tập, rèn luyện, trang bị những kiến thức cần thiết, cả về chuyên môn lẫn xã hội, có đạo đức, lối sống văn hóa, có ý chí vươn lên trong học tập, nghiên cứu khoa học để mai này có thể góp phần đưa đất nước tiến nhanh trên con đường xây dựng chủ nghĩa xã hội vừa "hồng" vừa "chuyên", như lời của Chủ tịch Hồ Chí Minh.\\
Sống và làm việc theo pháp luật, theo bản sắc dân tộc, thấm nhuần các truyền thống quý báu của dân tộc, nắm bắt những chủ trương, đường lối của Đảng, chính sách của Nhà nước, rèn luyện nhân sinh quan, thế giới quan và đạo đức của người cộng sản.\\
Trang bị những kiến thức đầy đủ, đúng đắn về lý luận chính trị, có bản lĩnh chính trị vững vàng, không bị các thế lực thù địch níu kéo, dụ dỗ, có nhận thức đúng đắn về chủ nghĩa xã hội và sự nghiệp xây dựng chủ nghĩa xã hội của nhân dân ta.
