\begin{mybox}
\textbf{Câu 2:} Phân tích bản chất, nguồn gốc và tính chất của tôn giáo. Sinh viên cần làm gì để không bị lôi cuốn vào các hình thức mê tín dị đoan?
\end{mybox}
\textbf{Trả lời.}\\
\textbf{Bản chất, nguồn gốc và tính chất của tôn giáo}\\
\textbf{Bản chất của tôn giáo}\\
Chủ nghĩa Mác $-$ Lênin cho rằng tôn giáo là một hình thái ý thức xã hội, phản ánh hư ảo hiện thực khách quan. Thông qua sự phản ánh đó, các lực lượng tự nhiên trở thành siêu tự nhiên, thần bí, ... Ăngghen cho rằng "... tất cả mọi tôn giáo chẳng qua chỉ là sự phản ánh hư ảo $-$ vào trong đầu óc con người $-$ của những lực lượng ở bên ngoài chi phối cuộc sống hằng ngày của họ; chỉ là sự phản ánh trong đó những lực lượng ở trần thế đã mang hình thức những lực lượng siêu trần thế" \supercite{Mac20}.\\
Ở một cách tiếp cận khác, tôn giáo còn được hiểu là một thực thể xã hội $-$ các tôn giáo cụ thể (ví dụ: Công giáo, Tin lành, Phật giáo, ...), với các tiêu chí cơ bản sau: có đấng siêu nhiên, đấng tối cao, thần linh để tôn thờ; có hệ thống giáo thuyết (giáo lý, giáo luật, lễ nghi) phản ánh thế giới quan, nhân sinh quan, đạo đức, lễ nghi của tôn giáo; có hệ thống cơ sở thờ tự; có tổ chức nhân sự, quản lý điều hành việc đạo (người hoạt động tôn giáo chuyên nghiệp hay không chuyên nghiệp); có hệ thống tín đồ đông đảo, những người tự nguyện tin theo một tôn giáo nào đó, và được tôn giáo đó thừa nhận.\\
Chỉ rõ bản chất tôn giáo, chủ nghĩa Mác $-$ Lênin khẳng định rằng: \textit{Tôn giáo là một hiện tượng xã hội $-$ văn hóa do con người sáng tạo ra}. Con người sáng tạo ra tôn giáo vì mục đích, lợi ích của họ, phản ánh những ước mơ, nguyện vọng, suy nghĩ của họ. Nhưng, sáng tạo ra tôn giáo, con người lại bị lệ thuộc vào tôn giáo, tuyệt đối hóa và phục tùng tôn giáo vô điều kiện. Chủ nghĩa Mác $-$ Lênin cũng cho rằng, sản xuất vật chất và các quan hệ kinh tế xét đến cùng là nhân tố quyết định sự tồn tại và phát triển của các hình thái ý thức xã hội, trong đó có tôn giáo. Do đó, mọi quan niệm về tôn giáo, các tổ chức, các thiết chế tôn giáo đều được sinh ra từ những hoạt động sản xuất, từ những điều kiện sống nhất định trong xã hội và thay đổi theo những thay đổi của cơ sở kinh tế. \textit{Về phương diện thế giới quan}, các tôn giáo mang thế giới quan duy tâm, có sự khác biệt với thế giới quan duy vật biện chứng, khoa học của chủ nghĩa Mác $-$ Lênin. Mặc dù có sự khác biệt về thế giới quan, nhưng những người cộng sản với lập trường mác xít không bao giờ có thái độ xem thường hoặc trấn áp những nhu cầu tín ngưỡng, tôn giáo của nhân dân; ngược lại, họ luôn tôn trọng quyền tự do tín ngưỡng, theo hoặc không theo tôn giáo của nhân dân. Trong những điều kiện cụ thể của xã hội, những người cộng sản và những người có tín ngưỡng tôn giáo có thể cùng nhau xây dựng một xã hội tốt đẹp hơn ở thế giới hiện thực. Xã hội ấy chính là xã hội mà quần chúng tín đồ cũng từng mơ ước và phản ánh nó qua một số tôn giáo.\\
Tôn giáo và tín ngưỡng không đồng nhất, nhưng có giao thoa nhất định.\\
\textbf{Nguồn gốc của tôn giáo}\\
\textbf{\textit{Nguồn gốc tự nhiên, kinh tế $-$ xã hội}}\\
Trong xã hội công xã nguyên thủy, do lực lượng sản xuất chưa phát triển, trước thiên nhiên hùng vĩ tác động và chi phối khiến cho con người cảm thấy yếu đuối và bất lực, không giải thích được, nên con người đã gán cho tự nhiên những sức mạnh, quyền lực thần bí.\\
Khi xã hội xuất hiện các giai cấp đối kháng, có áp bức, bất công, do không giải thích được nguồn gốc của sự phân hóa giai cấp và áp bức bóc lột, bất công, tội ác, ... cộng với lo sợ trước sự thống trị của các lực lượng xã hội, con người trông chờ vào sự giải phóng của một lực lượng siêu nhiên ngoài trần thế.\\
\textbf{\textit{Nguồn gốc nhận thức}}\\
Ở một giai đoạn lịch sử nhất định, sự nhận thức của con người về tự nhiên, xã hội và chính bản thân mình là có giới hạn. Khi mà khoảng cách giữa "biết" và "chưa biết" vẫn tồn tại, khi những điều mà khoa học chưa giải thích được, thì điều đó thường được giải thích thông qua lăng kính các tôn giáo. Ngay cả những vấn đề đã được khoa học chứng minh, nhưng do trình độ dân trí thấp, chưa thể nhận thức đầy đủ, thì đây vẫn là điều kiện, là mảnh đất cho tôn giáo ra đời, tồn tại và phát triển. Thực chất nguồn gốc nhận thức của tôn giáo chính là sự tuyệt đối hóa, sự cường điệu mặt chủ thể của nhận thức con người, biến cái nội dung khách quan thành cái siêu nhiên, thần thánh.\\
\textbf{\textit{Nguồn gốc tâm lý}}\\
Sự sợ hãi trước những hiện tượng tự nhiên, xã hội hay trong những lúc ốm đau, bệnh tật; ngay cả những may, rủi bất ngờ xảy ra, hoặc tâm lý muốn được bình yên khi làm việc lớn (ví dụ: ma chay, cưới xin, làm nhà, khởi đầu sự nghiệp kinh doanh...), con người cũng dễ tìm đến với tôn giáo. Thậm chí cả những tình cảm tích cực như tình yêu, lòng biết ơn, lòng kính trọng đối với những người có công với nước, với dân cũng dễ dẫn con người đến với tôn giáo (ví dụ: thờ các anh hùng dân tộc, thờ các thành hoàng làng, ...).\\
\textbf{Tính chất của tôn giáo}\\
\textbf{\textit{Tính lịch sử của tôn giáo}}\\
Tôn giáo là một hiện tượng xã hội có tính lịch sử, nghĩa là nó có sự hình thành, tồn tại và phát triển trong những giai đoạn lịch sử nhất định, nó có khả năng biến đổi để thích nghi với nhiều chế độ chính trị $-$ xã hội. Khi các điều kiện kinh tế $-$ xã hội, lịch sử thay đổi, tôn giáo cũng có sự thay đổi theo. Trong quá trình vận động của các tôn giáo, chính các điều kiện kinh tế $-$ xã hội, lịch sử cụ thể đã làm cho các tôn giáo bị phân liệt, chia tách thành nhiều tôn giáo, hệ phái khác nhau.\\
Theo quan điểm của chủ nghĩa Mác $-$ Lênin, đến một giai đoạn lịch sử nào đó, khi khoa học và giáo dục giúp cho đại đa số quần chúng nhân dân nhận thức được bản chất các hiện tượng tự nhiên và xã hội thì tôn giáo sẽ dần dần mất đi vị trí của nó trong đời sống xã hội và cả trong nhận thức, niềm tin của mỗi người.\\
\textbf{\textit{Tính quần chúng của tôn giáo}}\\
Tôn giáo là một hiện tượng xã hội phổ biến ở tất cả các dân tộc, quốc gia, châu lục. Tính quần chúng của tôn giáo không chỉ biểu hiện ở số lượng tín đồ đông đảo (gần 3/4 dân số thế giới); mà còn thể hiện ở chỗ, các tôn giáo là nơi sinh hoạt văn hóa, tinh thần của một bộ phận quần chúng nhân dân lao động. Dù tôn giáo hướng con người vào niềm tin hạnh phúc hư ảo của thế giới bên kia, song nó luôn luôn phản ánh khát vọng của những người lao động về một xã hội tự do, bình đẳng, bác ái. Mặt khác, nhiều tôn giáo có tính nhân văn, nhân đạo và hướng thiện, vì vậy, được nhiều người ở các tầng lớp khác nhau trong xã hội, đặc biệt là quần chúng lao động, tin theo.\\
\textbf{\textit{Tính chính trị của tôn giáo}}\\
Khi xã hội chưa có giai cấp, tôn giáo chỉ phản ánh nhận thức hồn nhiên, ngây thơ của con người về bản thân và thế giới xung quanh mình, tôn giáo chưa mang tính chính trị. Tính chất chính trị của tôn giáo chỉ xuất hiện khi xã hội đã phân chia giai cấp, có sự khác biệt, sự đối kháng về lợi ích. Trước hết, do tôn giáo là sản phẩm của những điều kiện kinh tế $-$ xã hội, phản ánh lợi ích, nguyện vọng của các giai cấp khác nhau trong cuộc đấu tranh giai cấp, đấu tranh dân tộc, nên tôn giáo mang tính chính trị. Mặt khác, khi các giai cấp bóc lột, thống trị sử dụng tôn giáo để phục vụ cho lợi ích giai cấp mình, chống lại các giai cấp lao động và tiến bộ xã hội, tôn giáo mang tính chính trị tiêu cực, phản tiến bộ.\\
Vì vậy, cần nhận rõ rằng, đa số quần chúng tín đồ đến với tôn giáo nhằm thỏa mãn nhu cầu tinh thần; song, trên thực tế, tôn giáo đã và đang bị các thế lực chính trị $-$ xã hội lợi dụng thực hiện mục đích ngoài tôn giáo của họ.\\
\textbf{Trách nhiệm của sinh viên để không bị lôi cuốn vào các hình thức mê tín dị đoan}\\
Tích cực ra sức học tập, rèn luyện để có trình độ chuyên môn cao, có lập trưởng chính trị vững vàng, có đạo đức, lối sống văn hóa, có ý chí vươn lên trong học tập, nghiên cứu khoa học.\\
Sống và làm việc theo pháp luật, theo bản sắc dân tộc, thấm nhuần các truyền thống quý báu của dân tộc, nắm bắt những chủ trương, đường lối của Đảng, chính sách của Nhà nước, đặc biệt là trong vấn đề tôn giáo, rèn luyện nhân sinh quan, thế giới quan và đạo đức của người cộng sản.\\
Tự chăm lo cho đời sống tinh thần của mình, không để các hình thức mê tín dị đoan níu kéo, dụ dỗ, tránh sa đà vào những hình thức "tà đạo".