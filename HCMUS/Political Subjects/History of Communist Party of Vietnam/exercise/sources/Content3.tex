\section{Đại hội đại biểu toàn quốc lần thứ III của Đảng và kế hoạch 5 năm lần thứ nhất (1960 $-$ 1965)}
\subsection{Đại hội đại biểu toàn quốc lần thứ III của Đảng (1960)}
Tháng 9/1960, \textit{Đại hội đại biểu toàn quốc lần thứ III} của Đảng họp tại Thủ đô Hà Nội. Trong diễn văn khai mạc Đại hội, Chủ tịch Hồ Chí Minh nêu rõ: "Đại hội lần thứ II là Đại hội kháng chiến, Đại hội lần này là Đại hội xây dựng chủ nghĩa xã hội ở miền Bắc và đấu tranh hòa bình thống nhất nước nhà" \supercite{dvkmdh3, HCMtt12}.\\ 
Đại hội đã thảo luận và thông qua Báo cáo chính trị của Ban Chấp hành Trung ương khóa II và thông qua Nghị quyết về Nhiệm vụ và đường lối của Đảng trong giai đoạn mới, thông qua Báo cáo về xây dựng Đảng và báo cáo về Kế hoạch 5 năm lần thứ nhất xây dựng chủ nghĩa xã hội ở miền Bắc, ...\\
Đại hội đã đề ra đường lối chung trong thời kỳ quá độ lên chủ nghĩa xã hội ở miền Bắc nước ta là: \textit{"Đoàn kết toàn dân, phát huy tinh thần yêu nước nồng nàn, truyền thống phấn đấu anh dũng và lao động cần cù của nhân dân ta, đồng thời tăng cường đoàn kết với các nước xã hội chủ nghĩa anh em do Liên Xô đứng đầu, để đưa miền Bắc tiến nhanh, tiến mạnh, tiến vững chắc lên chủ nghĩa xã hội, xây dựng đời sống ấm no, hạnh phúc ở miền Bắc và củng cố miền Bắc thành cơ sở vững mạnh cho cuộc đấu tranh thực hiện hòa bình thống nhất nước nhà, góp phần tăng cường phe xã hội chủ nghĩa, bảo vệ hòa bình ở Đông Nam Á và thế giới."} \supercite{nq1dh3}.\\
Tuy vẫn còn một số hạn chế trong đường lối cách mạng xã hội chủ nghĩa là nhận thức về con đường đi lên chủ nghĩa xã hội còn giản đơn, chưa có dự kiến về chặng đường đầu tiên của thời kỳ quá độ lên chủ nghĩa xã hội; song thành công cơ bản, to lớn nhất của Đại hội III là đã \textit{"hoàn chỉnh đường lối chiến lược chung của cách mạng Việt Nam trong giai đoạn mới, đường lối tiến hành đồng thời và kết hợp chặt chẽ hai chiến lược cách mạng khác nhau ở hai miền: cách mạng xã hội chủ nghĩa ở miền Bắc và cách mạng dân tộc dân chủ nhân dân ở miền Nam, nhằm thực hiện mục tiêu chung trước mắt của cả nước là giải phóng miền Nam, hòa bình thống nhất Tổ quốc"} \supercite{giaotrinh}.\\
Đó chính là \textit{đường lối giương cao ngọn cờ độc lập dân tộc và chủ nghĩa xã hội}, vừa phù hợp với miền Bắc vừa phù hợp với miền Nam, vừa phù hợp với cả nước Việt Nam vừa phù hợp với tình hình quốc tế. Đặt trong bối cảnh Việt Nam và quốc tế, đường lối chung của Đảng còn là sự thể hiện tinh thần độc lập, tự chủ, sáng tạo của Đảng trong việc giải quyết những vấn đề không có tiền lệ lịch sử, vừa đúng với thực tiễn Việt Nam vừa phù hợp với lợi ích của nhân loại và xu thế của thời đại.
\subsection{Kế hoạch 5 năm lần thứ nhất (1960 $-$ 1965)}
Trên cơ sở miền Bắc đã hoàn thành kế hoạch ba năm cải tạo xã hội chủ nghĩa (1958 $-$ 1960), Đại hội lần thứ III của Đảng đã đề ra và chỉ đạo thực hiện \textit{kế hoạch năm năm lần thứ nhất (1961 $-$ 1965)} nhằm xây dựng bước đầu cơ sở vật chất $-$ kỹ thuật của chủ nghĩa xã hội, thực hiện một bước công nghiệp hóa xã hội chủ nghĩa và hoàn thành công cuộc cải tạo xã hội chủ nghĩa, tiếp tục đưa miền Bắc tiến nhanh, tiến mạnh, tiến vững chắc lên chủ nghĩa xã hội. Mục tiêu, nhiệm vụ cụ thể của kế hoạch là "tiếp tục hoàn thiện quan hệ sản xuất xã hội chủ nghĩa; xây dựng một bước cơ sở vật chất của chủ nghĩa xã hội; cải thiện đời sống nhân dân; bảo đảm an ninh quốc phòng, làm hậu thuẫn cho cuộc đấu tranh thống nhất nước nhà" \supercite{giaotrinh}. \\
Để thực hiện các mục tiêu, nhiệm vụ đó, Ban Chấp hành Trung ương Đảng đã mở nhiều hội nghị chuyên đề \footnote{Như Hội nghị 4 (tháng 4/1961) về xây dựng Đảng, Hội nghị 5 (tháng 7/1961) về phát triển nông nghiệp, Hội nghị 7 (tháng 6/1962) về phát triển công nghiệp, Hội nghị tháng 4/1963 về kế hoạch nhà nước, Hội nghị 8 (tháng 12/1964) về lưu thông phân phối, giá cả, ...} nhằm cụ thể hóa đường lối, đưa nghị quyết của Đảng vào cuộc sống.\\
Trong quá trình thực hiện kế hoạch 5 năm lần thứ nhất (1961 $-$ 1965), nhiều cuộc vận động và phong trào thi đua được triển khai sôi nổi ở các ngành, các giới và các địa phương. Đặc biệt là phong trào "Mỗi người làm việc bằng hai để đền đáp lại cho đồng bào miền Nam ruột thịt" theo Lời kêu gọi của Chủ tịch Hồ Chí Minh tại Hội nghị chính trị đặc biệt tháng 3/1964, khi đế quốc Mỹ mở rộng chiến tranh ở miền Nam, đã làm tăng thêm không khí phấn khởi, hăng hái vươn lên hoàn thành kế hoạch.\\
Kế hoạch này mới được thực hiện hơn bốn năm (tính đến ngày 05/8/1964) thì được chuyển hướng do phải đối phó với chiến tranh phá hoại miền Bắc của đế quốc Mỹ, song những mục tiêu chủ yếu của kế hoạch đã cơ bản hoàn thành \supercite{giaotrinh}.\\
Trong những năm thực hiện kế hoạch 5 năm lần thứ nhất (1961 $-$ 1965), miền Bắc xã hội chủ nghĩa đã không ngừng tăng cường chi viện cho cách mạng miền Nam, qua các tuyến đường Trường Sơn và đường Hồ Chí Minh trên biển. Tính chung, năm 1965, "số bộ đội từ miền Bắc được đưa vào miền Nam tăng 9 lần, số vật chất tăng 10 lần" \supercite{giaotrinh} so với năm 1961. Đây là "một thành công lớn, có ý nghĩa chiến lược của hậu phương miền Bắc, góp phần vào chiến thắng của quân dân miền Nam đánh bại chiến tranh xâm lược thực dân mới của đế quốc Mỹ và tay sai giai đoạn 1961 $-$ 1965" \supercite{giaotrinh}.\\
Trải qua 10 năm khôi phục, cải tạo và xây dựng chế độ mới, "miền Bắc nước ta đã tiến những bước dài chưa từng có trong lịch sử dân tộc. Đất nước, xã hội, con người đều đổi mới" \supercite{HCMtt14}. Miền Bắc đã trở thành căn cứ địa vững chắc cho cách mạng cả nước với chế độ chính trị ưu việt, với lực lượng kinh tế và quốc phòng lớn mạnh.