\documentclass[12pt,a4paper]{article}
\usepackage[utf8]{vietnam}
\usepackage{amsmath}
\usepackage{amsfonts}
\usepackage{amssymb}
\usepackage{hyperref}
\usepackage{xcolor}
\usepackage{titlesec}
\usepackage{mdframed}
\usepackage{etoolbox}
\usepackage[thref,thmmarks,standard,amsmath,hyperref]{ntheorem}
\usepackage{graphicx}
\usepackage[left=2cm,right=2cm,top=2cm,bottom=2cm]{geometry}
\author{Nguyễn Văn Lộc - 20120131}
\theorembodyfont{\normalfont}
\theoremseparator {.}
\newtheorem{ques}{Câu}
\theoremstyle{nonumberplain}
\newtheorem{ans}{Đáp án}
\newmdenv[linecolor=black,skipabove=\topsep,skipbelow=\topsep,
leftmargin=-5pt,rightmargin=-5pt,
innerleftmargin=5pt,innerrightmargin=5pt]{mybox}
\begin{document}
\begin{center}
\textbf{ĐÁP ÁN VÒNG LOẠI\\}
\textbf{CUỘC THI ĐƯỜNG LÊN ĐỈNH OLYMPIA CẤP TRƯỜNG\\}
\textbf{TRƯỜNG THPT CHUYÊN LÊ QUÝ ĐÔN\\}
\textbf{NĂM HỌC 2018 - 2019}
\end{center}
\textbf{Lưu ý: đáp án có thể có sai sót không mong muốn. Mọi người cứ thoải mái comment các thắc mắc nhé.}
\begin{ques}
Họa sĩ Bùi Xuân Phái nổi tiếng với dòng tranh về thành phố nào?
\end{ques}
\begin{mybox}
\begin{ans}
Họa sĩ Bùi Xuân Phái nổi tiếng với dòng tranh về \textbf{phố cổ Hà Nội.}\\
\textbf{Giải thưởng Bùi Xuân Phái - Vì tình yêu Hà Nội} là một giải thưởng được trao hằng năm cho những tác giả, tâc phẩm, ý tưởng, việc làm có hàm lượng nghệ thuật và khoa học cao, gắn bó với các mặt của đời sống Hà Nội.\\
Đọc thêm về Giải thưởng Bùi Xuân Phái - Vì tình yêu Hà Nội tại: \url{https://bit.ly/GTBXPVTYHN}
\end{ans} \end{mybox}
\begin{ques}
Bốn cây đại thụ của nền mỹ thuật Việt Nam hiện đại là: \textit{"Nhất Trí, nhì Vân, tam Lân, tứ Cẩn"}. "Nhất Trí" ở đây là ai?
\end{ques}
\begin{mybox}
\begin{ans}
Bốn cây đại thụ của nền mỹ thuật Việt Nam hiện đại bao gồm:
\begin{itemize}
\item Nhất Trí: họa sĩ \textbf{Nguyễn Gia Trí.} Các tác phẩm nổi bật: \textit{Thiếu nữ trong vườn,} \textit{Vườn xuân Trung Nam Bắc} (được Chính phủ công nhận là Bảo vật quốc gia năm 2013).
\item Nhì Vân: họa sĩ \textbf{Tô Ngọc Vân.} Các tác phẩm nổi bật: \textit{Thiếu nữ bên hoa huệ,} \textit{Thiếu nữ bên hoa sen,} \textit{Hai thiếu nữ và em bé} (cũng được công nhận là Bảo vật quốc gia năm 2013).
\item Tam Lân: họa sĩ \textbf{Nguyễn Tường Lân} (cần lưu ý tránh nhầm lẫn với nhà văn Thạch Lam \(-\) người cũng có tên là Nguyễn Tường Lân). Tác phẩm nổi bật: \textit{Đôi bạn.}  
\item Tứ Cẩn: họa sĩ \textbf{Trần Văn Cẩn.} Các tác phẩm nổi bật: \textit{Em Thúy} (được công nhận là Bảo vật quốc gia năm 2013), \textit{Gội đầu.}
\end{itemize}
\end{ans} \end{mybox}
\begin{ques}
Đại tướng Võ Nguyên Giáp sinh ra ở tỉnh nào ngày nay?
\end{ques}
\begin{mybox}
\begin{ans}
Đại tướng Võ Nguyên Giáp sinh ngày 25/08/1911 tại xã Lộc Thủy, huyện Lệ Thủy, tỉnh \textbf{Quảng Bình,} là \textbf{Đại tướng đầu tiên} của Quân đội Nhân dân Việt Nam (1948), được phong hàm Đại tướng mà không cần qua các quân hàm nhỏ hơn.
\end{ans} \end{mybox}
\begin{ques}
Chùa Một Cột được vua Lý Thái Tông cho xây dựng vào năm nào?
\end{ques}
\begin{mybox}
\begin{ans}
Tháng mười âm lịch năm Kỷ Sửu \textbf{1049}, vua Lý Thái Tông cho xây dựng chùa Diên Hựu (chùa Một Cột ngày nay). Tương truyền là vua được Phật Bà Quan Âm báo mộng nên xây.
\end{ans} \end{mybox}
\begin{ques}
Ở cực Bắc của Trái Đất thì kim la bàn luôn chỉ về hướng nào?
\end{ques}
\begin{mybox}
\begin{ans}
Đề thi năm 2018 - 2019 đưa ra bốn đáp án cho câu hỏi này là: hướng Đông, hướng Tây, hướng Nam, hướng Bắc. Đáp án đúng được đưa ra là \textbf{hướng Nam.}\\
Ở đây có một số tranh cãi về việc ở cực Bắc của Trái Đất thì kim la bàn sẽ quay vòng vòng. Nhưng mà ở cực Bắc rồi thì hướng nào cũng là hướng Nam cả. :(\\
Còn ở \textbf{cực Bắc từ trường} của Trái Đất thì la bàn sẽ không hoạt động được. \\
\end{ans} \end{mybox}
Cần phân biệt \textbf{cực Bắc,} \textbf{Bắc Cực} và \textbf{cực Bắc từ trường.}
\begin{ques}
Quốc hội nước Việt Nam Dân chủ Cộng hòa quyết định đổi tên nước thành \textit{"Cộng hòa Xã hội Chủ nghĩa Việt Nam"} vào ngày tháng năm nào?
\end{ques}
\begin{mybox}
\begin{ans}
Ngày \textbf{02/07/1976}, tại kỳ họp đầu tiên, Quốc hội khóa VI đã quyết định:
\begin{itemize}
\item Đặt tên nước là \textit{"Cộng hòa Xã hội Chủ nghĩa Việt Nam"},
\item Quốc kỳ nền đỏ, ở giữa có ngôi sao vàng năm cánh,
\item Quốc huy là hình tròn, nền đỏ, ở giữa có ngôi sao vàng năm cánh, xung quanh có bông lúa, ở giữa có nửa bánh xe răng cưa và dòng chữ "Cộng hòa Xã hội Chủ nghĩa Việt Nam",
\item Thủ đô là thành phố Hà Nội,
\item Quốc ca là bài \textit{"Tiến quân ca"}.
\end{itemize}
\end{ans} \end{mybox}
\textbf{Lưu ý.} Có một số bạn đưa ra đáp án là ngày \textbf{25/04/1976} \(-\) đây là ngày \textbf{Tổng tuyển cử đầu tiên của nước Việt Nam thống nhất}, bầu Quốc hội khóa VI.
\begin{ques}
Tính đến hết tháng 7 năm 2021, trong các nước sau: Thụy Điển, Thụy Sĩ, Phần Lan, Hà Lan, nước nào không phải là thành viên của Liên minh Châu Âu (EU)?
\end{ques}
\begin{mybox}
\begin{ans}
Tính đến hết tháng 7 năm 2021, \textbf{Thụy Sĩ} là quốc gia duy nhất trong danh sách này không là thành viên của Liên minh Châu Âu (EU).\\
Thụy Sĩ luôn giữ lập trường là một \textbf{\textit{quốc gia trung lập}}, không tham gia các khối liên minh quân sự, không ký kết các hiệp ước dẫn đến xung đột vũ trang. Thụy Sĩ cũng không tham gia vào hai cuộc chiến tranh thế giới trong thế kỉ XX.
\end{ans} \end{mybox}
\begin{ques}
Huệ biển là loài sinh vật thuộc giới nào?
\end{ques}
\begin{mybox} \begin{ans}
Huệ biển là sinh vật thuộc giới \textbf{Động vật.}\\
Có 5 (năm) giới sinh vật bao gồm: giới Khởi sinh, giới Nguyên sinh, giới Nấm, giới Thực vật và giới Động vật.\\
Theo thứ tự nhỏ dần: giới \(-\) ngành \(-\) lớp \(-\) bộ \(-\) họ \(-\) chi \(-\) loài.
\end{ans} \end{mybox}
\begin{ques}
Máy tính điện tử đầu tiên trên thế giới có tên gọi là gì?
\end{ques}
\begin{mybox} \begin{ans}
Máy tính điện tử đầu tiên trên thế giới có tên là \textbf{ENIAC}, viết tắt của cụm từ \textit{"Electronic Numerical Integrator and Computer"}.
\end{ans} \end{mybox}
\begin{ques}
Máy tính điện tử đầu tiên trên thế giới ra đời tại quốc gia nào?
\end{ques}
\begin{mybox} \begin{ans}
Máy tính điện tử đầu tiên trên thế giới ra đời năm 1945 tại \textbf{Hợp chúng quốc Hoa Kỳ} (Mỹ).
\end{ans} \end{mybox}
\begin{ques}
Thiết bị nào được ví như là bộ não của máy tính?
\end{ques}
\begin{mybox} \begin{ans}
\textbf{Central Processing Unit} (CPU), tạm dịch là \textit{Bộ xử lí trung tâm}, là thiết bị được viết như bộ não của máy tính, có trách nhiệm chỉ huy các hoạt động của máy tính.
\end{ans} \end{mybox}
\begin{ques}
Trong Windows, muốn xóa hoàn toàn một đối tượng nào đó mà không lưu lại trong Recycle Bin thì ta dùng tổ hợp phím nào?
\end{ques}
\begin{mybox} \begin{ans}
Trong Windows, để xóa hoàn toàn một đối tượng mà không lưu lại trong Recycle Bin, ta dùng tổ hợp phím \textbf{Shift + Delete.}
\end{ans} \end{mybox}
\begin{ques}
Who said: \textit{"Stay foolish. Stay hungry"}?
\end{ques}
\begin{mybox} \begin{ans}
Câu này của \textbf{Steve Jobs} nha.
\end{ans} \end{mybox}
\begin{ques}
Ai là người châu Á đầu tiên đạt giải Nobel văn học?
\end{ques}
\begin{mybox} \begin{ans}
Năm 1913, \textbf{Rabindranath Tagore} trở thành người châu Á đầu tiên đoạt giải thưởng Nobel Văn học với tác phẩm \textit{"Thơ dâng"} (Gitanjali).
\end{ans} \end{mybox}
\begin{ques}
Điền vào chỗ trống trong câu ca dao sau: \textit{"Gió đông là ... lúa chiêm, gió bấc là duyên lúa mùa.}
\end{ques}
\begin{mybox} \begin{ans}
Gió đông là \textbf{chồng} lúa chiếm, gió bấc là duyên lúa mùa.
\end{ans} \end{mybox}
\begin{ques}
Tác giả của \textit{"Nhật kí ở rừng"} là ai?
\end{ques}
\begin{mybox} \begin{ans}
Đây là bút ký của \textbf{Nam Cao,} được viết vào mùa thu năm 1947, sau khi ông lên Việt Bắc.
\end{ans} \end{mybox}
\begin{ques}
Tác phẩm nào được Vũ Khâm Lân đánh giá là \textit{"một cuốn thiên cổ kì bút"}?
\end{ques}
\begin{mybox} \begin{ans}
Vũ Khâm Lân đã dùng cụm từ \textit{"một cuốn thiên cổ kì bút"} khi nói về \textbf{Truyền kỳ mạn lục} của Nguyễn Dữ. Tác phẩm được viết vào khoảng thế kỉ thứ XVI. \\
Trong sách giáo khoa thì ta được học 2 truyện trong tác phẩm này là \textit{"Chuyện người con gái Nam Xương"} và \textit{"Chuyện chức phán sự đền Tản Viên."}
\end{ans} \end{mybox}
\begin{ques}
Nhà văn nào được Nguyễn Minh Châu nhận xét là \textit{"một định nghĩa về người nghệ sĩ tài hoa"}?
\end{ques}
\begin{mybox} \begin{ans}
\textbf{Nguyễn Tuân} đã được Nguyễn Minh Châu nhận xét là \textit{"một định nghĩa về người nghệ sĩ tài hoa"}.\\
Một số bút danh khác của Nguyễn Tuân: Ân Ngũ Tuyên, Tuấn Thừa Sắc, ...\\
Một vài tác phẩm tiêu biểu: \textit{"Vang bóng một thời"}, Tùy bút \textit{"Sông Đà"}, Ký \textit{"Cô Tô"}, ...
\end{ans} \end{mybox}
\begin{ques}
Kim loại nào là kim loại có nhiệt độ nóng chảy cao nhất, được dùng làm dây tóc bóng đèn?
\end{ques}
\begin{mybox} \begin{ans}
\textbf{Wolfram}, còn có tên gọi khác là \textbf{Tungsten}, kí hiệu là \textbf{W}, là kim loại có nhiệt độ nóng chảy cao nhất, được dùng làm dây tóc bóng đèn.
\end{ans} \end{mybox}
\begin{ques}
Khi làm kem que người ta thường làm như sau: Cắm que tre vào ô đựng nước trái cây rồi đặt cả vào khay đá có đựng nước đá hòa tan nhiều muối ăn. Tất cả cho vào làm lạnh. Nước trái cây sẽ nhanh chóng đông lại thành kem que. Người ta đã lợi dụng tính chất gì khi dùng muối làm kem que?
\end{ques}
\begin{mybox} \begin{ans}
Bốn đáp án được đề thi đưa ra bao gồm:
\begin{itemize}
\item \textbf{Nhiệt độ của nước đá là \(0 ^ \circ \mathrm{C},\) nếu cho muối ăn, nhiệt độ sẽ giảm xuống dưới \(0 ^ \circ \mathrm{C}.\)}
\item Nhiệt độ phòng là \(25 ^ \circ \mathrm{C},\) nếu cho muối ăn vào nước đá, nhiệt độ phòng sẽ giảm xuống giúp kem chóng đông.
\item Muối ăn thu nhiệt cùng với độ lạnh của nước đá tác động làm trái cây nhanh chóng đông.
\item Muối ăn giúp duy trì nhiệt độ của nước đá ở \(0 ^ \circ \mathrm{C}\) giúp kem chóng đông.
\end{itemize}
Đáp án đúng là phương án đầu tiên.\\
Sau khi tham khảo trên mạng thì anh có thấy thêm một giải thích là nhắc tới tính chất \textit{tăng nhiệt độ sôi và giảm nhiệt độ đông đặc} của dung dịch: nhiệt độ đông đặc của nước đá là \(0 ^ \circ \mathrm{C},\) còn nhiệt độ đông đặc của nước muối là nhỏ hơn \(0 ^ \circ \mathrm{C}.\)
\end{ans} \end{mybox}
\begin{ques}
Dmitri Ivanovich Mendeleev - cha đẻ của bảng tuần hoàn các nguyên tố hóa học sinh ra và lớn lên ở thành phố nào của Nga ngày nay?
\end{ques}
\begin{mybox} \begin{ans}
Chẹp, câu này chắc là đề thi Olympia không có đâu.\\
Dmitri Ivanovich Mendeleev sinh ngày 08/02/1834 tại thành phố \textbf{Tobolsk} ngày nay. Đây là thành phố lớn thứ 179 của Nga.
\end{ans} \end{mybox}
\begin{ques}
Thủy ngân dễ bay hơi và rất độc. Nếu chẳng may nhiệt kế thủy ngân bị vỡ thì có thể dùng chất nào để khử độc và thu gom thủy ngân?
\end{ques}
\begin{mybox} \begin{ans}
Nếu chẳng may nhiệt kế thủy ngân bị vỡ vì ta có thể dùng \textbf{bột lưu huỳnh} để khử độc và thu gom thủy ngân, do lưu huỳnh và thủy ngân phản ứng với nhau ở nhiệt đô thường tạo ra \(\mathrm{HgS}.\)
\end{ans} \end{mybox}
\begin{ques}
Ông là nhà Toán học lỗi lạc của thế kỉ XIX, được mệnh danh là \textit{"Hoàng tử của các nhà Toán học"}. Ông là ai?
\end{ques}
\begin{mybox} \begin{ans}
\textit{"Hoàng tử của các nhà Toán học"} là \textbf{Carl Friderich Gauss}, nhà Toán học người Đức nè.
\end{ans} \end{mybox}
\begin{ques}
Bà Tư đi khoe khắp cả xóm rằng bà để dành được 1 000 000 đồng (một triệu đồng) gồm toàn những đồng xu mệnh giá 1000 đồng. Một hôm, thừa dịp bà Tư đi chợ, thằng Năm lẻn vào nhà lấy mất 750 000 (bảy mươi năm mươi ngàn đồng) từ chiếc hòm đựng của bà. Biết rằng cứ 30 giây nó đếm được 50 000 đồng. Hỏi cần ít nhất bao nhiêu thời gian để nó lấy được số tiền trên?
\end{ques}
\begin{mybox} \begin{ans}
Câu này anh thấy tất cả các bạn tham gia đều trả lời là 7 phút 30 giây (450 giây). Giờ chúng ta cùng đi thử xem có cách nào nhanh hơn không nha.\\
Trước hết phải nhận xét là \textbf{thằng trộm này rảnh thật}, sao không gom hết luôn đi mà phải đếm nữa?\\
Thay vì đếm 750 000 đồng, ta chỉ cần đếm số tiền là 1 000 000 - 750 000 = \textbf{250 000 đồng.} Sau đó thì ta trộm phần chưa đếm là được. \\
Có thể tính được thời gian đếm 250 000 đồng là \textbf{150 giây}, hoặc \textbf{2 phút 30 giây}.
\end{ans} \end{mybox}
\begin{ques}
Andrew Wiles là một Giáo sư Toán học người Mỹ. Cuối thế kỉ XX, ông được vinh danh như là người đầu tiên giải được trọn vẹn một bài toán đã tồn tại gần 4 thế kỉ. Đó là bài toán nào?
\end{ques}
\begin{mybox} \begin{ans}
Bài toán này là \textbf{định lí cuối cùng của Fermat}, còn được gọi là \textbf{định lí Fermat lớn} (do nó là cái duy nhất mà Fermat không chứng minh). Bài toán được phát biểu đơn giản như sau:
\begin{center}
Không có các số nguyên dương \(a,\) \(b,\) \(c\) nào thỏa mãn phương trình 
\[a^n + b^n = c^n,\]
với \(n > 2.\)
\end{center}
\end{ans} \end{mybox}
\begin{ques}
Những số có dạng \(2^{2^n} + 1\) với \(n\) là số tự nhiên được gọi là số Fermat. Fermat đã từng suy đoán rằng tất cả các số có dạng như vậy đều là số nguyên tố, nhưng sau đó Euler đã chỉ ra sai lầm của ông. Hỏi với \(n\) nhỏ nhất là bao nhiêu thì một số Fermat \textbf{không} phải là số nguyên tố?
\end{ques}
\begin{mybox} \begin{ans}
\(\mathbf{n = 5}\) là giá trị \(n\) nhỏ nhất để một số Fermat không là số nguyên tố.\\
\(2^{2^5} + 1\) là một bội của \(641.\)
\end{ans} \end{mybox}
\begin{ques}
Thế vận hội mùa hè năm 2020 được tổ chức tại quốc gia nào?
\end{ques}
\begin{mybox} \begin{ans}
Thế vận hội mùa hè 2020 (2020 Summer Olympics) được tổ chức tại thành phố Tokyo của \textbf{Nhật Bản.}
\end{ans} \end{mybox}
\begin{ques}
Môn thể thao nào được coi là \textit{"môn thể thao nữ hoàng"}?
\end{ques}
\begin{mybox} \begin{ans}
Môn thể thao vua là bóng đá, còn môn thể thao nữ hoàng là \textbf{điền kinh.}
\end{ans} \end{mybox}
\begin{ques}
Nam cầu thủ nào đã đoạt danh hiệu Cầu thủ xuất sắc nhất năm 2020 của FIFA?
\end{ques}
\begin{mybox} \begin{ans}
Giải thưởng cầu thủ xuất sắc nhất năm 2020 (FIFA The Best 2020) được trao cho cầu thủ người Ba Lan \textbf{Robert Lewandowski.}
\end{ans} \end{mybox}
\begin{ques}
Tính đến hết Wimbledon năm 2021, 3 tay vợt nào đang giữ kỉ lục giành 20 Grand Slam trong sự nghiệp?
\end{ques}
\begin{mybox} \begin{ans}
Với chức vô địch Wimbledon 2021, \textbf{Novak Djokovic} đã cân bằng kỉ lục 20 Grand Slam của \textbf{Roger Federer} và \textbf{Rafael Nadal.}
\end{ans} \end{mybox}
\begin{ques}
Hiện tượng các tia sáng bị lệch phương (gãy) khi truyền xiên góc qua mặt phân cách giữa hai môi trường trong suốt khác nhau gọi là gì?
\end{ques}
\begin{mybox} \begin{ans}
Hiện tượng này là \textbf{khúc xạ ánh sáng.} Hổng biết nói gì nữa vì đây là định nghĩa rồi.
\end{ans} \end{mybox}
\begin{ques}
Áp suất trong bình đựng chất lỏng lớn nhất ở vị trí nào của bình?
\end{ques}
\begin{mybox} \begin{ans}
Dựa vào công thức tính áp suất thì áp suất của nước ở \textbf{đáy bình} là lớn nhất.
\end{ans} \end{mybox}
\begin{ques}
\(1kg \times 1m \times 1{m \mathord{\left/
 {\vphantom {m {{s^2}}}} \right.
 \kern-\nulldelimiterspace} {{s^2}}}\) sẽ được đơn vị nào?
\end{ques}
\begin{mybox} \begin{ans}
Cái này nó thể coi nó như là \(m \cdot s \cdot a,\) nghĩa là \(F \cdot s,\) ra được công.\\
Vậy thì biểu thức này sẽ cho ta 1 \textbf{Jun} (J).
\end{ans} \end{mybox}
\begin{ques}
Hiện tượng nào chứng tỏ ánh sáng có tính chất hạt?
\end{ques}
\begin{mybox} \begin{ans}
Ở đây chúng ta sẽ nói luôn về lưỡng tính sóng \(-\) hạt của ánh sáng.\\
\textbf{Bản chất sóng} của ánh sáng có thể được chứng minh qua các hiện tượng sau:
\begin{itemize}
\item Hiện tượng tán sắc ánh sáng,
\item Hiện tượng nhiễu xạ ánh sáng,
\item Hiện tượng giao thoa ánh sáng.
\end{itemize}
\textbf{Bản chất hạt} của ánh sáng có thể được chứng minh qua các hiện tượng sau:
\begin{itemize}
\item Hiện tượng quang điện ngoài,
\item Hiện tượng quang điện trong,
\item Hiện tượng quang dẫn,
\item Hiện tượng quang trở,
\item Hiện tượng quang \(-\) phát quang.
\end{itemize}
Ngoài ra còn có thể có các hiện tượng khác.
\end{ans} \end{mybox}
\begin{ques}
Quyển sách Vật lý học đầu tiên trong lịch sử loài người do ai viết?
\end{ques}
\begin{mybox} \begin{ans}
Quyển sách này có tên là \textit{"Vật Lý học"}, được \textbf{Aristotle} viết vào thế kỉ thứ IV TCN.
\end{ans} \end{mybox}
\end{document}