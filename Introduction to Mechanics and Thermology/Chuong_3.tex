\chapter{Các định luật bảo toàn trong cơ học}
Trong quá trình chuyển động của chất điểm, có thể có một số đại lượng vật lý giữ nguyên không thay đổi theo thời gian, gọi là các đại lượng \textit{bảo toàn.}
\section{Định luật biến thiên và bảo toàn động lượng}
\subsection{Cho một chất điểm}
Người ta gọi động lượng $\overrightarrow{p}$ của một chất điểm khối lượng $m,$ chuyển động với vận tốc $\overrightarrow{v}$ là một vector được định nghĩa bằng tích số của $m$ và $\overrightarrow{v},$
$$\overrightarrow{p} = m \overrightarrow{v}.$$
Để xét sự biến thiên của động lượng theo thời gian, ta lấy đạo hàm của $\overrightarrow{p}$ theo biến $t,$ qua các phép biến đổi ta được:
$$\mathrm{d}\overrightarrow{p} = \overrightarrow{F} \mathrm{d}t.$$
Đại lượng $\overrightarrow{F} \mathrm{d}t$ được gọi là \textit{xung lượng} của lực $\overrightarrow{F}$ tác dụng lên chất điểm trong khoảng thời gian $\mathrm{d}t$ (còn gọi là \textit{xung lực}).\\
\textbf{\textit{Định luật biến thiên động lượng (Định lí động lượng):}} Độ biến thiên của động lượng $\overrightarrow{p}$ của chất điểm trong khoảng thời gian $\mathrm{d}t$ bằng xung lượng của ngoại lực tác dụng lên chất điểm trong thời gian đó.\\
\textbf{\textit{Định luật bảo toàn động lượng:}} Một chất điểm cô lập hoặc hợp lực tác dụng lên nó bằng không thì động lương của nó được bảo toàn.
\subsection{Cho hệ nhiều chất điểm}
Động lượng toàn phần của hệ nhiều chất điểm biến thiên và bảo toàn tương tự với một chất điểm.
\subsection{Ví dụ về định luật bảo toàn động lượng}
\begin{itemize}
\item Sự giật lùi của súng.
\item Chuyển động của vật có khối lượng thay đổi: chuyển động của con tàu vũ trụ.
\end{itemize}
\section{Định luật biến thiên và bảo toàn moment động lượng}
\subsection{Moment lực}
Moment của lực $\overrightarrow{F}$ đối với một chất điểm $O$ nào đó cho trước là một vector gốc $O,$ được xác định bởi tích có hướng của $\overrightarrow{r}$ và $\overrightarrow{F}.$ 
$$\overrightarrow{M} = \overrightarrow{r} \times \overrightarrow{F},$$
trong đó $\overrightarrow{r}$ là bán kính vector nối liền từ $O$ đến điểm đặt của lực $\overrightarrow{F}.$\\
Độ lớn của $\overrightarrow{M}$ được xác định bởi:
$$M = r F \sin \alpha.$$
Nếu $h$ là hình chiếu cảu $\overrightarrow{r}$ lên phương vuông góc với lực $\overrightarrow{F},$ thì $h = r \sin \alpha,$ thì
$$M = Fh.$$
\subsection{Moment động lượng của một chất điểm}
Moment của động lượng $\overrightarrow{p}$ đối với điểm $O$ nào đó cho trước và là một vector gốc $O$ được xác định bởi tích có hướng của $\overrightarrow{r}$ và $\overrightarrow{p}:$
$$ \overrightarrow{L} = \overrightarrow{r} \times \overrightarrow{p}.$$
\subsection{Định luật biến thiên và bảo toàn moment động lượng của chất điểm}
Đạo hàm công thức tính moment động lượng, ta có
$$\frac{{\mathrm{d} \overrightarrow L }}{\mathrm{d}t} = \frac{{\mathrm{d}}}{\mathrm{d}t}\left( {\overrightarrow r  \times  \overrightarrow p } \right) = \frac{{\mathrm{d}\overrightarrow r }}{\mathrm{d}t} \times \overrightarrow p  + \overrightarrow r  \times \frac{{\mathrm{d}\overrightarrow p }}{\mathrm{d}t}$$
$$ \Rightarrow \frac{{\mathrm{d}\overrightarrow L }}{\mathrm{d}t} = \overrightarrow v  \times \overrightarrow p  + \overrightarrow r  \times \overrightarrow F $$
$$ \Rightarrow \frac{{\mathrm{d}\overrightarrow L }}{\mathrm{d}t} = \overrightarrow 0  + \overrightarrow M  = \overrightarrow M $$
$$\Rightarrow \mathrm{d}\overrightarrow{L} = \overrightarrow{M}\mathrm{d}t.$$
Đại lượng $\overrightarrow{M}\mathrm{d}t$ được gọi là \textit{xung lượng} của moment lực $\overrightarrow{M}$ tác dụng lên chất điểm trong khoảng thời gian $\mathrm{d}t.$\\
\textbf{\textit{Định luât biến thiên moment động lượng:}} Độ biến thiên của moment động lượng của chất điểm trong khoảng thời gian $\mathrm{d}t$ bằng xung lượng của moment lực tác dụng lên chất điểm trong khoảng thời gian đó.\\
\textbf{\textit{Định luật bảo toàn moment động lượng:}} Mọi chất điểm cô lập hoặc moment lực tác dụng lên nó bằng không thì moment động lượng cảu nó được bảo toàn.
\subsection{Moment động lượng của một hệ các chất điểm}
Moment động lượng toàn phần của hệ nhiều chất điểm biến thiên và bảo toàn tương tự với một chất điểm.
\section{Định luật bảo toàn cơ năng}
\subsection{Công cơ học}
Công là đại lượng đặc trưng cho phần năng lượng chuyển đổi từ dạng năng lượng này sang dạng năng lượng khác, nói cách khác công là phần năng lượng trao đổi giữa các chất.\\
\textit{Công vi phân} $\delta A$ mà lực $\overrightarrow{F}$ thực hiện được trên đoạn đường $\mathrm{d}\overrightarrow{s}$ là tích vô hướng của hai vector:
$$\delta A = \overrightarrow{F} \cdot \mathrm{d}\overrightarrow{s},$$
hay
$$\delta A = F \mathrm{d}s \cos \alpha.$$
Đơn vị của công là Joule (J). \\
Lấy tích phân, ta có
$${A_{MN}} = \int\limits_M^N {\delta A}  = \int\limits_M^N {\overrightarrow F \mathrm{d}\overrightarrow s } .$$
\subsection{Động năng, định lí về động năng}
\subsubsection{Động năng}
Động năng của một chất điểm khối lượng $m$ có vận tốc $\overrightarrow{v}$ là một đại lượng vô hướng:
$$K = \frac{1}{2}mv^2.$$
Đơn vị của động năng là Joule (J).\\
Chữ K ở kí hiệu này là \textit{Kinetic energy}.
\subsubsection{Định lí về động năng}
Độ biến thiên của động năng trong một khoảng thời gian bằng công của \textit{\textbf{tất cả các lực}} đặt vào hệ thực hiện được trong khoảng thời gian đó:
$$A_{12} = K_2 - K_1.$$
\subsection{Trường lực thế, thế năng trong trường lực thế}
\subsubsection{Khái niệm về trường lực thế}
Một lực được gọi là lực thế (lực bảo toàn) nếu công do nó thực hiện trong sự chuyển dời một chất điểm chỉ phụ thuộc vào vị trí đầu và vị trí cuối mà không phụ thuộc dạng quỹ đạo giữa hai điểm này (có nghĩa là độc lập với lộ trình).\\
Lực hấp dẫn, lực phục hồi của lò xo, ... là những ví dụ về lực thế.
\subsubsection{Thế năng trong trường lực thế}
\textit{Định nghĩa thế năng.} Thế năng tại điểm $M$ trong trường lực thế là công làm dịch chuyển chất điểm từ vị trí $M$ đến điểm gốc của thế năng.\\
\textit{Định lí về thế năng.}
Công làm dịch chuyển chất điểm giữa hai điểm của trường thế bằng hiệu của thế năng giữa chất điểm đầu và cuối của quá trình chuyển động:
$$A_{MN}^ *  = {U_M} - {U_N}.$$
Ký hiệu $A^*$ để chỉ công của lực trường thế.
\subsubsection{Liên hệ giữa thế năng và lực trường thế}
$$\overrightarrow{F} = - \nabla U.$$
$\overrightarrow{F}$ bằng và trái dấu với gradient của thế năng $U.$
\subsection{Các lực bảo toàn và phi bảo toàn}
Lực đàn hồi, lực hấp dẫn là các luật dẫn xuất từ thế và còn gọi là các lực bảo toàn.\\
Một lực không có tính chất bảo toàn được gọi là lực phi bảo toàn. Một số lực phi bảo toàn như lực ma sát, lực nhớt của chất lưu làm tiêu tán một phần cơ năng của vật hay ta nói rằng làm tiêu tốn cơ năng, do đó các lực này còn được gọi là lực tiêu tán.
\subsection{Định luật biến thiên và bảo toàn cơ năng}
Độ biến thiên của cơ năng chất điểm bằng công của \textbf{\textit{lực phi bảo toàn.}}
$$E_2 - E_1 = A_{PBT}.$$
Trong trường hợp không có lực phi bảo toàn: thế năng và động năng của chất điểm sẽ biến đổi qua lại sao cho tổng động năng và thế năng là hằng, hay nói cách khác, cơ năng của chất điểm là hằng.
$$E = U + K = \mathrm{const}.$$
Đối với hệ vật, cơ năng của hệ bảo toàn khi ngoại lực là lực bảo toàn.
\section{Trường hấp dẫn}
\subsection{Lực hấp dẫn}
Lực hấp dẫn giữa hai chất điểm có khối lượng $m_1, m_2$ đặt cách nhau một khoảng $r$ có độ lớn
$$F = G \frac{m_1 m_2}{r^2},$$
với $G$ là hằng số hấp dẫn, trong hệ SI, $G \approx 6.67 \cdot 10^{-11} \mathrm{\frac{Nm^2}{kg^2}}.$
\subsection{Trường hấp dẫn}
Lực hấp dẫn giữa hai khối lượng là một lực tương tác từ xa, có nghĩa là hai vật tương tác không tiếp xúc. Trường hấp dẫn tác dụng lên tất cả các chất điểm trong không gian bao chung quanh Trái Đất là hình ảnh của hiện tượng tương tác hấp dẫn.
\subsection{Thế năng trong trường hấp dẫn}
Thế năng của vật ở gần mặt đất là
$$U \left( r \right) = mgh.$$
\section{Bài toán va chạm giữa hai vật}
\subsection{Định nghĩa} 
Là hiện tượng hai vật tiếp xúc với nhau trong một thời gian cực ngắn rồi tách rời nhau. Sự va chạm làm thay đổi vận tốc trong một thời gian ngắn, sự thay đổi có thể chia làm hai giai đoạn:
\begin{itemize}
\item Các vật va chạm bị biến dạng và ngừng lại, động năng giảm đi để cung cấp công làm vật va chạm biến dạng.
\item Các vật va chạm có thể trở lại dạng cũ và được hoàn lại động năng một phần hay tất cả.
\end{itemize}
\subsection{Các loại va chạm}
\begin{itemize}
\item Va chạm đàn hồi là va chạm trong đó các vật va chạm bị biến dạng, động năng lúc đó chuyển hoàn toàn thành thế năng đàn hồi, thế năng này liền chuyển trở lại động năng và các quả cầu bật lìa xa nhau. Sau đó hình dạng của chúng được phục hồi như cũ. Trong va chạm đàn hồi, động lượng và cơ năng được bảo toàn, mà thế năng không đổi nên ta có thể xem \textbf{\textit{động lượng và động năng bảo toàn.}}
\item Va chạm hoàn toàn không đàn hồi (va chạm mềm) là va chạm mà các vật sau va chạm dính vào nhau và chuyển động cùng vận tốc, \textbf{\textit{động lượng bảo toàn, cơ năng không bảo toàn.}}
\end{itemize}
\subsubsection{Va chạm đàn hồi giữa hai quả cầu}
$$\overrightarrow {{v_1}^\prime }  = \frac{{\left( {{m_1} - {m_2}} \right)\overrightarrow {{v_1}}  + 2{m_2}\overrightarrow {{v_2}} }}{{{m_1} + {m_2}}}.$$
$$\overrightarrow {{v_2}^\prime }  = \frac{{\left( {{m_2} - {m_1}} \right)\overrightarrow {{v_2}}  + 2{m_1}\overrightarrow {{v_1}} }}{{{m_1} + {m_2}}}.$$
\subsubsection{Va chạm hoàn toàn không đàn hồi (va chạm mềm)}
$$\overrightarrow v  = \frac{{{m_1}\overrightarrow {{v_1}}  + {m_2}\overrightarrow {{v_2}} }}{{{m_1} + {m_2}}}.$$
Năng lượng tiêu hao sau va chạm:
$$Q = \frac{{{m_1}{m_2}}}{{2\left( {{m_1} + {m_2}} \right)}}{\left( {\overrightarrow {{v_1}}  - \overrightarrow {{v_2}} } \right)^2}.$$