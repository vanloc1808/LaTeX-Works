\chapter{Nguyên lý thứ nhất nhiệt động học}
\section{Trạng thái cân bằng và quá trình cân bằng}
Trạng thái cân bằng của hệ là trạng thái mà các thông số trạng thái của hệ có giá trị hoàn toàn xác định, ngược lại khi các thông số trạng thái đang thay đổi tức là hệ ở trạng thái không cân bằng. Nếu hệ là một khối khí thì mỗi trạng thái cân bằng của nó đucợ xác định bởi hai trong ba thông số là $p, V, T.$\\
Quá trình cân bằng là quá trình biến đổi gồm một chuỗi liên tiếp các trạng thái cân bằng.
\section{Khái niệm về năng lượng, công, nhiệt lượng}
\subsection{Năng lượng}
Năng lượng của một hệ là đại lượng vật lý có thể dùng để chỉ mức độ vận động của hệ (động năng), mức độ tương tác của hệ với môi trường ngoài (thế năng) và khả năng tương tác lẫn nhau của các hạt tạo thành hệ (nội năng). Thông thường các đối tượng nghiên cứu xem là đứng yên và bỏ qua các trường ngoài, nghĩa là động năng và thế năng của hệ bằng không. Vậy \textit{năng lượng của hệ chính là nội năng của nó}.\\
Quan sát thực nghiệm, ta thu được \textit{nội năng (năng lượng) là hàm của trạng thái}.
Đơn vị của nội năng là Joule ($\mathrm{J}$) hoặc calories. 
\subsection{Công}
Nếu ta giả thiết là khối khí đứng yên thì khái niệm công đối với chất khí được xác định như sau: Lực tác dụng lên chất khí được xem là thực hiện một công nếu là \textit{thể tích chất khí thay đổi}. Do đó, khái niệm công gắn liền với quá trình biến đổi thể tích.\\
\textit{Công là hàm của quá trình.}
\subsubsection{Quy ước}
\begin{itemize}
\item Nếu hệ nhận công từ bên ngoài thì $A$ dương.
\item Nếu hệ sinh công thì $A$ âm.
\end{itemize}
\subsubsection{Biểu thức tính công trong một quá trình cân bằng}
Công nhỏ $\delta A:$
$$\delta A = -p \mathrm{d}V.$$
Công lớn $A:$ 
$$A =  - \int\limits_{{V_1}}^{{V_2}} {p \mathrm{d} V} .$$
\subsection{Nhiệt lượng}
Nhiệt lượng chỉ tồn tại khi có một quá trình biến đổi xảy ra.\\
\textit{Nhiệt lượng là hàm của quá trình.}
\subsubsection{Quy ước}
\begin{itemize}
\item Nếu hệ nhận nhiệt từ bên ngoài thì $Q$ dương.
\item Nếu hệ tỏa nhiệt thì $Q$ âm.
\end{itemize}
\subsubsection{Biểu thức tính nhiệt lượng trong một quá trình cân bằng}
Nhiệt lượng nhỏ $\delta Q:$
$$\delta Q = \frac{M}{\mu} C \mathrm{d}T.$$
Nhiệt lượng lớn $Q:$
$$Q = \frac{M}{\mu} C \Delta T.$$
\section{Nguyên lý thứ nhất nhiệt động học}
\subsection{Phát biểu và biểu thức}
\subsubsection{Phát biểu}
Độ biến thiên nội năng (năng lượng) của một hệ trong một quá trình biến đổi bằng tổng công và nhiệt lượng mà hệ nhận vào trong quá trình đó.
\subsubsection{Biểu thức}
Nếu quá trình nhỏ, độ biến thiên nội năng:
$$\mathrm{d}U = \delta A + \delta Q.$$
Quá trình hữu hạn:
$$\Delta U = A + Q.$$
Nếu hệ thực hiện một quá trình khép kín (một chu trình), nghĩa là quá trình mà trạng thái cuối cùng trùng với trạng thái đầu và nội năng là hàm trạng thái, thì $U_1 = U_2.$ Trong trường hợp này
$$\Delta U = A + Q = 0 \Rightarrow A = - Q.$$
\begin{itemize}
\item Nếu hệ nhận công $\left( {A > 0} \right)$ thì tỏa ra nhiệt lượng $Q$ $\left( {Q < 0} \right),$ có nghĩa là \textbf{môi trường bên ngoài} nhận được một nhiệt lượng $Q' = - Q > 0.$
\item Ngược lại, nếu hệ nhận nhiệt $\left( {Q > 0} \right)$ thì sinh công $A$ $\left( {A < 0} \right),$ có nghĩa là \textbf{môi trường bên ngoài} nhận được một công $A' = - A > 0.$
\end{itemize}
\subsubsection{Động cơ vĩnh cửu loại một}
Người ta gọi một động cơ có khả năng sinh ra công mà không cần nhận năng lượng ở đầu vào là \textit{động cơ vĩnh cửu loại một.}\\
Từ nguyên lý thứ nhất, ta có thể kết luận rằng không thể nào ché tạo được động cơ vĩnh cửu loại một.
\subsection{Ứng dụng ngueyen lý thứ nhất nhiệt động học để nghiên cứu các quá trình biến đổi của khí lý tưởng}
\subsubsection{Quá trình đẳng tích ($V = \mathrm{const}$)}
\textit{\textbf{Công mà hệ nhận được:}}
$$A =  - \int\limits_{{V_1}}^{{V_2}} {p \mathrm{d} V} ,$$
do $V = \mathrm{const}$ nên $\mathrm{d}V = 0,$ nghĩa là trong quá trình đẳng tích, công mà hệ nhận được là 
$$A = 0.$$
\textit{\textbf{Độ biến thiên nội năng:}}
$$ \Delta U = \frac{M}{\mu} \frac{i}{2} R \Delta T.$$
\textit{\textbf{Nhiệt lượng mà hệ nhận được:}}
$$\Delta U = A + Q$$
$$\Leftrightarrow Q = \Delta U - A = \Delta U.$$
$$\Leftrightarrow Q = \Delta U = \frac{M}{\mu} \frac{i}{2} R \Delta T.$$
Đặt $C_v = \frac{iR}{2},$  là \textit{nhiệt dung riêng phân tử đẩng tích.} Khi đó
$$Q = \frac{M}{\mu} C_v \Delta T.$$
\subsubsection{Quá trình đẳng áp ($p = \mathrm{const}$)}
\textbf{\textit{Công mà hệ nhận được:}}
$$A =  - \int\limits_{{V_1}}^{{V_2}} {p\mathrm{d}V} $$
Do $p = \mathrm{const}$ nên: 
$$A =  - p\int\limits_{{V_1}}^{{V_2}} {\mathrm{d}V}  =  - p\left( {{V_2} - {V_1}} \right).$$
Vậy công mà hệ nhận được trong quá trình đẳng áp là:
$$A = p \left( {V_1 - V_2} \right).$$
\textbf{\textit{Độ biến thiên nội năng:}}
$$ \Delta U = \frac{M}{\mu} \frac{i}{2} R \Delta T.$$
\textbf{\textit{Nhiệt lượng hệ nhận được:}}
$$Q = \Delta U - A = \frac{M}{\mu } \cdot \frac{{iR}}{2}\Delta T + p\left( {{V_2} - {V_1}} \right)$$
$$ \Leftrightarrow Q = \frac{M}{\mu } \cdot \frac{{iR}}{2}\Delta T + \frac{M}{\mu }R{T_2} - \frac{M}{\mu }R{T_1}$$
$$ \Leftrightarrow Q = \frac{M}{\mu } \cdot \frac{{iR}}{2}\Delta T + \frac{M}{\mu }R\Delta T$$
$$ \Leftrightarrow Q = \frac{M}{\mu }R\Delta T\left( {\frac{i}{2} + 1} \right)$$
$$ \Leftrightarrow Q = \frac{M}{\mu }\frac{{i + 2}}{2}R\Delta T.$$
Đặt $C_p = \frac{i + 2}{2} R.$ Khi đó, nhiệt lượng hệ nhận được là:
$$Q = \frac{M}{\mu }{C_p}\Delta T.$$
Suy ra hệ thức $$C_p - C_v = R.$$
Đặt 
$$\gamma = \frac{C_p}{C_v} = \frac{i + 2}{i},$$
gọi là \textit{hệ số Poisson.}
\subsubsection{Quá trình đẳng nhiệt ($T = \mathrm{const}$)}
\textbf{\textit{Công mà hệ nhận được:}}
$$A =  - \int\limits_{{V_1}}^{{V_2}} {p \mathrm{d} V} $$
Do quá trình đẳng nhiệt nên $pV = p_1 V_1 = p_2 V_2,$ vì vậy:
$$A =  - \int\limits_{{V_1}}^{{V_2}} {p\mathrm{d}V}  =  - {p_1}{V_1}\int\limits_{{V_1}}^{{V_2}} {\frac{1}{V}} \mathrm{d}V =  - {p_1}{V_1}\ln \frac{{{V_2}}}{{{V_1}}}$$
Vậy công mà hệ nhận được là:
$$A =  - \frac{M}{\mu }RT\ln \frac{{{V_2}}}{{{V_1}}}.$$
\textbf{\textit{Độ biến thiên nội năng:}}
$$ \Delta U = \frac{M}{\mu} \frac{i}{2} R \Delta T$$
Mà $\Delta T = 0$ nên
$$\Delta U = 0.$$
\textbf{\textit{Nhiệt lượng hệ nhận được:}}
$$Q = \Delta U - A = - A$$
$$\Leftrightarrow Q = \frac{M}{\mu }RT\ln \frac{{{V_2}}}{{{V_1}}}.$$
\subsubsection{Quá trình đoạn nhiệt}
Quá trình đoạn nhiệt là một quá trình mà trong đó không có sự truyền nhiệt vào trong cũng như mất nhiệt ra khỏi hệ nhiệt động đang xét. Nói cách khác, quá trình đoạn nhiệt là một quá trình hoàn toàn cách nhiêt ($Q = 0$).\\
Theo nguyên lý thứ nhất, bằng các phép biến đổi, ta được:
$$p V^\gamma = \mathrm{const}.$$
Phương trình trên được gọi là \textit{phương trình Poisson đối với quá trình đoạn nhiệt.}
Suy ra các phương trình:
$$TV^{\gamma - 1} =\mathrm{const},$$
$$PT^{\frac{\gamma}{1 - \gamma}} = \mathrm{const},$$
$$TP^{\frac{1 - \gamma}{\gamma}} = \mathrm{const}.$$
\textbf{\textit{Độ biến thiên nội năng:}}
$$ \Delta U = \frac{M}{\mu} \frac{i}{2} R \Delta T.$$
\textbf{\textit{Công mà hệ nhận được:}}\\
Bằng vài phép biến đổi, áp dụng công thức $pV = \frac{M}{\mu}RT,$ ta thu được công thức tính công mà hệ nhận được:
$$A = \frac{p_2 V_2 - p_1 V_1}{\gamma - 1}.$$