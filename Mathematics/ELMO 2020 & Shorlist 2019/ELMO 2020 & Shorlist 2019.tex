\documentclass[12pt,a4paper]{article}
\usepackage[utf8]{vietnam}
\usepackage{amsmath}
\usepackage{amsfonts}
\usepackage{amssymb}
\usepackage{graphicx}
\usepackage[left=2cm,right=2cm,top=2cm,bottom=2cm]{geometry}
\begin{document}
\begin{center}
\textbf{\fontsize{16}{14}\selectfont
\(22^{nd}\) Experimental Lincoln Math Olympiad \\
Year: 2020}
\end{center}
\begin{flushright}
\textit{Day 1: Monday, July 20, 2020}\\
\textit{2:00 PM \(-\) 6:30 PM EDT}
\end{flushright}
\begin{center}
\textit{Time: 4 hours 30 minutes.\\
Each problem is worth 7 points.}
\end{center}
\textbf{Problem 1.} Let \(\mathbb{N}\) be the set of all positive integers. Find all function: \(f: \mathbb{N}  \to \mathbb{N} \) such that
\[{f^{{f^{f\left( x \right)}}\left( y \right)}}\left( z \right) = x + y + z + 1\]
for all \(x, y, z \in \mathbb{N}.\)\\
Here, \({f^a}\left( b \right)\) denotes the result of repeated applications off \(f\) to \(b.\) Formally, we define \({f^1}\left( b \right) = f\left( b \right),\) and \({f^{a + 1}}\left( b \right) = f\left( {{f^a}\left( b \right)} \right)\) for all \(a > 0.\)\\
\newline
\textbf{Problem 2.} Define the Fibonacci numbers by \(F_1 = F_2 = 1\) and \(F_n = F_{n-1} + F_{n-2}\) for \(n \ge 3.\) Let \(k\) be a positive integer. Suppose that for every positive integer \(m\) there exists a positive integer \(n\) such that \(m|{F_n} - k.\) Must \(k\) be a Fibonacci number?\\
\newline
\textbf{Problem 3.} Janabel has a device that, when give two distinct points \(U\) and \(V\) in the plane, draws the perpendicular bisector of \(UV.\) Show that if three lines forming a triangle are drawn, Janabel can mark the orthocenter of the triangle using this device, a pencil, and no other tools.
\newpage
\begin{center}
\textbf{\fontsize{16}{14}\selectfont
\(22^{nd}\) Experimental Lincoln Math Olympiad \\
Year: 2020}
\end{center}
\begin{flushright}
\textit{Day 2: Monday, July 21, 2020}\\
\textit{2:00 PM \(-\) 6:30 PM EDT}
\end{flushright}
\begin{center}
\textit{Time: 4 hours 30 minutes.\\
Each problem is worth 7 points.}
\end{center}
\textbf{Problem 4.} Let acute scalene triangle \(ABC\) have orthocenter \(H\) and altitude \(AD\) with \(D\) on side \(BC.\) Let \(M\) be the midpoint of side \(BC\), and let \(D'\) be the reflection of \(D\) over \(M.\) Let \(P\) be a point on line \(D'H\) such that lines \(AP\) and \(BC\) are parallel, and let the circumcircles  of \(\bigtriangleup AHP\) and \(\bigtriangleup BHC\) meet again at \(G \ne H.\) Prove that  \(\angle MHG = {90^ \circ }.\)\\
\newline
\textbf{Problem 5.} Let \(m\) and \(n\) be positive integers. Find the smallest positive integer \(s\) for which there exists an \(m \times n\) rectangular array of positive integers such that\\
i. each row contains \(n\) distinct consecutive integers in some order, \\
ii. each column contains \(m\) distinct consecutive integers in some order, and\\
iii. each entry is less than or equal to \(s.\)\\
\newline
\textbf{Problem 6.} For any positive integer \(n\), let \\
i. \(\tau \left( n \right)\) denote the number of positive integer divisors of \(n\),\\
ii. \(\sigma \left( n \right)\) denote the sum of the positive integer divisors of \(n\), and\\
iii. \(\varphi \left( n \right)\) denote the number of positive integers less than or equal to \(n\) that are relatively prime to \(n.\)\\
Let \(a, b > 1\) be integers. Brandon has a calculator with three buttons that replace the integer \(n\) currently displayed with \(\tau \left( n \right)\), \(\sigma \left( n \right)\) or \(\varphi \left( n \right)\), respectively. Prove that if the calculator currently displays \(a,\) then Brandon can make the calculator display \(b\) after \(a\) finite (possibly empty) sequence of button presses.
\newpage
\begin{center}
\fontsize{18}{16}\selectfont
\textbf{Shorlisted Problems}\\
\fontsize{16}{14}\selectfont
\textbf{\({21^{st}}\) ELMO\\
Pittsburgh, PA, 2019}
\end{center}
\section*{Algebra}
\textbf{A1.} Let \(a, b, c\) be positive reals such that \(\frac{1}{a} + \frac{1}{b} + \frac{1}{c} = 1.\) Show that
\[{a^a}bc + {b^b}ca + {c^c}ab \ge 27 \left(ab + bc + ca \right).\]
\rightline{\textit{Milan Haiman}}\\
\textbf{A2.} Find all functions: \(f: \mathbb{Z} \to \mathbb{Z}\) with the property that for any surjective function \(g: \mathbb{Z} \to \mathbb{Z}\), the function \(f + g\) is also surjective.\\
\rightline{\textit{Sean Li}}\\
\textbf{A3.} Let \(n \ge 3\) be a fixed positive integer. Evan has a convex \(n\)-gon  in the plane and wishes to construct the centroid of its vertices. He has no standard ruler or compass, but he does have a device with which he can dissect the segment between two given points into \(m\) equal parts. For which \(m\) can Evan necessarily accomplish his task?\\
\rightline{\textit{Holden Mui and Carl Schildkraut}}\\
\textbf{A4.} Find all nondecreasing functions \(f: \mathbb{R} \to \mathbb{R}\) such that for all real numbers \(x, y\),
\[f\left( {f\left( x \right)} \right) + f\left( y \right) = f\left( {x + f\left( y \right)} \right) + 1.\]
\rightline{\textit{Carl Schildkraut}}\\
\textbf{A5.} Define the set of functional expressions to be the smallest set of expressions so that the following properties hold:\\
i. Any variable \(x,\) or any fixed real number, is a functional expression.\\
ii. Given any functional expression \(V,\) the expression \(f \left( V \right)\) is a functional expression, and given any two functional expressions \(V, W,\) the expressions \(V + W\) and \(V . W\) are functional expressions.\\
A functional equation is an equation of the form \(V = 0\) for any functional expression \(V:\) a function \textit{satisfies} it if that equation holds for all choices of each \(x,\) in the real numbers. \\
(For example, the equation \(f \left( x_1 \right) + f \left( x_2 \right) + \left( -1 \right) \left( {x_1} + {x_2} \right) = 0\) is a functional equation satisfied by only the identity function, while the equation \(f \left( x_1 \right) + f \left( x_2 \right) + \left( -1 \right) \left( {x_1} + {x_2} \right) = 0\) is a functional equation satisfied by infinitely many functions. The equation \(f\left( {\frac{1}{{1 + x_1^2}}} \right) = 0\) is not a functional equation at all.)\\
Does there exist a functional equation satisfied by a exactly one function \(f,\) and the function satisfies \(f \left( \mathbb{R} \right) = \mathbb{Z}?\)\\
\rightline{\textit{Carl Schildkraut}}\\
\newpage
\section*{Combinatorics}
\textbf{C1.} Let \(n \ge 3\) be fixed positive integer. Elmo is playing a game with his clone. Initially, \(n \ge 3\) points are given on a circle. On a player's turn, that player must draw a triangle using three unused points as vertices, without creating any crossing edges. The first player who cannot move loses. If Elmo's clone gones first and players alternate turns, which player wins for each \(n?\)\\
\rightline{\textit{Milan Haiman}}\\
\textbf{C2.} Adithya and Bill are playing a game on a connected graph with \(n > 2\) vertices and \(m\) edges. First, Adithya labels two of the vertices \(A\) and \(B,\) so that \(A\) and \(B\) are distinct and non-adjacent, and announces his choice to Bill. Then Adithya starts on vertex \(A\) and Bill starts on \(B.\)\\
Now the game proceeds in a series of rounds in which both players move simultaneously. In each round, Bill must move to an adjacent vertex, while Adithya may either move to an adjacent vertex or stay at his current vertex. Adithya loses if he is ever on the same vertex as Bill, and wins if he reaches \(B\) alone. Adithya cannot see where Bill is, but Bill can see where Adithya is.\\
Given that Adithya has a winning strategy, what is the maximum possible value of \(m,\) in terms of \(n?\)\\
\rightline{\textit{Steven Liu}}\\
\textbf{C3.} In the game of Ring Mafia, there are 2019 counters arranged in a circle, 673 of these which are mafia, and the remaining 1346 which are town. Two players, Tony and Madeline, take turns with Tony going first. Tony does not know which counters are mafia but Madeline does.\\
On Tony's turn, he selects any subset of the counters (possibly the empty set) and removes all counters in that set. On Madeline's turn, she selects a town counter which is adjacent to a mafia counter and remove it. (Whenever counters are removed, the remaining counters are brought closer together without changing their order so that they still form a circle). The game ends when either all mafia counters have been removed, or all town counters have been removed.\\
Is there a strategy for Tony that guarantees, no matter where the mafia counters are placed and what Madeline does, that at least one town counter remains at the end of the game?\\
\rightline{\textit{Andrew Gu}}\\
\textbf{C4.} Let \(n \ge 3\) be a positive integer. In a game, \(n\) players sit in a circle in that order. Initially, a deck of \(3n\) cards labeled \(\left\{ {1,...,3n} \right\}\) is shuffled and distributed among the players so that every player holds \(3\) cards in their hand. Then, every hour, every player simultaneously gives the smallest card in their hand to their left neighbor, and the largest card in their hand to their right neighbor. (Thus after each exchange, each player still has exactly \(3\) cards.)\\
Prove that each player's hand after the first \(n - 1\) exchanges is their same as their hand after the first \(2n - 1\) exchanges.\\
\rightline{\textit{Carl Schildkraut and Colin Tang}}\\
\textbf{C5}. Given a permutation of \(1, 2, 3, ..., n,\) with consecutive elements of \(a, b, c\) (in that order), we may perform either of the moves:\\
i. If \(a\) is the median of \(a, b\) and \(c\), we may replace \(a, b, c\) with \(b, c, a\) (in that order).\\
ii. If \(c\) is the median of \(a, b\) and \(c\), we may replace \(a, b, c\) with \(c, a, b\) (in that order).\\
What is the least number of sets in a partition of all \(n \! \) permutations, such that any two permutations in the same set are obtainable from each other by a sequence of moves?\\
\rightline{\textit{Milan Haiman}}\\
\newpage
\section*{Geometry}
\textbf{G1.} Let \(ABC\) be an acute triangle with orthocenter \(H\) and circumcircle \(\Gamma\). Let \(BH\) intersect \(AC\) at \(E\), and let \(CH\) intersect \(AB\) at \(F.\) Let \(AH\) intersect \(\Gamma\) again at \(P \ne A.\) Let \(PE\) intersect \(\Gamma\) again at \(Q \ne P.\) Prove that \(BQ\) bisects segment \(\overline{EF}.\)\\
\rightline{\textit{Luke Robitaille}}\\
\textbf{G2.} Snorlax is given three pairwise non-parallel lines \({\ell _1},{\ell _2},{\ell _3}\) and a circle \(\omega\) in the plane. In addition to a normal straightedge, Snorlex has a special straightedge which takes a line \(\ell\) and a point \(P\) and constructs a new line \(\ell '\) passing through \(P\) parallel to \(\ell.\) Determine if it is always possible for Snorlax to construct a triangle \(XYZ\) such that the side of \(\bigtriangleup XYZ\) are parallel to \({\ell _1},{\ell _2},{\ell _3}\) in some order, and \(X, Y, Z\) each lie on \(\omega.\)\\
\rightline{\textit{Vincent Huang}}\\
\textbf{G3.} Let \(\bigtriangleup ABC\) be an acute triangle with incenter \(I\) and circumcenter \(O.\) The incircle touches sides \(BC, CA,\) and \(AB\) at \(D, E,\) and \(F,\) respectively, and \(A'\) is the reflection of \(A\) over \(O.\) The circumcircles of \(ABC\) and \(A'EF\) meet at \(G,\) and the circumcircles of \(AMG\) and \(A'EF\) meet at a point \(H \ne G,\) where \(M\) is the midpoint of \(EF.\) Prove that if \(GH\) and \(EF\) meet at \(T,\) then \(DT \bot EF.\)\\
\rightline{\textit{Ankit Bisain}}\\
\textbf{G4.} Let triangle \(ABC\) have altitudes \(\overline{BE}\) and \(\overline{CF}\) which meet at \(H.\) The reflection of \(A\) over \(BC\) is \(A'.\) The circumcircles of \(\bigtriangleup AA'E\) and \(\bigtriangleup AA'F\) meet the circumcircle of \(\bigtriangleup ABC\) at \(P \ne A\) and \(Q \ne A,\) respectively. Lines \(BC\) and \(PQ\) meet at \(R.\) Prove that \(\overline {EF} \parallel \overline {HR} .\)\\
\rightline{\textit{Daniel Hu}}\\
\textbf{G5.} Given a triangle \(ABC\) for which \(\angle BAC \ne {90^ \circ},\) let \({B_1}, {C_1}\) be variable points on \(AB, AC,\) respectively. Let \({B_2}, {C_2}\) be the points on line \(BC\) such that a spiral similarity centered at \(A\) maps \({B_1}{C_1}\) to \({C_2}{B_2}.\) Denote the circumcircle of \(A{B_1}{C_1}\) by \(\omega.\) Show that if \({B_1}{B_2}\) and \({C_1}{C_2}\) concur on \(\omega\) at a point distinct from \(B_1\) and \(C_1\), then \(\omega\) passes through a fixed point other than \(A.\)\\
\rightline{\textit{Maxwell Jiang}}\\
\textbf{G6.} Let \(ABC\) be an acute scalene triangle and let \(P\) be a point in the plane. For any point \(Q \ne A, B, C,\) define \(T_A\) to be the unique point such that \(\bigtriangleup {T_A}BP \sim \bigtriangleup {T_A}QC\) and \(\bigtriangleup {T_A}BP, \bigtriangleup {T_A}QC\) are oriented in the same direction (clockwise or counter clockwise). Similarly define \({T_B}, {T_C}\).\\
(a). Find all \(P\) such that there exists a point \(Q\) with \({T_A}, {T_B}, {T_C}\) all lying on the circumcircle of \(\bigtriangleup ABC.\) Call such a pair \(\left( P, Q \right)\) \textit{a tasty pair} with respect to \(\bigtriangleup ABC.\)\\
(b). Keeping the notations form (a), determine if there exists a tasty pair which is also tasty with respect to \(\bigtriangleup {T_A}{T_B}{T_C}.\)\\
\rightline{\textit{Vincent Huang}}\\
\newpage
\section*{Number Theory}
\textbf{N1.} Let \(P\) be a polynomial with integer coefficients so that \(P \left( 0 \right) = 1.\) Let \({x_0} = 0,\) and let \({x_{i + 1}} = P \left( x_i \right)\) for all \(i \ge 0.\) Show that there are infinitely many positive integers \(n\) so that \(\gcd \left( {{x_n},n + 2019} \right) = 1.\)\\
\rightline{\textit{Carl Shcildkraut and Milan Haiman}}\\
\textbf{N2.} Let \(f: {\mathbb{Z}_{>0}} \to {\mathbb{Z}_{>0}}\) be a function. Prove that the following two conditions are equivalent:\\
i. \(f \left( m \right) + n\) divides  \(f \left( n \right) + m\) for all positive integers \(m \le n;\)\\
ii.  \(f \left( m \right) + n\) divides  \(f \left( n \right) + m\) for all positive integers \(m \ge n;\)\\
\rightline{\textit{Carl Shcildkraut}}\\
\textbf{N3.} Let \(S\) be a nonempty set of integers so that, for any (not necessarily distinct) integers \(a\) and \(b\) in \(S,\) \(ab + 1\) is also in \(S.\) Show that there are finitely many (possibly zero) primes which do not divide any element of \(S.\)\\
\rightline{\textit{Carl Shcildkraut}}\\
\textbf{N4.} A positive integers \(b \ge 2\) and a sequence \({a_0}, {a_1}, {a_2},...\) of base-\(b\) digits \(0 \le {a_i} < b\) is given. It is known that \({a_0} \ne 0\) and the sequence \(\left\{ {{a_i}} \right\}\) is eventually periodic but has infinitely many nonzero terms. Let \(S\) be the set of positive integers \(n\) so that the base-\(b\) number \({\left( {{a_0}{a_1}...{a_n}} \right)_b}\) is divisible by \(n.\) Given that \(S\) is infinite, show that there are infinitely many primes dividing at least one element of \(S.\)\\
\rightline{\textit{Carl Shcildkraut and Holden Mui}}\\
\textbf{N5.} Let \(m\) be a fixed even positive integer. Find all positive integers \(n\) for which there exists a bijection \(f\) from \(\left\{ {1,....,n} \right\}\) to itself such that for all \(x,y \in \left\{ {1,....,n} \right\}\) with \(mx - y\) divisible by \(n,\) we also have \[\left( {n + 1} \right)|f{\left( x \right)^m} - f\left( y \right).\]
\rightline{\textit{Milan Haiman and Carl Shcildkraut}}\\
\end{document}